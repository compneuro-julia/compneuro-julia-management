\section{Leaky integrate-and-fire モデル}\subsection{LIFモデルの定義}生理学的なイオンチャネルの挙動は考慮せず, 入力電流を膜電位が閾値に達するまで時間的に積分するというモデルを\textbf{Integrate-and-fire (IF, 積分発火)モデル} という.さらに, IFモデルにおいて膜電位の漏れ(leak)[^leak]も考慮したモデルを \textbf{Leaky integrate-and-fire (LIF, 漏れ積分発火) モデル} と呼ぶ.ここではLIFモデルのみを取り扱う.
ニューロンの膜電位を$V_m(t)$, 静止膜電位を$V_\text{rest}$, 入力電流[^isyn]を$I(t)$, 膜抵抗を$R_m$, 膜電位の時定数を$\tau_m\ (=R_m \cdot C_m)$とすると, 式は次のようになる[^vrest].
$$
\begin{equation}
\tau_m \frac{dV_{m}(t)}{dt}=-(V_{m}(t)-V_\text{rest})+R_mI(t)
\end{equation}
$$
ここで, $V_m$が閾値(threshold)[^theta]$V_{\text{th}}$を超えると, 脱分極が起こり, 膜電位はピーク電位 $V_{\text{peak}}$まで上昇する.発火後は再分極が起こり, 膜電位はリセット電位 $V_{\text{reset}}$まで低下すると仮定する[^reset].発火後, 一定の期間$\tau_{\text{ref}}$ の間は膜電位が変化しない[^ref], とする.これを \textbf{不応期(refractory time period)} と呼ぶ.
以上を踏まえてLIFモデルを実装してみよう.まず必要なパッケージを読み込む.
[^leak]: この漏れはイオンの拡散などによるもの. 
[^isyn]: シナプス入力による電流がどうなるかは,第三章「シナプス伝達のモデル」で扱う.
[^vrest]: $(V_{m}(t)-V_\text{rest})$の部分は膜電位の基準を静止膜電位としたことにして, 単に$V_m(t)$だけの場合もある. また, 右辺の$RI(t)$の部分は単に$I(t)$とされることもある. 同じ表記だが, この場合の$I(t)$はシナプス電流に比例する量, となっている(単位はmV). 
[^theta]: thから始まるので文字$\theta$が使われることもある.
[^reset]: リセット電位は静止膜電位と同じ場合もあれば, 過分極を考慮して静止膜電位より低めに設定することもある.
[^ref]: 実装によっては不応期の間は膜電位の変化は許容するが発火は生じないようにすることもある.