\subsubsection{解析的計算によるF-I curveの描画}
ここまでは数値的なシミュレーションによりF-I curveを求めた.以下では解析的にF-I curveの式を求めよう.具体的には,一定かつ持続的な入力電流を$I$としたときのLIFニューロンの発火率(firing rate)が


\begin{equation}
\text{rate}\approx \left(\tau_m \ln \frac{R_mI}{R_mI+V_\text{rest}-V_{\text{th}}}\right)^{-1}
\end{equation}


と近似できることを示す.まず,$t=t_1$にスパイクが生じたとする.このとき, 膜電位はリセットされるので$V_m(t_1)=V_\text{rest}$である(リセット電位と静止膜電位が同じと仮定する).$[t_1, t]$における膜電位はLIFの式を積分することで得られる.


\begin{equation}
\tau_m \frac{dV_{m}(t)}{dt}=-(V_{m}(t)-V_\text{rest})+R_m I
\end{equation}


の式を積分すると, 


\begin{align}
\int_{t_1}^{t} \frac{\tau_m dV_m}{R_mI+V_\text{rest}-V_m}&=\int_{t_1}^{t} dt\\
\ln \left(1-\frac{V_m(t)-V_\text{rest}}{R_mI}\right)&=-\frac{t-t_1}{\tau_m} \quad (\because V_m(t_1)=V_\text{rest})\\
V_m(t) &=V_\text{rest} + R_mI\left[1-\exp\left(-\frac{t-t_1}{\tau_m}\right)\right] 
\end{align}


となる.$t>t_1$における初めのスパイクが$t=t_2$に生じたとすると, そのときの膜電位は$V_m(t_2)=V_{\text{th}}$である (実際には閾値以上となっている場合もあるますが近似する).$t=t_2$を上の式に代入して


\begin{align}
V_{\text{th}}&=V_\text{rest} + R_mI\left[1-\exp\left(-\frac{t_2-t_1}{\tau}\right)\right] \\
T&= t_2-t_1 = \tau_m \ln \frac{R_mI}{R_mI+V_\text{rest}-V_{\text{th}}}
\end{align}

となる.ここで$T$は2つのスパイクの時間間隔 (spike interval)である.$t_1\leq t<t_2$におけるスパイクは$t=t_1$時の1つなので, 発火率は$1/T$となる.よって

\begin{equation}
\text{rate}\approx \frac{1}{T}=\left(\tau_m \ln \frac{R_mI}{R_mI+V_\text{rest}-V_{\text{th}}}\right)^{-1}
\end{equation}

となる.不応期$\tau_{\text{ref}}$を考慮すると, 持続的に入力がある場合は単純に$\tau_{\text{ref}}$だけ発火が遅れるので発火率は$1/(\tau_{\text{ref}}+T)$となる.

それではこの式に基づいてF-I curveを描画してみよう.
