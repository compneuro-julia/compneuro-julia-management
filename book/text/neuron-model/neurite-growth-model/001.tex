\subsection{神経突起の木構造}神経突起の形態は\textbf{樹状}突起 (dendrites; ギリシャ語で木を意味する*déndron*に由来) に代表されるように (生物としての) 木に類似している.さらに分節(segment)に離散化することでグラフ理論における\textbf{木}(tree; 連結で閉路を持たないグラフ)として捉えることができる.
シミュレーション用にデータ構造を作成しよう.なお,Juliaで木構造を扱うためのライブラリ\jl{AbstractTrees.jl}は使用しない.\jl{tree_info}はInt型vector (要素数3) のlistであり,接続している分節の番号,遠心性位数,分節の種類(1: 末端, 0:中間)を表す.\jl{seg_vec}は Float型vector (要素数2) のlistであり,分節の2次元極座標ベクトル(半径,角度)を表す.3次元に拡張することも可能であるが,本書では簡単のために2次元とする.多次元配列ではなくvectorのlistにしているのは,成長に伴って要素を追加していく際に配列に結合\jl{cat}するよりlist化して追加\jl{push!}する方が高速なためである.