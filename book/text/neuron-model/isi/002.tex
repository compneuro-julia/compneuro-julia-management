\subsubsection{定常ポアソン過程}
ここからポアソン過程によるスパイクのシミュレーションを実装する.実装方法にはISIが指数分布に従うことを利用したものと,ポアソン過程の条件2を利用したものの2通りがある.実装は後者が楽で計算量も少ないが,後のガンマ過程のために前者の実装を先に行う.

\paragraph{ISIの累積により発火時刻を求める手法}
ISIが指数分布に従うことを利用してポアソン過程モデルの実装を行う.まずISIを指数分布に従う乱数とする.次にISIを累積することで発火時刻を得る.最後に発火時間を整数値に丸めてindexとすることで$\{0, 1\}$のスパイク列が得られる.ISIの取得には\jl{Random.randexp()}を用いる.この関数は scale 1の指数分布に従う乱数を返す.このscaleは指数分布の確率密度関数を$f(t; \frac{1}{\beta}) = \frac{1}{\beta} e^{-t/\beta}$とした際の$\beta = 1/\lambda$である(この時,平均は$\beta$となる).よって発火率を\jl{fr}(1/s), 単位時間を\jl{dt}(s)としたときのISIは \jl{isi = 1/(fr*dt) * randexp()}として得ることができる.

まず必要なパッケージを読み込む.
