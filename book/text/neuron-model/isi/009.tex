\paragraph{$\Delta t$ 間の発火確率が $\lambda\Delta t$ であることを利用する方法}
次に2番目のポアソン過程モデルの実装を行う.こちらは$\lambda$を発火率とした場合, 区間$[t, t+\Delta t)$の間にポアソンスパイクが発生する確率は$\lambda \Delta t$となることを利用する.これはポアソン過程の条件だが,ポアソン分布から導けることを簡単に示しておく.事象が起こる確率が強度$\lambda$のポアソン分布に従う場合, 時刻$t$までに事象が$n$回起こる確率は$P[N(t)=n]=\dfrac{(\lambda t)^{n}}{n !} e^{-\lambda t}$となる.よって, 微小時間$\Delta t$において事象が$1$回起こる確率は


P[N(\Delta t)=1]=\dfrac{\lambda \Delta t}{1 !} e^{-\lambda \Delta t}\simeq \lambda \Delta t+o(\Delta t)


となる.ただし, $e^{-\lambda \Delta t}$についてはマクローリン展開による近似を行っている.このことから, 一様分布$U(0,1)$に従う乱数$\xi$を取得し, $\xi<\lambda dt$なら発火$(y=1)$, それ以外では$(y=0)$となるようにすればポアソンスパイクを実装できる.
