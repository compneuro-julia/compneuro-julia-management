\subsection{ポアソン過程モデル}
\subsubsection{点過程とポアソン過程}
時間に応じて変化する確率変数のことを\textbf{確率過程(stochastic process)} という.さらに確率過程の中で,連続時間軸上において離散的に生起する点事象の系列を\textbf{点過程(point process)} という.スパイクは離散的に起こるので,点過程を用いてモデル化ができるという話である.

ポアソン過程 (Poisson process)は点過程の1つである.ポアソン過程モデルはスパイクの発生をポアソン過程でモデル化したもので,このモデルによって生じるスパイクをポアソンスパイク(Poisson spike)と呼ぶ.ポアソン過程では,時刻$t$までに起こった点の数$N(t)$はポアソン分布に従う.すなわち,点が起こる確率が強度$\lambda$のポアソン分布に従う場合, 時刻$t$までに事象が$n$回起こる確率は$P[N(t)=n]=\dfrac{(\lambda t)^{n}}{n !} e^{-\lambda t}$となる. 

ポアソン過程において点が起こる回数がポアソン分布に従うことは,ポアソン過程という名称の由来となっている.これを定義とする場合もあれば,次の4条件を満たす点過程をポアソン過程とするという定義もある.

\begin{itemize}
\item 時刻0における初期の点の数は0 : $P[N(0)=0]=1$ 
\item $[t, t+\Delta t)$に点が1つ生じる確率 : $P[N(t+\Delta t)-N(t)=1]=\lambda(t)\Delta t+o(\Delta t)$
\item 微小時間$\Delta t$の間に点は2つ以上生じない : $P[N(t+\Delta t)-N(t)=2]=o(\Delta t)$
\item 任意の時点$t_1 < t_2 < \cdots< t_n$に対して,増分 $N(t_2)-N(t_1), N(t_3)-N(t_2), \cdots, N(t_n)-N(t_{n−1})$は互いに独立である.
\end{itemize}

ただし, $o(\cdot)$はLandauの記号(Landauのsmall o)であり, $o(x)$は$x\to 0$のとき,$o(x)/x\to 0$となる微小な量を表す.ポアソン過程に従ってスパイクが生じるとする場合,条件2の強度関数$\lambda(t)$は\textbf{発火率}を意味する (また実装において有用).条件3は不応期より小さいタイムステップにおいては,1つのタイムステップにおいて1つしかスパイクは生じないということを表す.条件4はスパイクは独立に発生する,ということを意味する.また,これらの条件から$N(t)$の分布は強度母数$\lambda(t)$のポアソン分布に従うことが示せる.

強度関数(点がスパイクの場合,発火率)が$\lambda(t)=\lambda$ (定数)となる場合は点の時間間隔(点がスパイクの場合,ISI)の確率変数$T$が強度母数$\lambda$の \textbf{指数分布}に従う.なお,指数分布の確率密度関数は確率変数を$T$とするとき,


f(t;\lambda )=\left\{{\begin{array}{ll}\lambda e^{-\lambda t}&(t\geq 0)\\0&(t<0)\end{array}}\right.


となる.このことは4条件とChapman-Kolmogorovの式により求められるが,ややこしいので, $P[N(t)=n]=\dfrac{(\lambda t)^{n}}{n !} e^{-\lambda t}$から導出できることを簡単に示す.指数分布の累積分布関数を$F(t; \lambda)$とすると,


F(t; \lambda) = P(T< t)=1-P(T\geq t)=1-P(N(t)=0)=1-e^{-\lambda t}


となる.よって


f(t; \lambda)=\frac{dF(t; \lambda)}{dt}=\lambda e^{-\lambda t}


が成り立つ.
