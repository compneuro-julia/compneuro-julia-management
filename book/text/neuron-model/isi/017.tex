\subsection{死時間付きポアソン過程モデル (Poisson process with dead time, PPD)}
ポアソン過程は簡易的で有用だが,不応期を考慮していない.そのため,時には生理的範疇を超えたバースト発火が起こる場合もある (複数のニューロンからの発火の重ね合わせ(superposition)であると考えることもできる.) .そこで,ポアソン過程において不応期のようなイベントの生起が起こらない \textbf{死時間(dead time)} \footnote{例えば,ガイガー・カウンター(Geiger counter)などの放射線の検出器には放射線の到達を機器の物理的特性として検出できない時間(つまり死時間)がある.そのため放射線の到達数がポアソン分布に従うとした場合,放射線測定装置のモデルとしてPPDが用いられる.}を考慮した\textbf{死時間付きポアソン過程 (Poisson process with dead time, PPD)} (またはdead time modified Poisson process)というモデルを導入する.

実装においてはLIFニューロンの時と同じような不応期の処理をする.つまり,現在が不応期かどうかを判断し,不応期なら発火を許可しないようにする.
