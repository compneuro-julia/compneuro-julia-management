\jl{spike_time}のように発火時刻で記録しておく方がメモリを節約できるが,シミュレーションにおいてはスパイク列$S$はタイムステップごとに発火しているかを表す$\{0,1\}$配列で保持しておくと楽に扱うことができる.そのため冗長ではあるが,発火時刻の配列を$\{0,1\}$配列\jl{spikes}に変換しスパイクの数と発火率を計算する.