なぜこのようなことが起こるか, というと過分極の状態から静止膜電位へと戻る際にNa$^+$チャネルが活性化 (Na$^+$チャネルの活性化パラメータ$m$が増加し, 不活性化パラメータ$h$が減少)し, 膜電位が脱分極することで再度Na$^+$チャネルが活性化する, というポジティブフィードバック過程(\textbf{自己再生的過程})に突入するためである (もちろん, この過程は通常の活動電位発生のメカニズムである). この際, 発火に必要な閾値が膜電位の低下に応じて下がった, ということもできる.

このように膜電位閾値は一定ではない.しかし, この後の節で紹介するLIFモデルなどでは簡略化のためにif文を用い, 膜電位閾値を超えたから発火, というものもある.実際には違うということを頭の片隅に残しながら読み進めることを推奨する.

}{Note}
PIRに関連する現象として抑制後促通 (Postinhibitory facilitation; PIF)がある.これは抑制入力の後に興奮入力がある一定の時間内で入ると発火が起こるという現象である ([Dolda et al., 2006](http://www.brain.riken.jp/en/summer/prev/2006/files/j_rinzel04.pdf), [Dodla, 2014](https://link.springer.com/referenceworkentry/10.1007%2F978-1-4614-7320-6_152-1)).
}
