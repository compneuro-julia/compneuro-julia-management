それでは, 等価回路モデルを用いて電位変化の式を立ててみよう.上図において, $C_m$は細胞膜のキャパシタンス(膜容量), $I_{m}(t)$は細胞膜を流れる電流(外部からの入力電流), $I_\text{Cap}(t)$は膜のコンデンサを流れる電流, $I_\text{Na}(t)$及び $I_K(t)$はそれぞれナトリウムチャネルとカリウムチャネルを通って膜から流出する電流, $I_\text{L}(t)$は漏れ電流である.このとき, 
$$
I_{m}(t)=I_\text{Cap}(t)+I_\text{Na}(t)+I_\text{K}(t)+I_\text{L}(t)    
$$
という仮定をしている.
膜電位を$V(t)$とすると, Kirchhoffの第二法則 (Kirchhoff's Voltage Law)より, 
$$
\underbrace{C_m\frac {dV(t)}{dt}}_{I_\text{Cap} (t)}=I_{m}(t)-I_\text{Na}(t)-I_\text{K}(t)-I_\text{L}(t)
$$
となる.Hodgkinらはチャネル電流$I_\text{Na}, I_K, I_\text{L}$が従う式を実験的に求めた.
$$
\begin{aligned}
I_\text{Na}(t) &= g_{\text{Na}}\cdot m^{3}h(V-E_{\text{Na}})\\
I_\text{K}(t) &= g_{\text{K}}\cdot n^{4}(V-E_{\text{K}})\\
I_\text{L}(t) &= g_{\text{L}}(V-E_{\text{L}})
\end{aligned}
$$
ただし, $g_{\text{Na}}, g_{\text{K}}$はそれぞれNa$^+$, K$^+$の最大コンダクタンスである.$g_{\text{L}}$はオームの法則に従うコンダクタンスで, Lコンダクタンスは時間的に変化はしないと仮定する.また, $m$はNa$^+$コンダクタンスの活性化パラメータ, $h$はNa$^+$コンダクタンスの不活性化パラメータ, $n$はK$^+$コンダクタンスの活性化パラメータであり, ゲートの開閉確率を表している.よって, HHモデルの状態は$V, m, h, n$の4変数で表される.これらの変数は以下の$x$を$m, n, h$に置き換えた3つの微分方程式に従う. 
$$
\frac{dx}{dt}=\alpha_{x}(V)(1-x)-\beta_{x}(V)x
$$
ただし, $V$の関数である$\alpha_{x}(V),\ \beta_{x}(V)$は$m, h, n$によって異なり, 次の6つの式に従う.
$$
\begin{array}{ll}
\alpha_{m}(V)=\dfrac {0.1(25-V)}{\exp \left[(25-V)/10\right]-1}, &\beta_{m}(V)=4\exp {(-V/18)}\\
\alpha_{h}(V)=0.07\exp {(-V/20)}, & \beta_{h}(V)={\dfrac{1}{\exp {\left[(30-V)/10 \right]}+1}}\\
\alpha_{n}(V)={\dfrac {0.01(10-V)}{\exp {\left[(10-V)/10\right]}-1}},& \beta_{n}(V)=0.125\exp {(-V/80)} 
\end{array}
$$
なお,この式は6.3℃の条件下においてイカの巨大軸索の活動から得たデータを用いて導かれたものであることに注意しよう.