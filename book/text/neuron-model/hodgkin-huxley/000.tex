\section{Hodgkin-Huxleyモデル}
\subsection{Hodgkin-Huxleyモデルにおける膜の等価回路モデル}
\textbf{Hodgkin-Huxleyモデル} (HH モデル)は, A.L. HodgkinとA.F. Huxleyによって1952年に考案されたニューロンの膜興奮を表すモデルである \cite{Hodgkin1952-gy}.Hodgkinらはヤリイカの巨大神経軸索に対する\textbf{電位固定法}(voltage-clamp)を用いた実験を行い, 実験から得られた観測結果を元にモデルを構築した[^hh].

[^hh]: HHモデルの構築に関する歴史については([Schwiening, 2012](https://www.ncbi.nlm.nih.gov/pmc/articles/PMC3424716/))を参照.

HHモデルには等価な電気回路モデルがあり, \textbf{膜の並列等価回路モデル} (parallel conductance model)と呼ばれている.膜の並列等価回路モデルでは, ニューロンの細胞膜をコンデンサ, 細胞膜に埋まっているイオンチャネルを可変抵抗 (動的に変化する抵抗) として置き換える.

\textbf{イオンチャネル} (ion channel)は特定のイオン(例えばナトリウムイオンやカリウムイオンなど)を選択的に通す膜輸送体の一種である.それぞれのイオンの種類において, 異なるイオンチャネルがある (同じイオンでも複数の種類のイオンチャネルがある).また, イオンチャネルにはイオンの種類に応じて異なる\textbf{コンダクタンス}(抵抗の逆数で電流の「流れやすさ」を意味する)と\textbf{平衡電位}(equilibrium potential)がある.HHモデルでは, ナトリウム(Na$^{+}$)チャネル, カリウム(K$^{+}$)チャネル, 漏れ電流(leak current)のイオンチャネルを仮定する.漏れ電流のイオンチャネルは当時特定できなかったチャネルで, 膜から電流が漏れ出すチャネルを意味する.なお, 現在では漏れ電流の多くはCl$^{-}$イオン(chloride ion)によることが分かっている.
