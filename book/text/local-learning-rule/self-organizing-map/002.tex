SOMにおける$n$番目の入力を $\mathbf{v}(t)=\mathbf{v}_n\in \mathbb{R}^{D} (n=1, \ldots, N)$,$m$番目のニューロン$(m=1, \ldots, M)$の重みベクトル (または活動ベクトル, 参照ベクトル)を$\mathbf{w}_m(t)\in \mathbb{R}^{D}$とする \cite{Kohonen2013-yt}.また,各ニューロンの物理的な位置を$\mathbf{x}_m$とする.このとき,$\mathbf{v}(t)$に対して$\mathbf{w}_m(t)$を次のように更新する.
まず,$\mathbf{v}(t)$と$\mathbf{w}_m(t)$の間の距離が最も小さい(類似度が最も大きい)ニューロンを見つける.距離や類似度としてはユークリッド距離やコサイン類似度などが考えられる.
$$
\begin{aligned}
&[\text{ユークリッド距離}]: c = \underset{m}{\operatorname{argmin}}\left[\|\mathbf{v}(t)-\mathbf{w}_m(t)\|^2\right]\\
&[\text{コサイン類似度}]: c  = \underset{m}{\operatorname{argmax}}\left[\frac{\mathbf{w}_m(t)^\top\mathbf{v}(t)}{\|\mathbf{w}_m(t)\|\|\mathbf{v}(t)\|}\right]
\end{aligned}
$$
この,$c$番目のニューロンを\textbf{勝者ユニット(best matching unit; BMU)} と呼ぶ.コサイン類似度において,$\mathbf{w}_m(t)^\top\mathbf{v}(t)$は線形ニューロンモデルの出力となる.このため,コサイン距離を採用する方が生理学的に妥当でありSOMの初期の研究ではコサイン類似度が用いられている \cite{Kohonen1982-mn}.しかし,コサイン類似度を用いる場合は$\mathbf{w}_m$および$\mathbf{v}$を正規化する必要がある.ユークリッド距離を用いると正規化なしでも学習できるため,SOMを応用する上ではユークリッド距離が採用される事が多い.ユークリッド距離を用いる場合,$\mathbf{w}_m$は重みベクトルではなくなるため,活動ベクトルや参照ベクトルと呼ばれる.ここでは結果の安定性を優先してユークリッド距離を用いることとする.
こうして得られた$c$を用いて$\mathbf{w}_m$を次のように更新する.
$$
\mathbf{w}_m(t+1)=\mathbf{w}_m(t)+h_{cm}(t)[\mathbf{v}(t)-\mathbf{w}_m(t)]
$$
ここで$h_{cm}(t)$は近傍関数 (neighborhood function) と呼ばれ,$c$番目と$m$番目のニューロンの距離が近いほど大きな値を取る.ガウス関数を用いるのが一般的である.
$$
h_{cm}(t)=\alpha(t)\exp\left(-\frac{\|\mathbf{x}_c-\mathbf{x}_m\|^2}{2\sigma^2(t)}\right)
$$
ここで$\mathbf{x}$はニューロンの位置を表すベクトルである.また,$\alpha(t), \sigma(t)$は単調に減少するように設定する.
\footnote{
Generative topographic map (GTM)を用いれば$\alpha(t), \sigma(t)$の縮小は必要ない.また,SOMとGTMの間を取ったモデルとしてS-mapがある.
}