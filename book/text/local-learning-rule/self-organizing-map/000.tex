\section{自己組織化マップと視覚野の構造}視覚野にはコラム構造が存在する.こうした構造は神経活動依存的な発生  (activity dependent development) により獲得される.本節では視覚野のコラム構造を生み出す数理モデルの中で,\textbf{自己組織化マップ (self-organizing map)} \cite{Kohonen1982-mn}, \cite{Kohonen2013-yt}を取り上げる.
自己組織化マップを視覚野の構造に適応したのは\cite{Obermayer1990-gq} \cite{N_V_Swindale1998-ri}などの研究である.視覚野マップの数理モデルとして自己組織化マップは受容野を考慮しないなどの簡略化がなされているが,単純な手法にして視覚野の構造に関する良い予測を与える.他の数理モデルとしては自己組織化マップと発想が類似している \textbf{Elastic net}  \cite{Durbin1987-bp} \cite{Durbin1990-xx} \cite{Carreira-Perpinan2005-gy} (ここでのElastic netは正則化手法としてのElastic net regularizationとは異なる)や受容野を明示的に設定した \cite{Tanaka2004-vz}, \cite{Ringach2007-oe}などのモデルがある.総説としては\cite{Das2005-mq},\cite{Goodhill2007-va} ,数理モデル同士の関係については\cite{2002-nm}が詳しい.
自己組織化マップでは「抹消から中枢への伝達過程で損失される情報量」,および「近い性質を持ったニューロン同士が結合するような配線長」の両者を最小化するような学習が行われる.包括性 (coverage) と連続性 (continuity) のトレードオフとも呼ばれる \cite{Carreira-Perpinan2005-gy}  (Elastic netは両者を明示的に計算し,線形結合で表されるエネルギー関数を最小化する.Elastic netは本書では取り扱わないが,MATLAB実装が公開されている
\url{https://faculty.ucmerced.edu/mcarreira-perpinan/research/EN.html}) . 連続性と関連する事項として,近い性質を持つ細胞が脳内で近傍に存在する現象があり,これを\textbf{トポグラフィックマッピング (topographic mapping)} と呼ぶ.トポグラフィックマッピングの数理モデルの初期の研究としては\cite{Von_der_Malsburg1973-bz} \cite{Willshaw1976-zo} \cite{Takeuchi1979-mi}などがある.
発生の数理モデルに関する総説 \cite{Van_Ooyen2011-fz}, \cite{Goodhill2018-ho}
