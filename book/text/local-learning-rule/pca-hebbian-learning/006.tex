\subsubsection{Oja則}
Hebb則を安定化させる別のアプローチとして,結合強度を正規化するという手法が考えられる.BCM則と同様に$\mathbf{x}\in \mathbb{R}^d, \mathbf{w}\in \mathbb{R}^d, y\in \mathbb{R}$とし,単一の出力$y = \mathbf{w}^\top \mathbf{x}=\mathbf{x}^\top \mathbf{w}$を持つ線形ニューロンを仮定する.$\eta$を学習率とすると,$\mathbf{w}\leftarrow\dfrac{\mathbf{w}+\eta \mathbf{x}y}{\|\mathbf{w}+\eta \mathbf{x}y\|}$とすれば正規化できる.ここで,$f(\eta):=\dfrac{\mathbf{w}+\eta \mathbf{x}y}{\|\mathbf{w}+\eta \mathbf{x}y\|}$とし,$\eta=0$においてTaylor展開を行うと,


\begin{align}
f(\eta)&\approx f(0) + \eta \left.\frac{df(\eta^*)}{d\eta^*}\right|_{\eta^*=0} + \mathcal{O}(\eta^2)\\
&=\frac{\mathbf{w}}{\|\mathbf{w}\|} + \eta \left(\frac{\mathbf{x}y}{\|\mathbf{w}\|}-\frac{y^2\mathbf{w}}{\|\mathbf{w}\|^3}\right)+ \mathcal{O}(\eta^2)
\end{align}


ここで$\|\mathbf{w}\|=1$として,1次近似すれば$f(\eta)\approx \mathbf{w} + \eta \left(\mathbf{x}y-y^2 \mathbf{w}\right)$となる.重みの変化が連続的であるとすると,


\begin{equation}
\frac{d\mathbf{w}}{dt} = \eta \left(\mathbf{x}y-y^2 \mathbf{w}\right)
\end{equation}


として重みの更新則が得られる.これを\textbf{Oja則 (Oja's rule)} と呼ぶ \cite{Oja1982-yd}.こうして得られた学習則において$\|\mathbf{w}\|\to 1$となることを確認しよう.


\begin{equation}
\frac{d\|\mathbf{w}\|^2}{dt}=2\mathbf{w}^\top\frac{d\mathbf{w}}{dt}= 2\eta y^2\left(1-\|\mathbf{w}\|^2\right)
\end{equation}


より,$\dfrac{d\|\mathbf{w}\|^2}{dt}=0$のとき,$\|\mathbf{w}\|= 1$となる.
