\subsubsection{Sanger則}
Oja則に複数の出力を持たせた場合であっても,出力が直交しないため,PCAの第1主成分しか求めることができない.\textbf{Sanger則 (Sanger's rule)},あるいは\textbf{一般化Hebb則 (generalized Hebbian algorithm; GHA)} は,Oja則に\textbf{Gram–Schmidtの正規直交化法(Gram–Schmidt orthonormalization)} を組み合わせた学習則であり,次式で表される.


\begin{equation}
\frac{d\mathbf{W}}{dt} = \eta \left(\mathbf{y}\mathbf{x}^\top - \mathrm{LT}\left[\mathbf{y}\mathbf{y}^\top\right] \mathbf{W}\right)
\end{equation}


$\mathrm{LT}(\cdot)$は行列の対角成分より上側の要素を0にした下三角行列(lower triangular matrix)を作り出す作用素である.Sanger則を用いればPCAの第2主成分以降も求めることができる.
