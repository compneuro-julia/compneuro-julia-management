\subsubsection{Oja則によるPCAの実行}
ここでOja則が主成分分析を実行できることを示す.重みの変化量の期待値を取る.


\begin{align}
\frac{d\mathbf{w}}{dt} &= \eta \left(\mathbf{x}y - y^2 \mathbf{w}\right)=\eta \left(\mathbf{x}\mathbf{x}^\top \mathbf{w} - \left[\mathbf{w}^\top \mathbf{x}\mathbf{x}^\top \mathbf{w}\right] \mathbf{w}\right)\\
\mathbb{E}\left[\frac{d\mathbf{w}}{dt}\right] &= \eta \left(\mathbf{C} \mathbf{w} - \left[\mathbf{w}^\top \mathbf{C} \mathbf{w}\right] \mathbf{w}\right)
\end{align}


$\mathbf{C}:=\mathbb{E}[\mathbf{x}\mathbf{x}^\top]\in \mathbb{R}^{d\times d}$とする.$\mathbf{x}$の平均が0の場合,$\mathbf{C}$は分散共分散行列である.$\mathbb{E}\left[\dfrac{d\mathbf{w}}{dt}\right]=0$となる$\mathbf{w}$が収束する固定点(fixed point)では次の式が成り立つ.


\begin{equation}
\mathbf{C}\mathbf{w} = \lambda \mathbf{w}
\end{equation}


これは固有値問題であり,$\lambda:=\mathbf{w}^\top \mathbf{C} \mathbf{w}$は固有値,$\mathbf{w}$は固有ベクトル(eigen vector)になる.

ここでサンプルサイズを$n$とし,$\mathbf{X} \in \mathbb{R}^{d\times n}, \mathbf{y}=\mathbf{X}^\top\mathbf{w} \in \mathbb{R}^n$とする.標本平均で近似して$\mathbf{C}\simeq \mathbf{X}\mathbf{X}^\top$とする.この場合,


\begin{align}
\mathbb{E}\left[\frac{d\mathbf{w}}{dt}\right] &\simeq \eta \left(\mathbf{X}\mathbf{X}^\top \mathbf{w} - \left[\mathbf{w}^\top \mathbf{X}\mathbf{X}^\top \mathbf{w}\right] \mathbf{w}\right)\\
&=\eta \left(\mathbf{X}\mathbf{y} - \left[\mathbf{y}^\top\mathbf{y}\right] \mathbf{w}\right)
\end{align}


となる.
