\subsection{Hebb則の安定化とLTP/LTD}
\subsubsection{BCM則}
Hebb則には問題点があり,シナプス結合強度が際限なく増大するか,0に近づくこととなってしまう.これを数式で確認しておこう.前細胞と後細胞がそれぞれ1つの場合を考える.2細胞間の結合強度を$w\ (>0)$ とし,$y=wx$が成り立つとすると,Hebb則は$\dfrac{dw}{dt}=\eta yx=\eta x^2w$となる.この場合,$\eta x^2>1$ なら $\lim_{t\to\infty} w= \infty$, $\eta x^2<1$ なら $\lim_{t\to\infty} w= 0$ となる.当然,生理的にシナプス結合強度が無限大となることはあり得ないが,不安定なほど大きくなってしまう可能性があることに違いはない.このため,Hebb則を安定化させるための修正が必要とされた.

Cooper, Liberman, Ojaらにより頭文字をとって\textbf{CLO則} (CLO rule) が提案された \cite{Cooper1979-wz}.その後,Bienenstock, Cooper, Munroらにより提案された学習則は同様に頭文字をとって\textbf{BCM則} (BCM rule) と呼ばれている\cite{Bienenstock1982-km} \cite{Cooper2012-ec}.

$\mathbf{x}\in \mathbb{R}^d, \mathbf{w}\in \mathbb{R}^d, y\in \mathbb{R}$とし,単一の出力$y = \mathbf{w}^\top \mathbf{x}=\mathbf{x}^\top \mathbf{w}$を持つ線形ニューロンを仮定する.重みの更新則は次のようにする.


\begin{equation}
\frac{d\mathbf{w}}{dt} = \eta_w \mathbf{x} \phi(y, \theta_m)
\end{equation}


ここで関数$\phi$は$\phi(y, \theta_m)=y(y-\theta_m)$などとする.また$\theta_m:=\mathbb{E}[y^2]$は閾値を決定するパラメータ,\textbf{修正閾値(modification threshold)} であり,


\begin{equation}
\frac{d\theta_m}{dt} = \eta_{\theta} \left(y^2-\theta_m\right)
\end{equation}


として更新される.

ToDo: 詳細
