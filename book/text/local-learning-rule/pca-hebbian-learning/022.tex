\subsubsection{非負主成分分析によるグリッドパターンの創発}
内側嗅内皮質(MEC)にある\textbf{グリッド細胞 (grid cells)} は六角形格子状の発火パターンにより自己位置等を符号化するのに貢献している.この発火パターンを生み出すモデルは多数あるが,\textbf{場所細胞(place cells)} の発火パターンを\textbf{非負主成分分析(nonnegative principal component analysis)} で次元削減するとグリッド細胞のパターンが生まれるというモデルがある \cite{Dordek2016-ff}.非線形Hebb学習を用いてこのモデルを実装しよう.なお,同様のことは\textbf{非負値行列因子分解 (NMF: nonnegative matrix factorization)} でも可能である.
