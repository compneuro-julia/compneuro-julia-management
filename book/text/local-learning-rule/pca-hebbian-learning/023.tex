\paragraph{場所細胞の発火パターン}
まず,訓練データとなる場所細胞の発火パターンを人工的に作成する.場所細胞の発火パターンは\textbf{Difference of Gaussians (DoG)} で近似する.DoGは大きさの異なる2つのガウス関数の差分を取った関数であり,画像に適応すればband-passフィルタとして機能する.また,DoGは網膜神経節細胞等の受容野のON中心OFF周辺型受容野のモデルとしても用いられる.受容野中央では活動が大きく,その周辺では活動が抑制される,という特性を持つ.2次元のガウス関数とDoG関数を実装する.
