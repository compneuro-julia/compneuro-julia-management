\section{Hebb則と教師なし学習}
\subsection{Hebb則}
神経回路はどのようにして自己組織化するのだろうか.1940年代にカナダの心理学者Donald O. Hebbにより著書"The Organization of Behavior"\cite{Hebb1949-iv} で提案された学習則は「細胞Aが反復的または持続的に細胞Bの発火に関与すると,細胞Aが細胞Bを発火させる効率が向上するような成長過程または代謝変化が一方または両方の細胞に起こる」というものであった.すなわち,発火に時間的相関のある細胞間のシナプス結合を強化するという学習則である.これを\textbf{Hebbの学習則 (Hebbian learning rule)} あるいは\textbf{Hebb則(Hebb's rule)} という.Hebb則は (Hebb自身ではなく) Shatzにより"cells that fire together wire together" (共に活動する細胞は共に結合する)と韻を踏みながら短く言い換えられている \cite{Shatz1992-he}.

\subsubsection{Hebb則の導出}
数式でHebb則を表してみよう.$n$個のシナプス前細胞と$m$個の後細胞の発火率をそれぞれ$\mathbf{x}\in \mathbb{R}^n, \mathbf{y}\in \mathbb{R}^m$ とする.前細胞と後細胞間のシナプス結合強度を表す行列を$\mathbf{W}\in \mathbb{R}^{m\times n}$とし,$\mathbf{y}=\mathbf{W}\mathbf{x}$が成り立つとする.このようなモデルを線形ニューロンモデル (Linear neuron model) という.このとき,Hebb則は


\begin{equation}
\tau\frac{d\mathbf{W}}{dt}=\phi(\mathbf{y})\varphi(\mathbf{x})^\top
\end{equation}


として表される.ただし,$\tau$は時定数であり,$\eta:=1/\tau$ は\textbf{学習率 (learning rate)} と呼ばれる学習の速さを決定するパラメータとなる.$\varphi(\cdot)$および$\phi(\cdot)$は,それぞれシナプス前細胞および後細胞の活動量に応じて重みの変化量を決定する関数である.ただし,$\varphi(\cdot), \phi(\cdot)$は基本的に恒等関数に設定される場合が多い.この場合,Hebb則は$
\tau\dfrac{d\mathbf{W}}{dt}=\mathbf{y}\mathbf{x}^\top=(\text{post})\cdot (\text{pre})^\top
$と簡潔に表現される.

このHebb則は数学的に導出されたものではないが,特定の目的関数を神経活動及び重みを変化させて最適化するようなネットワークを構築すれば自然に出現する.このようなネットワークを\textbf{エネルギーベースモデル (energy-based models)} といい,次章で扱う.エネルギーベースモデルでは,先にエネルギー関数 (あるいはコスト関数) $\mathcal{E}$ を定義し,その目的関数を最小化するような神経活動 $\mathbf{z}$ および重み行列 $\mathbf{W}$ のダイナミクスをそれぞれ,


\begin{equation}
\frac{d \mathbf{z}}{dt}\propto-\frac{\partial \mathcal{E}}{\partial \mathbf{z}},\ \frac{d \mathbf{W}}{dt}\propto-\frac{\partial \mathcal{E}}{\partial \mathbf{W}}
\end{equation}


として導出する.この手順の逆を行う,すなわち先に神経細胞の活動ダイナミクスを定義し,神経活動で積分することで神経回路のエネルギー関数$\mathcal{E}$を導出し,さらに $\mathcal{E}$ を重み行列で微分することでHebb則が導出できる \cite{Isomura2020-sn}.Hebb則の導出を連続時間線形ニューロンモデル $\dfrac{d\mathbf{y}}{dt}=\mathbf{W}\mathbf{x}$ を例にして考えよう.ここで$\dfrac{\partial\mathcal{E}}{\partial\mathbf{y}}:=-\dfrac{d\mathbf{y}}{dt}$となるようなエネルギー関数 $\mathcal{E}(\mathbf{x}, \mathbf{y}, \mathbf{W})$を仮定すると,


\begin{equation}
\mathcal{E}(\mathbf{x}, \mathbf{y}, \mathbf{W})=-\int \mathbf{W}\mathbf{x}\ d\mathbf{y}=-\mathbf{y}^\top \mathbf{W}\mathbf{x} \in \mathbb{R}
\end{equation}


となる.これをさらに$\mathbf{W}$で微分すると,


\begin{equation}
\dfrac{\partial\mathcal{E}}{\partial\mathbf{W}}=-\mathbf{y}\mathbf{x}^\top\Rightarrow
\frac{d\mathbf{W}}{dt}=-\dfrac{\partial\mathcal{E}}{\partial\mathbf{W}}=\mathbf{y}\mathbf{x}^\top
\end{equation}


となり,Hebb則が導出できる (簡単のため時定数は1とした).
