このコードを実行すると図\ref{fig:online_stdp2}のようになります。
\begin{figure}[htbp]
    \centering
    \includegraphics[scale=0.5]{figs/online_stdp2.pdf}
    \caption{Online STDP}
    \label{fig:online_stdp2}
\end{figure}

\subsection{重み依存的なSTDP}
生理学的にはシナプス強度$w$には$w_{\min} < w < w_{\max}$というような制限(bound)が存在すると考えられます\footnote{受容体の数が際限なく増加したり減少したりすることはないと考えられるためです。もしLTPによりシナプス強度が限りなく増大した場合、シナプス後細胞の発火頻度が高くなり、実際には発火を誘発していないシナプス前細胞とのシナプス強度も大きくなってしまいます。LTPの暴走を防ぐための機構の1つとして\textbf{恒常性可塑性}(homeostatic scaling)または\textbf{シナプススケーリング}(synaptic scaling)と呼ばれる現象があります。}。多くの場合では$w_{\min}=0$となっているので、以下では$w\in [0, w_{\max}]$となる場合を考えます。また、前節までは正の定数としていた$A_+, A_-$を重み依存的な関数であるとします(つまり$A_\pm=A_\pm(w)$とします)。\par
重みの制限には\textbf{ソフト制限(soft bound)}と\textbf{ハード制限(hard bound)}があります(Gerstner and Kistler, 2002, Chapter 11)。ソフト制限は、重みが上界 (または下界) に近づくにつれ重みの変化が小さくなるというものです。
\begin{align}
A_+(w) &= \eta_+\cdot (w_{\max}-w) \\
A_-(w) &= \eta_-\cdot w
\end{align}
ここで$\eta_+, \eta_-$は正の値で、\textbf{学習率(learning rate)}を意味します。\par
次に、ハード制限は重みが上限 (または下限) に達した際に、重みが増加 (または減少) しないというものです。Heavisideの階段関数$\Theta(x)$ (ただし$x<0$で$\Theta(x)=0$, $x\geq 0$で$\Theta(x)=1$)を用いて
\begin{align}
A_+(w) &= \eta_+\cdot \Theta(w_{\max}-w) \\
A_-(w) &= \eta_-\cdot \Theta(-w)
\end{align}
となります。
