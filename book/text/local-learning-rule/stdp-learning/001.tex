\section{STDP(spike-timing-dependent plasticity)則}
\subsection{Pair-based STDP則}
\textbf{Spike-timing-dependent plasticity} (STDP)はシナプス前細胞と後細胞の発火時刻の差によってシナプス強度が変化するという現象です(Markram et al. 1997; Bi and Poo 1998)。典型的なSTDP側は\textbf{Pair-based STDP則}と呼ばれ、シナプス前細胞と後細胞の2つのスパイクのペアの発火時刻によってLTPやLTDが起こります。この節ではこのPair-based STDP則について説明します。\par
シナプス後細胞におけるスパイク(postsynaptic spike)の発生時刻$t_\text{post}$とシナプス前細胞におけるスパイク(presynaptic spike)の発生時刻$t_\text{pre}$の差を$\Delta t_{\text{spike}}=t_\text{post}-t_\text{pre}$とします\footnote{$\Delta t_{\text{spike}}$の定義は元々は逆になっており、(Song et al,, 2000)では$\Delta t_{\text{spike}}=t_\text{pre}-t_\text{post}$としています。また、添え字は離散時のタイムステップと混同しないために付けています。}。$\Delta t_{\text{spike}}$はシナプス前細胞、後細胞の順で発火すれば正、逆なら負となります。Pair-based STDP則では、シナプス前細胞から後細胞へのシナプス強度($w$)\footnote{シナプス強度$w$に添え字をつけていませんが、この場合はシナプス前細胞と後細胞の2つの細胞しかないと仮定して考えています。}の変化$\Delta w$は$\Delta t_{\text{spike}}$に依存的に以下の式に従って変化します(Song et al., 2000)。
\begin{equation}
\Delta w = \begin{cases}
A_{+} \exp\left(-\dfrac{\Delta t_{\text{spike}}}{\tau_{+}}\right) &(\Delta t_{\text{spike}}> 0) \\
-A_{-} \exp\left(-\dfrac{|\Delta t_{\text{spike}}|}{\tau_{-}}\right) &(\Delta t_{\text{spike}}< 0)
\end{cases}
\end{equation}
$A_+, A_-$は正の定数、または重み依存的な関数 (後述) です。$\Delta t_{\text{spike}}>0$のときはLTPが起こり、$\Delta t_{\text{spike}}<0$のときはLTDが起こります。このタイプのSTDP則は\textbf{Hebbian STDP}と呼ばれ、Hebb則\footnote{「シナプス前細胞が発火してからシナプス後細胞が発火することによりシナプス結合が増強される」という法則です。1949年にDonald Hebbにより提唱されました。}に従うシナプス強度の変化が起こります\footnote{Hebb則に従わないSTDPもあり、例えばLTPとLTDの挙動が逆のものを\textbf{Anti-Hebbian STDP}と呼びます(Bell et al., 1997など)。}。
$A_+=0.01$, $A_-/A_+=1.05$, $\tau_{+}=\tau_{-}=20$ msとしたときの$\Delta t_{\text{spike}}$に対する$\Delta w$は図\ref{fig:stdp}のようになります。

\begin{figure}[htbp]
    \centering
    \includegraphics[scale=0.5]{figs/stdp.pdf}
    \caption{STDP}
    \label{fig:stdp}
\end{figure}
以下は図\ref{fig:stdp}を描画するためのコードです\footnote{コードは\texttt{./SingleFileSimulations/STDP/stdp.py}です。}
\begin{minted}[frame=lines, framesep=2mm, baselinestretch=1.2, bgcolor=shadecolor,fontsize=\small]{python}
tau_p = tau_m = 20 #ms
A_p = 0.01
A_m = 1.05*A_p
dt = np.arange(-50, 50, 1) #ms
dw = A_p*np.exp(-dt/tau_p)*(dt>0) - A_m*np.exp(dt/tau_p)*(dt<0) 

plt.figure(figsize=(5, 4))
plt.plot(dt, dw)
plt.hlines(0, -50, 50); plt.xlim(-50, 50)
plt.xlabel("$\Delta t$ (ms)"); plt.ylabel("$\Delta w$")
plt.show()
\end{minted}

\subsection{オンライン STDP則}
単に2つのニューロンを考えるなら上で紹介した式でも良いのですが、ネットワーク全体を考えると実装は複雑になり効率的ではありません。また、スパイク発生時刻を記憶しておくことは生物学的に妥当ではありません。そこで、スパイク活動のトレース(trace)というローカル変数を用いてSTDPを記述してみましょう。
\begin{align}
\frac{dx_\text{pre}}{dt}&=-\frac{x_\text{pre}}{\tau_+}+\sum_{t_{\text{pre}}^{(i)} <t} \delta \left(t-t_{\text{pre}}^{(i)}\right)\\
\frac{dx_\text{post}}{dt}&=-\frac{x_\text{post}}{\tau_-}+\sum_{t_{\text{post}}^{(j)}<t} \delta \left(t-t_{\text{post}}^{(j)}\right)
\end{align}
とします。ただし、$t_{\text{pre}}^{(i)}$はシナプス前細胞の$i$番目のスパイク、$t_{\text{post}}^{(j)}$はシナプス後細胞の$i$番目のスパイクを意味します。また、$x_\text{pre}$, $x_\text{post}$はそれぞれシナプス前細胞、後細胞のスパイクのトレースです。トレースはそれぞれの細胞においてスパイクが発生したときに1増加し\footnote{トレースの値域を$0\leq x\leq1$に制限するため、スパイクが発生したとき1にリセットするという場合もあります(Morrison et al., 2008)。その場合はx(t+\Delta t)=\left(1-\frac{\Delta t}{\tau}\right)x(t)\cdot(1-\delta_{t,t'})+\delta_{t,t'}のようにします ($t'$はスパイクの発生時刻) 。}、それ以外では時定数$\tau_+, \tau_-$で指数関数的に減少します。これは既に1章で説明した単一指数関数型シナプスと同じです。生理学的解釈ですが、$x_\text{pre}$はNMDA受容体のイオンチャネルの開口割合、$x_\text{post}$は逆伝搬活動電位 (back-propagating action potential; bAP) \footnote{誤差逆伝搬法(Back-propagation)とは異なります。}やbAPによるカルシウムの流入と捉えることができます(cf. 『標準生理学』)。\par
そして、$x_\text{pre}, x_\text{post}$を用いて重みの更新式は
\begin{equation}
\frac{dw}{dt}=A_+ x_\text{pre} \cdot \underbrace{\sum_{t_{\text{post}}^{(j)}<t} \delta \left(t-t_{\text{post}}^{(j)}\right)}_{\text{シナプス後細胞のスパイク}} - A_- x_{\text{post}} \cdot \underbrace{\sum_{t_{\text{pre}}^{(i)} <t} \delta \left(t-t_{\text{pre}}^{(i)}\right)}_{\text{シナプス前細胞のスパイク}}
\end{equation}
と表せます。\par
これらの式をEuler法によりタイムステップ$\Delta t$で離散化すると、
\begin{align}
x_{\text{pre}}(t+\Delta t)&=\left(1-\frac{\Delta t}{\tau_{+}}\right)\cdot x_{\text{pre}}(t)+
\delta_{t,t_{\text{pre}}^{(i)}}\\
x_{\text{post}}(t+\Delta t)&=\left(1-\frac{\Delta t}{\tau_{-}}\right)\cdot x_{\text{post}}(t)+\delta_{t,t_{\text{post}}^{(j)}}\\
w(t+\Delta t)&=w(t)+A_+ x_{\text{pre}}\cdot \delta_{t,t_{\text{post}}^{(j)}} - A_-x_{\text{post}}\cdot \delta_{t,t_{\text{pre}}^{(i)}}
\end{align}
となります。ただし、$\delta_{t,t'}$はKroneckerのdelta関数で、$t=t'$のときに1, それ以外は0となります。$\delta_{t,t'}$は実装時においてスパイクが起こったときに1, その他は0を取る変数を用いると良いでしょう。\par
それでは、Online STDPを実装してみましょう\footnote{コードは\texttt{./SingleFileSimulations/STDP/stdp2.py}です。}。
