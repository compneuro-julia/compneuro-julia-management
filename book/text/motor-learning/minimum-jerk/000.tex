\section{躍度最小モデル}
躍度最小モデル (minimum-jerk model; \cite{Flash1985-vj})を実装する.解析的に求まるが以下では二次計画法を用いて数値的に求める.

\subsection{等式制約下の二次計画法 (Equality Constrained Quadratic Programming)}

$n$個の変数があり,$m$個の制約条件がある等式制約二次計画問題を考える.$\mathbf {x}\in \mathbb{R}^n$, 対称行列$\mathbf{P}\in \mathbb{R}^{n\times n}$,  $\mathbf {q}\in \mathbb{R}^{n}$, $\mathbf{A}\in \mathbb{R}^{m\times n}$, $\mathbf {b}\in \mathbb{R}^m$.このとき,問題は次のようになる.


\begin{aligned}
&{\text{Minimize}}\quad {\frac {1}{2}}\mathbf {x}^\top \mathbf{P}\mathbf {x} +\mathbf {q} ^{\top}\mathbf {x}\\
&{\text{subject to}}\quad \mathbf{A}\mathbf {x} =\mathbf {b}
\end{aligned}


Lagrangeの未定乗数法を用いると解は


{\begin{bmatrix}\mathbf{P}&\mathbf{A}^\top\\\mathbf{A}&0\end{bmatrix}}{\begin{bmatrix}\mathbf {x} \\
\lambda \end{bmatrix}}={\begin{bmatrix}-\mathbf {q} \\\mathbf {b} \end{bmatrix}}


の解として与えられる.ここで $\lambda \in \mathbb{R}^{m}$  はLagrange乗数のベクトルである.
