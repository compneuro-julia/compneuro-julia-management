\subsection{躍度最小モデルの実装}
1次元における運動を考えよう.この仮定ではサッカードするときの眼球運動などが当てはまる.以下では \cite{Yazdani2012-sx} での問題設定を用いる.Toeplitz行列を用いた実装はYazdaniらのPythonでcvxoptを用いた実装を参考にして作成した.

問題設定は以下のようにする.
$$
\begin{align}
&\underset{u(t)}{\operatorname{minimize}}\quad \|u(t)\|_2 \\
&\text{subject to} \quad \dot{\mathbf{x}}(t)=A \mathbf{x}(t)+B u(t)
\end{align}
$$

ただし,$\|\cdot\|_{2}$は$L_{2}$ノルムを意味し,$A=\left[\begin{array}{lll}0 & 1 & 0 \\ 0 & 0 & 1 \\ 0 & 0 & 0\end{array}\right], B=\left[\begin{array}{l}0 \\ 0 \\ 1\end{array}\right], \mathbf{x}(t)=\left[\begin{array}{l}x(t) \\ \dot{x}(t) \\ \ddot{x}(t)\end{array}\right], u(t)=\dddot x(t)$とする.すなわち,制御信号$u(t)$は躍度$\dddot x(t)$と等しいとする.
