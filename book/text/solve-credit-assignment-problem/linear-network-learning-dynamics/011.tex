3層線形ネットワーク (deep)では大きな特異値から学習が始まっているのが分かる.また,それぞれの特異値の学習においてはシグモイド関数様の急速な学習段階が見られる.一方で2層線形ネットワーク (shallow)では全ての特異値の学習が初めから起こっていることがわかる.パラメータが少ないため,収束はこちらの方が速い.
このモデルが面白い理由の一つとして,知識の混同 (例えば『芋虫には骨がある』) の仕組みを提供することがある.発達において,大きい特異値から先に学習されるため,「動く」,「成長する」などの動物の要素が先に獲得される.身の回りの動物のほとんどが「骨を持つ」ので,\textbf{低ランク近似により,『芋虫にも骨がある』と錯覚してしまう}のではないか,という仮説が立てられている.