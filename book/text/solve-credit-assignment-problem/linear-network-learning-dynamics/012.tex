\subsection{線形多層ニューラルネットワークにおける勾配降下法による低ランク解の獲得}

> Jing, L., Zbontar, J. & LeCun, Y. \textbf{Implicit Rank-Minimizing Autoencoder}. *NeurIPS' 20*, 2020. \url{https://arxiv.org/abs/2010.00679}

([Arora et al., *NeurIPS' 19*. 2019](https://arxiv.org/abs/1905.13655))は深層線形ニューラルネットワークが低ランクの解を導出できることを理論的及び実験的に実証した.([Gunasekar et al., *NeurIPS' 18*. 2018](https://arxiv.org/abs/1806.00468))は,線形畳み込みニューラルネットワークにおいて勾配降下が正則化作用を持つことを示した.

証明は省略するが,([Arora et al., *NeurIPS' 19*. 2019](https://arxiv.org/abs/1905.13655))におけるTheorem 3.を紹介する.まず,$N$層の線形多層ニューラルネットワークを考え,$W_j \in \mathbb{R}^{d_j \times d_{j−1}}$を$j$層の重みとする.$t$を学習のタイムステップとし,$W(t) \in \mathbb{R}^{d \times d^\prime}$を重み行列を全て乗じた行列とする (ただし$d := d_N, d^\prime := d_0$).つまり$W(t):=\prod_{j=1}^N W_j(t)$である.

ここで$W(t)$を特異値分解し,$W(t) = U(t)S(t)V^\top(t)$と表現する.$S(t)$は対角行列で,その要素を$\sigma_1(t), \ldots , \sigma_{\min\{d, d^\prime\}}(t),$とする.これが$W(t)$の特異値となる.さらに$U(t), V (t)$の列ベクトルをそれぞれ $\mathbf{u}_1(t), \ldots, \mathbf{u}_{\min\{d, d^\prime\}}(t)$, および $\mathbf{v}_1(t), \ldots, \mathbf{v}_{\min\{d,d^\prime \}}(t)$とする.このとき,特異値$ \sigma_r(t)\ (r=1, \ldots, \min\{d,d^\prime \})$の損失関数$\mathcal{L}(W(t))$に対する勾配降下法による変化は


\frac{d \sigma_r(t)}{dt} = - N \cdot \left[\sigma_r(t)\right]^{1 - \frac{1}{N}} \cdot \left\langle \nabla \mathcal{L}(W(t)) , \mathbf{u}_r(t) \mathbf{v}_r^\top(t) \right\rangle


と表される (Arora et al., 2019; Theorem 3).(1)式で重要なのは$\left[\sigma_r(t)\right]^{1 - \frac{1}{N}}$の項である.これは$N\geq 2$のときに\textbf{特異値$\sigma_r(t)\ (\geq 0)$を小さくするような正則化作用が生じる}ことを意味している.一方で,隠れ層が1つのニューラルネットワーク ($N=1$)の場合 (1)式は


\frac{d \sigma_r(t)}{dt} = - \left\langle \nabla \mathcal{L}(W(t)) , \mathbf{u}_r(t) \mathbf{v}_r^\top(t) \right\rangle


となり,正則化作用は消失する.

このように線形多層ニューラルネットワークを勾配降下法で学習させると\textbf{陰的正則化(implicit regularization)} により低ランクの解が得られるということが複数の研究により明らかとなっている (線形多層ニューラルネットワークの陰的正則化に関して日本語で書かれた資料としては鈴木大慈先生の[深層学習の数理](https://www.slideshare.net/trinmu/ss-161240890)のスライドp.64, 65がある).Jingらはこの性質を用い,\textbf{Autoencoderに線形層を複数追加}するという簡便な方法で低次元表現を学習する決定論的Autoencoder (\textbf{Implicit Rank-Minimizing Autoencoder; IRMAE)} を考案した.
