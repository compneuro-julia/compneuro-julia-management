\subsection{線形多層ニューラルネットワークにおける学習ダイナミクスと知識の獲得}

> A. M. Saxe, J. L. McClelland, S. Ganguli. "\textbf{A mathematical theory of semantic development in deep neural networks}". *PNAS.* (2019). ([arXiv](https://arxiv.org/abs/1810.10531)). ([PNAS](https://www.pnas.org/content/early/2019/05/16/1820226116))

\subsubsection{モデルと学習}
入力 $\mathbf{x}$ は「もの」の項目(例えばカナリア,犬,サーモン,樫など),出力 $\mathbf{y}$はそれぞれの項目の性質・特性となっている.例えばカナリア(Canary)は成長し(Grow),動き(Move),空を飛べる(Fly)ので,Canaryという入力に対し,ネットワークが出力するのはGrow, Move, Flyとなる.モデルは3層の全結合線形ネットワークである.


\hat{\mathbf{y}}=\mathbf{W}_2 \mathbf{W}_1\mathbf{x} 


ただし非線形な活性化関数が無いことに注意しよう.このようなネットワークを線形ニューラルネットワーク (linear neural network)と呼ぶ.当然, $\mathbf{W}_s=\mathbf{W}_2 \mathbf{W}_1$として, 上のネットワークは


\hat{\mathbf{y}}=\mathbf{W}_s\mathbf{x}


とまとめることができる.このため,線形な活性化関数で深いニューラルネットワークを構築しても意味がなく,それゆえ非線形な活性化関数が必要となる.しかし,\textbf{勾配降下法で学習させると}3層と2層のネットワークの学習ダイナミクスはそれぞれ異なるものとなり,得られる解にも違いが生まれる.加えて,深い(3層の)ネットワークである場合のみ,幼児の発達における非線形な現象が説明できる.

3層ネットワークの学習(重みの更新)は誤差逆伝搬から導かれる次の2式により行う.


\begin{aligned} \tau \frac{d\mathbf{W}_1}{dt} &=\mathbf{W}_2^\top \left(\mathbf{\Sigma}^{yx} - \mathbf{W}_2 \mathbf{W}_1 \mathbf{\Sigma}^{x}\right)\\
\tau \frac{d\mathbf{W}_2}{dt} &=\left(\mathbf{\Sigma}^{yx} - \mathbf{W}_2 \mathbf{W}_1 \mathbf{\Sigma}^{x}\right) \mathbf{W}_1^\top
\end{aligned}


ただし,$ \mathbf{\Sigma}^{x}$は入力間の関係を表す行列,$\mathbf{\Sigma}^{yx}$は入出力の関係を表す行列である.

\subsubsection{特異値分解(SVD)による学習ダイナミクスの解析}
学習ダイナミクスは$ \mathbf{\Sigma}^{yx}$に対する特異値分解(singular value decomposition; SVD)を用いて説明できる.


\mathbf{\Sigma}^{yx}=\mathbf{USV}^\top


行列$ \mathbf{ S}$の対角成分の非ゼロ要素が特異値である.次に学習途中の時刻$(t)$における$\hat{\mathbf{\Sigma}}^{yx}(t)=\mathbf{W}_2 (t) \mathbf{W}_1(t) \mathbf{\Sigma}^{x}$に対してSVDを実行し,特異値$\mathbf{A}(t)=[a_{\alpha}(t)]$を得る.この $a _ {\alpha}(t)$だが,3層のネットワークでは大きな特異値から先に学習されるのに対し,2層のネットワークでは全ての特異値が同時に学習される.このダイナミクスだが,\textbf{低ランク近似} (low-rank approximation)が生じていて,特異値の大きな要素から学習されていると捉えることができる.学習が進むとランクが大きくなっていく,ということである.低ランク近似の例として,SVDによる画像の圧縮と復元を見てみよう.カメラマンの画像に対し,低ランク近似を行い,ランクを上げていく.するとランクが上がるにつれて,画像が鮮明になる.
