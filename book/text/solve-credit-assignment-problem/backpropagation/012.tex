$f(\cdot)$を活性化関数とする.順伝播(feedforward propagation)は以下のようになる.


\begin{align}
\text{入力層 : }&\mathbf{z}^{(0)}=\mathbf{x}\\
\text{隠れ層 : }&\mathbf{z}^{(\ell)}=f\left(\mathbf{a}^{(\ell)}\right)\\
&\mathbf{a}^{(\ell+1)}=W^{(\ell+1)}\mathbf{z}^{(\ell)}+\mathbf{b}^{(\ell+1)}\\
\text{出力層 : }&\hat{\mathbf{y}}=\mathbf{z}^{(L)}
\end{align}


逆伝播(backward propagation)


\begin{align}
\text{目的関数 : }&\mathcal{L}=\frac{1}{2}\left\|\hat{\mathbf{y}}-\mathbf{y}\right\|^{2}\\
\text{最急降下法 : }&\Delta W^{(\ell)}=-\eta \frac{\partial \mathcal{L}}{\partial W^{(\ell)}}\\
&\Delta \mathbf{b}^{(\ell)}=-\eta \frac{\partial \mathcal{L}}{\partial \mathbf{b}^{(\ell)}}\\
\text{誤差逆伝播法 : }&\frac{\partial \mathcal{L}}{\partial \hat{\mathbf{y}}}=\frac{\partial \mathcal{L}}{\partial \mathbf{z}^{(L)}}=\hat{\mathbf{y}}-\mathbf{y}\\
&\delta^{(L)}=\frac{\partial \mathcal{L}}{\partial \mathbf{z}^{(L)}} \frac{\partial \mathbf{z}^{(L)}}{\partial \mathbf{a}^{(L)}}=\left(\hat{\mathbf{y}}-\mathbf{y}\right) \odot f^{\prime}\left(\mathbf{a}^{(L)}\right)\\
&\mathbf{\delta}^{(\ell)}=\frac{\partial \mathcal{L}}{\partial \mathbf{z}^{(\ell)}} \frac{\partial \mathbf{z}^{(\ell)}}{\partial \mathbf{a}^{(\ell)}}=\left(W^{(\ell+1)}\right)^\top \delta^{(\ell+1)} \odot f^{\prime}\left(\mathbf{a}^{(\ell)}\right)\\
&\frac{\partial \mathcal{L}}{\partial W^{(\ell)}}=\frac{\partial \mathcal{L}}{\partial \mathbf{z}^{(\ell)}} \frac{\partial \mathbf{z}^{(\ell)}}{\partial \mathbf{a}^{(\ell)}} \frac{\partial \mathbf{a}^{(\ell)}}{\partial W^{(\ell)}}=\delta^{(\ell)}\left(\mathbf{z}^{(\ell-1)}\right)^\top\\
&\frac{\partial \mathcal{L}}{\partial \mathbf{b}^{(\ell)}}=\frac{\partial \mathcal{L}}{\partial \mathbf{z}^{(\ell)}} \frac{\partial \mathbf{z}^{(\ell)}}{\partial \mathbf{a}^{(\ell)}} \frac{\partial \mathbf{a}^{(\ell)}}{\partial \mathbf{b}^{(\ell)}}=\delta^{(\ell)}
\end{align}


バッチ処理を考慮すると,行列を乗ずる順番が変わる.
