入力は64(網膜座標系での位置)+2(眼球位置信号)=66とする.眼球位置信号は原著ではmonotonic形式による32(=8ユニット×2(x, y方向)×2 (傾き正負))ユニットで構成されるが,簡単のために眼球位置信号も$x, y$の2次元とする.視覚刺激は-40度から40度までの範囲であり,10度で離散化する.よって,網膜座標系での位置は$8\times 8$の行列で表現される.位置は2次元のGaussianで表現する.ただし,1/e幅 (ピークから1/eに減弱する幅) は15度である.$1/e$の代わりに$1/2$とすれば半値全幅(FWHM)となる.スポットサイズを$w$,Gaussianを$G(x)$とすると.$G(x+w/2)=G/e$より,$\sigma=\frac{\sqrt{2}w}{4}$と求まる.
