\subsection{補足:Monotonic formatによる位置のエンコーディング}
monotonic形式を入力の眼球位置と出力の頭部中心座標で用いるという仮定には,視覚刺激を中心窩で捉えた際,得られる眼球位置信号を頭部中心座標での位置の教師信号として使用できるという利点がある.([Andersen & Mountcastle, J. Neurosci. 1983](https://pubmed.ncbi.nlm.nih.gov/6827308/))では Parietal visual neurons (PVNs)の活動を調べ,傾き正あるいは負.0度をピークとして減少あるいは上昇の4種類あることを示した.前者は一次関数 (とReLU関数) で記述可能である.
