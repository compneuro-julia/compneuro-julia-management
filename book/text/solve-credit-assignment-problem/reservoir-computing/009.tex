途中で少し不思議に思われるようなことをしています。\\
\colorbox{shadecolor}{\texttt{PSC = synapses\_rec(JD*(len\_idx>0))}}の部分(とその少し上)ですが、これはデコードに用いる\texttt{r}を行列変換するよりも発火した結合重みの和を取り、再帰入力のシナプス後細胞のモデルに入力した方が速いという理由によります。\texttt{t}が一定のステップの範囲にある場合はFORCE法により学習を実行します。最後に各種変数を記録しています。\par
それでは学習後の結果を表示しましょう。初めに発火数と発火率を表示し、次に学習前と学習後の5つのニューロンの膜電位、最後に学習前/中間と学習後のデコード結果を描画します(なお、この本に記載はしていないですがコードには重みの固有値の描画も付けています)。
