\subsection{RLSフィルタのアルゴリズム}
\footnote{ModelDBにおいて公開されているMATLABのコード(\url{https://senselab.med.yale.edu/ModelDB/ShowModel.cshtml?model=190565})を参考にしました。}
FORCE法は\textbf{RLSフィルタ}(recursive least squares filter, 再帰的最小二乗法フィルタ)という\textbf{適応フィルタ}(adaptive filter)の一種を学習するアルゴリズムを、RNNの学習に適応したものです。
誤差を 
\begin{equation}
\boldsymbol{e}(t)=\hat{\boldsymbol{x}}(t)-\boldsymbol{x}(t)=\phi(t-\Delta t)^\intercal \boldsymbol{r}(t)-\boldsymbol{x}(t)    
\end{equation}
とした場合\footnote{実際にはこれは真の誤差ではなく、事前誤差(apriori error)と呼ばれるものです。真の誤差は$\phi(t)^\intercal \boldsymbol{r}(t)-\boldsymbol{x}(t)$と表されます。}、出力重み$\phi$を次の式で更新します。
\begin{align}
\phi(t)&=\phi(t-\Delta t)-\boldsymbol{P}(t) \boldsymbol{r}(t)\boldsymbol{e}(t)^\intercal\\
\boldsymbol{P}(t)&=\boldsymbol{P}(t-\Delta t)-\frac{\boldsymbol{P}(t-\Delta t) \boldsymbol{r}(t) \boldsymbol{r}(t)^\intercal \boldsymbol{P}(t-\Delta t)}{1+\boldsymbol{r}(t)^\intercal \boldsymbol{P}(t-\Delta
t) \boldsymbol{r}(t)} 
\end{align}
ここで$^\intercal$を転置記号とし、$\boldsymbol{x}$を列ベクトル、$\boldsymbol{x}^\intercal$を行ベクトルとします。また、初期値は$\phi(0)=0,
\boldsymbol{P}(0)=I_{N}\lambda^{-1}$です。$I_{N}$は$N$次の単位行列を意味します。$\lambda$は正則化のための定数です。
\section{RLSフィルタの導出}
ここからはRLSフィルタの導出を行います。まずReservoirニューロンの数を$N$とし、出力の数を$N_\text{out}$とします。RLSフィルタでは次の損失関数$C\in \mathbb{R}^{N_\text{out}}$を最小化するような重み$\phi=[\phi_j]\in \mathbb{R}^{N\times N_\text{out}}$を求めます。シミュレーション時間を$T$とすると、$C$は
\begin{equation}
C=\int_{0}^T(\hat{\boldsymbol{x}}(t)-\boldsymbol{x}(t))^{2} \mathrm{d} t+\lambda \phi^\intercal \phi
\end{equation}
です。ただし、$\hat{\boldsymbol{x}}(t), \boldsymbol{x}(t) \in \mathbb{R}^{N_\text{out}}$です。\par
さて、式の$C$を最小化するような$\phi$を数値的に求めるためには、損失関数の近似が必要です。まず、
時間幅$\Delta t$で離散化したステップ数を$n=T/\Delta t$とし、$C$を離散化します。さらに$n$ステップ目における重み$\phi(n)$により、$\hat{\boldsymbol{x}}(i)\simeq \phi(n)^\intercal \boldsymbol{r}(i)$と近似します。このとき、$n$ステップ目の損失関数$C(n)$は
\begin{align}
C(n)&\simeq \sum_{i=0}^{n}(\hat{\boldsymbol{x}}(i)-\boldsymbol{x}(i))^{2}+\lambda \phi(n)^\intercal \phi(n)\\     
&\simeq \sum_{i=0}^{n}(\phi(n)^\intercal \boldsymbol{r}(i)-\boldsymbol{x}(i))^{2}+\lambda \phi(n)^\intercal \phi(n)
\end{align}
となります。ここでL2正則化(ridge)付きの(通常の)最小二乗法の\textbf{正規方程式}(normal equation)により、$C(n)$を最小化する$\phi(n)$は
\begin{align}
\phi(n) &= \left[\sum_{i=0}^{n}(\boldsymbol{r}(i)\boldsymbol{r}(i)^\intercal+\lambda I_N)\right]^{-1}\left[\sum_{i=0}^{n}\boldsymbol{r}(i)\boldsymbol{x}(i)^\intercal\right]\\
&=P(n)\psi(n)
\end{align}
となります\footnote{重み$\phi$で$C$を微分し、勾配が0となるときの方程式の解です。}。ただし、
\begin{align}
P(n)^{-1}&= \sum_{i=0}^{n}(\boldsymbol{r}(i)\boldsymbol{r}(i)^\intercal+\lambda I_N)\ \left(=\int_{0}^T \boldsymbol{r}(t) \boldsymbol{r}(t)^\intercal \mathrm{d} t+\lambda I_{N}\right)\\
\psi(n)&=\sum_{i=0}^{n}\boldsymbol{r}(i)\boldsymbol{x}(i)^\intercal
\end{align}
です。$\boldsymbol{P}(n)$は$\boldsymbol{r}(n)$の相関行列の時間積分と係数倍した単位行列の和の逆行列となっています。また、
\begin{equation}
P(n)^{-1}=P(n-1)^{-1}+\boldsymbol{r}(n) \boldsymbol{r}(n)^\intercal
\end{equation}
となります。ここで、\textbf{逆行列の補助定理}(Matrix Inversion Lemma, またはSherman-Morrison-Woodbury Identity)より、
\begin{align}
X&=A+BCD\\
\Rightarrow X^{-1}&=A^{-1} - A^{-1}B(C^{-1}+DA^{-1}B)^{-1}DA^{-1}
\end{align}
となるので、$X={P}(n)^{-1}, A=\boldsymbol{P}(n-1)^{-1}, B= \boldsymbol{r}(n), C=I_{N}, D=\boldsymbol{r}(n)^\intercal$とすると、
\begin{align}
\boldsymbol{P}(n)&=\boldsymbol{P}(n-1)-\frac{\boldsymbol{P}(n-1) \boldsymbol{r}(n) \boldsymbol{r}(n)^\intercal \boldsymbol{P}(n-1)}{1+\boldsymbol{r}(n)^\intercal \boldsymbol{P}(n-1) \boldsymbol{r}(n)} 
\end{align}
が成り立ちます(右辺2項目の分母はスカラーとなります)。
さらに
\begin{align}
\psi(n)&=\psi(n-1)+\boldsymbol{r}(n)\boldsymbol{x}(n)^\intercal\\
&=P(n-1)^{-1}\phi(n-1)+\boldsymbol{r}(n)\boldsymbol{x}(n)^\intercal\\
&=\left\{P(n)^{-1}-\boldsymbol{r}(n) \boldsymbol{r}(n)^\intercal\right\}\phi(n-1)+\boldsymbol{r}(n)\boldsymbol{x}(n)^\intercal
\end{align}
となります。式から式へは
%%ここ式の番号入れる
\begin{equation}
\phi(n)=P(n)\psi(n) \Rightarrow \psi(n)=P(n)^{-1}\phi(n)
\end{equation}
であること、式から式へは式により、
\begin{equation}
P(n-1)^{-1}=P(n)^{-1}-\boldsymbol{r}(n) \boldsymbol{r}(n)^\intercal
\end{equation}
であることを用いています。よって、
\begin{align}
\phi(n)&=P(n)\psi(n)\\
&=P(n)\left[\left\{P(n)^{-1}-\boldsymbol{r}(n) \boldsymbol{r}(n)^\intercal\right\}\phi(n-1)+\boldsymbol{r}(n)\boldsymbol{x}(n)^\intercal\right]\\
&=\phi(n-1)-P(n)\boldsymbol{r}(n)\boldsymbol{r}(n)^\intercal\phi(n-1)+P(n)\boldsymbol{r}(n)\boldsymbol{x}(n)^\intercal\\
&=\phi(n-1)-P(n)\boldsymbol{r}(n)\left[\boldsymbol{r}(n)^\intercal\phi(n-1)-\boldsymbol{x}(n)^\intercal\right]\\
&=\phi(n-1)-P(n)\boldsymbol{r}(n)\boldsymbol{e}(n)^\intercal
\end{align}
となります。式と式を連続時間での表記法にすると、前節における式と式の更新式となります。
