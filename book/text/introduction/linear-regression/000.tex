\section{線形回帰と最小二乗法}
\subsection{線形回帰}
$n$個のデータ $\left(y_1,x_{11}, \ldots x_{1p}\right),\ldots \left(y_n,x_{n1},\ldots, x_{np}\right)$ を説明変数$p$個の線形モデル 


y=\theta_0+\theta_1x_1+\cdots+\theta_px_p+\varepsilon=\theta_0+\sum_{j=1}^p \theta_jx_j+\varepsilon


で説明することを考える (説明変数が単一の場合を単回帰,複数の場合を重回帰と呼ぶことがある).ここで, 


\mathbf{y}= \left[ \begin{array}{c} y_1\\ y_2\\ \vdots \\ y_n \end{array} \right],\quad 
\mathbf{X}=\left[ \begin{array}{ccccc} 1 & x_{11}& x_{12} &\cdots & x_{1p} \\ 1& x_{21}& x_{22}&\cdots & x_{2p}\\ \vdots & \vdots& \vdots& & \vdots \\1 &x_{n1} & x_{n2} &\cdots & x_{np} \end{array} \right],\quad
\mathbf{\theta}= \left[ \begin{array}{c} \theta_0\\ \theta_1\\ \vdots \\ \theta_p \end{array} \right]


とすると,線形回帰モデルは $\mathbf{y}=\mathbf{X}\mathbf{\theta}+\mathbf{\varepsilon}$と書ける.ただし,$\mathbf{X}$は計画行列 (design matrix),$\mathbf{\varepsilon}$は誤差項である.特に,$\mathbf{\varepsilon}$が平均0, 分散$\sigma^2$の独立な正規分布に従う場合,$\mathbf{y}\sim \mathcal{N}(\mathbf{X}\mathbf{\theta}, \sigma^2\mathbf{I})$と表せる.
