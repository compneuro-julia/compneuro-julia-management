\section{確率論}
\subsection{期待値 (Expectation)}


\mathbb{E}_{x\sim p(x)}\left[f(x)\right]:=\int f(x)p(x)dx


$x\sim p(x)$ が明示的な場合は $\mathbb{E}_{p(x)}\left[f(x)\right]$ や $\mathbb{E}\left[f(x)\right]$ と表す.

\subsection{情報量 (Information)}
出現頻度が低い事象は多くの情報量を持つ (Shannon, 1948).


\mathbb{I}(x):=\ln\left(\frac{1}{p(x)}\right)=-\ln p(x)


$\mathbf{I}$は単位行列なので注意.

\subsection{平均情報量 (エントロピー, entropy)}


\begin{aligned}
\mathbb{H}(x)&:=\mathbb{E}[-\ln p(x)]\\
\mathbb{H}(x\vert y)&:=\mathbb{E}[-\ln p(x\vert y)]
\end{aligned}


\subsection{Kullback-Leibler 情報量}
Kullback-Leibler (KL) divergence (Kullback and Leibler, 1951)


\begin{aligned}
D_{\text{KL}}\left(p(x) \Vert\ q(x)\right)&:=\int p(x) \ln \frac{p(x)}{q(x)} dx\\
&=\int p(x) \ln p(x) dx-\int p(x) \ln q(x) dx\\
&=\mathbb{E}_{x\sim p(x)}[\ln p(x)]-\mathbb{E}_{x\sim p(x)}[\ln q(x)]\\
&=-\mathbb{H}(x)-\mathbb{E}_{x\sim p(x)}[\ln q(x)]
\end{aligned}


\subsection{相互情報量 (Mutual information)}
aaa
