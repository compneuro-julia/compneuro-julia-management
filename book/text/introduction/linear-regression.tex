\section{線形回帰と最小二乗法}
\subsection{線形回帰}
$n$個のデータ $\left(y_1,x_{11}, \ldots x_{1p}\right),\ldots \left(y_n,x_{n1},\ldots, x_{np}\right)$ を説明変数$p$個の線形モデル 
\begin{equation}
y=\theta_0+\theta_1x_1+\cdots+\theta_px_p+\varepsilon=\theta_0+\sum_{j=1}^p \theta_jx_j+\varepsilon
\end{equation}
で説明することを考える (説明変数が単一の場合を単回帰,複数の場合を重回帰と呼ぶことがある).ここで, 
\begin{equation}
\mathbf{y}= \left[ \begin{array}{c} y_1\\ y_2\\ \vdots \\ y_n \end{array} \right],\quad 
\mathbf{X}=\left[ \begin{array}{ccccc} 1 & x_{11}& x_{12} &\cdots & x_{1p} \\ 1& x_{21}& x_{22}&\cdots & x_{2p}\\ \vdots & \vdots& \vdots& & \vdots \\1 &x_{n1} & x_{n2} &\cdots & x_{np} \end{array} \right],\quad
\mathbf{\theta}= \left[ \begin{array}{c} \theta_0\\ \theta_1\\ \vdots \\ \theta_p \end{array} \right]
\end{equation}
とすると,線形回帰モデルは $\mathbf{y}=\mathbf{X}\mathbf{\theta}+\mathbf{\varepsilon}$と書ける.ただし,$\mathbf{X}$は計画行列 (design matrix),$\mathbf{\varepsilon}$は誤差項である.特に,$\mathbf{\varepsilon}$が平均0, 分散$\sigma^2$の独立な正規分布に従う場合,$\mathbf{y}\sim \mathcal{N}(\mathbf{X}\mathbf{\theta}, \sigma^2\mathbf{I})$と表せる.
\subsection{最小二乗法によるパラメータの推定}
最小二乗法 (ordinary least squares)により線形回帰のパラメータを推定する.$y$の予測値は$\mathbf{X} \mathbf{\theta}$なので,誤差 $\mathbf{\delta} \in \mathbb{R}^n$は
$\mathbf{\delta} = \mathbf{y}-\mathbf{X} \mathbf{\theta}$と表せる.ゆえに目的関数$L(\mathbf{\theta})$は 
\begin{equation}
L(\theta)=\sum_{i=1}^n \delta_i^2 = \|\mathbf{\delta}\|^2=\mathbf{\delta}^\top \mathbf{\delta}
\end{equation}
となり, $L(\mathbf{\theta})$を最小化する$\mathbf{\theta}$, つまり $\hat {\mathbf {\theta }}={\underset {\mathbf {\theta}}{\operatorname {arg min} }}\,L({\mathbf{\theta}})$
を求める.
\subsubsection{勾配法を用いた推定}
最小二乗法による回帰直線を勾配法で求めてみよう.$\theta$の更新式は$\theta \leftarrow \theta + \alpha\cdot \dfrac{1}{n} \delta \mathbf{X}$と書ける.ただし,$\alpha$は学習率である.
\begin{lstlisting}[language=julia]
using PyPlot, LinearAlgebra, Random
rc("axes.spines", top=false, right=false)
\end{lstlisting}
\begin{lstlisting}[language=julia]
# Ordinary least squares regression
function OLSRegGradientDescent(X, y, initθ; lr=1e-4, num_iters=10000)
    θ = initθ
    for i in 1:num_iters
        ŷ = X * θ # predictions
        δ = y - ŷ  # error
        θ += lr * X' * δ # Update
    end
    return θ
end;
\end{lstlisting}
\begin{lstlisting}[language=julia]
# Generate Toy datas
num_train, num_test = 100, 500 # sample size
dims = 4 # dimensions
Random.seed!(0);

x = rand(num_train) #range(0.1, 0.9, length=num_train)
y =  sin.(2π*x) + 0.3randn(num_train);
X = hcat([x .^ p for p in 0:dims-1]...); # design matrix

xtest = range(0, 1, length=num_test)
ytest =  sin.(2π*xtest)
Xtest =hcat([xtest .^ p for p in 0:dims-1]...);
\end{lstlisting}
\begin{lstlisting}[language=julia]
# Gradient descent
initθ = zeros(dims) # init variables
θgd = OLSRegGradientDescent(X, y, initθ, lr=1e-2, num_iters=1e5)
ŷgd = Xtest * θgd; # predictions
\end{lstlisting}
\begin{lstlisting}[language=julia]
figure(figsize=(5,3.5))
title("Linear Regression with Gradient descent")
scatter(x, y, color="gray", s=10) # samples
plot(xtest, ytest, label="actual")  # regression line
plot(xtest, ŷgd, label="predicted")  # regression line
xlabel("x"); ylabel("y"); legend()
tight_layout()
\end{lstlisting}
\begin{figure}[ht]
	\centering
	\includegraphics[scale=0.8, max width=\linewidth]{./fig/introduction/linear-regression/cell007.png}
	\caption{cell007.png}
	\label{cell007.png}
\end{figure}
\subsubsection{正規方程式を用いた推定}
条件に基づいて目的関数$L(\mathbf{\theta})$を微分すると次のような方程式が得られる.
\begin{equation}
\mathbf{X}^\top\mathbf{X}\mathbf{\hat\theta}=\mathbf{X}^\top\mathbf{y}
\end{equation}
これを\textbf{正規方程式}\index{せいきほうていしき@正規方程式} (normal equation)と呼ぶ.この正規方程式より、係数の推定値は$\mathbf{\hat\theta}={(\mathbf{X}^\top\mathbf{X})}^{-1}X^\top\mathbf{y}$という式で得られる.なお,正規方程式自体は$\mathbf{y}=\mathbf{X}\mathbf{\theta}$の左から$\mathbf{X}^\top$をかける,と覚えると良い.
\begin{lstlisting}[language=julia]
θne = (X' * X) \ X' * y
ŷne = Xtest * θne; # predictions
\end{lstlisting}
\begin{lstlisting}[language=julia]
figure(figsize=(5,3.5))
title("Linear Regression with Normal equation")
scatter(x, y, color="gray", s=10) # samples
plot(xtest, ytest, label="actual")  # regression line
plot(xtest, ŷne, label="predicted")  # regression line
xlabel("x"); ylabel("y"); legend()
tight_layout()
\end{lstlisting}
\begin{figure}[ht]
	\centering
	\includegraphics[scale=0.8, max width=\linewidth]{./fig/introduction/linear-regression/cell010.png}
	\caption{cell010.png}
	\label{cell010.png}
\end{figure}
