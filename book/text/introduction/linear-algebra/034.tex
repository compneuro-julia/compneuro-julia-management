\subsection{Roth's column lemma}Roth's column lemmaは,例えば,$A, B, C$が与えられていて,$X$を未知とするときの方程式 $AXB = C$を考えると,この方程式は
$$
(B^\top \otimes A)\text{vec}(X) = \text{vec}(AXB)=\text{vec}(C)
$$
の形に書き下すことができる,というものである.$\text{vec}(\cdot)$はvec作用素 (行列を列ベクトル化する作用素) である.\jl{vec(X) = vcat(X...)}で実現できる.Roth's column lemmaを用いれば,$AXB = C$の解は
$$
X = \text{vec}^{-1}\left((B^\top \otimes A)^{-1}\text{vec}(C)\right)
$$
として得られる.ただし,$\text{vec}(\cdot)^{-1}$は列ベクトルを行列に戻す作用素(inverse of the vectorization operator)である.\jl{reshape()}で実現できる.2つの作用素をまとめると,
$$
\begin{align}
\text{vec} &: R^{m\times n}\to R^{mn}\\
\text{vec}^{−1} &: R^{mn}\to R^{m×n}
\end{align}
$$
であり,$\text{vec}^{−1}\left(\text{vec}(X)\right)=X\ (\text{for all}\ X\in R^{m\times n}),\text{vec}\left(\text{vec}^{−1}(x)\right)=x\ (\text{for all}\ x \in R^{mn})$となる.