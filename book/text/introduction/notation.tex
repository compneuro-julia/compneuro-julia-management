\section{記号の表記}
本書では次のような記号表記を用いる.
\begin{itemize}
\item 実数全体を$\mathbb{R}$, 複素数全体は$\mathbb{C}$と表記する.
\item スカラーは小文字・斜体で $x$ のように表記する.
\item ベクトルは小文字・立体・太字で $\mathbf{x}$ のように表記し,列ベクトル (縦ベクトル) として扱う.
\item 行列は大文字・立体・太字で $\mathbf{X}$ のように表記する.
\item $n\times 1$の実ベクトルの集合を $\mathbb{R}^n$, $n\times m$ の実行列の集合を $\mathbb{R}^{n\times m}$と表記する.
\item 行列 $\mathbf{X}$ の置換は $\mathbf{X}^\top$と表記する.ベクトルの要素を表す場合は $\mathbf{x} = (x_1, x_2,\cdots, x_n)^\top$のように表記する.
\item 単位行列を $\mathbf{I}$ と表記する.
\item ゼロベクトルは $\mathbf{0}$ , 要素が全て1のベクトルは $\mathbf{1}$ と表記する.  
\item $e$を自然対数の底とし,指数関数を $e^x=\exp(x)$と表記する.また,自然対数を $\ln(x)$と表記する.
\item 定義を$\triangleq$を用いて行う.例えば,$f(x)\triangleq2x$は$f(x)$という関数を$2x$として定義するという意味である.
\item 平均 $\mu$, 標準偏差 $\sigma$ の正規分布を $\mathcal{N}(\mu, \sigma^2)$ と表記する.
\end{itemize}
\subsection{変数の命名規則}
\begin{itemize}
\item \jl{tp1, tm1} : time plus one (t+1), time minus one (t-1)
\end{itemize}
