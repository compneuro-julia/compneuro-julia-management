\section{Julia言語の基本構文}
\subsection{変数}
aaa
\begin{lstlisting}[language=julia]
x = 1
α = 2 # \alpha + TAB key
\end{lstlisting}
|演算子|説明|使用例|結果|
|:-|:-|:-|:-|
|\jl{+}|和|aaa|aaa|
|\jl{-}|差|aaa|aaa|
|\jl{*}|積|aaa|aaa|
|\jl{.*}|配列の要素積|aaa|aaa|
|\jl{/}|除算,右から逆行列をかける|aaa|aaa|
|\jl{\}|左から逆行列をかける|aaa|aaa|
\jl{var}を用いることで,任意の文字列を変数にすることができる.
\begin{lstlisting}[language=julia]
var"log(1+θ)" = 10
\end{lstlisting}
Juliaの余りの関数は \jl{rem(x, y)} と \jl{mod(x, y)}がある.Juliaの\jl{x % y}は\jl{rem}と同じだが,Pythonの場合は\jl{mod}と同じなので注意.
\begin{lstlisting}[language=julia]
println("% : ", -1 % 2, ", rem : ", rem(-1, 2), ", mod : ", mod(-1, 2))
\end{lstlisting}
\subsection{for loop}
aaa
\begin{lstlisting}[language=julia]
x = 1
for i in 1:10
    x += 1
end
println(x)
\end{lstlisting}
\subsection{関数名の!記号}
単なる\textbf{慣習}\index{かんしゅう@慣習}として関数への入力を変更する場合に!を付ける.
関数内で配列を変更する場合には注意が必要である.以下に入力された配列を同じサイズの要素1の配列で置き換える,ということを目的として書かれた2つの関数がある.違いは\jl{v}の後に\jl{[:]}としているかどうかである.
\begin{lstlisting}[language=julia]
function wrong!(a::Array)
    a = ones(size(a))
end

function right!(a::Array)
    a[:] = ones(size(a))
end
\end{lstlisting}
実行すると\jl{wrong!}の場合には入力された配列が変更されていないことがわかる.
\begin{lstlisting}[language=julia]
using Random
v = rand(2, 2)
println("v : ", v)

wrong!(v)
println("wrong : ", v)

right!(v)
println("right : ", v)
\end{lstlisting}
\subsection{broadcastingの回避}
aaa
\begin{lstlisting}[language=julia]
foo(a,b) = sum(a) + b
\end{lstlisting}
\begin{lstlisting}[language=julia]
println(foo.(Ref([1,2]),[3,4,5]))
println(foo.(([1,2],), [3,4,5]))
println(foo.([[1,2]], [3,4,5]))
\end{lstlisting}
