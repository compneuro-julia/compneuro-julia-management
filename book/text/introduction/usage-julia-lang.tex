\section{Julia言語の基本構文}
\subsection{変数}
aaa
\lstinputlisting[language=julia]{./text/introduction/usage-julia-lang/001.jl}
|演算子|説明|使用例|結果|
|:-|:-|:-|:-|
|\jl{+}|和|aaa|aaa|
|\jl{-}|差|aaa|aaa|
|\jl{*}|積|aaa|aaa|
|\jl{.*}|配列の要素積|aaa|aaa|
|\jl{/}|除算,右から逆行列をかける|aaa|aaa|
|\jl{\}|左から逆行列をかける|aaa|aaa|
\jl{var}を用いることで,任意の文字列を変数にすることができる.
\lstinputlisting[language=julia]{./text/introduction/usage-julia-lang/004.jl}
Juliaの余りの関数は \jl{rem(x, y)} と \jl{mod(x, y)}がある.Juliaの\jl{x % y}は\jl{rem}と同じだが,Pythonの場合は\jl{mod}と同じなので注意.
\lstinputlisting[language=julia]{./text/introduction/usage-julia-lang/006.jl}
\subsection{for loop}
aaa
\lstinputlisting[language=julia]{./text/introduction/usage-julia-lang/008.jl}
\subsection{関数名の!記号}
単なる\textbf{\index{かんしゅう@慣習}}として関数への入力を変更する場合に!を付ける.

関数内で配列を変更する場合には注意が必要である.以下に入力された配列を同じサイズの要素1の配列で置き換える,ということを目的として書かれた2つの関数がある.違いは\jl{v}の後に\jl{[:]}としているかどうかである.
\lstinputlisting[language=julia]{./text/introduction/usage-julia-lang/010.jl}
実行すると\jl{wrong!}の場合には入力された配列が変更されていないことがわかる.
\lstinputlisting[language=julia]{./text/introduction/usage-julia-lang/012.jl}
\subsection{broadcastingの回避}
aaa
\lstinputlisting[language=julia]{./text/introduction/usage-julia-lang/014.jl}
\lstinputlisting[language=julia]{./text/introduction/usage-julia-lang/015.jl}
