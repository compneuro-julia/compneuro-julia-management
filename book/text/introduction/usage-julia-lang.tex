\section{Julia言語の基本構文}
\subsection{変数}
変数への代入は\jl{=}で行う.変数はUnicodeも使用可能であり,LaTeXコマンドとTABキーで入力することが可能である.
\begin{lstlisting}[language=julia]
x = 1
α = 2 # \alpha + TAB key
\end{lstlisting}
\begin{table}[h]
\centering
\begin{tabular}{|c|l|l|l|}
\hline
演算子 & 説明 & 使用例 & 結果 \\
\hline
\jl{+} & 和 & aaa & aaa \\
\hline
\jl{-} & 差 & aaa & aaa \\
\hline
\jl{*} & 積 & aaa & aaa \\
\hline
\jl{.*} & 配列の要素積 & aaa & aaa \\
\hline
\jl{/} & 除算,右から逆行列をかける & aaa & aaa \\
\hline
\jl{\} & 左から逆行列をかける & aaa & aaa \\
\hline
\end{tabular}
\end{table}
\jl{var}を用いることで,任意の文字列を変数にすることができる.
\begin{lstlisting}[language=julia]
var"log(1+θ)" = 3
\end{lstlisting}
\subsection{条件分岐}
\jl{if}構文を用いることで条件分岐が可能である.
\begin{lstlisting}[language=julia]
a = 2
if a > 0
    print("positive")
elseif a == 0
    print("zero")
else
    print("negative")
end
\end{lstlisting}
\subsection{再帰的処理}
再帰的な処理を行う場合には主に\jl{for}loop 構文を用いる.
\begin{lstlisting}[language=julia]
x = 1
for i in 1:10
    x += 1
end
println(x)
\end{lstlisting}
\subsection{関数}
\jl{function}により関数を定義する.
なお,慣習として関数への入力を変更する場合に!を付けることがある.関数内で配列を変更する場合には注意が必要である.以下に入力された配列を同じサイズの要素1の配列で置き換える,ということを目的として書かれた2つの関数がある.違いは\jl{v}の後に\jl{[:]}としているかどうかである.
\begin{lstlisting}[language=julia]
function wrong!(a::Array)
    a = ones(size(a))
end

function right!(a::Array)
    a[:] = ones(size(a))
end
\end{lstlisting}
実行すると\jl{wrong!}の場合には入力された配列が変更されていないことがわかる.
\begin{lstlisting}[language=julia]
using Random
v = rand(2, 2)
println("v : ", v)

wrong!(v)
println("wrong : ", v)

right!(v)
println("right : ", v)
\end{lstlisting}
\subsection{数値計算}
broadcastingの回避を行うには以下のような方法がある.
\begin{lstlisting}[language=julia]
foo(a,b) = sum(a) + b
\end{lstlisting}
\begin{lstlisting}[language=julia]
println(foo.(Ref([1,2]),[3,4,5]))
println(foo.(([1,2],), [3,4,5]))
println(foo.([[1,2]], [3,4,5]))
\end{lstlisting}
\subsection{その他の関数について}
Juliaの余りの関数は \jl{rem(x, y)} と \jl{mod(x, y)}がある.Juliaの\jl{x % y}は\jl{rem}と同じだが,Pythonの場合は\jl{mod}と同じなので注意.
\begin{lstlisting}[language=julia]
println("% : ", -1 % 2, ", rem : ", rem(-1, 2), ", mod : ", mod(-1, 2))
\end{lstlisting}
