\subsubsection{1階線形行列微分方程式の解}時不変 (time-invariant) の定数行列を$\mathbf{A} \in \mathbb{R}^{n\times n}, \mathbf{B} \in \mathbb{R}^{n\times m}$, 状態ベクトルを$\mathbf{x}(t)\in\mathbb{R}^n$, 入力ベクトルを$\mathbf{u}(t)\in\mathbb{R}^m$とする.
$$
\frac{d\mathbf{x}(t)}{dt} = \mathbf{A}\mathbf{x}(t) + \mathbf{B}\mathbf{u}(t)
$$
この線形行列微分方程式をLaplace変換 $\mathcal{L}$を用いて解こう.$\boldsymbol{X}(s) := \mathcal{L}(\mathbf{x}(t)), \boldsymbol{U}(s) := \mathcal{L}(\mathbf{u}(t))$とすると,
$$
\begin{aligned}
s\boldsymbol{X}(s) - \mathbf{x}(0) &= \mathbf{A}\boldsymbol{X}(s)+ \mathbf{B}\boldsymbol{U}(s)\\
(s\mathbf{I} - \mathbf{A}) \boldsymbol{X}(s) &= \mathbf{x}(0) + \mathbf{B}\boldsymbol{U}(s)\\
\boldsymbol{X}(s) &= (s\mathbf{I} - \mathbf{A})^{-1}(\mathbf{x}(0) + \mathbf{B}\boldsymbol{U}(s))\\
\end{aligned}
$$
行列指数関数 (matrix exponential)は
$$
e^\mathbf{A} = \exp(\mathbf{A}) := \sum_{k=0}^\infty \frac{1}{k!}\mathbf{A}^k = \mathbf{I}+\mathbf{A}+\frac{\mathbf{A}^2}{2!}+\cdots
$$
として定義される.
天下り的だが,
$$
\begin{aligned}
\mathcal{L}(e^{at})&=\frac{1}{s-a}\\
\mathcal{L}(e^{t\mathbf{A}})&=(s\mathbf{I} - \mathbf{A})^{-1}\\
\end{aligned}
$$
となる.よって
$$
\begin{aligned}
\boldsymbol{X}(s) &= (s\mathbf{I} - \mathbf{A})^{-1}(\mathbf{x}(0) + \mathbf{B}\boldsymbol{U}(s))\\
&= (s\mathbf{I} - \mathbf{A})^{-1}\mathbf{x}(0) + (s\mathbf{I} - \mathbf{A})^{-1}\mathbf{B}\boldsymbol{U}(s)\\
\mathbf{x}(t)&=e^{t\mathbf{A}}\mathbf{x}(0)+\int_0^t e^{(t-\tau)\mathbf{A}}\mathbf{B}\mathbf{u}(\tau) d\tau
\end{aligned}
$$
となる.最後の式は両辺を逆Laplace変換した.ここで,$\mathcal{L}^{-1}(F(s)G(s))=\int_0^tf(\tau)g(t-\tau)d\tau$であることを用いた.区間$[t, t+\Delta t]$において入力$\mathbf{u}(t)$が一定であると仮定すると,
$$
\begin{aligned}
\mathbf{x}(t+\Delta t)&=e^{(t+\Delta t)\mathbf{A}}\mathbf{x}(0)+\int_0^{t+\Delta t} e^{(t+\Delta t-\tau)\mathbf{A}}\mathbf{B}\mathbf{u}(\tau) d\tau\\
&=e^{\Delta t\mathbf{A}}e^{t\mathbf{A}}\mathbf{x}(0)+e^{\Delta t\mathbf{A}}\int_0^{t} e^{(t-\tau)\mathbf{A}}\mathbf{B}\mathbf{u}(\tau) d\tau + \int_t^{t+\Delta t} e^{(t+\Delta t-\tau)\mathbf{A}}\mathbf{B}\mathbf{u}(\tau) d\tau\\
&\approx \underbrace{e^{\Delta t\mathbf{A}}}_{=: \mathbf{A}_d}\mathbf{x}(t)+\underbrace{\left[\int_t^{t+\Delta t} e^{(t+\Delta t-\tau)\mathbf{A}} d\tau\right] \mathbf{B}}_{=: \mathbf{B}_d}\mathbf{u}(t)\\
&=\mathbf{A}_d\mathbf{x}(t)+\mathbf{B}_d\mathbf{u}(t)\\
\end{aligned}
$$
となる.添え字の$d$は離散化(discretization)を意味する.$\mathbf{A}_c$が正則行列の場合,
$$
\begin{aligned}
\mathbf{B}_d &= \left[\int_t^{t+\Delta t} e^{(t+\Delta t-\tau)\mathbf{A}} d\tau\right] \mathbf{B}\\
&=\mathbf{A}^{-1}\left[e^{\Delta t \mathbf{A}}-\mathbf{I}\right]\mathbf{B}
\end{aligned}
$$
が成り立つ.