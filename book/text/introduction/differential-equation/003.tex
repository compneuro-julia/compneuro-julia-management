
\subsection{連続時間モデルから離散時間モデルへの変換}
\subsubsection{離散化方法1: 解析解を用いた方法}
\paragraph{1次元の場合}

区間$[t, t+\Delta t]$において入力$u(t)$が一定であると仮定すると,


\begin{aligned}
x(t+\Delta t)&= \underbrace{e^{a_c \Delta t}}_{=: a_d}\mathbf{x}(t)+\underbrace{\left[\int_t^{t+\Delta t} e^{a_c(t+\Delta t-\tau)} d\tau\right] b_c}_{=: b_d}u(t)\\
&=a_d x(t)+b_d u(t)\\
\end{aligned}


\paragraph{n次元の場合}
区間$[t, t+\Delta t]$において入力$\mathbf{u}(t)$が一定であると仮定すると,


\begin{aligned}
\mathbf{x}(t+\Delta t)&=e^{(t+\Delta t)\mathbf{A}_c}\mathbf{x}(0)+\int_0^{t+\Delta t} e^{(t+\Delta t-\tau)\mathbf{A}_c}\mathbf{B}_c\mathbf{u}(\tau) d\tau\\
&=e^{\Delta t\mathbf{A}_c}e^{t\mathbf{A}_c}\mathbf{x}(0)+e^{\Delta t\mathbf{A}_c}\int_0^{t} e^{(t-\tau)\mathbf{A}_c}\mathbf{B}_c\mathbf{u}(\tau) d\tau + \int_t^{t+\Delta t} e^{(t+\Delta t-\tau)\mathbf{A}_c}\mathbf{B}_c\mathbf{u}(\tau) d\tau\\
&\approx \underbrace{e^{\Delta t\mathbf{A}_c}}_{=: \mathbf{A}_d}\mathbf{x}(t)+\underbrace{\left[\int_t^{t+\Delta t} e^{(t+\Delta t-\tau)\mathbf{A}_c} d\tau\right] \mathbf{B}_c}_{=: \mathbf{B}_d}\mathbf{u}(t)\\
&=\mathbf{A}_d\mathbf{x}(t)+\mathbf{B}_d\mathbf{u}(t)\\
\end{aligned}



離散化した場合の


\mathbf{x}_{t+1} = \mathbf{A}_c\mathbf{x}_t + \mathbf{B}_c\mathbf{u}_t


状態遷移方程式 (dynamics equations) とも呼ぶ.
