\section{微分方程式}
\subsection{微分方程式の基礎}
微分方程式はある関数とそれを微分した導関数の関係式であり,関数の特定の変数に対する変化を記述することができる.まず,1階線形微分方程式を例として見てみよう.


\frac{dx(t)}{dt}=a_c x(t)+b_c u(t)


状態変数 $x(t)$は,時間$t$に対する関数である.

添え字の$c$は連続 (continuous) を意味するが,これは後で離散化する際に区別するためである.この方程式においては$b_c=0$の場合を\textbf{同次方程式}, $b_c\neq 0$の場合を\textbf{非同次方程式}という.

\subsubsection{微分方程式の解}
微分方程式を解くとは$x(t)$のような関数の具体的な式を求めることである.上式の解は


x(t)=e^{a_c t}x(0)+\int_0^t e^{a_c (t-\tau)}b_c u(\tau) d\tau


として与えられる.微分方程式を解く手法は様々で,それぞれの方程式について適切な手法を選択する.本書ではLaplace変換を多用するが,細かい説明は付録にて行う.

\subsubsection{連立線形微分方程式}
$n$個の微分方程式

連立線形微分方程式という.これをベクトル,行列を用いて

時不変 (time-invariant) の定数行列を$\mathbf{A}_c \in \mathbb{R}^{n\times n}, \mathbf{B}_c \in \mathbb{R}^{n\times m}$, 状態ベクトルを$\mathbf{x}(t)\in\mathbb{R}^n$, 入力ベクトルを$\mathbf{u}(t)\in\mathbb{R}^m$とする.


\frac{d\mathbf{x}(t)}{dt} = \mathbf{A}_c\mathbf{x}(t) + \mathbf{B}_c\mathbf{u}(t)


解は


\mathbf{x}(t)=e^{t\mathbf{A}_c}\mathbf{x}(0)+\int_0^t e^{(t-\tau)\mathbf{A}_c}\mathbf{B}_c\mathbf{u}(\tau) d\tau
