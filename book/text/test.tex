\section{神経サンプリング}
サンプリングに基づく符号化(sampling-based coding; SBC or neural sampling model)をガウス尺度混合モデルを例にとり実装する.

\subsection{ガウス尺度混合モデル}
\textbf{ガウス尺度混合 (Gaussian scale mixture; GSM) モデル}は確率的生成モデルの一種である{cite:p}\jl{Wainwright1999-cl}{cite:p}\jl{Orban2016-tm}.GSMモデルでは入力を次式で予測する:

$$
\text{入力}={z}\left(\sum \text{神経活動} \times \text{基底} \right) + \text{ノイズ}
$$

前節までのスパース符号化モデル等と同様に,入力が基底の線形和で表されるとしている.ただし,尺度(scale)パラメータ$z$が基底の線形和に乗じられている点が異なる.

\subsubsection{事前分布}$\mathbf{x} \in \mathbb{R}^{N_x}$, $\mathbf{A} \in \mathbb{R}^{N_x\times N_y}$, $\mathbf{y} \in \mathbb{R}^{N_y}$, $\mathbf{z} \in \mathbb{R}$とする.

$$
p\left(\mathbf{x}\mid\mathbf{y}, z\right)=\mathcal{N}\left(z \mathbf{A} \mathbf{y}, \sigma_{\mathbf{x}}^{2} \mathbf{I}\right)
$$

事前分布を

$$
\begin{aligned}
p\left(\mathbf{y}\right)&=\mathcal{N}\left(\mathbf{0}, \mathbf{C}\right)\\
p\left(z\right)&=\Gamma (k, \vartheta)
\end{aligned}
$$

とする.$\Gamma(k, \vartheta)$はガンマ分布であり,$k$は形状(shape)パラメータ,$\vartheta$は尺度(scale)パラメータである.$p\left(\mathbf{y}\right)$は$\mathbf{y}$の事前分布であり,刺激がない場合の自発活動の分布を表していると仮定する.