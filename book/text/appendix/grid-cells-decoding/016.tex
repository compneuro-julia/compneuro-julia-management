\subsection{発火率マップ}

発火率$\lambda(\boldsymbol{x})$は,場所$\boldsymbol{x}=(x,y)$で記録されたスパイクの回数を,場所$\boldsymbol{x}$における滞在時間(s)で割ることで得られる. 

 
\lambda(\boldsymbol{x})=\frac{\displaystyle \sum_{i=1}^n
g\left(\frac{\boldsymbol{s}_i-\boldsymbol{x}}{h}\right)}{\displaystyle \int_0^T g\left(\frac{\boldsymbol{y}(t)-\boldsymbol{x}}{h}\right)dt} 
 

ただし,$n$はスパイクの回数,$T$は計測時間,$g(\cdot)$はGaussain
Kernel (中身の分子が平均,分母が標準偏差) ,$\boldsymbol{s}_i$は$i$番目のスパイクの発生した位置,$\boldsymbol{y}(t)$は時刻$t$でのラットの位置である.分母は積分になっているが,実際には離散的に記録をするので,累積和に変更し,$dt$を時間のステップ幅(今回は0.02s)とする.

この式の分母はマウスの位置,分子はニューロンが発火したときのマウスの位置についてそれぞれカーネル密度推定 (kernel density estimation)を行うことを意味する.今回はヒストグラムを求め,描画の際にGaussianで平滑化することで計算量を下げることとする.
