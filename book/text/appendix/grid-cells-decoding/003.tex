格子細胞の活動データはMoser研が公開しており,\url{https://www.ntnu.edu/kavli/research/grid-cell-data}からダウンロードできる.公開されているデータはMATLABのmatファイル形式である.使用するデータ:[10704-07070407_POS.mat](https://github.com/Salad-bowl-of-knowledge/hp/blob/master/_notebooks/data/grid_cells_data/10704-07070407_POS.mat), [10704-07070407_T2C3.mat](https://github.com/Salad-bowl-of-knowledge/hp/blob/master/_notebooks/data/grid_cells_data/10704-07070407_T2C3.mat)
これらのファイルは\url{https://archive.norstore.no/pages/public/datasetDetail.jsf?id=8F6BE356-3277-475C-87B1-C7A977632DA7}からダウンロードできるファイルの一部である.以下では\jl{./data/grid_cells_data/}ディレクトリの下にファイルを置いている.
データの末尾の"POS"と"T2C3"の意味について説明しておく.まず,"POS"はpost, posx, posyを含む構造体でそれぞれ試行の経過時間,x座標, y座標である.座標は$[-50, 50]$で記録されている.1m四方の正方形の部屋で,原点を部屋の中心としている."T2C3"はtがtetrode (テトロード電極) でcがcell (細胞) を意味する.後ろの数字は番号付けである. 