\section{有用なリンク集
}
この説では計算論的神経科学を学ぶのに有用なWeb上の資料のリンクをまとめた。随時追加。
\subsection{書籍
}
\begin{itemize}
\item [Theoretical Neuroscience](http://www.gatsby.ucl.ac.uk/~dayan/book/):理論神経科学の教科書の中で一番有名な本。pdfとexercises (MATLAB)などが公開されている。
\item [Neuronal Dynamics](https://neuronaldynamics.epfl.ch/index.html):こちらも計算論的神経科学の教科書。書籍の内容がすべてブラウザで見れるようになっている。Pythonのコードも少しついている。また、[edXの講義](https://www.classcentral.com/course/edx-neuronal-dynamics-2685)もある。
\item [「計算神経科学への招待」脳の学習機構の理解を目指して](https://groups.oist.jp/sites/default/files/imce/u194/Books/Doya2007icns.pdf):OISTの[銅谷先生](https://groups.oist.jp/ja/ncu)の書かれた「数理科学」の別冊のpdf版 (銅谷先生が公開されている) 
\end{itemize}
\subsection{講義資料
}
\begin{itemize}
\item [Introduction to Theoretical Neuroscience](https://warwick.ac.uk/fac/sci/systemsbiology/staff/richardson/teaching/ma4g4/):M. Richardson先生の講義資料
\item [David Heeger](https://www.cns.nyu.edu/~david/):D. Heeger先生の講義資料 (ホームページの"Teaching"を参照) 
\end{itemize}
\subsection{オンライン講義
}
\begin{itemize}
\item [Computational Neuroscience](https://www.coursera.org/learn/computational-neuroscience/):Courseraの講義。
\item [Computational Neuroscience: Neuronal Dynamics of Cognition](https://www.classcentral.com/course/edx-computational-neuroscience-neuronal-dynamics-of-cognition-104230):edXの講義。
\end{itemize}
\subsection{ワークショップの資料
}
\begin{itemize}
\item [Neuromatch Academy Tutorial](http://www.neuromatchacademy.org/syllabus/):Pythonのnotebookと講義動画が大変よくまとまっている ([GitHub](https://github.com/NeuromatchAcademy/course-content/tree/master/tutorials))。
\item [neurotheory-seminar-2019](https://github.com/RainerEngelken/neurotheory-seminar-2019):一部、スライドとコードが公開されている。
\item [Computational Psychiatry Course](https://www.translationalneuromodeling.org/cpcourse/):[code](https://bitbucket.org/fpetzschner/workspace/projects/CPC)
\end{itemize}
\subsection{その他
}
\begin{itemize}
\item [Insights from the brain: The road towards Machine Intelligence](https://www.insightsfromthebrain.com/):神経科学と人工知能の関連についてまとまった資料。
\end{itemize}
\subsection{Juliaの参考資料
}
\begin{itemize}
\item [MATLAB–Python–Julia cheatsheet](https://cheatsheets.quantecon.org/)
\end{itemize}
