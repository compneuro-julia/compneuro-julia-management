\section{Jupyter bookの使い方 (Julia言語版)
}
このサイトの構築方法について (2020/09/02)。



\subsection{Julia言語のサイトを構築する方法
}
Julia言語でサイトを構築するには[\textbf{Documenter.jl}\index{Documenter.jl}](https://github.com/JuliaDocs/Documenter.jl)や[\textbf{Franklin.jl}\index{Franklin.jl}](https://github.com/tlienart/Franklin.jl), [\textbf{Weave.jl}\index{Weave.jl}](https://github.com/JunoLab/Weave.jl)などのパッケージを用いるのが一般的である。これらはMarkdown (.mdや.jmd)などのファイルをHTMLに変換できるが、Jupyter notebookからHTMLに変換するのはやや手間である (と自分は思っているが、今後改善される可能性は十分にある)。詳細は次項で説明するが、このサイトを構築している[\textbf{Jupyter book}\index{Jupyter book}](https://jupyterbook.org/intro.html)はJuliaで完結してはいないものの、より簡単にJupyter notebookをHTMLに変換できる。また、LaTeX形式やpdf形式に変換することもできる (これはWeave.jlでも同じことができるということは述べておく)。



\begin{itemize}
\item [Julia の Documenter.jl でホームページを作成する準備. - Qiita](https://qiita.com/SatoshiTerasaki/items/b0ac17088f3b2c374099)

\item [Weave.jl で Markdown + Julia の文章をHTMLに変換して自分のホームページで公開しよう - Qiita](https://qiita.com/SatoshiTerasaki/items/3a913f897b2ef4b82979)

\item [Weave.jlを使ってJuliaのノートブックを作成する - システムとモデリング](http://otepipi.hatenablog.com/entry/2019/03/30/221635)

\end{itemize}


\subsection{Jupyter bookの導入
}
[Jupyter book](https://jupyterbook.org/intro.html)は[Sphinx](https://www.sphinx-doc.org/ja/master/)を用いてMarkdownやJupyter notebookからサイトを生成するツールである (以前は[Jekyll](https://jekyllrb.com/)が用いられていた)。Sphinx自体はPythonで書かれた文章作成ツールだが、導入においてPythonコードを編集することは基本的にない。



また、重要な点として、\textbf{Jupyter notebookのカーネルはPythonでなくてもよい}\index{Jupyter notebookのかーねるはPythonでなくてもよい@Jupyter notebookのカーネルはPythonでなくてもよい}ということが挙げられる (Python, R, Julia, Ruby, Go, Scala, Perlなどなど)。カーネルとして用いることのできる言語の一覧は[Jupyter kernels · jupyter/jupyter Wiki](https://github.com/jupyter/jupyter/wiki/Jupyter-kernels)を参照。Python以外の言語を用いている例としては[Other Programming Languages](https://myst-nb.readthedocs.io/en/latest/examples/coconut-lang.html)がある。ここではCoconutというPython製の関数型言語が用いられている。



なお、同じサイト内で異なる言語のカーネルを持つJupyter notebookを元としたページも作成できる (例えばあるページはPython, 次のページはJuliaといったように)。同じページ内で複数の言語を用いるのはまだできないようだが。



とはいえ、Pythonカーネルでやるのが一番楽である。以下にJuliaカーネルを用いる場合の手順を記すが、Pythonカーネルの場合はJuliaのinstallの手順を飛ばせばよい。



\subsubsection{1. Anacondaをinstall
}
[Anaconda](https://www.anaconda.com/products/individual) または [miniconda](https://docs.conda.io/en/latest/miniconda.html)をinstallする (使用しているOSに応じてinstallerを選択する)。



\subsubsection{2. jupyter-notebookをinstall
}
pip (またはconda)で[Jupyter notebook](https://jupyter.org/)をinstallする。



}

$ pip install jupyter

}



installしたTerminal (Windowsならanaconda prompt)でJupyter notebookを立ち上げられることを確認する。



}

$ jupyter notebook

}



\subsubsection{3. Juliaをinstallし、Ijuliaをinstall
}
[Julia](https://julialang.org/)をinstallする。次にJuliaを立ち上げて、REPLで\jl{]}を入力してpkg modeにし、



}

pkg> add IJulia

}



により[Ijulia.jl](https://github.com/JuliaLang/IJulia.jl)をinstallする。この段階で、Jupyter notebookを立ち上げて新規にnotebookを作成する際にJuliaカーネルを選択できるはずである。うまくいっていない場合はJuliaのPATHが通っていないなどが考えられる。



\subsubsection{4. jupyter bookをinstall
}
pip (またはconda)で\jl{jupyter-book} をinstallする。



}

$ pip install jupyter-book

}



\subsubsection{5. テンプレートの作成
}
[Overview and installation](https://jupyterbook.org/start/overview.html)を参照。Terminal (Windowsならanaconda prompt)で次のようにしてテンプレートを作成する。



}

$ cd hogehoge

$ jupyter-book create mybookname

}



ただし、\jl{mybookname}はサイトの構成ファイルを保存するディレクトリ名であり、変更可能である。\jl{jupyter-book}は\jl{jb}と省略できるので、



}

$ jb create mybookname

}



としてもよい。実行後は次のようなファイルが\jl{mybookname}の下に生成される。



}

mybookname/

├── _config.yml

├── _toc.yml

├── content.md

├── intro.md

├── markdown.md

├── notebooks.ipynb

└── references.bib

}



このうち、Markdownファイル (.md)とJupyter notebookファイル (.ipynb)および参考文献を記述するためのBiBTeXファイル(.bib)が実際のサイトの元ファイルとなる。\jl{_config.yml}はサイトの情報を指定するファイルであり、\jl{_toc.yml}はサイトの構成を指定するファイルである。また、ディレクトリとファイルの構成例として\jl{quantecon-mini-example} ([Github pages](https://executablebooks.github.io/quantecon-mini-example/docs/index.html), [Github](https://github.com/executablebooks/quantecon-mini-example)) が用意されている。



Jupyter bookの機能を確認する場合は先に [8. サイトのbuild](#build)を参照。



\subsubsection{6. \jl{_config.yml}にサイトの情報を記述
}
[Configure book settings](https://jupyterbook.org/customize/config.html)を参照。サイトの名前、著者、ロゴ、リポジトリへのリンク、colabで立ち上げるボタンの追加、など様々なことが設定できる。



\subsubsection{7. \jl{_toc.yml}にサイトの構成を記述
}
[Table of Contents structure](https://jupyterbook.org/customize/toc.html)を参照。各ファイルをどのような構成でサイトに変換するか、ということを指定する。



\subsubsection{8. サイトのbuild
}
コンテンツを含むディレクトリを\jl{./mybookname}としたとき



}

jb build mybookname

}



によりbuildする。このとき、Jupyter notebookはbuild時に実行される。Ijulia.jlが入っていなかったり、実行不可能であればエラーが生じる。build完了後、\jl{./build/html}というディレクトリの下に他の依存ファイルを含めたサイトのHTMLが生成される。



\subsubsection{9. GitHub pagesでサイトを公開する
}
\jl{./build/html}の中身を同じリポジトリの\jl{gh-pages}ブランチにcommitするか、別のリポジトリを用意してcommitする。本サイトの場合はコンテンツの管理は\url{https://github.com/compneuro-julia/compneuro-julia-management}で、サイトのホスティングは\url{https://github.com/compneuro-julia/compneuro-julia.github.io}で行っている。



なお、GitHub pagesはデフォルトではJekyllで処理されるので、\jl{.nojekyll}という名前の空ファイルを作成しておく (空ファイルだとuploadされない場合もあるので注意)。



\begin{itemize}
\item [GitHub Pagesで普通の静的ホスティングをしたいときは .nojekyll ファイルを置く - Qiita](https://qiita.com/sky_y/items/b96ae52c90457bcd7846)

\end{itemize}


\subsection{MyST Markdown形式について
}
Jupyter bookは通常のMarkdown記法に加え、\textbf{MyST}\index{MyST} (Markedly Structured Text)と呼ばれる形式のMarkdown記法に対応している。例えば



}`

\footnote{

これはノートです。

}

}`



と記述すれば次のように変換される。



\footnote{

これはノートです。

}



これはMarkdownファイルにも、Jupyter notebookのMarkdown cellにも記述できる。詳細は[MyST Markdown Overview](https://jupyterbook.org/content/myst.html)や[MyST Cheat Sheet](https://jupyterbook.org/reference/cheatsheet.html)などを参照 (特に後者はMySTでどのようなことができるか一目でわかる)。



\subsubsection{数式について
}
通常のMarkdownやJupyter notebookと同様に行える。次のように入力すれば



}



F(\omega) = \cfrac{1}{\sqrt{2\pi}}\int_{-\infty}^{+\infty}f(t)e^{i\omega t}dt



}



以下のように出力される。





F(\omega) = \cfrac{1}{\sqrt{2\pi}}\int_{-\infty}^{+\infty}f(t)e^{i\omega t}dt





\subsubsection{入力・出力を隠す
}
Jupyter notebookのcellのtagを変更してcellの入力や出力を隠すことができる。Tagの変更はJupyter notebookで\jl{View > Cell Toolbar > Tags}とすることで各cellにtagを追加する欄が表示される。詳細は[Hiding cell contents](https://myst-nb.readthedocs.io/en/latest/use/hiding.html)を参照。







\subsection{Syntax highlightingについて
}
Sphinxはコードの構文解析を[Pygments](https://pygments.org/)によって行っている (cf. [sphinx/highlighting.py](https://github.com/sphinx-doc/sphinx/blob/3.x/sphinx/highlighting.py))。これまでは全てPythonLexerでsyntax highlightされていたが、Jupyter book v0.8.0でMyST-NBのv0.10に対応し、Jupyter notebookのカーネルの言語に応じてcode cellがsyntax highlightされるようになった。

\begin{itemize}
\item  [Change Log - Jupyter book v0.8.0 2020-09-01](https://jupyterbook.org/reference/_changelog.html#v0-8-0-2020-09-01)

\end{itemize}


Syntax highlightの色などはカスタムCSSの追加して調整している (Qiitaの配色が見やすいので参考にした)。具体的な調整については\url{https://compneuro-julia.github.io/_static/custom.css}および次項を参照。



\subsection{カスタムCSSの追加
}
本の構成が



}

mybook/

├── _config.yml

├── _toc.yml

└── page1.md

}



のようであった場合、



}

mybook/

├── _config.yml

├── _toc.yml

├── page1.md

└── _static

    └── myfile.css

}



のように \jl{_static}ディレクトリを作成し、その下にCSSやJavascriptなどのファイルを置いておけばbuildするときに自動で読み込まれる。\jl{ _static}ディレクトリには他に画像ファイルやpdfファイルなどを置くことも可能である。今回、試験的にfontに[JuliaMono](https://juliamono.netlify.app/)を用いた。custom.cssに以下を記述し、fontに\jl{JuliaMono-Regular}を指定した。



}css

@font-face {

    font-family: JuliaMono-Regular;

    src: url("https://cdn.jsdelivr.net/gh/cormullion/juliamono/webfonts/JuliaMono-Regular.woff2");

}

}
