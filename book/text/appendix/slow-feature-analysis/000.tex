\section{Slow Feature Analysis (SFA)}\textbf{Slow Feature Analysis (SFA)} とは, 複数の時系列データの中から低速に変化する成分 (slow feature) を抽出する教師なし学習のアルゴリズムである (Laurenz Wiskott, Berkes, Franzius, Sprekeler, & Wilbert, 2011; L. Wiskott & Sejnowski, 2002).
潜在変数 $y$ の時間変化の2乗である $\left(\frac{dy}{dt}\right)^2$を最小にするように教師なし学習を行う.初期視覚野の受容野や格子細胞・場所細胞などのモデルに応用がされている (Franzius, Sprekeler, & Wiskott, 2007).
生理学的妥当性についてはいくつかの検討がされている.(Sprekeler, Michaelis, & Wiskott, 2007)ではSTDP則によりSFAが実現できることを報告している.ただし,in vivoにおけるSTDPの存在については近年疑問視されている.これまでのin vitroでの実験は細胞外Ca濃度が高かったために、pre/postのスパイクの時間差でLTD/LTPが生じるという「古典的STDP則」が生じていた可能性があり,細胞外Ca濃度をin vivoの水準まで下げると古典的STDP則は起こらないという報告がある (Inglebert, Aljadeff, Brunel, & Debanne, 2020).古典的な線形Recurrent neural networkでの実装も提案されている ([Lipshutz, Windolf, Golkar, & Chklovskii, 2020](https://arxiv.org/abs/2010.12644)).
