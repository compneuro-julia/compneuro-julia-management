\section{シナプスの形態と生理
}
スパイクが生じたことによる膜電位変化は軸索を伝播し, \textbf{\index{しなぷす@シナプス}}という構造により, 次のニューロンへと興奮が伝わる. このときの伝達の仕組みとして, シナプスには\textbf{\index{かがくしなぷす@化学シナプス}}(chemical synapse)とGap junctionによる\textbf{\index{でんきしなぷす@電気シナプス}}(electrical synapse)がある.  



化学シナプスの場合, シナプス前膜からの\textbf{\index{しんけいでんたつぶっしつ@神経伝達物質}}の放出, シナプス後膜の受容体への神経伝達物質の結合, イオンチャネル開口による\textbf{\index{しなぷすのちでんりゅう@シナプス後電流}}(postsynaptic current; PSC)の発生, という過程が起こる.



しかし, これらの過程を全てモデル化するのは計算量がかなり大きくなるので, 基本的には簡易的な現象論的なモデルを用いる.



このように, シナプス前細胞のスパイク列(spike train)は次のニューロンにそのまま伝わるのではなく, ある種の時間的フィルターをかけられて伝わる.このフィルターを\textbf{\index{しなぷすふぃるたー@シナプスフィルター}}(synaptic filter)と呼ぶ.本章では, このようにシナプス前細胞で生じた発火が, シナプス後細胞の膜電位に与える過程のモデルについて説明する.



