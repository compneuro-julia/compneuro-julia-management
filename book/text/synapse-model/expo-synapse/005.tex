\subsection{単一指数関数型モデル(Single exponential model)}
シナプス前ニューロンにおいてスパイクが生じてからのシナプス後電流の変化はおおよそ指数関数的に減少する, というのが単一指数関数型モデルである [^comp]. 式は次のようになる.

[^comp]: 薬学動態の静注1コンパートメントモデルと同じ式である.


\begin{equation}
f(t)=\frac{1}{\tau_{s}}\exp\left(-\frac{t}{\tau_s}\right)    
\end{equation}


この関数を時間的なフィルターとして, 過去の全てのスパイクについての総和を取る.


\begin{equation}
r(t)=\sum_{t_{k}< t} f\left(t-t_{k}\right)
\end{equation}


ここで${r(t)}$は前節におけるシナプス動態($s_{\text{syn}}$)で, $t_{k}$はあるニューロンの$k$番目のスパイクの発生時刻である.${t_{k}<t}$の意味は現在の時刻$t$までに発生したスパイクについての和を取るという意味である.なお,スパイクが生じてから, ある程度の時間が経過した後はそのスパイクの影響はないと見なせるので, 一定の時間までの総和を取るのがよい.

別の表記法としてスパイク列に対する畳み込みを行うというものもある.畳み込み演算子を$*$とし, シナプス前細胞のスパイク列を$S(t)=\sum_{t_{k}< t} \delta\left(t-t_{k}\right)$とする (ただし, $\delta$はDiracのdelta関数において$\delta(0)=1$とした関数).このとき, $r(t)=f*S(t)$と表すことができる.畳み込み演算子を用いると簡略な表記ができるが,実装上は他と同じ手法を用いる.

\subsubsection{微分方程式による表現}
上の手法ではニューロンの発火時刻を記憶し, 時間毎に全てのスパイクについての和を取る必要がある.そこで, 実装する場合は次の等価な微分方程式を用いる.


\begin{equation}
\frac{dr}{dt}=-\frac{r}{\tau_{s}}+\frac{1}{\tau_{s}} \sum_{t_{k}< t} \delta\left(t-t_{k}\right)   
\end{equation}


ここで$\tau_s$はシナプスの時定数(synaptic time constant)である. また, $\delta(\cdot)$はDiracのdelta関数です(ただし$\delta(0)=1$です). これをEuler法で差分化すると 


\begin{equation}
r(t+\Delta t)=\left(1-\frac{\Delta t}{\tau_{s}}\right)r(t)+\frac{1}{\tau_{s}}\delta_{t,t_{k}} 
\end{equation}


となる.ここで$\delta_{t,t_{k}}$はKroneckerのdelta関数で, $t=t_{k}$のときに1, それ以外は0となる.また減衰度として$\left(1-\Delta  t/\tau_{d}\right)$の代わりに$\exp\left(-\Delta t/\tau_{d}\right)$を用いる場合もある.
