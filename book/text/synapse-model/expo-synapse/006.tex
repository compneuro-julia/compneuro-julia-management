\subsection{二重指数関数型モデル(Double exponential model)}
2重の指数関数によりシナプス後電流の立ち上がりも考慮するのが, 二重指数関数型モデル(Double exponential model)である[^comp2].$t=0$にシナプス前細胞が発火したときのシナプス後電流の時間変化の関数は次のようになる.
[^comp2]: 薬学動態の内服1コンパートメントモデルと同じ式である.


\begin{equation}
f(t)=A\left[\exp\left(-\frac{t}{\tau_d}\right)-\exp\left(-\frac{t}{\tau_r}\right)\right]    
\end{equation}


ただし, ${\tau_r}$は立ち上がり時定数(synaptic rise time constant), ${\tau_d}$は減衰時定数(synaptic decay time constant)である.$\tau_{d}$は$\tau_{s}$と同じく神経伝達物質の減少速度を決定している.$A$は規格化定数で次のように表される.


\begin{equation}
A=\frac{\tau_d}{\tau_d-\tau_r}\cdot \left(\frac{\tau_r}{\tau_d}\right)^\frac{\tau_r}{\tau_r-\tau_d}    
\end{equation}


規格化定数$A$を乗じることで最大値が1となる.ただし, シミュレーションをする上で実際に規格化をする場合は少ない.

\subsubsection{$\alpha$関数}
上記の式において, $\tau=\tau_{r}=\tau_{d}$の場合は $\boldsymbol{\alpha}$ \textbf{関数} (alpha function, alpha synapse)と呼ぶ ([Rall, 1967](https://pubmed.ncbi.nlm.nih.gov/6055351/)).式としては次のようになる.


\begin{equation}
\alpha(t)=\frac{t}{\tau}\exp\left(1-\frac{t}{\tau}\right)    
\end{equation}


この式は二重指数関数型シナプスの式に単に代入するだけでは導出できない.これらの式の対応については後述する.

\subsubsection{微分方程式による表現}
ここで, 二重指数関数型シナプスの式に対応する, 補助変数$h$を用いた微分方程式を導入する. 


\begin{align} 
\frac{dr}{dt}&=-\frac{r}{\tau_{d}}+h\\
\frac{dh}{dt}&=-\frac{h}{\tau_{r}}+\frac{1}{\tau_{r} \tau_{d}} \sum_{t_{k}< t} \delta\left(t-t_{k}\right) 
\end{align} 


単一指数関数型シナプスの場合と同様にEuler法で差分化すると 


\begin{align} 
r(t+\Delta t)&=\left(1-\frac{\Delta t}{\tau_{d}}\right)r(t)+h(t)\cdot \Delta t\\ 
h(t+\Delta t)&=\left(1-\frac{\Delta t}{\tau_{r}}\right)h(t)+\frac{1}{\tau_{r}\tau_{d}} \delta_{t,t_{j k}}
\end{align}


となる.

念のため, 微分方程式と元の式が一致することを確認しておこう.$t=0$のときにシナプス前細胞が発火したとし, それ以降の発火はないとする.このとき, $h(0)=1/\tau_{r}\tau_{d}, r(0)=0$ である.$h$についての微分方程式の解は


\begin{equation}
h(t)=h(0)\cdot \exp\left(-\frac{t}{\tau_r}\right)    
\end{equation}


となるので, これを$r$についての式に代入して


\begin{equation}
\frac{dr}{dt}=-\frac{r}{\tau_{d}}+h(0)\cdot \exp\left(-\frac{t}{\tau_r}\right) 
\end{equation}


となる.これを解くには両辺に積分因子$\exp({t}/{\tau_d})$をかけてから積分をするかLaplace変換をするかである.今回はLaplace変換を用いる.右辺一項目を移行した後に両辺をLaplace変換すると以下のようになる.


\begin{align}
\mathcal{L}\left[\frac{dr}{dt}+r/\tau_{d}\right]&=\mathcal{L}\left[h(0)\cdot \exp\left(-t/\tau_r\right)\right]\\
sF(s)-r(0)+\frac{1}{\tau_{d}}F(s)&=\frac{h(0)}{s+1/\tau_r}\\
F(s)&=\frac{h(0)}{(s+1/\tau_r)(s+1/\tau_d)}
\end{align}


ただし$r(t)$のLaplace変換を$F(s)$とした. ここで逆Laplace変換を行うと次のようになる.


\begin{align}
r(t)&=\mathcal{L}^{-1}(F(s))\\
&=\mathcal{L}^{-1}\left[\frac{h(0)}{(s+1/\tau_r)(s+1/\tau_d)}\right]\\
&=\mathcal{L}^{-1}\left[\frac{h(0)}{1/\tau_r-1/\tau_d}\left(\frac{1}{s+1/\tau_d}-\frac{1}{s+1/\tau_r}\right)\right]\\
&=\frac{1}{\tau_d-\tau_r}\left[\exp(-t/\tau_d)-\exp(-t/\tau_r)\right]
\end{align}


この式の最大値$r_{\max}$を求めておこう. $r(t)$を微分して0と置いた式の解$t_{\max}$を代入すれば求められる.計算すると, 


\begin{equation}
t_{\max}=\dfrac{\ln(\tau_d/\tau_r)}{1/\tau_r-1/\tau_d},\ \ r_{\max}=\dfrac{1}{\tau_{d}}\cdot \left(\dfrac{\tau_{r}}{\tau_{d}}\right)^{\frac{\tau_{r}}{\tau_d-\tau_{r}}}    
\end{equation}


となる.

なお, $\alpha$関数の導出は逆Laplace変換をする前に$\tau=\tau_d=\tau_r$とすればよく, 


\begin{align}
F_\alpha(s)&=\frac{h(0)}{(s+1/\tau)^2}\\
\alpha(t)&=\frac{t}{\tau^2}\exp\left(-\frac{t}{\tau}\right)
\end{align}

となる.若干の係数の違いはあるが, 同じ形の関数が導出された. 
