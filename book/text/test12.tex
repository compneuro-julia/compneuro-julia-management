\subsubsection{事後分布の計算}事後分布は$z$と$\mathbf{y}$のそれぞれについて次のように求められる.
$$
\begin{aligned}
p(z \mid \mathbf{x}) &\propto p(z) \mathcal{N}\left(0, z^{2} \mathbf{A C A}^{\top}+\sigma_{x}^{2} \mathbf{I}\right)\\
p(\mathbf{y} \mid z, \mathbf{x})& = \mathcal{N}\left(\mu(z, \mathbf{x}), \Sigma(z)\right)
\end{aligned}
$$
ただし,
$$
\begin{aligned}
\Sigma(z)&=\left(\mathbf{C}^{-1}+\frac{z^{2}}{\sigma_{x}^{2}} \mathbf{A}^{\top} \mathbf{A}\right)^{-1}\\
\mu(z, \mathbf{x})&=\frac{z}{\sigma_{x}^{2}} \Sigma(z) \mathbf{A}^{\top} \mathbf{x}
\end{aligned}
$$
である.
最終的な予測において$z$の事後分布は必要でないため,$p(\mathbf{y} \mid z, \mathbf{x})$から$z$を消去することを考えよう.厳密に行う場合,次式のように周辺化(marginalization)により,$z$を(積分)消去する必要がある.
$$
p(\mathbf{y} \mid \mathbf{x}) = \int dz\ p(z\mid \mathbf{x})\cdot p(\mathbf{y} \mid z, \mathbf{x})
$$
周辺化においては,まず$z$のMAP推定(最大事後確率推定)値 $z_{\mathrm{MAP}}$を求める.
$$
z_{\mathrm{MAP}} = \underset{z}{\operatorname{argmax}} p(z\mid \mathbf{x})
$$
次に$z_{\mathrm{MAP}}$の周辺で$p(z\mid \mathbf{x})$を積分し,積分値が一定の閾値を超える$z$の範囲を求め,この範囲で$z$を積分消去してやればよい.しかし,$z$は単一のスカラー値であり,この手法で推定するのは煩雑であるために近似手法が\cite{Echeveste2017-wu}において提案されている.Echevesteらは第一の近似として,$z$の分布を$z_{\mathrm{MAP}}$でのデルタ関数に置き換える,すなわち,$p(z\mid \mathbf{x})\simeq \delta (z-z_{\mathrm{MAP}})$とすることを提案している.この場合,$z$は定数とみなせ,$p(\mathbf{y} \mid \mathbf{x})\simeq p(\mathbf{y} \mid \mathbf{x}, z=z_{\mathrm{MAP}})$となる.第二の近似として,$z_{\mathrm{MAP}}$を真のコントラスト$z^*$で置き換えることが提案されている.GSMへの入力$\mathbf{x}$は元の画像を$\mathbf{\tilde x}$とすると,$\mathbf{x}=z^* \mathbf{\tilde x}$としてスケーリングされる.この入力の前処理の際に用いる$z^*$を用いてしまおうということである.この場合,$p(\mathbf{y} \mid \mathbf{x})\simeq p(\mathbf{y} \mid \mathbf{x}, z=z^*)$となる.しかし,入力を任意の画像とする場合,$z^*$は未知である.簡便さと精度のバランスを取り,ここでは第一の近似,$z=z_{\mathrm{MAP}}$とする手法を用いることにする.