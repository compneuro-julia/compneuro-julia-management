\subsubsection{事前分布の設定}事前分布$p(\mathbf{r})$としては,0においてピークがあり,裾の重い(heavy tail)を持つsparse distributionあるいは \textbf{super-Gaussian distribution} (Laplace 分布やCauchy分布などGaussian分布よりもkurtoticな分布)を用いるのが良い.このような分布では,$\mathbf{r}$の各要素$r_i$はほとんど0に等しく,ある入力に対しては大きな値を取る.$p(\mathbf{r})$は一般化して式(4), (5)のように表記する.
$$
\begin{align}
p(\mathbf{r})&=\prod_j p(r_j) \quad \tag{4}\\
p(r_j)&=\frac{1}{Z_{\beta}}\exp \left[-\beta S(r_j)\right] \quad \tag{5}
\end{align}
$$
ただし,$\beta$は逆温度(inverse temperature), $Z_{\beta}$は規格化定数 (分配関数) である[^can].$S(x)$と分布の関係をまとめた表が以下となる (cf. [Harpur, 1997](https://pdfs.semanticscholar.org/be08/da912362bf40fe3ded78bdadc644f921b4e7.pdf)).
[^can]: これらの用語は統計力学における正準分布 (ボルツマン分布)から来ている.