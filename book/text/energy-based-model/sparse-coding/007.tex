\subsection{ Locally competitive algorithm (LCA) }$\mathbf{r}$の勾配法による更新則は,$E$の微分により次のように得られる.
$$
\frac{d \mathbf{r}}{dt}= -\frac{\eta_\mathbf{r}}{2}\frac{\partial E}{\partial \mathbf{r}}=\eta_\mathbf{r} \cdot\left[\mathbf{\Phi}^\top (\mathbf{x}-\mathbf{\Phi}\mathbf{r})- \frac{\lambda}{2}S'\left(\mathbf{r}\right)\right]
$$
ただし,$\eta_{\mathbf{r}}$は学習率である.この式により$\mathbf{r}$が収束するまで最適化するが,単なる勾配法ではなく,\cite{Olshausen1996-xe}では\textbf{共役勾配法} (conjugate gradient method)を用いている.しかし,共役勾配法は実装が煩雑で非効率であるため,より効率的かつ生理学的な妥当性の高い学習法として,\textbf{LCA}  (locally competitive algorithm)が提案されている \cite{Rozell2008-wp}.LCAは\textbf{側抑制} (local competition, lateral inhibition)と\textbf{閾値関数} (thresholding function)を用いる更新則である.LCAによる更新を行うRNNは通常のRNNとは異なり,コスト関数(またはエネルギー関数)を最小化する動的システムである.このような機構はHopfield networkで用いられているために,Olshausenは\textbf{Hopfield trick}と呼んでいる.