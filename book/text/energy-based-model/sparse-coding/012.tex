なお,軟判定閾値関数は次の目的関数$C$を最小化する$x$を求めることで導出できる.


C=\frac{1}{2}(y-x)^2+\lambda |x|


ただし,$x, y, \lambda$はスカラー値とする.$|x|$が微分できないが,これは場合分けを考えることで解決する.$x\geq 0$を考えると,(6)式は


C=\frac{1}{2}(y-x)^2+\lambda x = \{x-(y-\lambda)\}^2+\lambda(y-\lambda)


となる.(7)式の最小値を与える$x$は場合分けをして考えると,$y-\lambda\geq0$のとき二次関数の頂点を考えて$x=y-\lambda$となる. 一方で$y-\lambda<0$のときは$x\geq0$において単調増加な関数となるので,最小となるのは$x=0$のときである.同様の議論を$x\leq0$に対しても行うことで (5)式が得られる.

なお,閾値関数としては軟判定閾値関数だけではなく,硬判定閾値関数や$y=x - \text{tanh}(x)$ (Tanh-shrink)など様々な関数を用いることができる.
