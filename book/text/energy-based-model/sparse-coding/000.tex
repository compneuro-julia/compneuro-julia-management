\section{スパース符号化}\subsection{Sparse codingと生成モデル}\textbf{Sparse codingモデル} \cite{Olshausen1996-xe} \cite{Olshausen1997-qu}はV1のニューロンの応答特性を説明する\textbf{線形生成モデル} (linear generative model)である.まず,画像パッチ $\mathbf{x}$ が基底関数(basis function) $\mathbf{\Phi} = [\phi_j]$ のノイズを含む線形和で表されるとする (係数は $\mathbf{r}=[r_j]$ とする).
$$
\mathbf{x} = \sum_j r_j \phi_j +\boldsymbol{\epsilon}= \mathbf{\Phi} \mathbf{r}+ \boldsymbol{\epsilon} \quad \tag{1}
$$
ただし,$\boldsymbol{\epsilon} \sim \mathcal{N}(\mathbf{0}, \sigma^2 \mathbf{I})$ である.このモデルを神経ネットワークのモデルと考えると, $\mathbf{\Phi}$ は重み行列,係数 $\mathbf{r}$ は入力よりも高次の神経細胞の活動度を表していると解釈できる.ただし,$r_j$ は負の値も取るので単純に発火率と捉えられないのはこのモデルの欠点である.
Sparse codingでは神経活動 $\mathbf{r}$ が潜在変数の推定量を表現しているという仮定の下,少数の基底で画像 (や目的変数)を表すことを目的とする.要は上式において,ほとんどが0で,一部だけ0以外の値を取るという疎 (=sparse)な係数$\mathbf{r}$を求めたい.