\subsubsection{確率的モデルの記述}入力される画像パッチ $\mathbf{x}_i\ (i=1, \ldots, N)$ の真の分布を $p_{data}(\mathbf{x})$ とする.また,$\mathbf{x}$ の生成モデルを $p(\mathbf{x}|\mathbf{\Phi})$ とする.さらに潜在変数 $\mathbf{r}$ の事前分布 (prior)を $p(\mathbf{r})$, 画像パッチ $\mathbf{x}$ の尤度 (likelihood)を $p(\mathbf{x}|\mathbf{r}, \mathbf{\Phi})$ とする.このとき,
$$
p(\mathbf{x}|\mathbf{\Phi})=\int p(\mathbf{x}|\mathbf{r}, \mathbf{\Phi})p(\mathbf{r})d\mathbf{r} \quad \tag{2}
$$
が成り立つ.$p(\mathbf{x}|\mathbf{r}, \mathbf{\Phi})$は,(1)式においてノイズ項を$\boldsymbol{\epsilon} \sim\mathcal{N}(\mathbf{0}, \sigma^2 \mathbf{I})$としたことから,
$$
p(\mathbf{x}|\ \mathbf{r}, \mathbf{\Phi})=\mathcal{N}\left(\mathbf{x}|\ \mathbf{\Phi} \mathbf{r}, \sigma^2 \mathbf{I} \right)=\frac{1}{Z_{\sigma}} \exp\left(-\frac{\|\mathbf{x} - \mathbf{\Phi} \mathbf{r}\|^2}{2\sigma^2}\right)\quad \tag{3}
$$
と表せる.ただし,$Z_{\sigma}$は規格化定数である.