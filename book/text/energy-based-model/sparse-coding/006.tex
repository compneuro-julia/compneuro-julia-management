\subsection{目的関数の設定と最適化}
最適な生成モデルを得るために,入力される画像パッチの真の分布 $p_{data}(\mathbf{x})$と$\mathbf{x}$の生成モデル $p(\mathbf{x}|\mathbf{\Phi})$を近づける.このために,2つの分布のKullback-Leibler ダイバージェンス $D_{\text{KL}}\left(p_{data}(\mathbf{x}) \Vert\ p(\mathbf{x}|\mathbf{\Phi})\right)$を最小化したい.しかし,真の分布は得られないので,経験分布 


\hat{p}_{data}(\mathbf{x}):=\frac{1}{N}\sum_{i=1}^N \delta(\mathbf{x}-\mathbf{x}_i)


を近似として用いる ($\delta(\cdot)$ はDiracのデルタ関数である).ゆえに$D_{\text{KL}}\left(\hat{p}_{data}(\mathbf{x}) \Vert\ p(\mathbf{x}|\mathbf{\Phi})\right)$を最小化する.


\begin{aligned}
D_{\text{KL}}\left(\hat{p}_{data}(\mathbf{x}) \Vert\ p(\mathbf{x}|\mathbf{\Phi})\right)&=\int \hat{p}_{data}(\mathbf{x}) \log \frac{\hat{p}_{data}(\mathbf{x})}{p(\mathbf{x}|\mathbf{\Phi})} d\mathbf{x}\\
&=\mathbb{E}_{\hat{p}_{data}} \left[\ln \frac{\hat{p}_{data}(\mathbf{x})}{p(\mathbf{x}|\mathbf{\Phi})}\right]\\
&=\mathbb{E}_{\hat{p}_{data}} \left[\ln \hat{p}_{data}(\mathbf{x})\right]-\mathbb{E}_{\hat{p}_{data}} \left[\ln p(\mathbf{x}|\mathbf{\Phi})\right]
\end{aligned}


が成り立つ.(7)式の1番目の項は一定なので,$D_{\text{KL}}\left(\hat{p}_{data}(\mathbf{x}) \Vert\ p(\mathbf{x}|\mathbf{\Phi})\right)$ を最小化するには$\mathbb{E}_{\hat{p}_{data}} \left[\ln p(\mathbf{x}|\mathbf{\Phi})\right]$を最大化すればよい.ここで,


\mathbb{E}_{\hat{p}_{data}} \left[\ln p(\mathbf{x}|\mathbf{\Phi})\right]=\sum_{i=1}^N \hat{p}_{data}(\mathbf{x}_i)\ln p(\mathbf{x}_i|\mathbf{\Phi})=\frac{1}{N}\sum_{i=1}^N \ln p(\mathbf{x}_i|\mathbf{\Phi})


が成り立つ.また,(2)式より


\ln p(\mathbf{x}|\mathbf{\Phi})=\ln \int p(\mathbf{x}|\mathbf{r}, \mathbf{\Phi})p(\mathbf{r})d\mathbf{r}


が成り立つので,近似として $\displaystyle \int p(\mathbf{x}|\mathbf{r}, \mathbf{\Phi})p(\mathbf{r})d\mathbf{r}$ を $p(\mathbf{x}|\mathbf{r}, \mathbf{\Phi})p(\mathbf{r}) \left(=p(\mathbf{x}, \mathbf{r}| \mathbf{\Phi})\right)$ で評価する.これらの近似の下,最適な$\mathbf{\Phi}=\mathbf{\Phi}^*$は次のようにして求められる.


\begin{aligned}
\mathbf{\Phi}^*&=\text{arg} \min_{\mathbf{\Phi}} \min_{\mathbf{r}} D_{\text{KL}}\left(\hat{p}_{data}(\mathbf{x}) \| p(\mathbf{x}|\mathbf{\Phi})\right)\\
&=\text{arg} \max_{\mathbf{\Phi}} \max_{\mathbf{r}} \mathbb{E}_{\hat{p}_{data}} \left[\ln p(\mathbf{x}|\mathbf{\Phi})\right]\\
&= \text{arg} \max_{\mathbf{\Phi}}\sum_{i=1}^N \max_{\mathbf{r}_i} \ln p(\mathbf{x}_i|\mathbf{\Phi})\\
&\approx \text{arg} \max_{\mathbf{\Phi}}\sum_{i=1}^N \max_{\mathbf{r}_i} \ln p(\mathbf{x}_i|\mathbf{r}_i, \mathbf{\Phi})p(\mathbf{r}_i)\\
&=\text{arg}\min_{\mathbf{\Phi}} \sum_{i=1}^N \min_{\mathbf{r}_i}\ E(\mathbf{x}_i, \mathbf{r}_i|\mathbf{\Phi})
\end{aligned}


ただし,$\mathbf{x}_i$に対する神経活動を $\mathbf{r}_i$とした.また,$E(\mathbf{x}, \mathbf{r}|\mathbf{\Phi})$はコスト関数であり,次式のように表される.


\begin{aligned}
E(\mathbf{x}, \mathbf{r}|\mathbf{\Phi}):=&-\ln p(\mathbf{x}|\mathbf{r}, \mathbf{\Phi})p(\mathbf{r})\\
=&\underbrace{\left\|\mathbf{x}-\mathbf{\Phi} \mathbf{r}\right\|^2}_{\text{preserve information}} + \lambda \underbrace{\sum_j S\left(r_j\right)}_{\text{sparseness of}\ r_j}
\end{aligned}


ただし,$\lambda=2\sigma^2\beta$は正則化係数(この式から逆温度$\beta$が正則化の度合いを調整するパラメータであることがわかる.)であり,1行目から2行目へは式(3), (4), (5)を用いた.ここで,第1項が復元損失,第2項が罰則項 (正則化項)となっている.

式(9)で表される最適化手順を最適な$\mathbf{r}$と$\mathbf{\Phi}$を求める過程に分割しよう.まず, $\mathbf{\Phi}$を固定した下で$E(\mathbf{x}_n, \mathbf{r}_i|\mathbf{\Phi})$を最小化する$\mathbf{r}_i=\hat{\mathbf{r}}_i$を求める.


\hat{\mathbf{r}}_i=\text{arg}\min_{\mathbf{r}_i}E(\mathbf{x}_i, \mathbf{r}_i|\mathbf{\Phi})\ \left(= \text{arg}\max_{\mathbf{r}_i}p(\mathbf{r}_i|\mathbf{x}_i)\right)


これは $\mathbf{r}$ について \textbf{MAP推定} (maximum a posteriori estimation)を行うことに等しい.次に$\hat{\mathbf{r}}$を用いて


\mathbf{\Phi}^*=\text{arg}\min_{\mathbf{\Phi}} \sum_{i=1}^N E(\mathbf{x}_i, \hat{\mathbf{r}}_i|\mathbf{\Phi})\ \left(= \text{arg}\max_{\mathbf{\Phi}} \prod_{i=1}^N p(\mathbf{x}_i|\hat{\mathbf{r}}_i, \mathbf{\Phi})\right)


とすることにより,$\mathbf{\Phi}$を最適化する.こちらは $\mathbf{\Phi}$ について \textbf{最尤推定} (maximum likelihood estimation)を行うことに等しい.
