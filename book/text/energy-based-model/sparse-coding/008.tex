\subsubsection{軟判定閾値関数を用いる場合 (ISTA)}
$S(x)=|x|$とした場合の閾値関数を用いる手法として\textbf{ISTA}(Iterative Shrinkage Thresholding Algorithm)がある.ISTAはL1-norm正則化項に対する近接勾配法で,要はLasso回帰に用いる勾配法である.

解くべき問題は次式で表される.


\mathbf{r} = \mathop{\rm arg~min}\limits_{\mathbf{r}}\left\{\|\mathbf{x}-\mathbf{\Phi}\mathbf{r}\|^2_2+\lambda\|\mathbf{r}\|_1\right\}


詳細は後述するが,次のように更新することで解が得られる.

\begin{itemize}
\item $\mathbf{r}(0)$を要素が全て0のベクトルで初期化:$\mathbf{r}(0)=\mathbf{0}$
\item $\mathbf{r}_*(t+1)=\mathbf{r}(t)+\eta_\mathbf{r}\cdot \mathbf{\Phi}^\top(\mathbf{x}-\mathbf{\Phi}\mathbf{r}(t))$
\item $\mathbf{r}(t+1) = \Theta_\lambda(\mathbf{r}_*(t+1))$
\item $\mathbf{r}$が収束するまで2と3を繰り返す
\end{itemize}

ここで$\Theta_\lambda(\cdot)$は\textbf{軟判定閾値関数} (Soft thresholding function)と呼ばれ,次式で表される.


\Theta_\lambda(y)= 
\begin{cases} 
y-\lambda & (y>\lambda)\\ 
0 & (-\lambda\leq y\leq\lambda)\\ 
 y+\lambda & (y<-\lambda) 
\end{cases}


$\Theta_\lambda(\cdot)$を関数として定義すると次のようになる.また,ReLU (ランプ関数)は\jl{max(x, 0)}で実装できる.この点から考えればReLUを軟判定非負閾値関数 (soft nonnegative thresholding function)と捉えることもできる \cite{Papyan2018-yr}.
