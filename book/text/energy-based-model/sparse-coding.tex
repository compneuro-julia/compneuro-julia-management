\section{スパース符号化}
\subsection{Sparse codingと生成モデル}
\textbf{\index{Sparse codingもでる@Sparse codingモデル}} \cite{Olshausen1996-xe} \cite{Olshausen1997-qu}はV1のニューロンの応答特性を説明する\textbf{\index{せんけいせいせいもでる@線形生成モデル}} (linear generative model)である.まず,画像パッチ $\mathbf{x}$ が基底関数(basis function) $\mathbf{\Phi} = [\phi_j]$ のノイズを含む線形和で表されるとする (係数は $\mathbf{r}=[r_j]$ とする).


\begin{equation}
\mathbf{x} = \sum_j r_j \phi_j +\boldsymbol{\epsilon}= \mathbf{\Phi} \mathbf{r}+ \boldsymbol{\epsilon}
\end{equation}


ただし,$\boldsymbol{\epsilon} \sim \mathcal{N}(\mathbf{0}, \sigma^2 \mathbf{I})$ である.このモデルを神経ネットワークのモデルと考えると, $\mathbf{\Phi}$ は重み行列,係数 $\mathbf{r}$ は入力よりも高次の神経細胞の活動度を表していると解釈できる.ただし,$r_j$ は負の値も取るので単純に発火率と捉えられないのはこのモデルの欠点である.

Sparse codingでは神経活動 $\mathbf{r}$ が潜在変数の推定量を表現しているという仮定の下,少数の基底で画像 (や目的変数)を表すことを目的とする.要は上式において,ほとんどが0で,一部だけ0以外の値を取るという疎 (=sparse)な係数$\mathbf{r}$を求めたい.
\subsubsection{確率的モデルの記述}
入力される画像パッチ $\mathbf{x}_i\ (i=1, \ldots, N)$ の真の分布を $p_{data}(\mathbf{x})$ とする.また,$\mathbf{x}$ の生成モデルを $p(\mathbf{x}|\mathbf{\Phi})$ とする.さらに潜在変数 $\mathbf{r}$ の事前分布 (prior)を $p(\mathbf{r})$, 画像パッチ $\mathbf{x}$ の尤度 (likelihood)を $p(\mathbf{x}|\mathbf{r}, \mathbf{\Phi})$ とする.このとき,


\begin{equation}
p(\mathbf{x}|\mathbf{\Phi})=\int p(\mathbf{x}|\mathbf{r}, \mathbf{\Phi})p(\mathbf{r})d\mathbf{r}
\end{equation}


が成り立つ.$p(\mathbf{x}|\mathbf{r}, \mathbf{\Phi})$は,(1)式においてノイズ項を$\boldsymbol{\epsilon} \sim\mathcal{N}(\mathbf{0}, \sigma^2 \mathbf{I})$としたことから,


\begin{equation}
p(\mathbf{x}|\ \mathbf{r}, \mathbf{\Phi})=\mathcal{N}\left(\mathbf{x}|\ \mathbf{\Phi} \mathbf{r}, \sigma^2 \mathbf{I} \right)=\frac{1}{Z_{\sigma}} \exp\left(-\frac{\|\mathbf{x} - \mathbf{\Phi} \mathbf{r}\|^2}{2\sigma^2}\right)
\end{equation}


と表せる.ただし,$Z_{\sigma}$は規格化定数である.
\subsubsection{事前分布の設定}
事前分布$p(\mathbf{r})$としては,0においてピークがあり,裾の重い(heavy tail)を持つsparse distributionあるいは \textbf{\index{super-Gaussian distribution}} (Laplace 分布やCauchy分布などGaussian分布よりもkurtoticな分布)を用いるのが良い.このような分布では,$\mathbf{r}$の各要素$r_i$はほとんど0に等しく,ある入力に対しては大きな値を取る.$p(\mathbf{r})$は一般化して式(4), (5)のように表記する.


\begin{align}
p(\mathbf{r})&=\prod_j p(r_j)\\
p(r_j)&=\frac{1}{Z_{\beta}}\exp \left[-\beta S(r_j)\right]
\end{align}


ただし,$\beta$は逆温度(inverse temperature), $Z_{\beta}$は規格化定数 (分配関数) である.これらの用語は統計力学における正準分布 (ボルツマン分布)から来ている.$S(x)$と分布の関係をまとめた表が以下となる (cf. \url{https://pdfs.semanticscholar.org/be08/da912362bf40fe3ded78bdadc644f921b4e7.pdf}).
|$S(r)$|$\dfrac{dS(r)}{dr}$|$p(r)$|分布名|尖度(kurtosis)|
|:-:|:-:|:-:|:-:|:-:|
|$r^2$|$2r$|$\dfrac{1}{\alpha \sqrt{2\pi}}\exp\left(-\dfrac{r^2}{2\alpha^2}\right)$|Gaussian 分布|0|
|$\vert r\vert$|$\text{sign}(r)$|$\dfrac{1}{2\alpha}\exp\left(-\dfrac{\vert r\vert}{\alpha}\right)$|Laplace 分布|3.0|
|$\ln (\alpha^2+r^2)$|$\dfrac{2r}{\alpha^2+r^2}$|$\dfrac{\alpha}{\pi}\dfrac{1}{\alpha^2+r^2}=\dfrac{\alpha}{\pi}\exp[-\ln (\alpha^2+r^2)]$|Cauchy 分布|-|
分布$p(r)$や$S(r)$を描画すると次のようになる.
\lstinputlisting[language=julia]{./text/energy-based-model/sparse-coding/005.jl}
\begin{figure}[ht]
	\centering
	\includegraphics[scale=0.8, max width=\linewidth]{./fig/solve-credit-assignment-problem/linear-network-learning-dynamics/cell005.png}
	\caption{cell005.png}
	\label{cell005.png}
\end{figure}
\subsection{目的関数の設定と最適化}
最適な生成モデルを得るために,入力される画像パッチの真の分布 $p_{data}(\mathbf{x})$と$\mathbf{x}$の生成モデル $p(\mathbf{x}|\mathbf{\Phi})$を近づける.このために,2つの分布のKullback-Leibler ダイバージェンス $D_{\text{KL}}\left(p_{data}(\mathbf{x}) \Vert\ p(\mathbf{x}|\mathbf{\Phi})\right)$を最小化したい.しかし,真の分布は得られないので,経験分布 


\begin{equation}
\hat{p}_{data}(\mathbf{x})\triangleq\frac{1}{N}\sum_{i=1}^N \delta(\mathbf{x}-\mathbf{x}_i)
\end{equation}


を近似として用いる ($\delta(\cdot)$ はDiracのデルタ関数である).ゆえに$D_{\text{KL}}\left(\hat{p}_{data}(\mathbf{x}) \Vert\ p(\mathbf{x}|\mathbf{\Phi})\right)$を最小化する.


\begin{align}
D_{\text{KL}}\left(\hat{p}_{data}(\mathbf{x}) \Vert\ p(\mathbf{x}|\mathbf{\Phi})\right)&=\int \hat{p}_{data}(\mathbf{x}) \log \frac{\hat{p}_{data}(\mathbf{x})}{p(\mathbf{x}|\mathbf{\Phi})} d\mathbf{x}\\
&=\mathbb{E}_{\hat{p}_{data}} \left[\ln \frac{\hat{p}_{data}(\mathbf{x})}{p(\mathbf{x}|\mathbf{\Phi})}\right]\\
&=\mathbb{E}_{\hat{p}_{data}} \left[\ln \hat{p}_{data}(\mathbf{x})\right]-\mathbb{E}_{\hat{p}_{data}} \left[\ln p(\mathbf{x}|\mathbf{\Phi})\right]
\end{align}


が成り立つ.(7)式の1番目の項は一定なので,$D_{\text{KL}}\left(\hat{p}_{data}(\mathbf{x}) \Vert\ p(\mathbf{x}|\mathbf{\Phi})\right)$ を最小化するには$\mathbb{E}_{\hat{p}_{data}} \left[\ln p(\mathbf{x}|\mathbf{\Phi})\right]$を最大化すればよい.ここで,


\begin{equation}
\mathbb{E}_{\hat{p}_{data}} \left[\ln p(\mathbf{x}|\mathbf{\Phi})\right]=\sum_{i=1}^N \hat{p}_{data}(\mathbf{x}_i)\ln p(\mathbf{x}_i|\mathbf{\Phi})=\frac{1}{N}\sum_{i=1}^N \ln p(\mathbf{x}_i|\mathbf{\Phi})
\end{equation}


が成り立つ.また,(2)式より


\begin{equation}
\ln p(\mathbf{x}|\mathbf{\Phi})=\ln \int p(\mathbf{x}|\mathbf{r}, \mathbf{\Phi})p(\mathbf{r})d\mathbf{r}
\end{equation}


が成り立つので,近似として $\displaystyle \int p(\mathbf{x}|\mathbf{r}, \mathbf{\Phi})p(\mathbf{r})d\mathbf{r}$ を $p(\mathbf{x}|\mathbf{r}, \mathbf{\Phi})p(\mathbf{r}) \left(=p(\mathbf{x}, \mathbf{r}| \mathbf{\Phi})\right)$ で評価する.これらの近似の下,最適な$\mathbf{\Phi}=\mathbf{\Phi}^*$は次のようにして求められる.


\begin{align}
\mathbf{\Phi}^*&=\text{arg} \min_{\mathbf{\Phi}} \min_{\mathbf{r}} D_{\text{KL}}\left(\hat{p}_{data}(\mathbf{x}) \| p(\mathbf{x}|\mathbf{\Phi})\right)\\
&=\text{arg} \max_{\mathbf{\Phi}} \max_{\mathbf{r}} \mathbb{E}_{\hat{p}_{data}} \left[\ln p(\mathbf{x}|\mathbf{\Phi})\right]\\
&= \text{arg} \max_{\mathbf{\Phi}}\sum_{i=1}^N \max_{\mathbf{r}_i} \ln p(\mathbf{x}_i|\mathbf{\Phi})\\
&\approx \text{arg} \max_{\mathbf{\Phi}}\sum_{i=1}^N \max_{\mathbf{r}_i} \ln p(\mathbf{x}_i|\mathbf{r}_i, \mathbf{\Phi})p(\mathbf{r}_i)\\
&=\text{arg}\min_{\mathbf{\Phi}} \sum_{i=1}^N \min_{\mathbf{r}_i}\ E(\mathbf{x}_i, \mathbf{r}_i|\mathbf{\Phi})
\end{align}


ただし,$\mathbf{x}_i$に対する神経活動を $\mathbf{r}_i$とした.また,$E(\mathbf{x}, \mathbf{r}|\mathbf{\Phi})$はコスト関数であり,次式のように表される.


\begin{align}
E(\mathbf{x}, \mathbf{r}|\mathbf{\Phi})\triangleq&-\ln p(\mathbf{x}|\mathbf{r}, \mathbf{\Phi})p(\mathbf{r})\\
=&\underbrace{\left\|\mathbf{x}-\mathbf{\Phi} \mathbf{r}\right\|^2}_{\text{preserve information}} + \lambda \underbrace{\sum_j S\left(r_j\right)}_{\text{sparseness of}\ r_j}
\end{align}


ただし,$\lambda=2\sigma^2\beta$は正則化係数(この式から逆温度$\beta$が正則化の度合いを調整するパラメータであることがわかる.)であり,1行目から2行目へは式(3), (4), (5)を用いた.ここで,第1項が復元損失,第2項が罰則項 (正則化項)となっている.

式(9)で表される最適化手順を最適な$\mathbf{r}$と$\mathbf{\Phi}$を求める過程に分割しよう.まず, $\mathbf{\Phi}$を固定した下で$E(\mathbf{x}_n, \mathbf{r}_i|\mathbf{\Phi})$を最小化する$\mathbf{r}_i=\hat{\mathbf{r}}_i$を求める.


\begin{equation}
\hat{\mathbf{r}}_i=\text{arg}\min_{\mathbf{r}_i}E(\mathbf{x}_i, \mathbf{r}_i|\mathbf{\Phi})\ \left(= \text{arg}\max_{\mathbf{r}_i}p(\mathbf{r}_i|\mathbf{x}_i)\right)
\end{equation}


これは $\mathbf{r}$ について \textbf{\index{MAPすいてい@MAP推定}} (maximum a posteriori estimation)を行うことに等しい.次に$\hat{\mathbf{r}}$を用いて


\begin{equation}
\mathbf{\Phi}^*=\text{arg}\min_{\mathbf{\Phi}} \sum_{i=1}^N E(\mathbf{x}_i, \hat{\mathbf{r}}_i|\mathbf{\Phi})\ \left(= \text{arg}\max_{\mathbf{\Phi}} \prod_{i=1}^N p(\mathbf{x}_i|\hat{\mathbf{r}}_i, \mathbf{\Phi})\right)
\end{equation}


とすることにより,$\mathbf{\Phi}$を最適化する.こちらは $\mathbf{\Phi}$ について \textbf{\index{さいゆうすいてい@最尤推定}} (maximum likelihood estimation)を行うことに等しい.
\subsection{ Locally competitive algorithm (LCA) }
$\mathbf{r}$の勾配法による更新則は,$E$の微分により次のように得られる.


\begin{equation}
\frac{d \mathbf{r}}{dt}= -\frac{\eta_\mathbf{r}}{2}\frac{\partial E}{\partial \mathbf{r}}=\eta_\mathbf{r} \cdot\left[\mathbf{\Phi}^\top (\mathbf{x}-\mathbf{\Phi}\mathbf{r})- \frac{\lambda}{2}S'\left(\mathbf{r}\right)\right]
\end{equation}


ただし,$\eta_{\mathbf{r}}$は学習率である.この式により$\mathbf{r}$が収束するまで最適化するが,単なる勾配法ではなく,\cite{Olshausen1996-xe}では\textbf{\index{きょうやくこうばいほう@共役勾配法}} (conjugate gradient method)を用いている.しかし,共役勾配法は実装が煩雑で非効率であるため,より効率的かつ生理学的な妥当性の高い学習法として,\textbf{\index{LCA}}  (locally competitive algorithm)が提案されている \cite{Rozell2008-wp}.LCAは\textbf{\index{そくよくせい@側抑制}} (local competition, lateral inhibition)と\textbf{\index{いきちかんすう@閾値関数}} (thresholding function)を用いる更新則である.LCAによる更新を行うRNNは通常のRNNとは異なり,コスト関数(またはエネルギー関数)を最小化する動的システムである.このような機構はHopfield networkで用いられているために,Olshausenは\textbf{\index{Hopfield trick}}と呼んでいる.
\subsubsection{軟判定閾値関数を用いる場合 (ISTA)}
$S(x)=|x|$とした場合の閾値関数を用いる手法として\textbf{\index{ISTA}}(Iterative Shrinkage Thresholding Algorithm)がある.ISTAはL1-norm正則化項に対する近接勾配法で,要はLasso回帰に用いる勾配法である.

解くべき問題は次式で表される.


\begin{equation}
\mathbf{r} = \mathop{\rm arg~min}\limits_{\mathbf{r}}\left\{\|\mathbf{x}-\mathbf{\Phi}\mathbf{r}\|^2_2+\lambda\|\mathbf{r}\|_1\right\}
\end{equation}


詳細は後述するが,次のように更新することで解が得られる.

\begin{itemize}
\item $\mathbf{r}(0)$を要素が全て0のベクトルで初期化:$\mathbf{r}(0)=\mathbf{0}$
\item $\mathbf{r}_*(t+1)=\mathbf{r}(t)+\eta_\mathbf{r}\cdot \mathbf{\Phi}^\top(\mathbf{x}-\mathbf{\Phi}\mathbf{r}(t))$
\item $\mathbf{r}(t+1) = \Theta_\lambda(\mathbf{r}_*(t+1))$
\item $\mathbf{r}$が収束するまで2と3を繰り返す
\end{itemize}

ここで$\Theta_\lambda(\cdot)$は\textbf{\index{なんはんていいきちかんすう@軟判定閾値関数}} (Soft thresholding function)と呼ばれ,次式で表される.


\begin{equation}
\Theta_\lambda(y)= 
\begin{cases} 
y-\lambda & (y>\lambda)\\ 
0 & (-\lambda\leq y\leq\lambda)\\ 
 y+\lambda & (y<-\lambda) 
\end{cases}
\end{equation}


$\Theta_\lambda(\cdot)$を関数として定義すると次のようになる.また,ReLU (ランプ関数)は\jl{max(x, 0)}で実装できる.この点から考えればReLUを軟判定非負閾値関数 (soft nonnegative thresholding function)と捉えることもできる \cite{Papyan2018-yr}.
\lstinputlisting[language=julia]{./text/energy-based-model/sparse-coding/009.jl}
次に$\Theta_\lambda(\cdot)$を描画すると次のようになる.
\lstinputlisting[language=julia]{./text/energy-based-model/sparse-coding/011.jl}
\begin{figure}[ht]
	\centering
	\includegraphics[scale=0.8, max width=\linewidth]{./fig/local-learning-rule/pca-hebbian-learning/cell011.png}
	\caption{cell011.png}
	\label{cell011.png}
\end{figure}
なお,軟判定閾値関数は次の目的関数$C$を最小化する$x$を求めることで導出できる.


\begin{equation}
C=\frac{1}{2}(y-x)^2+\lambda |x|
\end{equation}


ただし,$x, y, \lambda$はスカラー値とする.$|x|$が微分できないが,これは場合分けを考えることで解決する.$x\geq 0$を考えると,(6)式は


\begin{equation}
C=\frac{1}{2}(y-x)^2+\lambda x = \{x-(y-\lambda)\}^2+\lambda(y-\lambda)
\end{equation}


となる.(7)式の最小値を与える$x$は場合分けをして考えると,$y-\lambda\geq0$のとき二次関数の頂点を考えて$x=y-\lambda$となる. 一方で$y-\lambda<0$のときは$x\geq0$において単調増加な関数となるので,最小となるのは$x=0$のときである.同様の議論を$x\leq0$に対しても行うことで (5)式が得られる.

なお,閾値関数としては軟判定閾値関数だけではなく,硬判定閾値関数や$y=x - \text{tanh}(x)$ (Tanh-shrink)など様々な関数を用いることができる.
\subsection{重み行列の更新則}
$\mathbf{r}$が収束したら勾配法により$\mathbf{\Phi}$を更新する.


\begin{equation}
\Delta \phi_i(\boldsymbol{x}) = -\eta \frac{\partial E}{\partial \mathbf{\Phi}}=\eta\cdot\left[\left([\mathbf{x}-\mathbf{\Phi}\mathbf{r}\right)\mathbf{r}^\top\right]
\end{equation}
\subsection{Sparse coding networkの実装}
ネットワークは入力層を含め2層の単純な構造である.今回は,入力はランダムに切り出した16×16 (=256)の画像パッチとし,これを入力層の256個のニューロンが受け取るとする.入力層のニューロンは次層の100個のニューロンに投射するとする.100個のニューロンが入力をSparseに符号化するようにその活動および重み行列を最適化する.
\subsubsection{画像データの読み込み}
データは\url{http://www.rctn.org/bruno/sparsenet/}からダウンロードできる.これはアメリカ北西部で撮影された自然画像であり,van Hateren's Natural Image Dataset \url{http://bethgelab.org/datasets/vanhateren/} から取得されたものである.\jl{IMAGES_RAW.mat}は10枚の自然画像で,\jl{IMAGES.mat}はそれを白色化したものである.\jl{mat}ファイルの読み込みには MAT.jl \url{https://github.com/JuliaIO/MAT.jl}を用いる.
\lstinputlisting[language=julia]{./text/energy-based-model/sparse-coding/016.jl}
\lstinputlisting[language=julia]{./text/energy-based-model/sparse-coding/017.jl}
画像データを描画する.
\lstinputlisting[language=julia]{./text/energy-based-model/sparse-coding/019.jl}
\begin{figure}[ht]
	\centering
	\includegraphics[scale=0.8, max width=\linewidth]{./fig/energy-based-model/sparse-coding/cell019.png}
	\caption{cell019.png}
	\label{cell019.png}
\end{figure}
\subsubsection{モデルの定義}
必要なパッケージを読み込む.
\lstinputlisting[language=julia]{./text/energy-based-model/sparse-coding/021.jl}
モデルを定義する.
\lstinputlisting[language=julia]{./text/energy-based-model/sparse-coding/023.jl}
パラメータを更新する関数を定義する.今回はより生理学的に妥当にするため,軟判定非負閾値関数を用いる.
\lstinputlisting[language=julia]{./text/energy-based-model/sparse-coding/025.jl}
行ごとに正規化する関数を定義する.
\lstinputlisting[language=julia]{./text/energy-based-model/sparse-coding/027.jl}
損失関数を定義する.
\lstinputlisting[language=julia]{./text/energy-based-model/sparse-coding/029.jl}
シミュレーションを実行する関数を定義する.外側の\jl{for loop}では画像パッチの作成と\jl{r}の初期化を行う.内側の\jl{for loop}では\jl{r}が収束するまで更新を行い,収束したときに重み行列\jl{Phi}を更新する.
\lstinputlisting[language=julia]{./text/energy-based-model/sparse-coding/031.jl}
\jl{r_tm1 .= model.r}の部分は,要素ごとのコピーを実行している.\jl{r_tm1 = copy(model.r)}でもよいが,新たなメモリ割り当てが生じるので避けている.\jl{@. r_tm1 = model.r}としてもよい.シミュレーションの実行をする.
\lstinputlisting[language=julia]{./text/energy-based-model/sparse-coding/033.jl}
\subsubsection{訓練中の損失の描画}
訓練中の損失の変化を描画してみよう.損失が低下し,学習が進行したことが分かる.
\lstinputlisting[language=julia]{./text/energy-based-model/sparse-coding/035.jl}
\begin{figure}[ht]
	\centering
	\includegraphics[scale=0.8, max width=\linewidth]{./fig/energy-based-model/sparse-coding/cell035.png}
	\caption{cell035.png}
	\label{cell035.png}
\end{figure}
\subsubsection{重み行列 (受容野)の描画}
学習後の重み行列 \jl{Phi} ($\mathbf{\Phi}$)を可視化してみよう.
\lstinputlisting[language=julia]{./text/energy-based-model/sparse-coding/037.jl}
\begin{figure}[ht]
	\centering
	\includegraphics[scale=0.8, max width=\linewidth]{./fig/energy-based-model/sparse-coding/cell037.png}
	\caption{cell037.png}
	\label{cell037.png}
\end{figure}
白色が\textbf{\index{ONりょういき@ON領域}}(興奮),黒色が\textbf{\index{OFFりょういき@OFF領域}}(抑制)を表す.Gaborフィルタ様の局所受容野が得られており,これは一次視覚野(V1)における単純型細胞(simple cells)の受容野に類似している.
\subsubsection{画像の再構成}
学習したモデルを用いて入力画像が再構成されるか確認しよう.
\lstinputlisting[language=julia]{./text/energy-based-model/sparse-coding/040.jl}
神経活動 $\mathbf{r}$がスパースになっているか確認しよう.
\lstinputlisting[language=julia]{./text/energy-based-model/sparse-coding/042.jl}
\begin{figure}[ht]
	\centering
	\includegraphics[scale=0.8, max width=\linewidth]{./fig/bayesian-brain/neural-sampling/cell042.png}
	\caption{cell042.png}
	\label{cell042.png}
\end{figure}
要素がほとんど0のスパースなベクトルになっていることがわかる.次に画像を再構成する.
\lstinputlisting[language=julia]{./text/energy-based-model/sparse-coding/044.jl}
再構成した結果を描画する.
\lstinputlisting[language=julia]{./text/energy-based-model/sparse-coding/046.jl}
\begin{figure}[ht]
	\centering
	\includegraphics[scale=0.8, max width=\linewidth]{./fig/energy-based-model/sparse-coding/cell046.png}
	\caption{cell046.png}
	\label{cell046.png}
\end{figure}
上段が入力画像,下段が再構成された画像である.差異はあるものの,概ね再構成されていることがわかる.
論文以外の参考資料
\begin{itemize}
\item \url{http://www.scholarpedia.org/article/Sparse_coding}
\item Bruno Olshausen: “Sparse coding in brains and machines”(\url{https://talks.stanford.edu/bruno-olshausen-sparse-coding-in-brains-and-machines/}), \url{http://www.rctn.org/bruno/public/Simons-sparse-coding.pdf}
\item \url{https://redwood.berkeley.edu/wp-content/uploads/2018/08/sparse-coding-ICA.pdf}
\item \url{https://redwood.berkeley.edu/wp-content/uploads/2018/08/sparse-coding-LCA.pdf}
\item \url{https://redwood.berkeley.edu/wp-content/uploads/2018/08/Dylan-lca_overcompleteness_09-27-2018.pdf}
\end{itemize}
