\section{Hopfield モデル}


\cite{Hopfield1982-vu}で提案.始めは1と0の状態を取った.

Hopfieldモデルと呼ばれることが多いが,Amariの先駆的研究\cite{Amari1972-fq}を踏まえAmari-Hopfieldモデルと呼ばれることもある.

次のような連続時間線形モデルを考える.シナプス前活動を$\mathbf{x}\in \mathbb{R}^n$, 後活動を$\mathbf{y}\in \mathbb{R}^m$, 重み行列を$\mathbf{W}\in \mathbb{R}^{m\times n}$とする.


\frac{d\mathbf{y}}{dt}=-\mathbf{y}+\mathbf{W}\mathbf{x}+\mathbf{b}


ここで$\dfrac{\partial\mathcal{L}}{\partial\mathbf{y}}:=-\dfrac{d\mathbf{y}}{dt}$となるような$\mathcal{L}\in \mathbb{R}$を仮定すると,


\mathcal{L}=\int \left(\mathbf{y}-\mathbf{W}\mathbf{x}-\mathbf{b}\right)\ d\mathbf{y}=\frac{1}{2}\|\mathbf{y}\|^2-\mathbf{y}^\top \mathbf{W}\mathbf{x}-\mathbf{y}^\top \mathbf{b}


となる. これをさらに$\mathbf{W}$で微分すると,


\dfrac{\partial\mathcal{L}}{\partial\mathbf{W}}=-\mathbf{y}\mathbf{x}^\top\Rightarrow
\frac{d\mathbf{W}}{dt}=-\dfrac{\partial\mathcal{L}}{\partial\mathbf{W}}=\mathbf{y}\mathbf{x}^\top=(\text{post})\cdot (\text{pre})^\top


となり,Hebb則が導出できる.
