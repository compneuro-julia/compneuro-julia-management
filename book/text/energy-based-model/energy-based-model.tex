\section{エネルギーベースモデル}
本章では\textbf{\index{えねるぎーべーすもでる (energy-based models; EBMs)@エネルギーベースモデル (energy-based models; EBMs)}} という枠組みに含まれるモデルを紹介する.エネルギーベースモデルではネットワークの状態をスカラー値に変換するエネルギー関数 (あるいはコスト関数) を定義し,推論時と学習時の双方においてエネルギーを最小化するようにネットワークの状態を更新する.\cite{LeCun2006-dt}

入力 $\mathbf{x}\in \mathbb{R}^d$, エネルギー関数 $E_\theta: \mathbb{R}^d\to \mathbb{R}$を考える.


\begin{align}
p_\theta(\mathbf{x})&=\frac{\exp(-E_\theta(\mathbf{x}))}{Z_\theta}\\
Z_\theta &= \int \exp(-E_\theta(\mathbf{x})) d\mathbf{x}
\end{align}


$Z_\theta$は分配関数.
