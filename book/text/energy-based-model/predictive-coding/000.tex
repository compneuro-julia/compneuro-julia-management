\section{予測符号化}
\subsection{観測世界の階層的予測}
\textbf{階層的予測符号化(hierarchical predictive coding; HPC)} は\cite{Rao1999-zv}により導入された.構築するネットワークは入力層を含め,3層のネットワークとする.LGNへの入力として画像 $\mathbf{x} \in \mathbb{R}^{n_0}$を考える.画像 $\mathbf{x}$ の観測世界における隠れ変数,すなわち\textbf{潜在変数} (latent variable)を$\mathbf{r} \in \mathbb{R}^{n_1}$とし,ニューロン群によって発火率で表現されているとする (真の変数と $\mathbf{r}$は異なるので文字を分けるべきだが簡単のためにこう表す).このとき,


\mathbf{x} = f(\mathbf{U}\mathbf{r}) + \boldsymbol{\epsilon}


が成立しているとする.ただし,$f(\cdot)$は活性化関数 (activation function),$\mathbf{U} \in \mathbb{R}^{n_0 \times n_1}$は重み行列である.
$\boldsymbol{\epsilon} \in \mathbb{R}^{n_0}$ は $\mathcal{N}(\mathbf{0}, \sigma^2 \mathbf{I})$ からサンプリングされるとする.

潜在変数 $\mathbf{r}$はさらに高次 (higher-level)の潜在変数 $\mathbf{r}^h$により,次式で表現される.


\mathbf{r} = \mathbf{r}^{td}+\boldsymbol{\epsilon}^{td}=f(\mathbf{U}^h \mathbf{r}^h)+\boldsymbol{\epsilon}^{td}


ただし,Top-downの予測信号を $\mathbf{r}^{td}:=f(\mathbf{U}^h \mathbf{r}^h)$とした.また,$\mathbf{r}^{td} \in \mathbb{R}^{n_1}$, $\mathbf{r}^{h} \in \mathbb{R}^{n_2}$, $\mathbf{U}^h \in \mathbb{R}^{n_1 \times n_2}$ である.
$\boldsymbol{\epsilon}^{td} \in \mathbb{R}^{n_1}$は$\mathcal{N}(\mathbf{0}$, $\sigma_{td}^2 \mathbf{I}$) からサンプリングされるとする.

話は飛ぶが,Predictive codingのネットワークの特徴は
\begin{itemize}
\item 階層的な構造
\item 高次による低次の予測 (Feedback or Top-down信号)
\item 低次から高次への誤差信号の伝搬 (Feedforward or Bottom-up 信号)
\end{itemize}

である.ここまでは高次表現による低次表現の予測,というFeedback信号について説明してきたが,この部分はSparse codingでも同じである.それではPredictive codingのもう一つの要となる,低次から高次への予測誤差の伝搬というFeedforward信号はどのように導かれるのだろうか.結論から言えば,これは\textbf{復元誤差 (reconstruction error)の最小化を行う再帰的ネットワーク (recurrent network)を考慮することで自然に導かれる}.
