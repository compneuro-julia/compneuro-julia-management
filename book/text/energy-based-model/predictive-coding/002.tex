\subsection{損失関数と学習則}\subsubsection{事前分布の設定}$\mathbf{r}$の事前分布$p(\mathbf{r})$はCauchy分布を用いる.$p(\mathbf{r})$の負の対数事前分布を$g(\mathbf{r}):=-\log p(\mathbf{r})$としておく.
$$
\begin{align}
p(\mathbf{r})&=\prod_i p(r_i)=\prod_i \exp\left[-\alpha \ln(1+r_i^2)\right]\\
g(\mathbf{r})&=-\ln p(\mathbf{r})=\alpha \sum_i \ln(1+r_i^2)\\
g'(\mathbf{r})&=\frac{\partial g(\mathbf{r})}{\partial \mathbf{r}}=\left[\frac{2\alpha r_i}{1+r_i^2}\right]_i
\end{align}
$$
次に重み行列$\mathbf{U}$の事前分布 $p(\mathbf{U})$はGaussian分布とする.$p(\mathbf{U})$の負の対数事前分布を$h(\mathbf{U}):=-\ln p(\mathbf{U})$とすると,次のように表される.
$$
\begin{align}
p(\mathbf{U})&=\exp(-\lambda\|\mathbf{U}\|^2_F)\\
h(\mathbf{U})&=-\ln p(\mathbf{U})=\lambda\|\mathbf{U}\|^2_F\\
h'(\mathbf{U})&=\frac{\partial h(\mathbf{U})}{\partial \mathbf{U}}=2\lambda \mathbf{U}
\end{align}
$$
ただし,$\|\cdot \| _ F^2$はフロベニウスノルムを意味する.
\subsubsection{損失関数の設定}[11-2](https://compneuro-julia.github.io/11-2_sparse-coding.html)と同様に考えることにより,損失関数 $E$を次のように定義する.
$$
\begin{align}
E=\underbrace{\frac{1}{\sigma^{2}}\|\mathbf{x}-f(\mathbf{U} \mathbf{r})\|^2+\frac{1}{\sigma_{t d}^{2}}\left\|\mathbf{r}-f(\mathbf{U}^h \mathbf{r}^h)\right\|^2}_{\text{reconstruction error}}+\underbrace{g(\mathbf{r})+g(\mathbf{r}^{h})+h(\mathbf{U})+h(\mathbf{U}^h)}_{\text{sparsity penalty}}\tag{4}
\end{align}
$$
潜在変数 $\mathbf{r}, \mathbf{r}^h$ と 重み行列 $\mathbf{U}, \mathbf{U}^h$ のそれぞれに事前分布を仮定しているため,これらについてのMAP推定を行うことに相当する.
\subsubsection{再帰ネットワークの更新則}簡単のために$\mathbf{z}:=\mathbf{U}\mathbf{r}, \mathbf{z}^h:=\mathbf{U}^h\mathbf{r}^h$とする.
$$
\begin{align}
\frac{d \mathbf{r}}{d t}&=-\frac{k_{1}}{2} \frac{\partial E}{\partial \mathbf{r}}=k_{1}\cdot\Bigg(\frac{1}{\sigma^{2}} \mathbf{U}^{T}\bigg[\frac{\partial f(\mathbf{z})}{\partial \mathbf{z}}\odot\underbrace{(\mathbf{x}-f(\mathbf{z}))}_{\text{bottom-up error}}\bigg]-\frac{1}{\sigma_{t d}^{2}}\underbrace{\left(\mathbf{r}-f(\mathbf{z}^h)\right)}_{\text{top-down error}}-\frac{1}{2}g'(\mathbf{r})\Bigg)\tag{5}\\
\frac{d \mathbf{r}^h}{d t}&=-\frac{k_{1}}{2} \frac{\partial E}{\partial \mathbf{r}^h}=k_{1}\cdot\Bigg(\frac{1}{\sigma_{t d}^{2}}(\mathbf{U}^h)^\top\bigg[\frac{\partial f(\mathbf{z}^h)}{\partial \mathbf{z}^h}\odot\underbrace{\left(\mathbf{r}-f(\mathbf{z}^h)\right)}_{\text{bottom-up error}}\bigg]-\frac{1}{2}g'(\mathbf{r}^h)\Bigg)\tag{6}
\end{align}
$$
ただし,$k_1$は更新率 (updating rate)である.または,発火率の時定数を$\tau:=1/k_1$として,$k_1$は発火率の時定数$\tau$の逆数であると捉えることもできる.ここで(5)式において,中間表現 $\mathbf{r}$ のダイナミクスはbottom-up errorとtop-down errorで記述されている.このようにbottom-up errorが $\mathbf{r}$ への入力となることは自然に導出される.なお,top-down errorに関しては高次からの予測 (prediction)の項 $f(\mathbf{x}^h)$とleaky-integratorとしての項 $-\mathbf{r}$に分割することができる.また$\mathbf{U}^\top, (\mathbf{U}^h)^\top$は重み行列の転置となっており,bottom-upとtop-downの投射において対称な重み行列を用いることを意味している.$-g'(\mathbf{r})$は発火率を抑制してスパースにすることを目的とする項だが,無理やり解釈をすると自己再帰的な抑制と言える.