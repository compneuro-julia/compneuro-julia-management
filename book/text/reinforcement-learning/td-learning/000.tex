\section{TD学習}

TD (Temporal difference) learningにおいて,\textbf{報酬予測誤差}(reward prediction error, \textbf{RPE}) $\delta_{i}$は次のように計算される. 

 
\delta_{i}=r+\gamma V_{j}\left(x^{\prime}\right)-V_{i}(x) 
 

ただし,現在の状態を$x$, 次の状態を$x'$, 予測価値分布を$V(x)$, 報酬信号を$r$, 時間割引率(time discount)を$\gamma$としました.
また,$V_{j}\left(x^{\prime}\right)$は予測価値分布$V\left(x^{\prime}\right)$からのサンプルです. このRPEは脳内において主に中脳の\textbf{VTA}(腹側被蓋野)や\textbf{SNc}(黒質緻密部)における\textbf{ドパミン(dopamine)ニューロン}の発火率として表現されています.

ただし,VTAとSNcのドパミンニューロンの役割は同一ではありません.ドパミンニューロンへの入力が異なっています [(Watabe-Uchida et al., _Neuron._ 2012)](https://www.cell.com/neuron/fulltext/S0896-6273(12)00281-4). また,細かいですがドパミンニューロンの発火は報酬量に対して線形ではなく,やや飽和する非線形な応答関数 (Hill functionで近似可能)を持ちます([Eshel et al., _Nat. Neurosci._ 2016](https://www.nature.com/articles/nn.4239)).このため著者実装では報酬 $r$に非線形関数がかかっているものもあります.

先ほどRPEはドパミンニューロンの発火率で表現されている,といいました.RPEが正の場合はドパミンニューロンの発火で表現できますが,単純に考えると負の発火率というものはないため,負のRPEは表現できないように思います.ではどうしているかというと,RPEが0 (予想通りの報酬が得られた場合) でもドパミンニューロンは発火しており,RPEが正の場合にはベースラインよりも発火率が上がるようになっています.逆にRPEが負の場合にはベースラインよりも発火率が減少する(抑制される)ようになっています
    ([Schultz et al., \url{span style="font-style: italic;"}Science.\url{/span} 1997](https://science.sciencemag.org/content/275/5306/1593.long "https://science.sciencemag.org/content/275/5306/1593.long"); [Chang et al., \url{span style="font-style: italic;"}Nat Neurosci\url{/span}. 2016](https://www.nature.com/articles/nn.4191 "https://www.nature.com/articles/nn.4191")).発火率というのを言い換えればISI (inter-spike interval, 発火間隔)の長さによってPREが符号化されている(ISIが短いと正のRPE, ISIが長いと負のRPEを表現)ともいえます ([Bayer et al., \url{span style="font-style: italic;"}J.
    Neurophysiol\url{/span}. 2007](https://www.physiology.org/doi/full/10.1152/jn.01140.2006 "https://www.physiology.org/doi/full/10.1152/jn.01140.2006")).

予測価値(分布) $V(x)$ですが,これは線条体(striatum)のパッチ (SNcに抑制性の投射をする)やVTAのGABAニューロン (VTAのドパミンニューロンに投射して減算抑制をする, ([Eshel, et al., _Nature_. 2015](https://www.nature.com/articles/nature14855 "https://www.nature.com/articles/nature14855")))などにおいて表現されている. この予測価値は通常のTD learningでは次式により更新されます. 

 
V_{i}(x) \leftarrow V_{i}(x)+\alpha_{i} f\left(\delta_{i}\right) 
 

ただし,$\alpha_{i}$は学習率(learning rate), $f(\cdot)$はRPEに対する応答関数である.生理学的には$f(\delta)=\delta$を使うのが妥当である.
