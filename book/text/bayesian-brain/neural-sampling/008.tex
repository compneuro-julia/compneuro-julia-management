\subsubsection{分散共分散行列$\mathbf{C}$の作成}
$\mathbf{C}$は$y$の事前分布の分散共分散行列である.\cite{Orban2016-tm}では自然画像を用いて作成しているが,ここでは簡単のため$\mathbf{A}$と同様に\cite{Echeveste2020-sh}に従って作成する.前項で作成した通り,$\mathbf{A}$の各基底には周期性があるため,類似した基底を持つニューロン同士は類似した出力をすると考えられる.Echevesteらは$\theta\in[-\pi/2, \pi/2)$の範囲においてFourier基底を複数作成し,そのグラム行列(Gram matrix)を係数倍したものを$\mathbf{C}$と設定している.ここではガウス過程(Gaussian process)モデルとの類似性から,周期カーネル(periodic kernel) 


\begin{equation}
K(\theta, \theta')=\exp\left[\phi_1 \cos \left(\dfrac{|\theta-\theta'|}{\phi_2}\right)\right]
\end{equation}


を用いる.ここでは$|\theta-\theta'|=m\pi\ (m=0,1,\ldots)$の際に類似度が最大になればよいので,$\phi_2=0.5$とする.これが正定値行列になるように単位行列の係数倍$\epsilon\mathbf{I}$を加算し,スケーリングした上で,\jl{Symmetric(C)}や\jl{Matrix(Hermitian(C)))}により実対象行列としたものを$\mathbf{C}$とする.$\mathbf{C}$を正定値行列にする理由はJuliaの\jl{MvNormal}がCholesky分解を用いて多変量正規分布の乱数を生成するためである. 事前に\jl{cholesky(C)}が実行できるか確認するのもよい.
