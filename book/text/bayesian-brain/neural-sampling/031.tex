\subsection{興奮性・抑制性神経回路によるサンプリング}
前節で実装したMCMCを\textbf{興奮性・抑制性神経回路 (excitatory-inhibitory (E-I) network)} で実装する.HMCとLMCの両方を神経回路で実装する.

Hamiltonianを用いる場合,一般化座標$\mathbf{q}$を興奮性神経細胞の活動$\mathbf{u}$, 一般化運動量$\mathbf{p}$を抑制性神経細胞の活動$\mathbf{v}$に対応させる.ToDo: 詳しい説明.

簡単のため,前項で用いた入力刺激のうち,最も$z$が大きいサンプルのみを使用する.
