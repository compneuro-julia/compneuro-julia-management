\subsection{ハミルトニアン・モンテカルロ法 (HMC法)}ハミルトニアン・モンテカルロ法(Hamiltonian Monte Calro)あるいはハイブリッド・モンテカルロ法(Hybrid Monte Calro)という
一般化座標を$\mathbf{q}$, 一般化運動量を$\mathbf{p}$とする.ポテンシャルエネルギーを$U(\mathbf{q})$としたとき,古典力学 (解析力学) において保存力のみが作用する場合の\textbf{ハミルトニアン (Hamiltonian)} $\mathcal{H}(\mathbf{q}, \mathbf{p})$は
$$
\mathcal{H}(\mathbf{q}, \mathbf{p}):=U(\mathbf{q})+\frac{1}{2}\|\mathbf{p}\|^2
$$
となる.このとき,次の2つの方程式が成り立つ.
$$
\frac{d\mathbf{q}}{dt}=\frac{\partial \mathcal{H}}{\partial \mathbf{p}}=\mathbf{p},\quad\frac{d\mathbf{p}}{dt}=-\frac{\partial \mathcal{H}}{\partial \mathbf{q}}=-\frac{\partial U}{\partial \mathbf{q}}
$$
これを\textbf{ハミルトンの運動方程式(hamilton's equations of motion)} あるいは\textbf{正準方程式 (canonical equations)} という.
この処理をMetropolis-Hastings法における採用・不採用アルゴリズムという.
リープフロッグ(leap frog)法により離散化する.