\subsection{分位点・エクスペクタイル回帰}
\subsubsection{分位点回帰 (Quantile Regression)}
線形回帰(linear regression)は,誤差が正規分布と仮定したとき(必ずしも正規分布を仮定しなくてもよい)の$X$(説明変数)に対する$Y$(目的変数)の期待値$E[Y]$を求める,というものであった.\textbf{分位点回帰(quantile regression)} では,Xに対するYの分布における分位点を通るような直線を引く.

\textbf{分位点}(または分位数)において,代表的なものが\textbf{四分位数}である.四分位数は箱ひげ図などで用いるが,例えば第一四分位数は分布を25:75に分ける数,第二四分位数(中央値)は分布を50:50に分ける数である.同様に$q$分位数($q$-quantile)というと分布を$q:1-q$に分ける数となっている.分位点回帰の話に戻る.下図は$x\sim U(0, 5),\quad y=3x+x\cdot \xi,\quad \xi\sim N(0,1)$とした500個の点に対する分位点回帰である.赤い領域はX=1,2,3,4でのYの分布を示している.深緑,緑,黄色の直線はそれぞれ10, 50, 90%tile回帰の結果である.例えば50%tile回帰の結果は,Xが与えられたときのYの中央値(50%tile点)を通るような直線となっている.同様に90%tile回帰の結果は90%tile点を通るような直線となっている.
