Classical TD learningではRPEに比例して発火する細胞しかないが,Distributional TD learningではRPEの正負に応じて発火率応答が変化していることがわかる. 特に$\alpha_{i}^{+} \gt \alpha_{i}^{-}$の細胞を\textbf{楽観的細胞 (optimistic cells)},$\alpha_{i}^{+}\lt
\alpha_{i}^{-}$の細胞を**悲観的細胞 (pessimistic
cells)** と著者らは呼んでいる.実際には2群に分かれているわけではなく,gradientに遷移している.収束する予測価値が細胞ごとに異なることで,$V$には報酬の期待値ではなく複雑な形状の報酬分布が符号化される.その仕組みについて,次項から見ていこう.