\subsubsection{勾配法を用いた分位点回帰・エクスペクタイル回帰}
予測誤差$\delta$と$\tau$の関数を


\begin{align}
\text{分位点回帰:}&\quad
\rho_q(\delta; \tau)=\left|\tau-\mathbb{I}_{\delta \leq 0}\right|\cdot |\delta|=\left(\tau-\mathbb{I}_{\delta \leq 0}\right)\cdot \delta\\
\text{エクスペクタイル回帰:}&\quad
\rho_e(\delta; \tau)=\left|\tau-\mathbb{I}_{\delta \leq 0}\right|\cdot \delta^2
\end{align}


と定義する.$\rho_q(\delta; \tau)$のみ,チェック関数 (check function)あるいは非対称絶対損失関数(asymmetric absolute loss function)と呼ぶ.ただし,$\tau$は分位点(quantile),$\mathbb{I}$は指示関数(indicator function)である.この場合,$\mathbb{I}_{\delta \leq 0}$は$\delta \gt 0$なら0, $\delta \leq 0$なら1となる.このとき,目的関数は 


L_{\tau}(\delta)
=\sum_{i=1}^n \rho(\delta_i; \tau)


である.$\rho(\delta; \tau)$を色々な $\tau$についてplotすると次図のようになる.
