\subsubsection{分位数(Quantile)モデルと報酬分布の符号化}

\paragraph{RPEに対する応答がsign関数のモデルと報酬分布の分位点への予測価値の収束}
さて,Distributional RLモデルでどのようにして報酬分布が学習されるかについてみていこう.この項ではRPEに対する応答関数$f(\cdot)$が符合関数(sign function)の場合を考える.結論から言うと,この場合はasymmetric scaling factor $\tau_i$は分位数(quantile)となり,**予測価値
$V_i$は報酬分布の$\tau_i$分位数に収束する**.
    
どういうことかを簡単なシミュレーションで見てみよう.今,報酬分布を平均2, 標準偏差5の正規分布とする (すなわち$r \sim N(2, 5^2)$となります).また,$\tau_i = 0.25, 0.5, 0.75 (i=1,2,3)$とする.このとき,3つの予測価値 $V_i \ (i=1,2,3)$はそれぞれ$N(2, 5^2)$の0.25, 0.5,
0.75分位数に収束する.下図はシミュレーションの結果である.左が$V_i$の変化で,右が報酬分布と0.25, 0.5, 0.75分位数の位置 (黒短線)となっています.対応する分位数に見事に収束していることが分かる.
