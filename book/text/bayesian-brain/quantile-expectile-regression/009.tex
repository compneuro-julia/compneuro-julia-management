$L_\tau$を最小化するような$\theta$の更新式について考える.まず,



\begin{align}
\text{分位点回帰:}&\quad
\frac{\partial \rho_q(\delta; \tau)}{\partial \delta}= \rho_q^{\prime}(\delta; \tau)=\left|\tau-\mathbb{I}_{\delta \leq 0}\right| \cdot
\operatorname{sign}(\delta)\\
\text{エクスペクタイル回帰:}&\quad
\frac{\partial \rho_e(\delta; \tau)}{\partial \delta}= \rho_e^{\prime}(\delta; \tau)=2\left|\tau-\mathbb{I}_{\delta \leq 0}\right| \cdot
\delta
\end{align}


である (ただし$\text{sign}(\cdot)$は符号関数).さらに


\frac{\partial L_{\tau}}{\partial \theta}=\frac{\partial L_{\tau}}{\partial \delta}\frac{\partial \delta(\theta)}{\partial \theta}=-\frac{1}{n} \rho^{\prime}(\delta; \tau) X
 

が成り立つので,$\theta$の更新式は$\theta \leftarrow \theta + \alpha\cdot \dfrac{1}{n} \rho^{\prime}(\delta; \tau) X$と書ける ($\alpha$は学習率である).分位点回帰を単純な勾配法で求める場合,勾配が0となって解が求まらない可能性があるが,目的関数を滑らかにすることで回避できるという研究もある ([Zheng. *IJMLC*. 2011](https://link.springer.com/article/10.1007/s13042-011-0031-2)).この点,Expectileならこの問題を回避できる (?).
