\subsubsection{sign関数を用いたDistributional RLと分位点回帰}それでは,なぜ予測価値 $V_i$は$\tau_i$ 分位点に収束するのでしょうか.Extended Data Fig.1のように平衡点で考えてもよいのですが,後のために分位点回帰との関連について説明します.
実はDistributional RL (かつ,RPEの応答関数にsign関数を用いた場合)における予測報酬 $V_i$の更新式は,分位点回帰(Quantile
regression)を勾配法で行うときの更新式とほとんど同じです.分位点回帰では$\delta$の関数$\rho_{\tau}(\delta)$を次のように定義します. $$ \rho_{\tau}(\delta)=\left|\tau-\mathbb{I}_{\delta \leq 0}\right|\cdot |\delta|=\left(\tau-\mathbb{I}_{\delta
\leq 0}\right)\cdot \delta $$ そして,この関数を最小化することで回帰を行います.ここで$\tau$は分位点です.また$\delta=r-V$としておきます.今回,どんな行動をしても未来の報酬に影響はないので$\gamma=0$としています.\url{br/}
\url{br/}
ここで, $$ \frac{\partial \rho_{\tau}(\delta)}{\partial \delta}=\rho_{\tau}^{\prime}(\delta)=\left|\tau-\mathbb{I}_{\delta \leq 0}\right| \cdot \operatorname{sign}(\delta) $$ なので,$r$を観測値とすると, $$
\frac{\partial \rho_{\tau}(\delta)}{\partial V}=\frac{\partial \rho_{\tau}(\delta)}{\partial \delta}\frac{\partial \delta(V)}{\partial V}=-\left|\tau-\mathbb{I}_{\delta \leq 0}\right| \cdot
\operatorname{sign}(\delta) $$ となります.ゆえに$V$の更新式は $$ V \leftarrow V - \beta\cdot\frac{\partial \rho_{\tau}(\delta)}{\partial V}=V+\beta \left|\tau-\mathbb{I}_{\delta \leq 0}\right| \cdot
\operatorname{sign}(\delta) $$ です.ただし,$\beta$はベースラインの学習率です.個々の$V_i$について考え,符号で場合分けをすると
$$ \begin{cases} V_{i} \leftarrow V_{i}+\beta\cdot |\tau_i|\cdot\operatorname{sign}\left(\delta_{i}\right)
&\text { for } \delta_{i}>0\\ V_{i} \leftarrow V_{i}+\beta\cdot |\tau_i-1|\cdot\operatorname{sign}\left(\delta_{i}\right) &\text { for } \delta_{i} \leq 0 \end{cases} $$ となります.$0 \leq
\tau_i \leq 1$であり,$\tau_i=\alpha_{i}^{+} / \left(\alpha_{i}^{+} + \alpha_{i}^{-}\right)$であることに注意すると上式は次のように書けます. $$ \begin{cases} V_{i} \leftarrow V_{i}+\beta\cdot
\frac{\alpha_{i}^{+}}{\alpha_{i}^{+}+\alpha_{i}^{-}}\cdot\operatorname{sign}\left(\delta_{i}\right) &\text { for } \delta_{i}>0\\ V_{i} \leftarrow V_{i}+\beta\cdot
\frac{\alpha_{i}^{-}}{\alpha_{i}^{+}+\alpha_{i}^{-}}\cdot\operatorname{sign}\left(\delta_{i}\right) &\text { for } \delta_{i} \leq 0 \end{cases} $$ これは前節で述べたDistributional
RLの更新式とほぼ同じです.いくつか違う点もありますが,RPEが正の場合と負の場合に更新される値の比は同じとなっています.
このようにRPEの応答関数にsign関数を用いた場合,報酬分布を上手く符号化することができます.しかし実際のドパミンニューロンはsign関数のような生理的に妥当でない応答はせず,RPEの大きさに応じた活動をします.そこで次節ではRPEの応答関数を線形にしたときの話をします.