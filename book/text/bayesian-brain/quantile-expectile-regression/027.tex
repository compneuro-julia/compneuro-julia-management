ここでoptimisticな細胞($\tau=0.75$)は中央値よりも高い予測価値,pessimisticな細胞($\tau=0.25$)は中央値よりも低い予測価値に収束しています. つまり細胞の楽観度というものは,細胞が期待する報酬が大きいほど上がります.

同様のシミュレーションを今度は200個の細胞 (ユニット)で行います.報酬は0.1, 1, 2 μLのジュースがそれぞれ確率0.3, 0.6, 0.1で出るとします (Extended Data Fig.1と同じような分布にしています).なお,著者らはシミュレーションとマウスに対して\textbf{Variable-magnitude task}
(異なる量の報酬(ジュース)が異なる確率で出る)と\textbf{Variable-probability task} (一定量の報酬がある確率で出る)を行っています.以下はVariable-magnitude taskを行う,ということです.学習結果は次図のようになります.左はGround Truthの報酬分布で,右は$V_i$に対してカーネル密度推定
(KDE)することによって得た予測価値分布です.2つの分布はほぼ一致していることが分かります.
