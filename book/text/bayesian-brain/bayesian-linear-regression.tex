\section{ベイズ線形回帰 (Bayesian linear regression)}
共役事前分布 (conjugate prior) を
\begin{equation}
p(\mathbf{w})=\mathcal{N}(\mathbf{w}|\boldsymbol{\mu}_0, \boldsymbol{\Sigma}_0)
\end{equation}
と定義し,事後分布 (posterior) を
\begin{equation}
p(\mathbf{w}|\mathbf{Y}, \mathbf{X})=\mathcal{N}(\mathbf{w}|\hat{\boldsymbol{\mu}}, \hat{\boldsymbol{\Sigma}})
\end{equation}
とする.ただし,
\begin{align}
\hat{\boldsymbol{\Sigma}}^{-1}&= \boldsymbol{\Sigma}_0^{-1}+ \beta \Phi^\top\Phi\\
\hat{\boldsymbol{\mu}}&=\hat{\boldsymbol{\Sigma}} (\beta \Phi^\top \mathbf{y}+\boldsymbol{\Sigma}_0^{-1}\boldsymbol{\mu}_0)
\end{align}
である.また,$\Phi=\phi.(\mathbf{x})$であり,$\phi(x)=[1, x, x^2, x^3]$, $\boldsymbol{\mu}_0=\mathbf{0}, \boldsymbol{\Sigma}_0= \alpha^{-1} \mathbf{I}$とする.テストデータを$\mathbf{x}^*$とした際,予測分布は
\begin{equation}
p(y^*|\mathbf{x}^*, \mathbf{Y}, \mathbf{X})=\mathcal{N}(y^*|\boldsymbol{\mu}^*, \boldsymbol{\Sigma}^*)
\end{equation}
となる.ただし,
\begin{align}
\boldsymbol{\mu}^*&=\hat{\boldsymbol{\mu}}^\top \phi(\mathbf{x}^*)\\
\boldsymbol{\Sigma}^* &= \frac{1}{\beta} +  \phi(\mathbf{x}^*)^\top\hat{\boldsymbol{\Sigma}}\phi(\mathbf{x}^*)\\
\end{align}
である.
\begin{lstlisting}[language=julia]
using Base: @kwdef
using Parameters: @unpack
using PyPlot, LinearAlgebra, Random, Distributions
rc("axes.spines", top=false, right=false)
\end{lstlisting}
\begin{lstlisting}[language=julia]
# Generate Toy datas
num_train, num_test = 20, 100 # sample size
dims = 4 # dimensions
σy = 0.3

polynomial_expansion(x; degree=3) = stack([x .^ p for p in 0:degree]);

Random.seed!(0);
x = rand(num_train)
y = sin.(2π*x) + σy * randn(num_train);
ϕ = polynomial_expansion(x, degree=dims-1) # design matrix

xtest = range(-0.1, 1.1, length=num_test)
ytest = sin.(2π*xtest)
ϕtest = polynomial_expansion(xtest, degree=dims-1);
\end{lstlisting}
\begin{lstlisting}[language=julia]
@kwdef mutable struct BayesianLinearReg
    μ_hat::Array
    Σ_hat::Array
end

# Training params & definition of model
function BayesianLinearReg(ϕ, y, α, β)
    Σ_hat = inv(α * I + β * ϕ' * ϕ)
    μ_hat = β * Σ_hat  * ϕ' * y;
    return BayesianLinearReg(μ_hat=μ_hat, Σ_hat=Σ_hat)
end;

function predict(ϕ, blr::BayesianLinearReg, β)
    @unpack μ_hat, Σ_hat = blr
    μp = ϕ * μ_hat
    σp = sqrt.(1/β .+ diag(ϕ * Σ_hat * ϕ'));
    return μp, σp
end;

function sampling_func(ϕ, blr::BayesianLinearReg, num_sampling::Int)
    @unpack μ_hat, Σ_hat = blr
    dist = MvNormal(μ_hat, Matrix(Hermitian(Σ_hat)))
    sampled_params = rand(dist, num_sampling);
    return ϕ * sampled_params 
end;
\end{lstlisting}
\begin{lstlisting}[language=julia]
α, β = 1e-3, 5.0;

blr = BayesianLinearReg(ϕ, y, α, β);
μtest, σtest = predict(ϕtest, blr, β);

num_sampling = 5
sampled_func = sampling_func(ϕtest, blr, num_sampling);
\end{lstlisting}
\begin{lstlisting}[language=julia]
figure(figsize=(5,3.5))
title("Bayesian Linear Regression")
scatter(x, y, facecolor="None", edgecolors="black", s=25) # samples
plot(xtest, ytest, "--", label="Actual", color="tab:red")  # regression line
plot(xtest, μtest, label="Predicted mean", color="tab:blue")  # regression line
fill_between(xtest, μtest+σtest, μtest-σtest, alpha=0.5, color="tab:gray", label="Predicted std.")
for i in 1:num_sampling
    plot(xtest, sampled_func[:, i], alpha=0.3, color="tab:green")
end
xlabel("x"); ylabel("y"); legend()
xlim(-0.1, 1.1); tight_layout()
\end{lstlisting}
\begin{figure}[ht]
	\centering
	\includegraphics[scale=0.8, max width=\linewidth]{./fig/bayesian-brain/bayesian-linear-regression/cell005.png}
	\caption{cell005.png}
	\label{cell005.png}
\end{figure}
