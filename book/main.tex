\documentclass[titlepage]{ltjsbook}
\usepackage[
  paperheight=232truemm, paperwidth=182truemm,
  top=20truemm, bottom=15truemm, inner=15truemm, outer=15truemm
  ]{geometry}

%\documentclass[tombow, paper={182truemm, 232truemm}, titlepage]{ltjsbook}
\usepackage{amsmath}
\usepackage{amsfonts}
\usepackage{amssymb}
\usepackage{mathtools}

\usepackage{textgreek}
\usepackage[luatex]{graphicx} 
\usepackage[svgnames]{xcolor}
\usepackage{sty/julia-syntax-highlighting} % 
\usepackage{sty/indexing} % 

\usepackage[export]{sty/adjustbox} % added

\usepackage{fancyhdr}
\pagestyle{fancy}
\usepackage{hyperref}

\newcommand{\jl}{\lstinline[language=julia]}

\title{\Huge \textbf{Juliaで学ぶ計算論的神経科学}}
\author{\huge 山本 拓都}
\date{\huge \today} 
\begin{document}
\maketitle

\setcounter{tocdepth}{2}
\tableofcontents
\clearpage
\chapter{はじめに}
% \section{予測符号化}
\subsection{観測世界の階層的予測}
\textbf{階層的予測符号化(hierarchical predictive coding; HPC)} は\cite{Rao1999-zv}により導入された.構築するネットワークは入力層を含め,3層のネットワークとする.LGNへの入力として画像 $\mathbf{x} \in \mathbb{R}^{n_0}$を考える.画像 $\mathbf{x}$ の観測世界における隠れ変数,すなわち\textbf{潜在変数} (latent variable)を$\mathbf{r} \in \mathbb{R}^{n_1}$とし,ニューロン群によって発火率で表現されているとする (真の変数と $\mathbf{r}$は異なるので文字を分けるべきだが簡単のためにこう表す).このとき,


\mathbf{x} = f(\mathbf{U}\mathbf{r}) + \boldsymbol{\epsilon}


が成立しているとする.ただし,$f(\cdot)$は活性化関数 (activation function),$\mathbf{U} \in \mathbb{R}^{n_0 \times n_1}$は重み行列である.
$\boldsymbol{\epsilon} \in \mathbb{R}^{n_0}$ は $\mathcal{N}(\mathbf{0}, \sigma^2 \mathbf{I})$ からサンプリングされるとする.

潜在変数 $\mathbf{r}$はさらに高次 (higher-level)の潜在変数 $\mathbf{r}^h$により,次式で表現される.


\mathbf{r} = \mathbf{r}^{td}+\boldsymbol{\epsilon}^{td}=f(\mathbf{U}^h \mathbf{r}^h)+\boldsymbol{\epsilon}^{td}


ただし,Top-downの予測信号を $\mathbf{r}^{td}:=f(\mathbf{U}^h \mathbf{r}^h)$とした.また,$\mathbf{r}^{td} \in \mathbb{R}^{n_1}$, $\mathbf{r}^{h} \in \mathbb{R}^{n_2}$, $\mathbf{U}^h \in \mathbb{R}^{n_1 \times n_2}$ である.
$\boldsymbol{\epsilon}^{td} \in \mathbb{R}^{n_1}$は$\mathcal{N}(\mathbf{0}$, $\sigma_{td}^2 \mathbf{I}$) からサンプリングされるとする.

話は飛ぶが,Predictive codingのネットワークの特徴は
\begin{itemize}
\item 階層的な構造
\item 高次による低次の予測 (Feedback or Top-down信号)
\item 低次から高次への誤差信号の伝搬 (Feedforward or Bottom-up 信号)
\end{itemize}

である.ここまでは高次表現による低次表現の予測,というFeedback信号について説明してきたが,この部分はSparse codingでも同じである.それではPredictive codingのもう一つの要となる,低次から高次への予測誤差の伝搬というFeedforward信号はどのように導かれるのだろうか.結論から言えば,これは\textbf{復元誤差 (reconstruction error)の最小化を行う再帰的ネットワーク (recurrent network)を考慮することで自然に導かれる}.

 % ok
% \section{予測符号化}
\subsection{観測世界の階層的予測}
\textbf{階層的予測符号化(hierarchical predictive coding; HPC)} は\cite{Rao1999-zv}により導入された.構築するネットワークは入力層を含め,3層のネットワークとする.LGNへの入力として画像 $\mathbf{x} \in \mathbb{R}^{n_0}$を考える.画像 $\mathbf{x}$ の観測世界における隠れ変数,すなわち\textbf{潜在変数} (latent variable)を$\mathbf{r} \in \mathbb{R}^{n_1}$とし,ニューロン群によって発火率で表現されているとする (真の変数と $\mathbf{r}$は異なるので文字を分けるべきだが簡単のためにこう表す).このとき,


\mathbf{x} = f(\mathbf{U}\mathbf{r}) + \boldsymbol{\epsilon}


が成立しているとする.ただし,$f(\cdot)$は活性化関数 (activation function),$\mathbf{U} \in \mathbb{R}^{n_0 \times n_1}$は重み行列である.
$\boldsymbol{\epsilon} \in \mathbb{R}^{n_0}$ は $\mathcal{N}(\mathbf{0}, \sigma^2 \mathbf{I})$ からサンプリングされるとする.

潜在変数 $\mathbf{r}$はさらに高次 (higher-level)の潜在変数 $\mathbf{r}^h$により,次式で表現される.


\mathbf{r} = \mathbf{r}^{td}+\boldsymbol{\epsilon}^{td}=f(\mathbf{U}^h \mathbf{r}^h)+\boldsymbol{\epsilon}^{td}


ただし,Top-downの予測信号を $\mathbf{r}^{td}:=f(\mathbf{U}^h \mathbf{r}^h)$とした.また,$\mathbf{r}^{td} \in \mathbb{R}^{n_1}$, $\mathbf{r}^{h} \in \mathbb{R}^{n_2}$, $\mathbf{U}^h \in \mathbb{R}^{n_1 \times n_2}$ である.
$\boldsymbol{\epsilon}^{td} \in \mathbb{R}^{n_1}$は$\mathcal{N}(\mathbf{0}$, $\sigma_{td}^2 \mathbf{I}$) からサンプリングされるとする.

話は飛ぶが,Predictive codingのネットワークの特徴は
\begin{itemize}
\item 階層的な構造
\item 高次による低次の予測 (Feedback or Top-down信号)
\item 低次から高次への誤差信号の伝搬 (Feedforward or Bottom-up 信号)
\end{itemize}

である.ここまでは高次表現による低次表現の予測,というFeedback信号について説明してきたが,この部分はSparse codingでも同じである.それではPredictive codingのもう一つの要となる,低次から高次への予測誤差の伝搬というFeedforward信号はどのように導かれるのだろうか.結論から言えば,これは\textbf{復元誤差 (reconstruction error)の最小化を行う再帰的ネットワーク (recurrent network)を考慮することで自然に導かれる}.

 % ok
% \section{Julia言語の基本構文}
\subsection{変数}
aaa
\lstinputlisting[language=julia]{./text/introduction/usage-julia-lang/001.jl}
|演算子|説明|使用例|結果|
|:-|:-|:-|:-|
|\jl{+}|和|aaa|aaa|
|\jl{-}|差|aaa|aaa|
|\jl{*}|積|aaa|aaa|
|\jl{.*}|配列の要素積|aaa|aaa|
|\jl{/}|除算,右から逆行列をかける|aaa|aaa|
|\jl{\}|左から逆行列をかける|aaa|aaa|
\jl{var}を用いることで,任意の文字列を変数にすることができる.
\lstinputlisting[language=julia]{./text/introduction/usage-julia-lang/004.jl}
Juliaの余りの関数は \jl{rem(x, y)} と \jl{mod(x, y)}がある.Juliaの\jl{x % y}は\jl{rem}と同じだが,Pythonの場合は\jl{mod}と同じなので注意.
\lstinputlisting[language=julia]{./text/introduction/usage-julia-lang/006.jl}
\subsection{for loop}
aaa
\lstinputlisting[language=julia]{./text/introduction/usage-julia-lang/008.jl}
\subsection{関数名の!記号}
単なる\textbf{\index{かんしゅう@慣習}}として関数への入力を変更する場合に!を付ける.

関数内で配列を変更する場合には注意が必要である.以下に入力された配列を同じサイズの要素1の配列で置き換える,ということを目的として書かれた2つの関数がある.違いは\jl{v}の後に\jl{[:]}としているかどうかである.
\lstinputlisting[language=julia]{./text/introduction/usage-julia-lang/010.jl}
実行すると\jl{wrong!}の場合には入力された配列が変更されていないことがわかる.
\lstinputlisting[language=julia]{./text/introduction/usage-julia-lang/012.jl}
\subsection{broadcastingの回避}
aaa
\lstinputlisting[language=julia]{./text/introduction/usage-julia-lang/014.jl}
\lstinputlisting[language=julia]{./text/introduction/usage-julia-lang/015.jl}
 % ok
% \section{予測符号化}
\subsection{観測世界の階層的予測}
\textbf{階層的予測符号化(hierarchical predictive coding; HPC)} は\cite{Rao1999-zv}により導入された.構築するネットワークは入力層を含め,3層のネットワークとする.LGNへの入力として画像 $\mathbf{x} \in \mathbb{R}^{n_0}$を考える.画像 $\mathbf{x}$ の観測世界における隠れ変数,すなわち\textbf{潜在変数} (latent variable)を$\mathbf{r} \in \mathbb{R}^{n_1}$とし,ニューロン群によって発火率で表現されているとする (真の変数と $\mathbf{r}$は異なるので文字を分けるべきだが簡単のためにこう表す).このとき,


\mathbf{x} = f(\mathbf{U}\mathbf{r}) + \boldsymbol{\epsilon}


が成立しているとする.ただし,$f(\cdot)$は活性化関数 (activation function),$\mathbf{U} \in \mathbb{R}^{n_0 \times n_1}$は重み行列である.
$\boldsymbol{\epsilon} \in \mathbb{R}^{n_0}$ は $\mathcal{N}(\mathbf{0}, \sigma^2 \mathbf{I})$ からサンプリングされるとする.

潜在変数 $\mathbf{r}$はさらに高次 (higher-level)の潜在変数 $\mathbf{r}^h$により,次式で表現される.


\mathbf{r} = \mathbf{r}^{td}+\boldsymbol{\epsilon}^{td}=f(\mathbf{U}^h \mathbf{r}^h)+\boldsymbol{\epsilon}^{td}


ただし,Top-downの予測信号を $\mathbf{r}^{td}:=f(\mathbf{U}^h \mathbf{r}^h)$とした.また,$\mathbf{r}^{td} \in \mathbb{R}^{n_1}$, $\mathbf{r}^{h} \in \mathbb{R}^{n_2}$, $\mathbf{U}^h \in \mathbb{R}^{n_1 \times n_2}$ である.
$\boldsymbol{\epsilon}^{td} \in \mathbb{R}^{n_1}$は$\mathcal{N}(\mathbf{0}$, $\sigma_{td}^2 \mathbf{I}$) からサンプリングされるとする.

話は飛ぶが,Predictive codingのネットワークの特徴は
\begin{itemize}
\item 階層的な構造
\item 高次による低次の予測 (Feedback or Top-down信号)
\item 低次から高次への誤差信号の伝搬 (Feedforward or Bottom-up 信号)
\end{itemize}

である.ここまでは高次表現による低次表現の予測,というFeedback信号について説明してきたが,この部分はSparse codingでも同じである.それではPredictive codingのもう一つの要となる,低次から高次への予測誤差の伝搬というFeedforward信号はどのように導かれるのだろうか.結論から言えば,これは\textbf{復元誤差 (reconstruction error)の最小化を行う再帰的ネットワーク (recurrent network)を考慮することで自然に導かれる}.

\lstinputlisting[language=julia]{./text/introduction/linear-algebra/001.jl}
\subsubsection{事前分布の設定}
事前分布$p(\mathbf{r})$としては,0においてピークがあり,裾の重い(heavy tail)を持つsparse distributionあるいは \textbf{super-Gaussian distribution} (Laplace 分布やCauchy分布などGaussian分布よりもkurtoticな分布)を用いるのが良い.このような分布では,$\mathbf{r}$の各要素$r_i$はほとんど0に等しく,ある入力に対しては大きな値を取る.$p(\mathbf{r})$は一般化して式(4), (5)のように表記する.


\begin{aligned}
p(\mathbf{r})&=\prod_j p(r_j)\\
p(r_j)&=\frac{1}{Z_{\beta}}\exp \left[-\beta S(r_j)\right]
\end{aligned}


ただし,$\beta$は逆温度(inverse temperature), $Z_{\beta}$は規格化定数 (分配関数) である.これらの用語は統計力学における正準分布 (ボルツマン分布)から来ている.$S(x)$と分布の関係をまとめた表が以下となる (cf. \url{https://pdfs.semanticscholar.org/be08/da912362bf40fe3ded78bdadc644f921b4e7.pdf}).

\lstinputlisting[language=julia]{./text/introduction/linear-algebra/003.jl}
2種類の指数関数型シナプスの動態.破線は単一指数関数型シナプスで, 実線は二重指数関数型シナプスである.
\lstinputlisting[language=julia]{./text/introduction/linear-algebra/005.jl}
いくつかの処理について解説しておく.まず,一番目のforループ内の\jl{v[i]}の\jl{((dt*tcount) > (tlast[i] + tref))}は最後にニューロンが発火した時刻\jl{tlast[i]}に不応期\jl{tref}を足した時刻よりも現在の時刻\jl{dt*tcount[1]}が大きければ膜電位の更新を許可し,小さければ更新しない.二番目のforループにおける\jl{fire[i]}はニューロンの膜電位が閾値電位\jl{vthr}を超えたら\jl{True}となる.\jl{v[i]}などの更新式にある\jl{ifelse(a, b, c)}はaが\jl{True}の時はbを返し,\jl{False}の時はcを返す関数であり,\jl{v[i] = ifelse(fire[i], vreset, v[i])}は\jl{fire[i]}が\jl{True}なら\jl{v[i]}をリセット電位\jl{vreset}とし,そうでなければそのままの値を返すという処理である.同様にして\jl{tlast[i]}は発火したときにその時刻を記録する変数となっている.なお,\jl{v_[i] = ifelse(fire[i], vpeak, v[i])}は実際のシミュレーションにおいて意味をなさない.単に発火時の電位\jl{vpeak}を含めて記録すると描画時の見栄えが良いというだけである.

これらの\jl{struct}と関数を用いてシミュレーションを実行する.\jl{I} はHHモデルのときと同じように矩形波を入力する.実は\jl{I}は入力電流ではなく入力電流に比例する量となっているが,これは膜抵抗を乗じた後の値であると考えるとよい.

\lstinputlisting[language=julia]{./text/introduction/linear-algebra/007.jl}
\lstinputlisting[language=julia]{./text/introduction/linear-algebra/008.jl}
\lstinputlisting[language=julia]{./text/introduction/linear-algebra/009.jl}
\subsection{ランジュバン・モンテカルロ法 (LMC)}拡散過程
$$
{\frac{d\theta}{dt}}=\nabla \log p (\theta)+{\sqrt 2}{d{W}}
$$
Euler–Maruyama法により,
\lstinputlisting[language=julia]{./text/introduction/linear-algebra/011.jl}
\lstinputlisting[language=julia]{./text/introduction/linear-algebra/012.jl}
損失関数を定義する.

\lstinputlisting[language=julia]{./text/introduction/linear-algebra/014.jl}
\lstinputlisting[language=julia]{./text/introduction/linear-algebra/015.jl}
50msから200msまでで11回, 250msから400msまでで16回発火しているので発火回数は計27回であり,この結果は正しい.
\lstinputlisting[language=julia]{./text/introduction/linear-algebra/017.jl}
入力は64(網膜座標系での位置)+2(眼球位置信号)=66とする.眼球位置信号は原著ではmonotonic形式による32(=8ユニット×2(x, y方向)×2 (傾き正負))ユニットで構成されるが,簡単のために眼球位置信号も$x, y$の2次元とする.視覚刺激は-40度から40度までの範囲であり,10度で離散化する.よって,網膜座標系での位置は$8\times 8$の行列で表現される.位置は2次元のGaussianで表現する.ただし,1/e幅 (ピークから1/eに減弱する幅) は15度である.$1/e$の代わりに$1/2$とすれば半値全幅(FWHM)となる.スポットサイズを$w$,Gaussianを$G(x)$とすると.$G(x+w/2)=G/e$より,$\sigma=\frac{\sqrt{2}w}{4}$と求まる.

\lstinputlisting[language=julia]{./text/introduction/linear-algebra/019.jl}
\lstinputlisting[language=julia]{./text/introduction/linear-algebra/020.jl}
\lstinputlisting[language=julia]{./text/introduction/linear-algebra/021.jl}
\subsubsection{非負主成分分析によるグリッドパターンの創発}
内側嗅内皮質(MEC)にある\textbf{グリッド細胞 (grid cells)} は六角形格子状の発火パターンにより自己位置等を符号化するのに貢献している.この発火パターンを生み出すモデルは多数あるが,\textbf{場所細胞(place cells)} の発火パターンを\textbf{非負主成分分析(nonnegative principal component analysis)} で次元削減するとグリッド細胞のパターンが生まれるというモデルがある \cite{Dordek2016-ff}.非線形Hebb学習を用いてこのモデルを実装しよう.なお,同様のことは\textbf{非負値行列因子分解 (NMF: nonnegative matrix factorization)} でも可能である.

\lstinputlisting[language=julia]{./text/introduction/linear-algebra/023.jl}
\lstinputlisting[language=julia]{./text/introduction/linear-algebra/024.jl}
損失の変化を描画する.

\lstinputlisting[language=julia]{./text/introduction/linear-algebra/026.jl}
\lstinputlisting[language=julia]{./text/introduction/linear-algebra/027.jl}
\subsubsection{対角行列}

aaa

\subsubsection{線形行列方程式}


\mathbf{A}\mathbf{x}=\mathbf{b}


$\mathbf{A}$が正則の場合,逆行列が存在し,


\mathbf{x}=\mathbf{A}^{-1}\mathbf{b}

\lstinputlisting[language=julia]{./text/introduction/linear-algebra/030.jl}
\lstinputlisting[language=julia]{./text/introduction/linear-algebra/031.jl}
Juliaではバックスラッシュ演算子 \jl{\}を用いることで明示的に逆行列を計算せずに解を求めることができる.

\lstinputlisting[language=julia]{./text/introduction/linear-algebra/033.jl}
\subsubsection{sign関数を用いたDistributional RLと分位点回帰}それでは,なぜ予測価値 $V_i$は$\tau_i$ 分位点に収束するのでしょうか.Extended Data Fig.1のように平衡点で考えてもよいのですが,後のために分位点回帰との関連について説明します.
実はDistributional RL (かつ,RPEの応答関数にsign関数を用いた場合)における予測報酬 $V_i$の更新式は,分位点回帰(Quantile
regression)を勾配法で行うときの更新式とほとんど同じです.分位点回帰では$\delta$の関数$\rho_{\tau}(\delta)$を次のように定義します. $$ \rho_{\tau}(\delta)=\left|\tau-\mathbb{I}_{\delta \leq 0}\right|\cdot |\delta|=\left(\tau-\mathbb{I}_{\delta
\leq 0}\right)\cdot \delta $$ そして,この関数を最小化することで回帰を行います.ここで$\tau$は分位点です.また$\delta=r-V$としておきます.今回,どんな行動をしても未来の報酬に影響はないので$\gamma=0$としています.\url{br/}
\url{br/}
ここで, $$ \frac{\partial \rho_{\tau}(\delta)}{\partial \delta}=\rho_{\tau}^{\prime}(\delta)=\left|\tau-\mathbb{I}_{\delta \leq 0}\right| \cdot \operatorname{sign}(\delta) $$ なので,$r$を観測値とすると, $$
\frac{\partial \rho_{\tau}(\delta)}{\partial V}=\frac{\partial \rho_{\tau}(\delta)}{\partial \delta}\frac{\partial \delta(V)}{\partial V}=-\left|\tau-\mathbb{I}_{\delta \leq 0}\right| \cdot
\operatorname{sign}(\delta) $$ となります.ゆえに$V$の更新式は $$ V \leftarrow V - \beta\cdot\frac{\partial \rho_{\tau}(\delta)}{\partial V}=V+\beta \left|\tau-\mathbb{I}_{\delta \leq 0}\right| \cdot
\operatorname{sign}(\delta) $$ です.ただし,$\beta$はベースラインの学習率です.個々の$V_i$について考え,符号で場合分けをすると
$$ \begin{cases} V_{i} \leftarrow V_{i}+\beta\cdot |\tau_i|\cdot\operatorname{sign}\left(\delta_{i}\right)
&\text { for } \delta_{i}>0\\ V_{i} \leftarrow V_{i}+\beta\cdot |\tau_i-1|\cdot\operatorname{sign}\left(\delta_{i}\right) &\text { for } \delta_{i} \leq 0 \end{cases} $$ となります.$0 \leq
\tau_i \leq 1$であり,$\tau_i=\alpha_{i}^{+} / \left(\alpha_{i}^{+} + \alpha_{i}^{-}\right)$であることに注意すると上式は次のように書けます. $$ \begin{cases} V_{i} \leftarrow V_{i}+\beta\cdot
\frac{\alpha_{i}^{+}}{\alpha_{i}^{+}+\alpha_{i}^{-}}\cdot\operatorname{sign}\left(\delta_{i}\right) &\text { for } \delta_{i}>0\\ V_{i} \leftarrow V_{i}+\beta\cdot
\frac{\alpha_{i}^{-}}{\alpha_{i}^{+}+\alpha_{i}^{-}}\cdot\operatorname{sign}\left(\delta_{i}\right) &\text { for } \delta_{i} \leq 0 \end{cases} $$ これは前節で述べたDistributional
RLの更新式とほぼ同じです.いくつか違う点もありますが,RPEが正の場合と負の場合に更新される値の比は同じとなっています.
このようにRPEの応答関数にsign関数を用いた場合,報酬分布を上手く符号化することができます.しかし実際のドパミンニューロンはsign関数のような生理的に妥当でない応答はせず,RPEの大きさに応じた活動をします.そこで次節ではRPEの応答関数を線形にしたときの話をします.
\lstinputlisting[language=julia]{./text/introduction/linear-algebra/035.jl}
\lstinputlisting[language=julia]{./text/introduction/linear-algebra/036.jl}
\lstinputlisting[language=julia]{./text/introduction/linear-algebra/037.jl}
\lstinputlisting[language=julia]{./text/introduction/linear-algebra/038.jl}
非負PCAの場合

\lstinputlisting[language=julia]{./text/introduction/linear-algebra/040.jl}
$z$の推定過程を描画する.また,$z$を除いた$\mathbf{u}$を平均化し,自己相関の度合いを確認する.
\lstinputlisting[language=julia]{./text/introduction/linear-algebra/042.jl}
Hamiltonianネットワークは自己相関を振動により低下させることで,効率の良いサンプリングを実現している.ToDo: 普通にMCMCやる場合も自己相関は確認したほうがいいという話をどこかに書く.

推定された事後分布を特定の神経細胞のペアについて確認する.

\lstinputlisting[language=julia]{./text/introduction/linear-algebra/044.jl}
\subsubsection{reshapeにおける残りの次元の指定}numpyにおいては(2, 3, 5)次元の配列に対し,reshape(-1, 5)を行うと(6, 5)次元の配列となった.これと同様なことは,Juliaでは:を使うことで実装できる.
\lstinputlisting[language=julia]{./text/introduction/linear-algebra/046.jl}
上段が入力画像,下段が再構成された画像である.差異はあるものの,概ね再構成されていることがわかる.

\lstinputlisting[language=julia]{./text/introduction/linear-algebra/048.jl}
\lstinputlisting[language=julia]{./text/introduction/linear-algebra/049.jl}
}{admonition} 論文以外の参考資料
- \url{http://www.scholarpedia.org/article/Sparse_coding}
- Bruno Olshausen: “Sparse coding in brains and machines”([Stanford talks](https://talks.stanford.edu/bruno-olshausen-sparse-coding-in-brains-and-machines/)), [Slide](http://www.rctn.org/bruno/public/Simons-sparse-coding.pdf)
- \url{https://redwood.berkeley.edu/wp-content/uploads/2018/08/sparse-coding-ICA.pdf}
- \url{https://redwood.berkeley.edu/wp-content/uploads/2018/08/sparse-coding-LCA.pdf}
- \url{https://redwood.berkeley.edu/wp-content/uploads/2018/08/Dylan-lca_overcompleteness_09-27-2018.pdf}
}
\lstinputlisting[language=julia]{./text/introduction/linear-algebra/051.jl}
\lstinputlisting[language=julia]{./text/introduction/linear-algebra/052.jl}
 % ok
% \section{予測符号化}
\subsection{観測世界の階層的予測}
\textbf{階層的予測符号化(hierarchical predictive coding; HPC)} は\cite{Rao1999-zv}により導入された.構築するネットワークは入力層を含め,3層のネットワークとする.LGNへの入力として画像 $\mathbf{x} \in \mathbb{R}^{n_0}$を考える.画像 $\mathbf{x}$ の観測世界における隠れ変数,すなわち\textbf{潜在変数} (latent variable)を$\mathbf{r} \in \mathbb{R}^{n_1}$とし,ニューロン群によって発火率で表現されているとする (真の変数と $\mathbf{r}$は異なるので文字を分けるべきだが簡単のためにこう表す).このとき,


\mathbf{x} = f(\mathbf{U}\mathbf{r}) + \boldsymbol{\epsilon}


が成立しているとする.ただし,$f(\cdot)$は活性化関数 (activation function),$\mathbf{U} \in \mathbb{R}^{n_0 \times n_1}$は重み行列である.
$\boldsymbol{\epsilon} \in \mathbb{R}^{n_0}$ は $\mathcal{N}(\mathbf{0}, \sigma^2 \mathbf{I})$ からサンプリングされるとする.

潜在変数 $\mathbf{r}$はさらに高次 (higher-level)の潜在変数 $\mathbf{r}^h$により,次式で表現される.


\mathbf{r} = \mathbf{r}^{td}+\boldsymbol{\epsilon}^{td}=f(\mathbf{U}^h \mathbf{r}^h)+\boldsymbol{\epsilon}^{td}


ただし,Top-downの予測信号を $\mathbf{r}^{td}:=f(\mathbf{U}^h \mathbf{r}^h)$とした.また,$\mathbf{r}^{td} \in \mathbb{R}^{n_1}$, $\mathbf{r}^{h} \in \mathbb{R}^{n_2}$, $\mathbf{U}^h \in \mathbb{R}^{n_1 \times n_2}$ である.
$\boldsymbol{\epsilon}^{td} \in \mathbb{R}^{n_1}$は$\mathcal{N}(\mathbf{0}$, $\sigma_{td}^2 \mathbf{I}$) からサンプリングされるとする.

話は飛ぶが,Predictive codingのネットワークの特徴は
\begin{itemize}
\item 階層的な構造
\item 高次による低次の予測 (Feedback or Top-down信号)
\item 低次から高次への誤差信号の伝搬 (Feedforward or Bottom-up 信号)
\end{itemize}

である.ここまでは高次表現による低次表現の予測,というFeedback信号について説明してきたが,この部分はSparse codingでも同じである.それではPredictive codingのもう一つの要となる,低次から高次への予測誤差の伝搬というFeedforward信号はどのように導かれるのだろうか.結論から言えば,これは\textbf{復元誤差 (reconstruction error)の最小化を行う再帰的ネットワーク (recurrent network)を考慮することで自然に導かれる}.

\subsubsection{重み行列$\mathbf{A}$の作成}

\subsubsection{事前分布の設定}
事前分布$p(\mathbf{r})$としては,0においてピークがあり,裾の重い(heavy tail)を持つsparse distributionあるいは \textbf{super-Gaussian distribution} (Laplace 分布やCauchy分布などGaussian分布よりもkurtoticな分布)を用いるのが良い.このような分布では,$\mathbf{r}$の各要素$r_i$はほとんど0に等しく,ある入力に対しては大きな値を取る.$p(\mathbf{r})$は一般化して式(4), (5)のように表記する.


\begin{aligned}
p(\mathbf{r})&=\prod_j p(r_j)\\
p(r_j)&=\frac{1}{Z_{\beta}}\exp \left[-\beta S(r_j)\right]
\end{aligned}


ただし,$\beta$は逆温度(inverse temperature), $Z_{\beta}$は規格化定数 (分配関数) である.これらの用語は統計力学における正準分布 (ボルツマン分布)から来ている.$S(x)$と分布の関係をまとめた表が以下となる (cf. \url{https://pdfs.semanticscholar.org/be08/da912362bf40fe3ded78bdadc644f921b4e7.pdf}).

UNDERSTANDING STRAIGHT-THROUGH ESTIMATOR IN TRAINING ACTIVATION QUANTIZED NEURAL NETS

Yoshua Bengio, Nicholas L´eonard, and Aaron Courville. Estimating or propagating gradients through stochastic neurons for conditional computation. arXiv preprint arXiv:1308.3432, 2013.

2種類の指数関数型シナプスの動態.破線は単一指数関数型シナプスで, 実線は二重指数関数型シナプスである.
変更しない定数を保持する \jl{struct} の \jl{FHNParameter} と, 変数を保持する \jl{mutable struct} の \jl{FHN} を作成する.
\lstinputlisting[language=julia]{./text/introduction/differential-equation/006.jl}
 % ok
% \section{予測符号化}
\subsection{観測世界の階層的予測}
\textbf{階層的予測符号化(hierarchical predictive coding; HPC)} は\cite{Rao1999-zv}により導入された.構築するネットワークは入力層を含め,3層のネットワークとする.LGNへの入力として画像 $\mathbf{x} \in \mathbb{R}^{n_0}$を考える.画像 $\mathbf{x}$ の観測世界における隠れ変数,すなわち\textbf{潜在変数} (latent variable)を$\mathbf{r} \in \mathbb{R}^{n_1}$とし,ニューロン群によって発火率で表現されているとする (真の変数と $\mathbf{r}$は異なるので文字を分けるべきだが簡単のためにこう表す).このとき,


\mathbf{x} = f(\mathbf{U}\mathbf{r}) + \boldsymbol{\epsilon}


が成立しているとする.ただし,$f(\cdot)$は活性化関数 (activation function),$\mathbf{U} \in \mathbb{R}^{n_0 \times n_1}$は重み行列である.
$\boldsymbol{\epsilon} \in \mathbb{R}^{n_0}$ は $\mathcal{N}(\mathbf{0}, \sigma^2 \mathbf{I})$ からサンプリングされるとする.

潜在変数 $\mathbf{r}$はさらに高次 (higher-level)の潜在変数 $\mathbf{r}^h$により,次式で表現される.


\mathbf{r} = \mathbf{r}^{td}+\boldsymbol{\epsilon}^{td}=f(\mathbf{U}^h \mathbf{r}^h)+\boldsymbol{\epsilon}^{td}


ただし,Top-downの予測信号を $\mathbf{r}^{td}:=f(\mathbf{U}^h \mathbf{r}^h)$とした.また,$\mathbf{r}^{td} \in \mathbb{R}^{n_1}$, $\mathbf{r}^{h} \in \mathbb{R}^{n_2}$, $\mathbf{U}^h \in \mathbb{R}^{n_1 \times n_2}$ である.
$\boldsymbol{\epsilon}^{td} \in \mathbb{R}^{n_1}$は$\mathcal{N}(\mathbf{0}$, $\sigma_{td}^2 \mathbf{I}$) からサンプリングされるとする.

話は飛ぶが,Predictive codingのネットワークの特徴は
\begin{itemize}
\item 階層的な構造
\item 高次による低次の予測 (Feedback or Top-down信号)
\item 低次から高次への誤差信号の伝搬 (Feedforward or Bottom-up 信号)
\end{itemize}

である.ここまでは高次表現による低次表現の予測,というFeedback信号について説明してきたが,この部分はSparse codingでも同じである.それではPredictive codingのもう一つの要となる,低次から高次への予測誤差の伝搬というFeedforward信号はどのように導かれるのだろうか.結論から言えば,これは\textbf{復元誤差 (reconstruction error)の最小化を行う再帰的ネットワーク (recurrent network)を考慮することで自然に導かれる}.

\subsubsection{重み行列$\mathbf{A}$の作成}

\subsubsection{事前分布の設定}
事前分布$p(\mathbf{r})$としては,0においてピークがあり,裾の重い(heavy tail)を持つsparse distributionあるいは \textbf{super-Gaussian distribution} (Laplace 分布やCauchy分布などGaussian分布よりもkurtoticな分布)を用いるのが良い.このような分布では,$\mathbf{r}$の各要素$r_i$はほとんど0に等しく,ある入力に対しては大きな値を取る.$p(\mathbf{r})$は一般化して式(4), (5)のように表記する.


\begin{aligned}
p(\mathbf{r})&=\prod_j p(r_j)\\
p(r_j)&=\frac{1}{Z_{\beta}}\exp \left[-\beta S(r_j)\right]
\end{aligned}


ただし,$\beta$は逆温度(inverse temperature), $Z_{\beta}$は規格化定数 (分配関数) である.これらの用語は統計力学における正準分布 (ボルツマン分布)から来ている.$S(x)$と分布の関係をまとめた表が以下となる (cf. \url{https://pdfs.semanticscholar.org/be08/da912362bf40fe3ded78bdadc644f921b4e7.pdf}).

UNDERSTANDING STRAIGHT-THROUGH ESTIMATOR IN TRAINING ACTIVATION QUANTIZED NEURAL NETS

Yoshua Bengio, Nicholas L´eonard, and Aaron Courville. Estimating or propagating gradients through stochastic neurons for conditional computation. arXiv preprint arXiv:1308.3432, 2013.

\lstinputlisting[language=julia]{./text/introduction/linear-regression/004.jl}
\lstinputlisting[language=julia]{./text/introduction/linear-regression/005.jl}
\lstinputlisting[language=julia]{./text/introduction/linear-regression/006.jl}
\lstinputlisting[language=julia]{./text/introduction/linear-regression/007.jl}
\lstinputlisting[language=julia]{./text/introduction/linear-regression/008.jl}
\begin{figure}[ht]
	\centering
	\includegraphics[scale=0.8, max width=\linewidth]{./fig/appendix/slow-feature-analysis/cell008.png}
	\caption{cell008.png}
	\label{cell008.png}
\end{figure}
\subsubsection{正規方程式を用いた推定}条件に基づいて目的関数$L(\mathbf{\theta})$を微分すると次のような方程式が得られる.
$$
\mathbf{X}^\top\mathbf{X}\mathbf{\hat\theta}=\mathbf{X}^\top\mathbf{y}
$$
これを\textbf{正規方程式} (normal equation)と呼ぶ.この正規方程式より、係数の推定値は$\mathbf{\hat\theta}={(\mathbf{X}^\top\mathbf{X})}^{-1}X^\top\mathbf{y}$という式で得られる.なお,正規方程式自体は$\mathbf{y}=\mathbf{X}\mathbf{\theta}$の左から$\mathbf{X}^\top$をかける,と覚えると良い.
\lstinputlisting[language=julia]{./text/introduction/linear-regression/010.jl}
\lstinputlisting[language=julia]{./text/introduction/linear-regression/011.jl}
\begin{figure}[ht]
	\centering
	\includegraphics[scale=0.8, max width=\linewidth]{./fig/local-learning-rule/pca-hebbian-learning/cell011.png}
	\caption{cell011.png}
	\label{cell011.png}
\end{figure}
 % ok
% \section{予測符号化}
\subsection{観測世界の階層的予測}
\textbf{階層的予測符号化(hierarchical predictive coding; HPC)} は\cite{Rao1999-zv}により導入された.構築するネットワークは入力層を含め,3層のネットワークとする.LGNへの入力として画像 $\mathbf{x} \in \mathbb{R}^{n_0}$を考える.画像 $\mathbf{x}$ の観測世界における隠れ変数,すなわち\textbf{潜在変数} (latent variable)を$\mathbf{r} \in \mathbb{R}^{n_1}$とし,ニューロン群によって発火率で表現されているとする (真の変数と $\mathbf{r}$は異なるので文字を分けるべきだが簡単のためにこう表す).このとき,


\mathbf{x} = f(\mathbf{U}\mathbf{r}) + \boldsymbol{\epsilon}


が成立しているとする.ただし,$f(\cdot)$は活性化関数 (activation function),$\mathbf{U} \in \mathbb{R}^{n_0 \times n_1}$は重み行列である.
$\boldsymbol{\epsilon} \in \mathbb{R}^{n_0}$ は $\mathcal{N}(\mathbf{0}, \sigma^2 \mathbf{I})$ からサンプリングされるとする.

潜在変数 $\mathbf{r}$はさらに高次 (higher-level)の潜在変数 $\mathbf{r}^h$により,次式で表現される.


\mathbf{r} = \mathbf{r}^{td}+\boldsymbol{\epsilon}^{td}=f(\mathbf{U}^h \mathbf{r}^h)+\boldsymbol{\epsilon}^{td}


ただし,Top-downの予測信号を $\mathbf{r}^{td}:=f(\mathbf{U}^h \mathbf{r}^h)$とした.また,$\mathbf{r}^{td} \in \mathbb{R}^{n_1}$, $\mathbf{r}^{h} \in \mathbb{R}^{n_2}$, $\mathbf{U}^h \in \mathbb{R}^{n_1 \times n_2}$ である.
$\boldsymbol{\epsilon}^{td} \in \mathbb{R}^{n_1}$は$\mathcal{N}(\mathbf{0}$, $\sigma_{td}^2 \mathbf{I}$) からサンプリングされるとする.

話は飛ぶが,Predictive codingのネットワークの特徴は
\begin{itemize}
\item 階層的な構造
\item 高次による低次の予測 (Feedback or Top-down信号)
\item 低次から高次への誤差信号の伝搬 (Feedforward or Bottom-up 信号)
\end{itemize}

である.ここまでは高次表現による低次表現の予測,というFeedback信号について説明してきたが,この部分はSparse codingでも同じである.それではPredictive codingのもう一つの要となる,低次から高次への予測誤差の伝搬というFeedforward信号はどのように導かれるのだろうか.結論から言えば,これは\textbf{復元誤差 (reconstruction error)の最小化を行う再帰的ネットワーク (recurrent network)を考慮することで自然に導かれる}.

\subsubsection{重み行列$\mathbf{A}$の作成}

 % ok
% \section{予測符号化}
\subsection{観測世界の階層的予測}
\textbf{階層的予測符号化(hierarchical predictive coding; HPC)} は\cite{Rao1999-zv}により導入された.構築するネットワークは入力層を含め,3層のネットワークとする.LGNへの入力として画像 $\mathbf{x} \in \mathbb{R}^{n_0}$を考える.画像 $\mathbf{x}$ の観測世界における隠れ変数,すなわち\textbf{潜在変数} (latent variable)を$\mathbf{r} \in \mathbb{R}^{n_1}$とし,ニューロン群によって発火率で表現されているとする (真の変数と $\mathbf{r}$は異なるので文字を分けるべきだが簡単のためにこう表す).このとき,


\mathbf{x} = f(\mathbf{U}\mathbf{r}) + \boldsymbol{\epsilon}


が成立しているとする.ただし,$f(\cdot)$は活性化関数 (activation function),$\mathbf{U} \in \mathbb{R}^{n_0 \times n_1}$は重み行列である.
$\boldsymbol{\epsilon} \in \mathbb{R}^{n_0}$ は $\mathcal{N}(\mathbf{0}, \sigma^2 \mathbf{I})$ からサンプリングされるとする.

潜在変数 $\mathbf{r}$はさらに高次 (higher-level)の潜在変数 $\mathbf{r}^h$により,次式で表現される.


\mathbf{r} = \mathbf{r}^{td}+\boldsymbol{\epsilon}^{td}=f(\mathbf{U}^h \mathbf{r}^h)+\boldsymbol{\epsilon}^{td}


ただし,Top-downの予測信号を $\mathbf{r}^{td}:=f(\mathbf{U}^h \mathbf{r}^h)$とした.また,$\mathbf{r}^{td} \in \mathbb{R}^{n_1}$, $\mathbf{r}^{h} \in \mathbb{R}^{n_2}$, $\mathbf{U}^h \in \mathbb{R}^{n_1 \times n_2}$ である.
$\boldsymbol{\epsilon}^{td} \in \mathbb{R}^{n_1}$は$\mathcal{N}(\mathbf{0}$, $\sigma_{td}^2 \mathbf{I}$) からサンプリングされるとする.

話は飛ぶが,Predictive codingのネットワークの特徴は
\begin{itemize}
\item 階層的な構造
\item 高次による低次の予測 (Feedback or Top-down信号)
\item 低次から高次への誤差信号の伝搬 (Feedforward or Bottom-up 信号)
\end{itemize}

である.ここまでは高次表現による低次表現の予測,というFeedback信号について説明してきたが,この部分はSparse codingでも同じである.それではPredictive codingのもう一つの要となる,低次から高次への予測誤差の伝搬というFeedforward信号はどのように導かれるのだろうか.結論から言えば,これは\textbf{復元誤差 (reconstruction error)の最小化を行う再帰的ネットワーク (recurrent network)を考慮することで自然に導かれる}.

 % ok

\chapter{神経細胞のモデル}
% \section{予測符号化}
\subsection{観測世界の階層的予測}
\textbf{階層的予測符号化(hierarchical predictive coding; HPC)} は\cite{Rao1999-zv}により導入された.構築するネットワークは入力層を含め,3層のネットワークとする.LGNへの入力として画像 $\mathbf{x} \in \mathbb{R}^{n_0}$を考える.画像 $\mathbf{x}$ の観測世界における隠れ変数,すなわち\textbf{潜在変数} (latent variable)を$\mathbf{r} \in \mathbb{R}^{n_1}$とし,ニューロン群によって発火率で表現されているとする (真の変数と $\mathbf{r}$は異なるので文字を分けるべきだが簡単のためにこう表す).このとき,


\mathbf{x} = f(\mathbf{U}\mathbf{r}) + \boldsymbol{\epsilon}


が成立しているとする.ただし,$f(\cdot)$は活性化関数 (activation function),$\mathbf{U} \in \mathbb{R}^{n_0 \times n_1}$は重み行列である.
$\boldsymbol{\epsilon} \in \mathbb{R}^{n_0}$ は $\mathcal{N}(\mathbf{0}, \sigma^2 \mathbf{I})$ からサンプリングされるとする.

潜在変数 $\mathbf{r}$はさらに高次 (higher-level)の潜在変数 $\mathbf{r}^h$により,次式で表現される.


\mathbf{r} = \mathbf{r}^{td}+\boldsymbol{\epsilon}^{td}=f(\mathbf{U}^h \mathbf{r}^h)+\boldsymbol{\epsilon}^{td}


ただし,Top-downの予測信号を $\mathbf{r}^{td}:=f(\mathbf{U}^h \mathbf{r}^h)$とした.また,$\mathbf{r}^{td} \in \mathbb{R}^{n_1}$, $\mathbf{r}^{h} \in \mathbb{R}^{n_2}$, $\mathbf{U}^h \in \mathbb{R}^{n_1 \times n_2}$ である.
$\boldsymbol{\epsilon}^{td} \in \mathbb{R}^{n_1}$は$\mathcal{N}(\mathbf{0}$, $\sigma_{td}^2 \mathbf{I}$) からサンプリングされるとする.

話は飛ぶが,Predictive codingのネットワークの特徴は
\begin{itemize}
\item 階層的な構造
\item 高次による低次の予測 (Feedback or Top-down信号)
\item 低次から高次への誤差信号の伝搬 (Feedforward or Bottom-up 信号)
\end{itemize}

である.ここまでは高次表現による低次表現の予測,というFeedback信号について説明してきたが,この部分はSparse codingでも同じである.それではPredictive codingのもう一つの要となる,低次から高次への予測誤差の伝搬というFeedforward信号はどのように導かれるのだろうか.結論から言えば,これは\textbf{復元誤差 (reconstruction error)の最小化を行う再帰的ネットワーク (recurrent network)を考慮することで自然に導かれる}.

 % ok
% \section{Hodgkin-Huxleyモデル}
\subsection{Hodgkin-Huxleyモデル}
\textbf{Hodgkin-Huxleyモデル}\index{Hodgkin-Huxleyもでる@Hodgkin-Huxleyモデル} (HH モデル)は, ニューロンの膜興奮を表現する,初めに導出された数理モデルである \citep{Hodgkin1952-gy}\footnote{\citep{Hopper2022-xj}はHodgkinおよびHuxleyの論文の図をカラー化して分かりやすくしたものである.}.HodgkinおよびHuxleyはヤリイカの巨大神経軸索に対して\textbf{電位固定法}\index{でんいこていほう@電位固定法}(voltage-clamp)を用いた実験を行い, 実験から得られた観測結果を元にモデルを構築した \citep{Schwiening2012-pi}.HHモデルには等価な電気回路モデルがあり, \textbf{膜の並列等価回路モデル}\index{まくのへいれつとうかかいろもでる@膜の並列等価回路モデル} (parallel conductance model)と呼ばれている.膜の並列等価回路モデルでは, ニューロンの細胞膜をコンデンサ, 細胞膜に埋まっているイオンチャネルを可変抵抗 (動的に変化する抵抗) として置き換える.

\textbf{イオンチャネル}\index{いおんちゃねる@イオンチャネル} (ion channel)は特定のイオン(例えばナトリウムイオンやカリウムイオンなど)を選択的に通す膜輸送体の一種である.それぞれのイオンの種類において, 異なるイオンチャネルがある (同じイオンでも複数の種類のイオンチャネルがある).また, イオンチャネルにはイオンの種類に応じて異なる\textbf{コンダクタンス}\index{こんだくたんす@コンダクタンス}(抵抗の逆数で電流の「流れやすさ」を意味する)と\textbf{平衡電位}\index{へいこうでんい@平衡電位}(equilibrium potential)がある.HHモデルでは, ナトリウム(Na$^{+}$)チャネル, カリウム(K$^{+}$)チャネル, 漏れ電流(leak current)のイオンチャネルを仮定する.漏れ電流のイオンチャネルは当時特定できなかったチャネルで, 膜から電流が漏れ出すチャネルを意味する.なお, 現在では漏れ電流の多くはCl$^{-}$イオン(chloride ion)によることが分かっている.
それでは, 等価回路モデルを用いて電位変化の式を立ててみよう.上図において, $C_m$は細胞膜のキャパシタンス(膜容量), $I_{m}(t)$は細胞膜を流れる電流(外部からの入力電流), $I_\text{Cap}(t)$は膜のコンデンサを流れる電流, $I_\text{Na}(t)$及び $I_K(t)$はそれぞれナトリウムチャネルとカリウムチャネルを通って膜から流出する電流, $I_\text{L}(t)$は漏れ電流である.このとき, 


\begin{equation}
I_{m}(t)=I_\text{Cap}(t)+I_\text{Na}(t)+I_\text{K}(t)+I_\text{L}(t)    
\end{equation}


という仮定をしている.膜電位を$V(t)$とすると, Kirchhoffの第二法則 (Kirchhoff's Voltage Law)より, 


\begin{equation}
\underbrace{C_m\frac {dV(t)}{dt}}_{I_\text{Cap} (t)}=I_{m}(t)-I_\text{Na}(t)-I_\text{K}(t)-I_\text{L}(t)
\end{equation}


となる.Hodgkinらはチャネル電流$I_\text{Na}, I_K, I_\text{L}$が従う式を実験的に求めた.


\begin{align}
I_\text{Na}(t) &= \bar{g}_{\text{Na}}\cdot m^{3}h(V-E_{\text{Na}})\\
I_\text{K}(t) &= \bar{g}_{\text{K}}\cdot n^{4}(V-E_{\text{K}})\\
I_\text{L}(t) &= \bar{g}_{\text{L}}(V-E_{\text{L}})
\end{align}


ただし, $\bar{g}_{\text{Na}}, \bar{g}_{\text{K}}$はそれぞれNa$^+$, K$^+$の最大コンダクタンスである (ここで上付き棒は上限値であることを示す).$\bar{g}_{\text{L}}$はオームの法則に従うコンダクタンスで, Lコンダクタンスは時間的に変化はしないと仮定する.また, $m$はNa$^+$コンダクタンスの活性化パラメータ, $h$はNa$^+$コンダクタンスの不活性化パラメータ, $n$はK$^+$コンダクタンスの活性化パラメータであり, ゲートの開閉確率を表している.よって, HHモデルの状態は$V, m, h, n$の4変数で表される.これらの変数は以下の$x$を$m, n, h$に置き換えた3つの微分方程式に従う. 


\begin{equation}
\frac{dx}{dt}=\alpha_{x}(V)(1-x)-\beta_{x}(V)x
\end{equation}


ただし, $V$の関数である$\alpha_{x}(V),\ \beta_{x}(V)$は$m, h, n$によって異なり, 次の6つの式に従う.


\begin{equation}
\begin{array}{ll}
\alpha_{m}(V)=\dfrac {0.1(V+40)}{1-\exp (-0.1(V+40))}, &\beta_{m}(V)=4\exp {(-(V+65)/18)}\\
\alpha_{h}(V)=0.07\exp {(-0.05(V+65))}, & \beta_{h}(V)=1/(1+{\exp {\left(-0.1(V+35)\right)}})\\
\alpha_{n}(V)={\dfrac {0.01(V+55)}{1-\exp {\left(-0.1(V+55)\right)}}},& \beta_{n}(V)=0.125\exp {(-0.0125(V+65))} 
\end{array}
\end{equation}


これまでに説明した式を用いてHHモデルを実装する.まず必要なパッケージを読み込む.
\lstinputlisting[language=julia]{./text/neuron-model/hodgkin-huxley/002.jl}
\lstinputlisting[language=julia]{./text/neuron-model/hodgkin-huxley/003.jl}
変更しない定数を保持する \jl{struct} の \jl{HHParameter} と, 変数を保持する \jl{mutable struct} の \jl{HH} を作成する.定数は次のように設定する. 


\begin{equation}
\begin{array}{l}
C_m=1.0\ \mu\textrm{F/cm}^2, \bar{g}_{\text{Na}}=120\ \textrm{mS/cm}^2, \bar{g}_{\text{K}}=36\ \textrm{mS/cm}^2, \bar{g}_{\text{L}}=0.3\ \textrm{mS/cm}^2\\
E_{\text{Na}}=50.0\ \textrm{mV}, E_{\text{K}}=-77\ \textrm{mV}, E_{\text{L}}=-54.387\ \textrm{mV} 
\end{array}
\end{equation}

\lstinputlisting[language=julia]{./text/neuron-model/hodgkin-huxley/005.jl}
\lstinputlisting[language=julia]{./text/neuron-model/hodgkin-huxley/006.jl}
次に変数を更新する関数\jl{update!}を書く.ソルバーとしては陽的Euler法または4次のRunge-Kutta法を用いる.以下ではEuler法を用いている.Juliaではforループを用いて1つのニューロンごとにパラメータを更新する方がベクトルを用いるよりも高速である.
\subsubsection{Hodgkin-Huxleyモデルのシミュレーションの実行}
いくつかの定数を設定してシミュレーションを実行する.
\lstinputlisting[language=julia]{./text/neuron-model/hodgkin-huxley/009.jl}
ニューロンの膜電位 \jl{v}, ゲート変数 \jl{m, h, n}, 刺激電流 \jl{Ie}の描画をする.
\lstinputlisting[language=julia]{./text/neuron-model/hodgkin-huxley/011.jl}
\begin{figure}[ht]
	\centering
	\includegraphics[scale=0.8, max width=\linewidth]{./fig/local-learning-rule/pca-hebbian-learning/cell011.png}
	\caption{cell011.png}
	\label{cell011.png}
\end{figure}
次項で用いるために発火回数を求める.ここでは膜電位が0を超えた点を数えることで,簡易的に求める.
\lstinputlisting[language=julia]{./text/neuron-model/hodgkin-huxley/013.jl}
50msから200msまでで11回, 250msから400msまでで16回発火しているので発火回数は計27回であり,この結果は正しい.
\subsection{Connor-Stevensモデル}
HHモデルはイカの巨大軸索の活動を再現したものであるが,脊椎動物のニューロンの神経活動を再現するためにHHモデルを修正したモデル (modified Hodgkin-Huxley model) が提案されてきた.その一種である,\textbf{Connor-Stevensモデル}\index{Connor-Stevensもでる@Connor-Stevensモデル} はHHモデルに2つ目のカリウム電流 (A型カリウム電流) を追加し,低い発火率でも活動を維持できる (振動を維持できる) ようにしたものである \citep{Connor1971-rs,Connor1977-qo}.ここでパラメータは\citep{Dayan2005-ib}に記載のものを使用する.


\begin{equation}
\begin{array}{l}
C_m=1.0\ \mu\textrm{F/cm}^2,\\ 
\bar{g}_{\text{Na}}=120\ \textrm{mS/cm}^2, \bar{g}_{\text{K}}=20\ \textrm{mS/cm}^2, \bar{g}_{\text{A}}=47.7\ \textrm{mS/cm}^2, \bar{g}_{\text{L}}=0.3\ \textrm{mS/cm}^2\\
E_{\text{Na}}=55.0\ \textrm{mV}, E_{\text{K}}=-72\ \textrm{mV}, E_{\text{A}}=-75\ \textrm{mV},E_{\text{L}}=-17\ \textrm{mV} 
\end{array}
\end{equation}



\begin{equation}
\begin{array}{ll}
\alpha_m=\dfrac{0.38(V+29.7)}{1-\exp (-0.1(V+29.7))} & \beta_m=15.2 \exp (-(V+54.7)/18) \\
\alpha_h=0.266 \exp (-0.05(V+48)) & \beta_h=3.8 /(1+\exp (-0.1(V+18))) \\ 
\alpha_n=\dfrac{0.02(V+45.7)}{1-\exp (-0.1(V+45.7))} & \beta_n=0.25 \exp (-0.0125(V+55.7))
\end{array}
\end{equation}



\begin{equation}
\frac{dx}{dt}=\frac{x_\infty-x}{\tau_x}\ (x=a, b)
\end{equation}



\begin{equation}
\begin{array}{l}
a_{\infty}=\left(\dfrac{0.0761 \exp [(V+94.22)/31.84]}{1+\exp ((V+1.17)/28.93)}\right)^{\frac{1}{3}}\\
\tau_a=0.3632+1.158 /(1+\exp ((V+55.96)/20.12)) \\
b_{\infty}=\left[1+\exp ((V+53.3)/14.54)\right]^{-4}\\
\tau_b=1.24+2.678 /(1+\exp [(V+50)/16.027])
\end{array}
\end{equation}
\lstinputlisting[language=julia]{./text/neuron-model/hodgkin-huxley/016.jl}
\lstinputlisting[language=julia]{./text/neuron-model/hodgkin-huxley/017.jl}
\lstinputlisting[language=julia]{./text/neuron-model/hodgkin-huxley/018.jl}
\lstinputlisting[language=julia]{./text/neuron-model/hodgkin-huxley/019.jl}
\begin{figure}[ht]
	\centering
	\includegraphics[scale=0.8, max width=\linewidth]{./fig/energy-based-model/sparse-coding/cell019.png}
	\caption{cell019.png}
	\label{cell019.png}
\end{figure}
\subsection{F-I曲線}
HHモデルにおいて,入力電流に対する発火率がどのように変化するかを調べる.次のコードのように入力電流を徐々に増加させたときの発火率を見てみよう.
\lstinputlisting[language=julia]{./text/neuron-model/hodgkin-huxley/021.jl}
\lstinputlisting[language=julia]{./text/neuron-model/hodgkin-huxley/022.jl}
発火率を計算して結果を描画する.
\lstinputlisting[language=julia]{./text/neuron-model/hodgkin-huxley/024.jl}
\begin{figure}[ht]
	\centering
	\includegraphics[scale=0.8, max width=\linewidth]{./fig/neuron-model/hodgkin-huxley/cell024.png}
	\caption{cell024.png}
	\label{cell024.png}
\end{figure}
このような曲線を\textbf{frequency-current (F-I) 曲線}\index{frequency-current (F-I) きょくせん@frequency-current (F-I) 曲線} (または neuronal input/output (I/O) 曲線)と呼ぶ.$I_\theta$は閾値電流を意味する (ここでは発火率が1Hz以上になる点を閾値と設定している) .F-I曲線の種類に応じてType IおよびIIに分けられる\footnote{Type IIIニューロンも存在する}.
\subsection{全か無かの法則の反例}
ニューロンは電流が流入することで膜電位が変化し, 膜電位がある一定の閾値を超えると活動電位が発生する, というのはニューロンの活動電位発生についての典型的な説明である.膜電位が閾値を超えるか超えないかで活動電位の発生が決まるという法則を, \textbf{全か無かの法則}\index{まったかなかのほうそく@全か無かの法則} (all-or-none principle) と呼ぶ.後に説明するLIFモデルなどは,全か無かの法則に従って神経活動のモデル化を行っている.しかし,この全か無かの法則の法則は必ずしも成立するわけではない.反例として \textbf{抑制後リバウンド}\index{よくせいのちりばうんど@抑制後リバウンド} (Postinhibitory rebound; PIR)という現象がある.抑制後リバウンドは過分極性の電流の印加を止めた際に膜電位が静止膜電位に回復するのみならず, さらに脱分極をして発火をするという現象である.この時生じる発火を\textbf{リバウンド発火}\index{りばうんどはっか@リバウンド発火} (rebound spikes)と呼ぶ.この現象が生じる要因として\textbf{アノーダルブレイク}\index{あのーだるぶれいく@アノーダルブレイク} (anodal break, またはanode break excitation; ABE)や,遅いT型カルシウム電流 (slow T-type calcium current) が考えられている \citep{Chik2004-ka}.HH モデルはこのうちアノーダルブレイクを再現できるため, シミュレーションによりどのような現象か確認してみよう.これは入力電流を変更するだけで行える.
\lstinputlisting[language=julia]{./text/neuron-model/hodgkin-huxley/027.jl}
結果は次のようになる.
\lstinputlisting[language=julia]{./text/neuron-model/hodgkin-huxley/029.jl}
\begin{figure}[ht]
	\centering
	\includegraphics[scale=0.8, max width=\linewidth]{./fig/local-learning-rule/pca-hebbian-learning/cell029.png}
	\caption{cell029.png}
	\label{cell029.png}
\end{figure}
なぜこのようなことが起こるか, というと過分極の状態から静止膜電位へと戻る際にNa$^+$チャネルが活性化 (Na$^+$チャネルの活性化パラメータ$m$が増加し, 不活性化パラメータ$h$が減少)し, 膜電位が脱分極することで再度Na$^+$チャネルが活性化する, というポジティブフィードバック過程(\textbf{自己再生的過程}\index{じこさいせいてきかてい@自己再生的過程})に突入するためである (もちろん, この過程は通常の活動電位発生のメカニズムである). この際, 発火に必要な閾値が膜電位の低下に応じて下がった, ということもできる.

なお,PIRに関連する現象として抑制後促通 (Postinhibitory facilitation; PIF)がある.これは抑制入力の後に興奮入力がある一定の時間内で入ると発火が起こるという現象である \citep{Dodla2006-fj}.

% \section{FitzHugh-Nagumoモデル}
\subsection{FitzHugh-Nagumoモデルの定義}

前節では神経活動のダイナミクスを微分方程式で表したHodgkin-Huxley(HH)モデルを扱った.HHモデルの特徴は,4変数で構成され,各変数が膜電位およびNaチャネルやKチャネルなどの活性/不活性状態を意味することである.このHHモデルをより簡易化し,2変数で神経活動の興奮とその伝播を表そうと提案されたのが\textbf{FitzHugh-Nagumo (FHN)モデル}\index{FitzHugh-Nagumo (FHN)もでる@FitzHugh-Nagumo (FHN)モデル} である.FHNモデルはvan der Pol振動子をFitzHughが修正し\cite{FitzHugh1955-bx} \cite{Fitzhugh1961-fp},南雲らによりトンネル (江崎) ダイオードを用いて電子回路上に実装\footnote{神経活動を再現する電子回路を\textbf{ニューリスタ}\index{にゅーりすた@ニューリスタ}  (neuristor) という.}された \cite{Nagumo1962-ob}という経緯がある.FHNモデルは以下で表される.


\begin{align} 
\frac{dv}{dt} &= c\left(v-\frac{v^3}{3}-u+I_e\right)\\ 
\frac{du}{dt} &= v-bu+a 
\end{align}


$v$は膜電位で,$u$は回復変数(recovery variable)と呼ばれる. FitzHughにより,HHモデルにおける$(V, m)$および$(n, h)$がそれぞれFHNモデルの$v$および$u$に対応すると説明されている \cite{Fitzhugh1961-fp} \footnote{HHモデルにおける$V$と$m$は強い正の相関があり,$n$と$h$は強い負の相関があるため,それぞれの変数の組は1つの変数に縮約されうる.}.$a,b,c$は定数であり,$a=0.7, b=0.8, c=10$がよく使われる.$I_e$は外部刺激電流に対応する.
\lstinputlisting[language=julia]{./text/neuron-model/fhn/001.jl}
変更しない定数を保持する \jl{struct} の \jl{FHNParameter} と, 変数を保持する \jl{mutable struct} の \jl{FHN} を作成する.
\lstinputlisting[language=julia]{./text/neuron-model/fhn/003.jl}
次に変数を更新する関数\jl{update!}を書く.ソルバーとしては陽的Euler法または4次のRunge-Kutta法を用いる.以下ではEuler法を用いている.Juliaではforループを用いて1つのニューロンごとにパラメータを更新する方がベクトルを用いるよりも高速である.
\lstinputlisting[language=julia]{./text/neuron-model/fhn/005.jl}
\subsection{FitzHugh-Nagumoモデルのシミュレーションの実行}
いくつかの定数を設定してシミュレーションを実行する.
\lstinputlisting[language=julia]{./text/neuron-model/fhn/007.jl}
結果を描画する.
\lstinputlisting[language=julia]{./text/neuron-model/fhn/009.jl}
\begin{figure}[ht]
	\centering
	\includegraphics[scale=0.8, max width=\linewidth]{./fig/bayesian-brain/mcmc/cell009.png}
	\caption{cell009.png}
	\label{cell009.png}
\end{figure}
\subsection{相図の描画}
phase plot
\lstinputlisting[language=julia]{./text/neuron-model/fhn/011.jl}
\lstinputlisting[language=julia]{./text/neuron-model/fhn/012.jl}
\begin{figure}[ht]
	\centering
	\includegraphics[scale=0.8, max width=\linewidth]{./fig/neuron-model/isi/cell012.png}
	\caption{cell012.png}
	\label{cell012.png}
\end{figure}

% \section{予測符号化}
\subsection{観測世界の階層的予測}
\textbf{階層的予測符号化(hierarchical predictive coding; HPC)} は\cite{Rao1999-zv}により導入された.構築するネットワークは入力層を含め,3層のネットワークとする.LGNへの入力として画像 $\mathbf{x} \in \mathbb{R}^{n_0}$を考える.画像 $\mathbf{x}$ の観測世界における隠れ変数,すなわち\textbf{潜在変数} (latent variable)を$\mathbf{r} \in \mathbb{R}^{n_1}$とし,ニューロン群によって発火率で表現されているとする (真の変数と $\mathbf{r}$は異なるので文字を分けるべきだが簡単のためにこう表す).このとき,


\mathbf{x} = f(\mathbf{U}\mathbf{r}) + \boldsymbol{\epsilon}


が成立しているとする.ただし,$f(\cdot)$は活性化関数 (activation function),$\mathbf{U} \in \mathbb{R}^{n_0 \times n_1}$は重み行列である.
$\boldsymbol{\epsilon} \in \mathbb{R}^{n_0}$ は $\mathcal{N}(\mathbf{0}, \sigma^2 \mathbf{I})$ からサンプリングされるとする.

潜在変数 $\mathbf{r}$はさらに高次 (higher-level)の潜在変数 $\mathbf{r}^h$により,次式で表現される.


\mathbf{r} = \mathbf{r}^{td}+\boldsymbol{\epsilon}^{td}=f(\mathbf{U}^h \mathbf{r}^h)+\boldsymbol{\epsilon}^{td}


ただし,Top-downの予測信号を $\mathbf{r}^{td}:=f(\mathbf{U}^h \mathbf{r}^h)$とした.また,$\mathbf{r}^{td} \in \mathbb{R}^{n_1}$, $\mathbf{r}^{h} \in \mathbb{R}^{n_2}$, $\mathbf{U}^h \in \mathbb{R}^{n_1 \times n_2}$ である.
$\boldsymbol{\epsilon}^{td} \in \mathbb{R}^{n_1}$は$\mathcal{N}(\mathbf{0}$, $\sigma_{td}^2 \mathbf{I}$) からサンプリングされるとする.

話は飛ぶが,Predictive codingのネットワークの特徴は
\begin{itemize}
\item 階層的な構造
\item 高次による低次の予測 (Feedback or Top-down信号)
\item 低次から高次への誤差信号の伝搬 (Feedforward or Bottom-up 信号)
\end{itemize}

である.ここまでは高次表現による低次表現の予測,というFeedback信号について説明してきたが,この部分はSparse codingでも同じである.それではPredictive codingのもう一つの要となる,低次から高次への予測誤差の伝搬というFeedforward信号はどのように導かれるのだろうか.結論から言えば,これは\textbf{復元誤差 (reconstruction error)の最小化を行う再帰的ネットワーク (recurrent network)を考慮することで自然に導かれる}.

\lstinputlisting[language=julia]{./text/neuron-model/lif/001.jl}
\subsubsection{事前分布の設定}
事前分布$p(\mathbf{r})$としては,0においてピークがあり,裾の重い(heavy tail)を持つsparse distributionあるいは \textbf{super-Gaussian distribution} (Laplace 分布やCauchy分布などGaussian分布よりもkurtoticな分布)を用いるのが良い.このような分布では,$\mathbf{r}$の各要素$r_i$はほとんど0に等しく,ある入力に対しては大きな値を取る.$p(\mathbf{r})$は一般化して式(4), (5)のように表記する.


\begin{aligned}
p(\mathbf{r})&=\prod_j p(r_j)\\
p(r_j)&=\frac{1}{Z_{\beta}}\exp \left[-\beta S(r_j)\right]
\end{aligned}


ただし,$\beta$は逆温度(inverse temperature), $Z_{\beta}$は規格化定数 (分配関数) である.これらの用語は統計力学における正準分布 (ボルツマン分布)から来ている.$S(x)$と分布の関係をまとめた表が以下となる (cf. \url{https://pdfs.semanticscholar.org/be08/da912362bf40fe3ded78bdadc644f921b4e7.pdf}).

\lstinputlisting[language=julia]{./text/neuron-model/lif/003.jl}
2種類の指数関数型シナプスの動態.破線は単一指数関数型シナプスで, 実線は二重指数関数型シナプスである.
\lstinputlisting[language=julia]{./text/neuron-model/lif/005.jl}
いくつかの処理について解説しておく.まず,一番目のforループ内の\jl{v[i]}の\jl{((dt*tcount) > (tlast[i] + tref))}は最後にニューロンが発火した時刻\jl{tlast[i]}に不応期\jl{tref}を足した時刻よりも現在の時刻\jl{dt*tcount[1]}が大きければ膜電位の更新を許可し,小さければ更新しない.二番目のforループにおける\jl{fire[i]}はニューロンの膜電位が閾値電位\jl{vthr}を超えたら\jl{True}となる.\jl{v[i]}などの更新式にある\jl{ifelse(a, b, c)}はaが\jl{True}の時はbを返し,\jl{False}の時はcを返す関数であり,\jl{v[i] = ifelse(fire[i], vreset, v[i])}は\jl{fire[i]}が\jl{True}なら\jl{v[i]}をリセット電位\jl{vreset}とし,そうでなければそのままの値を返すという処理である.同様にして\jl{tlast[i]}は発火したときにその時刻を記録する変数となっている.なお,\jl{v_[i] = ifelse(fire[i], vpeak, v[i])}は実際のシミュレーションにおいて意味をなさない.単に発火時の電位\jl{vpeak}を含めて記録すると描画時の見栄えが良いというだけである.

これらの\jl{struct}と関数を用いてシミュレーションを実行する.\jl{I} はHHモデルのときと同じように矩形波を入力する.実は\jl{I}は入力電流ではなく入力電流に比例する量となっているが,これは膜抵抗を乗じた後の値であると考えるとよい.

5分間のシミュレーションを行う.
\lstinputlisting[language=julia]{./text/neuron-model/lif/008.jl}
\subsubsection{正規方程式を用いた推定}条件に基づいて目的関数$L(\mathbf{\theta})$を微分すると次のような方程式が得られる.
$$
\mathbf{X}^\top\mathbf{X}\mathbf{\hat\theta}=\mathbf{X}^\top\mathbf{y}
$$
これを\textbf{正規方程式} (normal equation)と呼ぶ.この正規方程式より、係数の推定値は$\mathbf{\hat\theta}={(\mathbf{X}^\top\mathbf{X})}^{-1}X^\top\mathbf{y}$という式で得られる.なお,正規方程式自体は$\mathbf{y}=\mathbf{X}\mathbf{\theta}$の左から$\mathbf{X}^\top$をかける,と覚えると良い.
\lstinputlisting[language=julia]{./text/neuron-model/lif/010.jl}
\begin{figure}[ht]
	\centering
	\includegraphics[scale=0.8, max width=\linewidth]{./fig/bayesian-brain/neural-sampling/cell010.png}
	\caption{cell010.png}
	\label{cell010.png}
\end{figure}
## 画像の復元
\lstinputlisting[language=julia]{./text/neuron-model/lif/012.jl}
損失関数を定義する.

\lstinputlisting[language=julia]{./text/neuron-model/lif/014.jl}
\lstinputlisting[language=julia]{./text/neuron-model/lif/015.jl}
\begin{figure}[ht]
	\centering
	\includegraphics[scale=0.8, max width=\linewidth]{./fig/neuron-model/isi/cell015.png}
	\caption{cell015.png}
	\label{cell015.png}
\end{figure}
50msから200msまでで11回, 250msから400msまでで16回発火しているので発火回数は計27回であり,この結果は正しい.
発火数のヒストグラムを描画する.PyPlotで\jl{hist2D(posx[idx], posy[idx], bins=10, cmap="jet")}などとする方が簡便だが,今回はhistgramの各binの値を用いるために\jl{StatsBase.jl}を用いる.
\lstinputlisting[language=julia]{./text/neuron-model/lif/018.jl}
\subsection{Target jump}

target jumpする場合の最適制御 \cite{Li2018-qt}. 状態にtarget位置も含むモデルであればtarget位置をずらせばよいが,ここでは自己位置をずらしtargetとの相対位置を変化させることでtarget jumpを実現する.

\lstinputlisting[language=julia]{./text/neuron-model/lif/020.jl}
\begin{figure}[ht]
	\centering
	\includegraphics[scale=0.8, max width=\linewidth]{./fig/energy-based-model/predictive-coding/cell020.png}
	\caption{cell020.png}
	\label{cell020.png}
\end{figure}
 % ok
% \section{Izhikevich モデル}
\subsection{Izhikevich モデルの定義}
\textbf{Izhikevich モデル} (または\textbf{Simple model})は([Izhikevich, 2003](https://www.izhikevich.org/publications/spikes.htm))で考案されたモデルである.HHモデルのような生理学的な知見に基づいたモデルは実際のニューロンの発火特性をよく再現できるが,式が複雑化するため,数学的な解析が難しく,計算量が増えるために大規模なシミュレーションも困難となる\footnote{これに関しては必ずしも正しくない.計算機の発達によりHHモデルで大きなモデルをシミュレーションすることも可能である.}.そこで,生理学的な正しさには目をつぶり,生体内でのニューロンの発火特性を再現するモデルが求められた.その特徴を持つのがIzhikevich モデルである (以下ではIzモデルと表記する).Izモデルは 2変数しかない\footnote{数値計算をする上では簡易的だが,if文が入るために解析をするのは難しくなる.([Bernardo, et al., 2008](https://www.springer.com/gp/book/9781846280399))を読むといいらしい.}
簡素な微分方程式だが, 様々なニューロンの活動を模倣することができる.定式化には主に2種類ある.まず,([Izhikevich, 2003](https://www.izhikevich.org/publications/spikes.htm))で提案されたのが次式である.


\begin{align}
\frac{dv(t)}{dt}&=0.04v(t)^2 + 5v(t)+140-u(t)+I(t) \\
\frac{du(t)}{dt}&=a(bv(t)-u(t))
\end{align} 


ここで,$v$と$u$が変数であり, $v$は膜電位(membrane potential;単位はmV), $u$は回復電流(recovery current; 単位はpA)\footnote{ここでの「回復」というのは脱分極した後の膜電位が静止膜電位へと戻る,という意味である (対義語はactivationで膜電位の上昇を意味する).$u$は$v$の導関数において$v$の上昇を抑制するように$-u$で入っているため,$u$としてはK$^+$チャネル電流やNa$^+$チャネルの不活性化動態などが考えられる.}
である.また,$a$は回復時定数(recovery time constant; 単位はms$^{-1}$)の逆数 (これが大きいと$u$が元に戻る時間が短くなる), $b$は$u$の$v$に対する感受性(共鳴度合い,  resonance; 単位はpA/mV)である.

この式は簡便だが,生理学的な意味づけが分かりにくい.改善された式として["Dynamical Systems in Neuroscience" (Izhikevich, 2007)](https://mitpress.mit.edu/books/dynamical-systems-neuroscience)のChapter 8で紹介されているのが次式である.


\begin{align}
C\frac{dv(t)}{dt}&=k\left(v(t)-v_r\right)\left(v(t)-v_t\right)-u(t)+I(t) \\
\frac{du(t)}{dt}&=a\left\{b\left(v(t)-v_{r}\right)-u(t)\right\}
\end{align} 


ここで,$C$は膜容量(membrane capacitance; 単位はpF), $v_r$は静止膜電位(resting membrane potential; 単位はmV), $v_t$は閾値電位(instantaneous threshold potential; 単位はmV), $k$はニューロンのゲインに関わる定数で,小さいと発火しやすくなる (単位はpA/mV).以後はこちらの式を用いる.

Izモデルの\textbf{閾値の取り扱い}はLIFモデルと異なり,HHモデルに近い.LIFモデルでは閾値を超えた時に膜電位をピーク電位まで上昇させ (この過程は無くてもよい),続いて膜電位をリセットする.Izモデルの閾値は$v_t$だが, 膜電位のリセットは閾値を超えたかで判断せず,膜電位$v$がピーク電位$v_{\text{peak}}$になったとき (または超えた時) に行う.そのためIzモデルの実際の閾値は膜電位の挙動が変化する(発火状態に移行する),つまり分岐(bifurcation) が生じる点であり,パラメータの閾値$v_t$との間には差異がある.

さて,膜電位がピーク電位$v_{\text{peak}}$に達したとき (すなわち \jl{if} $v \geq v_{\text{peak}}$),$u, v$を次のようにリセットする\footnote{バースト発火(bursting)の挙動を表現するためには,速い回復変数(fast recovery variable)と遅い回復変数(slow recovery variable)の2つが必要となる(従って膜電位も合わせて全部で3変数必要).一方で,IzモデルではLIFモデルのようなif文によるリセットを用いているため,速い回復変数が必要なく,遅い回復変数$u$のみでバースト発火を表現できる.}.


\begin{align} 
u&\leftarrow u+d\\
v&\leftarrow v_{\text{reset}}
\end{align}


とする.ただし, $v_{\text{reset}}$は過分極を考慮して静止膜電位$v_r$よりも小さい値とする.また,$d$はスパイク発火中に活性化される正味の外向き電流の合計を表し,発火後の膜電位の挙動に影響する (単位はpA).

以上を踏まえて, シミュレーションを行う.まず,必要なパッケージを読み込む.
\lstinputlisting[language=julia]{./text/neuron-model/izhikevich/001.jl}
変更しない定数を保持する\jl{struct}の\jl{IZParameter}と,変数を保持する\jl{mutable struct}の\jl{IZ}を作成する.2つの定式化でパラメータの値が異なるので注意すること.
\lstinputlisting[language=julia]{./text/neuron-model/izhikevich/003.jl}
次に変数を更新する関数\jl{update!}を書く.LIFの場合と異なり,\jl{v[i] >= vpeak}であることに注意する (\jl{v[i] >= vthr}ではない).
\lstinputlisting[language=julia]{./text/neuron-model/izhikevich/005.jl}
\subsection{Izhikevich モデルのシミュレーションの実行}
いくつかの定数を設定してシミュレーションを実行する.
\lstinputlisting[language=julia]{./text/neuron-model/izhikevich/007.jl}
\jl{Plots}を読み込み,膜電位\jl{v}, 回復変数\jl{u}, 入力電流\jl{I}を描画する.
\lstinputlisting[language=julia]{./text/neuron-model/izhikevich/009.jl}
\begin{figure}[ht]
	\centering
	\includegraphics[scale=0.8, max width=\linewidth]{./fig/bayesian-brain/mcmc/cell009.png}
	\caption{cell009.png}
	\label{cell009.png}
\end{figure}
\subsection{様々な発火パターンのシミュレーション}
次に様々な発火パターンを模倣するようにIzモデルの定数を変化させてみよう.Intrinsically Bursting (IB)ニューロンとChattering (CH) ニューロン(または fast rhythmic bursting (FRB) ニューロン)のシミュレーションを行う.基本的には定数を変えるだけである.

本書で用いている式における発火パターンに対するパラメータは([Izhikevich, 2003](https://www.izhikevich.org/publications/spikes.htm))では得られないが,["Dynamical Systems in Neuroscience" (Izhikevich, 2007)](https://mitpress.mit.edu/books/dynamical-systems-neuroscience)には記載がある.他の発火パターンに関してはこの本を参照のこと.
\lstinputlisting[language=julia]{./text/neuron-model/izhikevich/011.jl}
これまでと異なり,モデルの定義時に\jl{param}を設定していることに注意しよう.最後に膜電位変化を描画する.
\lstinputlisting[language=julia]{./text/neuron-model/izhikevich/013.jl}
\begin{figure}[ht]
	\centering
	\includegraphics[scale=0.8, max width=\linewidth]{./fig/neuron-model/hodgkin-huxley/cell013.png}
	\caption{cell013.png}
	\label{cell013.png}
\end{figure}
\subsection{ランダムネットワークのシミュレーション}
1000個のIzニューロン(興奮性800個, 抑制性200個)によるランダムネットワークのシミュレーションを行う.これは([Izhikevich, 2003](https://www.izhikevich.org/publications/spikes.htm))においてMATLABコードが示されており,それをJuliaに移植したものである.このシミュレーションではRS(regular spiking)ニューロンを興奮性細胞,FS(fast spiking)ニューロンを抑制性細胞のモデルとして用いている.
\lstinputlisting[language=julia]{./text/neuron-model/izhikevich/015.jl}
膜電位の更新の際,\jl{v}を2回に分けて更新しているが,これは数値的な安定性を高めるためである.計算量は上がるが,前述したモデルにおいても同様の処理を行う実装もある.

シミュレーションの実行後,ネットワークを構成するニューロンの発火を描画する.これを\textbf{ラスタープロット} (raster plot)という.この図は横軸が時間,縦軸がニューロンの番号となっており,各ニューロンが発火したことを点で表している.
\lstinputlisting[language=julia]{./text/neuron-model/izhikevich/017.jl}
\begin{figure}[ht]
	\centering
	\includegraphics[scale=0.8, max width=\linewidth]{./fig/motor-learning/optimal-feedback-control/cell017.png}
	\caption{cell017.png}
	\label{cell017.png}
\end{figure}
初めの400msぐらいまでは100msごとに約10Hzの$\alpha$波が見られ,800ms付近には約40Hzの$\gamma$波が見られる.
 
% \section{Inter-spike interval モデル}
これまで紹介したモデルでは,入力に対する膜電位などの時間変化に基づき発火が起こるかどうか,ということを考えてきた.この節では,発火が生じるまでの過程を考慮せず,発火の時間間隔(\textbf{inter-spike interval, ISI}\index{inter-spike interval, ISI})の統計による現象論的モデルを考える.これを\textbf{Inter-spike interval (ISI)}\index{Inter-spike interval (ISI)} モデルと呼ぶ.ISIモデルは\textbf{点過程(point process)}\index{てんかてい(point process)@点過程(point process)} という統計的モデルに基づいており,各モデルにはISIが従う分布の名称がついている.
この節では,使用頻度の高い \textbf{ポアソン過程 (Poisson process) モデル}\index{ぽあそんかてい (Poisson process) もでる@ポアソン過程 (Poisson process) モデル},ポアソン過程モデルにおいて不応期を考慮した \textbf{死時間付きポアソン過程 (Poisson process with dead time, PPD) モデル}\index{しにどきかんつきぽあそんかてい (Poisson process with dead time, PPD) もでる@死時間付きポアソン過程 (Poisson process with dead time, PPD) モデル},皮質の定常発火においてポアソン過程モデルよりも当てはまりがよいとされる \textbf{ガンマ過程 (Gamma process) モデル}\index{がんまかてい (Gamma process) もでる@ガンマ過程 (Gamma process) モデル}について説明する.
なお,SNNにおいて,ISIモデルは主に画像入力の際に\textbf{連続値からスパイク列へのエンコード}\index{れんぞくちからすぱいくれつへのえんこーど@連続値からスパイク列へのエンコード}に用いられる.これに限らず入力として用いられることが多い.
この節は \citep{Shimazaki_undated-ko}, \citep{Pachitariu2010-pm} を参考に執筆した.
\subsection{ポアソン過程モデル}
\subsubsection{点過程とポアソン過程}
時間に応じて変化する確率変数のことを\textbf{確率過程(stochastic process)}\index{かくりつかてい(stochastic process)@確率過程(stochastic process)} という.さらに確率過程の中で,連続時間軸上において離散的に生起する点事象の系列を\textbf{点過程(point process)}\index{てんかてい(point process)@点過程(point process)} という.スパイクは離散的に起こるので,点過程を用いてモデル化ができるという話である.
ポアソン過程 (Poisson process)は点過程の1つである.ポアソン過程モデルはスパイクの発生をポアソン過程でモデル化したもので,このモデルによって生じるスパイクをポアソンスパイク(Poisson spike)と呼ぶ.ポアソン過程では,時刻$t$までに起こった点の数$N(t)$はポアソン分布に従う.すなわち,点が起こる確率が強度$\lambda$のポアソン分布に従う場合, 時刻$t$までに事象が$n$回起こる確率は$P[N(t)=n]=\dfrac{(\lambda t)^{n}}{n !} e^{-\lambda t}$となる. 
ポアソン過程において点が起こる回数がポアソン分布に従うことは,ポアソン過程という名称の由来となっている.これを定義とする場合もあれば,次の4条件を満たす点過程をポアソン過程とするという定義もある.
\begin{enumerate}
\item 時刻0における初期の点の数は0 : $P[N(0)=0]=1$ 
\item $[t, t+\Delta t)$に点が1つ生じる確率 : $P[N(t+\Delta t)-N(t)=1]=\lambda(t)\Delta t+o(\Delta t)$
\item 微小時間$\Delta t$の間に点は2つ以上生じない : $P[N(t+\Delta t)-N(t)=2]=o(\Delta t)$
\item 任意の時点$t_1 < t_2 < \cdots< t_n$に対して,増分 $N(t_2)-N(t_1), N(t_3)-N(t_2), \cdots, N(t_n)-N(t_{n−1})$は互いに独立である.
\end{enumerate}
ただし, $o(\cdot)$はLandauの記号(Landauのsmall o)であり, $o(x)$は$x\to 0$のとき,$o(x)/x\to 0$となる微小な量を表す.ポアソン過程に従ってスパイクが生じるとする場合,条件2の強度関数$\lambda(t)$は\textbf{発火率}\index{はっかりつ@発火率}を意味する (また実装において有用).条件3は不応期より小さいタイムステップにおいては,1つのタイムステップにおいて1つしかスパイクは生じないということを表す.条件4はスパイクは独立に発生する,ということを意味する.また,これらの条件から$N(t)$の分布は強度母数$\lambda(t)$のポアソン分布に従うことが示せる.
強度関数(点がスパイクの場合,発火率)が$\lambda(t)=\lambda$ (定数)となる場合は点の時間間隔(点がスパイクの場合,ISI)の確率変数$T$が強度母数$\lambda$の \textbf{指数分布}\index{しすうぶんぷ@指数分布}に従う.なお,指数分布の確率密度関数は確率変数を$T$とするとき,
\begin{equation}
f(t;\lambda )=\left\{{\begin{array}{ll}\lambda e^{-\lambda t}&(t\geq 0)\\0&(t<0)\end{array}}\right.
\end{equation}
となる.このことは4条件とChapman-Kolmogorovの式により求められるが,ややこしいので, $P[N(t)=n]=\dfrac{(\lambda t)^{n}}{n !} e^{-\lambda t}$から導出できることを簡単に示す.指数分布の累積分布関数を$F(t; \lambda)$とすると,
\begin{equation}
F(t; \lambda) = P(T< t)=1-P(T\geq t)=1-P(N(t)=0)=1-e^{-\lambda t}
\end{equation}
となる.よって
\begin{equation}
f(t; \lambda)=\frac{dF(t; \lambda)}{dt}=\lambda e^{-\lambda t}
\end{equation}
が成り立つ.
\subsubsection{定常ポアソン過程}
ここからポアソン過程によるスパイクのシミュレーションを実装する.実装方法にはISIが指数分布に従うことを利用したものと,ポアソン過程の条件2を利用したものの2通りがある.実装は後者が楽で計算量も少ないが,後のガンマ過程のために前者の実装を先に行う.
\paragraph{ISIの累積により発火時刻を求める手法}
ISIが指数分布に従うことを利用してポアソン過程モデルの実装を行う.まずISIを指数分布に従う乱数とする.次にISIを累積することで発火時刻を得る.最後に発火時間を整数値に丸めてindexとすることで$\{0, 1\}$のスパイク列が得られる.ISIの取得には\jl{Random.randexp()}を用いる.この関数は scale 1の指数分布に従う乱数を返す.このscaleは指数分布の確率密度関数を$f(t; \frac{1}{\beta}) = \frac{1}{\beta} e^{-t/\beta}$とした際の$\beta = 1/\lambda$である(この時,平均は$\beta$となる).よって発火率を\jl{fr}(1/s), 単位時間を\jl{dt}(s)としたときのISIは \jl{isi = 1/(fr*dt) * randexp()}として得ることができる.
まず必要なパッケージを読み込む.
\begin{lstlisting}[language=julia]
using Random, PyPlot, Distributions
rc("axes.spines", top=false, right=false)
\end{lstlisting}
乱数のseed値を設定し,必要な定数を定義した後に\jl{isi}を計算する.\jl{isi}を累積することでスパイクの生じた時刻を記録する配列\jl{spike_time}を作成する.作成後,\jl{spike_time}を用いてラスタープロットを描画する.
\begin{lstlisting}[language=julia]
Random.seed!(0) # set random seed

T = 1000 # ms
dt = 1f0 # ms
nt = Int32(T/dt) # number of timesteps

n_neurons = 10 # ニューロンの数
fr = 30 # ポアソンスパイクの発火率(Hz)

isi = 1/(fr*dt*1e-3) * randexp(Int(nt*1.5/fr), n_neurons)
spike_time = cumsum(isi, dims=1); # ISIを累積
\end{lstlisting}
\begin{lstlisting}[language=julia]
figure(figsize=(6, 2))
for i=1:n_neurons
    scatter(spike_time[:, i], i*ones(Int(nt*1.5/fr)), c="k", s=1)
end
xlabel("Time (ms)"); ylabel("Neuron index"); xlim(0, T+10); tight_layout()
\end{lstlisting}
\begin{figure}[ht]
	\centering
	\includegraphics[scale=0.8, max width=\linewidth]{./fig/neuron-model/isi/cell006.png}
	\caption{cell006.png}
	\label{cell006.png}
\end{figure}
\jl{spike_time}のように発火時刻で記録しておく方がメモリを節約できるが,シミュレーションにおいてはスパイク列$S$はタイムステップごとに発火しているかを表す$\{0,1\}$配列で保持しておくと楽に扱うことができる.そのため冗長ではあるが,発火時刻の配列を$\{0,1\}$配列\jl{spikes}に変換しスパイクの数と発火率を計算する.
\begin{lstlisting}[language=julia]
spike_time[spike_time .> nt - 1] .= 1 # ntを超える場合を1に
spike_time = round.(Int, spike_time) # float to int
spikes = zeros(nt, n_neurons) # スパイク記録変数

for i=1:n_neurons    
    spikes[spike_time[:, i], i] .= 1
end

spikes[1] = 0 # (spike_time=1)の発火を削除
println("Num. of spikes : ", sum(spikes))
println("Firing rate : ", sum(spikes)/(n_neurons*T)*1e3, "Hz")
\end{lstlisting}
\paragraph{$\Delta t$ 間の発火確率が $\lambda\Delta t$ であることを利用する方法}
次に2番目のポアソン過程モデルの実装を行う.こちらは$\lambda$を発火率とした場合, 区間$[t, t+\Delta t)$の間にポアソンスパイクが発生する確率は$\lambda \Delta t$となることを利用する.これはポアソン過程の条件だが,ポアソン分布から導けることを簡単に示しておく.事象が起こる確率が強度$\lambda$のポアソン分布に従う場合, 時刻$t$までに事象が$n$回起こる確率は$P[N(t)=n]=\dfrac{(\lambda t)^{n}}{n !} e^{-\lambda t}$となる.よって, 微小時間$\Delta t$において事象が$1$回起こる確率は
\begin{equation}
P[N(\Delta t)=1]=\dfrac{\lambda \Delta t}{1 !} e^{-\lambda \Delta t}\simeq \lambda \Delta t+o(\Delta t)
\end{equation}
となる.ただし, $e^{-\lambda \Delta t}$についてはマクローリン展開による近似を行っている.このことから, 一様分布$U(0,1)$に従う乱数$\xi$を取得し, $\xi<\lambda dt$なら発火$(y=1)$, それ以外では$(y=0)$となるようにすればポアソンスパイクを実装できる.
\begin{lstlisting}[language=julia]
Random.seed!(0) # set random seed

T = 1000 # ms
dt = 1f0 # ms
nt = Int(T/dt) # number of timesteps

n_neurons = 10 # ニューロンの数
fr = 30 # ポアソンスパイクの発火率(Hz)

spikes = rand(nt, n_neurons) .< fr*dt*1e-3

println("Num. of spikes : ", sum(spikes))
println("Firing rate : ", sum(spikes)/(n_neurons*T)*1e3, "Hz")
\end{lstlisting}
\begin{lstlisting}[language=julia]
function rasterplot(spikes)
    # input spike -> time, #neurons
    spike_inds = Tuple.(findall(x -> x > 0, spikes))
    spike_time = first.(spike_inds)
    neuron_inds = last.(spike_inds)
    
    # raster plot
    scatter(spike_time, neuron_inds, s=1, c="black")
    xlabel("Time (ms)"); ylabel("Neuron index")
    tight_layout()
end
\end{lstlisting}
\begin{lstlisting}[language=julia]
figure(figsize=(5,2))
rasterplot(spikes)
\end{lstlisting}
\begin{figure}[ht]
	\centering
	\includegraphics[scale=0.8, max width=\linewidth]{./fig/neuron-model/isi/cell012.png}
	\caption{cell012.png}
	\label{cell012.png}
\end{figure}
なお,ここでは全時間における発火をまとめて計算しているが,タイムステップごとに発火の有無を計算することもできる.前者は発火情報を保持するためのメモリが必要だが,計算時間は短くなる.後者はメモリの節約になるが,計算時間は長くなる.そのため,これら2つの方法はメモリと計算時間のトレードオフとなる.また,他には発火情報を疎行列(sparse matrix)の形式で保持しておくとメモリの節約になる.
\subsubsection{非定常ポアソン過程}
これまでの実装は発火率$\lambda$が一定であるとする,定常ポアソン過程 (homogeneous poisson process)であったが,ここからは発火率$\lambda(t)$が時間変化するとする\textbf{非定常ポアソン過程}\index{ひていじょうぽあそんかてい@非定常ポアソン過程} (inhomogeneous poisson process)について考える.とはいえ,定常ポアソン過程における発火率を,時間についての関数で置き換えるだけで実装できる.以下は$\lambda(t)=\sin^2(\alpha t)$(ただし$\alpha$は定数)とした場合の実装である.
\begin{lstlisting}[language=julia]
Random.seed!(0) # set random seed

T = 1000 # ms
dt = 1f0 # ms
nt = Int32(T/dt) # number of timesteps

n_neurons = Int32(10) # ニューロンの数

t = Array{Float32}(1:nt)*dt
fr = 30(sin.(1e-2t)).^2 # ポアソンスパイクの発火率(Hz)

spikes = rand(nt, n_neurons) .< fr*dt*1e-3

figure(figsize=(5,3))
subplot(2,1,1); plot(t, fr); ylabel("Firing rate (Hz)")
subplot(2,1,2); rasterplot(spikes)
\end{lstlisting}
\begin{figure}[ht]
	\centering
	\includegraphics[scale=0.8, max width=\linewidth]{./fig/neuron-model/isi/cell015.png}
	\caption{cell015.png}
	\label{cell015.png}
\end{figure}
上が発火率$\lambda(t)$の時間変化, 下がラスタープロットである.
\subsection{死時間付きポアソン過程モデル (Poisson process with dead time, PPD)}
ポアソン過程は簡易的で有用だが,不応期を考慮していない.そのため,時には生理的範疇を超えたバースト発火が起こる場合もある (複数のニューロンからの発火の重ね合わせ(superposition)であると考えることもできる.) .そこで,ポアソン過程において不応期のようなイベントの生起が起こらない \textbf{死時間}\index{しにどきかん@死時間} (dead time) \footnote{例えば,ガイガー・カウンター(Geiger counter)などの放射線の検出器には放射線の到達を機器の物理的特性として検出できない時間(つまり死時間)がある.そのため放射線の到達数がポアソン分布に従うとした場合,放射線測定装置のモデルとしてPPDが用いられる.}を考慮した\textbf{死時間付きポアソン過程}\index{しにどきかんつきぽあそんかてい@死時間付きポアソン過程} (PPD: Poisson process with dead time または dead time modified Poisson process)というモデルを導入する.
実装においてはLIFニューロンの時と同じような不応期の処理をする.つまり,現在が不応期かどうかを判断し,不応期なら発火を許可しないようにする.
\begin{lstlisting}[language=julia]
Random.seed!(0) # set random seed

T = 1000 # ms
dt = 1f0 # ms
nt = Int32(T/dt) # number of timesteps
tref = 5f0 # 不応期 (ms)

n_neurons = Int32(10) # ニューロンの数
fr = 30 # ポアソンスパイクの発火率(Hz)

tlast, spikes = zeros(n_neurons), zeros(nt, n_neurons) # 発火時刻の記録変数

# simulation
@time for i=1:nt
    fire = rand(n_neurons) .< fr*dt*1e-3
    spikes[i, :] = ((dt*i) .> (tlast .+ tref)) .* fire
    tlast[:] = tlast .* (1 .- fire) + dt*i * fire # 発火時刻の更新
end

println("Num. of spikes : ", sum(spikes))
println("Firing rate : ", sum(spikes)/(n_neurons*T)*1e3, "Hz")
\end{lstlisting}
不応期があるために発火率は設定値の30Hzよりも低くなっていることが分かる.次にラスタープロットを描画する.
\begin{lstlisting}[language=julia]
figure(figsize=(5,2))
rasterplot(spikes)
\end{lstlisting}
\begin{figure}[ht]
	\centering
	\includegraphics[scale=0.8, max width=\linewidth]{./fig/neuron-model/isi/cell020.png}
	\caption{cell020.png}
	\label{cell020.png}
\end{figure}
通常のPoisson spikeと差はあまり感じられないが,高頻度発火の場合に通常のモデルとの違いが明瞭となる.
\subsection{ガンマ過程モデル}
ガンマ過程(gamma process)は点の時間間隔がガンマ分布に従うとするモデルである.ガンマ過程はポアソン過程よりも皮質における定常発火への当てはまりが良いとされている \citep{Shinomoto2003-lz,Maimon2009-mb}.時間間隔の確率変数を$T$とした場合,ガンマ分布の確率密度関数は
\begin{equation}
f(t;k,\theta) =  t^{k-1}\frac{e^{-t/\theta}}{\theta^k\Gamma(k)}
\end{equation}
と表される.ただし,$t > 0$であり, 2つの母数は$k, \theta > 0$である.また,$\Gamma (\cdot)$はガンマ関数であり,
\begin{equation}
\Gamma (k)=\int _{0}^{\infty }x^{k-1}e^{-x}\,dx
\end{equation}
と定義される.ガンマ分布の平均は$k\theta$だが,発火率はISIの平均の逆数なので,$\lambda=1/k\theta$となる.また,$k=1$のとき,ガンマ分布は指数分布となる.さらに$k$が正整数のとき,ガンマ分布はアーラン分布となる.
ガンマ過程モデルの実装はポアソン過程モデルのISIを累積する手法と同様に書くことができる.ただしこの時,\jl{Distributions.jl}を用いる.基本的には\jl{randexp(shape)}を\jl{rand(Gamma(a,b), shape)}に置き換えればよい (もちろん多少の修正は必要とする).
スパイク列を生成する関数を書く.
\begin{lstlisting}[language=julia]
function gamma_spike(T, dt, n_neurons, fr, k)
    nt = Int32(T/dt) # number of timesteps
    theta = 1/(k*(fr*dt*1e-3)) # fr = 1/(k*theta)

    isi = rand(Gamma(k, theta), Int32(round(nt*1.5/fr)), n_neurons)
    spike_time = cumsum(isi, dims=1) # ISIを累積

    spike_time[spike_time .> nt - 1] .= 1 # ntを超える場合を1に
    spike_time = round.(Int32, spike_time) # float to int
    spikes = zeros(Bool, nt, n_neurons) # スパイク記録変数

    for i=1:n_neurons    
        spikes[spike_time[:, i], i] .= 1
    end

    spikes[1] = 0 # (spike_time=1)の発火を削除
    return spikes
end
\end{lstlisting}
\jl{gamma_spike} 関数を用いて $k=1, 12$ の場合のシミュレーションを実行する.なお,$k=1$のときはポアソン過程に一致することに注意しよう.
\begin{lstlisting}[language=julia]
Random.seed!(0) # set random seed

T = 1000 # ms
dt = 1f0 # ms
nt = Int32(T/dt) # number of timesteps

n_neurons = 10 # ニューロンの数
fr = 30 # ガンマスパイクの発火率(Hz)

# k=1のときはポアソン過程に一致
spikes1 = gamma_spike(T, dt, n_neurons, fr, 1)
spikes2 = gamma_spike(T, dt, n_neurons, fr, 12)

println("Num. of spikes : ", sum(spikes1), ", ",sum(spikes2))
println("Firing rate : ", sum(spikes1)/(n_neurons*T)*1e3, "Hz, ", sum(spikes2)/(n_neurons*T)*1e3, "Hz")
\end{lstlisting}
ISIの分布を描画するための関数を定義する.
\begin{lstlisting}[language=julia]
function gamma_isi_plot(dt, fr, k, n=1000)
    theta = 1/(k*(fr*dt*1e-3)) # fr = 1/(k*theta)
    isi = rand(Gamma(k, theta), n)
    gamma_pdf = pdf.(Gamma(k, theta), minimum(isi):maximum(isi))

    hist(isi, bins=20, density=true, alpha=0.5, ec="black"); 
    plot(minimum(isi):maximum(isi), gamma_pdf, color="black"); 
    xlabel("ISI (ms)"); ylabel("Density");
end
\end{lstlisting}
結果を描画する.上段はISIの分布,下段はラスタープロットである.左の$k=1$の場合をポアソン過程モデルのスパイク列と比較しよう (同じ外観になっていることが分かる).右は$k=12$とした場合である.
\begin{lstlisting}[language=julia]
figure(figsize=(6, 4))
subplot(2,2,1); gamma_isi_plot(dt, fr, 1)
subplot(2,2,2); gamma_isi_plot(dt, fr, 12)
subplot(2,2,3); rasterplot(spikes1)
subplot(2,2,4); rasterplot(spikes2)
tight_layout()
\end{lstlisting}
\begin{figure}[ht]
	\centering
	\includegraphics[scale=0.8, max width=\linewidth]{./fig/neuron-model/isi/cell030.png}
	\caption{cell030.png}
	\label{cell030.png}
\end{figure}
なお,前述したようにガンマ過程モデルの方がポアソン過程モデルよりも皮質ニューロンのモデルとしては優れているが,入力画像のエンコーディングをガンマ過程モデルにすることでSNNの認識精度が向上するかどうかはまだ十分に研究されていない.また,\citep{Deger2012-ai}ではPPDやガンマ過程の重ね合わせによるスパイク列を生成するアルゴリズムを考案している.
 % ok
% \section{神経突起の成長モデル}
神経細胞は他の細胞に比して特異な形態を持つ.またニューロンの種類およびその役割により基本的な細胞体,樹状突起,軸索等の構造は共通するものの,各部分の形態は異なる.このような形態はどのようにして発達するのだろうか.本節では\textbf{神経突起(neurite)}\index{しんけいとっき(neurite)@神経突起(neurite)} の\textbf{成長モデル(growth model)}\index{せいちょうもでる(growth model)@成長モデル(growth model)} を取り扱う.神経突起とは神経細胞において細胞体から伸びる突起の総称である.
\subsection{神経突起の木構造}
神経突起の形態は\textbf{樹状}\index{じゅじょう@樹状}突起 (dendrites; ギリシャ語で木を意味する*déndron*に由来) に代表されるように (生物としての) 木に類似している.さらに分節(segment)に離散化することでグラフ理論における\textbf{木}\index{き@木}(tree; 連結で閉路を持たないグラフ)として捉えることができる.
シミュレーション用にデータ構造を作成しよう.なお,Juliaで木構造を扱うためのライブラリ\jl{AbstractTrees.jl}は使用しない.\jl{tree_info}はInt型vector (要素数3) のlistであり,接続している分節の番号,遠心性位数,分節の種類(1: 末端, 0:中間)を表す.\jl{seg_vec}は Float型vector (要素数2) のlistであり,分節の2次元極座標ベクトル(半径,角度)を表す.3次元に拡張することも可能であるが,本書では簡単のために2次元とする.多次元配列ではなくvectorのlistにしているのは,成長に伴って要素を追加していく際に配列に結合\jl{cat}するよりlist化して追加\jl{push!}する方が高速なためである.
\begin{lstlisting}[language=julia]
using PyPlot, ProgressMeter, Distributions, Random
using PyPlot: matplotlib

rc("axes.spines", top=false, right=false)
\end{lstlisting}
\begin{lstlisting}[language=julia]
tree_info_eg = Vector{Int64}[[1, 0, 0], [1, 0, 0], [2, 1, 0], [2, 1, 0], [3, 2, 1], 
                             [3, 2, 1], [4, 2, 1], [4, 2, 1], [8, 2, 1], [9, 2, 1], [9, 2, 1]];
seg_vec_eg = Vector{Float64}[[0.0, 0.0], [1, π/2], [√2, 3π/4], [√2, π/4], [√2/2, 3π/4],
                             [√2/2, π/4], [√2/2, 3π/4], [√2/2, π/4], [√2/2, π/4], [√2/2, 3π/4], [√2/2, π/4]];
\end{lstlisting}
木構造を描画するための関数を作成する.以下の\jl{segments_lines}は\jl{tree_info}と\jl{seg_vec}から節点位置\jl{pos}と各分節の両端点\jl{lines}を返す関数である.\jl{lines}は主に\jl{matplotlib.collections.LineCollection}で用いる (\jl{plot}を用いるより高速である).
\begin{lstlisting}[language=julia]
function segments_lines(tree_info, seg_vec, init_pos=nothing)
    num_segments = size(tree_info)[1]
    
    pos = zeros(num_segments, 2);
    if init_pos != nothing
        pos[1, :] = init_pos
    end
    
    for j in 1:num_segments
        pos[j, :] = pos[tree_info[j][1], :] + seg_vec[j][1] * [cos(seg_vec[j][2]), sin(seg_vec[j][2])]
    end
    
    lines = []
    for j in 1:num_segments
        x1, y1 = pos[tree_info[j][1], :]
        x2, y2 = pos[j, :]
        push!(lines, [(x1, y1), (x2, y2)])
    end
    return lines, pos
end;
\end{lstlisting}
\begin{lstlisting}[language=julia]
lines_eg, _ = segments_lines(tree_info_eg, seg_vec_eg);
\end{lstlisting}
木構造を描画してみよう.以下では各部位の説明を加えており,\citep{Koene2009-hv}, \citep{Cuntz2010-in}を参考に作成した.
\begin{lstlisting}[language=julia]
rc("font", family="Meiryo") # 日本語用フォント
\end{lstlisting}
\begin{lstlisting}[language=julia]
figure(figsize=(6, 6), dpi=100)
ax = PyPlot.axes()
for i in 1:length(tree_info_eg)
    x, y = lines_eg[i][1]
    dx, dy = lines_eg[i][2] .- lines_eg[i][1]
    arrow(x, y, dx, dy,width=0.01,head_width=0.1,head_length=0.1,length_includes_head=true,color="k")
end
scatter(0, 0, s=100, color="k") # root

text(3, 4, L"遠心性位数 $\gamma$"*"\n (centrifugal order)", size=10, color="tab:red", ha="center", va="center")
hlines_y = [0, 0, -1, 2];
for i in 1:4
    hlines(i-1, hlines_y[i], 3, color="gray", linestyle="dashed", linewidth=2, alpha=0.5)
    text(3, (i-1)+0.2, string(i-1), size=10, color="tab:red", ha="center", va="center")
end

arrowprops=Dict("arrowstyle" => "->", "color" => "tab:blue");
annotate("根 (root)", xy=(-0.1, 0), xytext=(-1.0, 0), size=10, color="tab:blue", ha="center", va="center", arrowprops=arrowprops)
annotate("根分節\n (root segments)", xy=(0, 0.5), xytext=(-1.6, 0.5), size=10, color="tab:blue", ha="center", va="center", arrowprops=arrowprops)
annotate("中間分節\n (internal segments)", xy=(-0.5, 1.5), xytext=(-2.2, 1.5), size=10, color="tab:blue", ha="center", va="center", arrowprops=arrowprops)
annotate("末端分節\n (terminal segments)", xy=(-1.25, 2.25), xytext=(-2.8, 2.25), size=10, color="tab:blue", ha="center", va="center", arrowprops=arrowprops)
annotate("分岐点\n (branch point)", xy=(1.8, 3), xytext=(0.6, 3), size=10, color="tab:blue", ha="center", va="center", arrowprops=arrowprops)
annotate("成長円錐\n (growth cone)", xy=(1.5, 3.5), xytext=(0.3, 3.5), size=10, color="tab:blue", ha="center", va="center", arrowprops=arrowprops)
annotate("継続点\n (continuation point)", xy=(1.5, 2.4), xytext=(1.5, 1.5), size=10, color="tab:blue", ha="center", va="center", arrowprops=arrowprops)
annotate("一組の娘枝\n (daughter branches)", xy=(-0.75, 2.25), xytext=(-1.25, 3), size=10, color="tab:purple", ha="center", va="center", arrowprops=Dict("arrowstyle" => "->", "color" => "tab:purple"))
annotate("", xy=(-1.25, 2.25), xytext=(-1.1, 2.74), ha="center", va="center", arrowprops=Dict("arrowstyle" => "->", "color" => "tab:purple"))

xlim(-4, 3); ylim(-0.3, 4); axis("off")
ax.set_aspect("equal")
tight_layout()
\end{lstlisting}
\begin{figure}[ht]
	\centering
	\includegraphics[scale=0.8, max width=\linewidth]{./fig/neuron-model/neurite-growth-model/cell009.png}
	\caption{cell009.png}
	\label{cell009.png}
\end{figure}
\subsection{Van Peltモデル}
Van PeltモデルはVan Peltらによって構築された,神経突起の成長についての現象論的モデルである \citep{Van_Pelt2002-vm}.以下では\citep{Koene2009-hv}に基づいて記述する.なお,このモデルでは軸索誘導分子 (axon guidance molecules) 等の存在は無視している.
神経突起の成長の過程には分岐(branching),伸長(elongation),転向(turn)が含まれる.簡略化のため,空間を2次元にし,分節の太さおよび成長円錐が向きを変える時のsegment history tension model (後述) を省略する.またVan Peltモデルを元にした神経回路構築ソフトウェア\textbf{NETMORPH}\index{NETMORPH} \citep{Koene2009-hv}ではシナプス結合の形成も含めたシミュレーションを行っている.
\subsubsection{分岐 (branching)}
時刻$[t_i, t_i + \Delta t]$において,$j$番目の末端分節(terminal segment)が分岐する確率は
\begin{equation}
p_{i,j} = n_i^{-E}\cdot B_{\infty} e^{\frac{-t_i}{\tau}} \left(e^{\frac{\Delta t}{\tau}} - 1\right)\cdot \frac{2^{-S\gamma_j}}{C_{n_i}}
\end{equation}
で表される.ここで,$B_{\infty}, E, S, \tau$は定数である.$\gamma_j$は$j$番目の末端分節の遠心性位数(centrifugal order)であり,$n_i$は時刻$t_i$における末端分節の総計である.さらに
\begin{equation}
{C_{n_i}} = \frac{1}{n_i}\sum\nolimits_{k = 1}^{n_i} {{2^{ - S{\gamma_k}}}}
\end{equation}
とする.$n_i^{-E}$は末端分節の総計に応じて分岐確率を変化させる項であり,$E$は競合変数(competition parameter)と呼ばれる.
$B_{\infty} e^{\frac{-t_i}{\tau}} \left(e^{\frac{\Delta t}{\tau}} - 1\right)$は経過時間に応じて分岐確率を変化させる項であり,$B_{\infty}$は$E=0$の場合の末端分節での分岐数の漸近的な期待値である.
$\frac{2^{-S\gamma_j}}{C_{n_i}}$の項は末端分節の遠心性位数に応じて分岐確率を変化させる項であり,$C_{n_i}$は正規化定数である.
$S=0$のときは末端分節は全て同じ確率で分岐するが,$S>0$のときは近位の末端分節,$S<0$のときは遠位の末端分節における分岐確率が大きくなる.
\begin{lstlisting}[language=julia]
function branching_prob(t, dt, γ, n, C, B∞, E, S, τ)
    return (n^(-E))*B∞*exp(-t/τ)*(exp(dt/τ) - 1)*(2^(-S*γ))/C
end;
\end{lstlisting}
\subsubsection{伸長 (elongation) }
末端分節が伸長する速さ$\nu_e(t_i)\ [\mu m/s]$は正規分布 $\mathcal{N}(\mu_e, \sigma_e^2)$に従うとする \citep{Van_Ooyen2014-fb}.伸長する長さは$\Delta L_j(t_i)=\nu_e(t_i) \cdot \Delta t$となる.
\subsubsection{転向 (turn)}
神経突起は真っ直ぐに伸び続けるわけではなく,向きを時折変えながら伸長する.伸長時に転向するかどうかの確率$p_d(t_i)$を次のようにする.
\begin{equation}
p_d(t_i) = r_L\Delta L_j(t_i)
\end{equation}
ただし,$r_L\ [\mu m^{-1}]$は回転率を表す.確率$p_d(t_i)$により転向する部分は新しい分節として定義する.転向する角度は\citep{Koene2009-hv}では転向角度の履歴を考慮したsegment history tension modelが導入されているが,本書では前述のように省略する.代わりに転向角度は一様分布$U(-\alpha, \alpha)\ \left(\alpha\in \left[0, \frac{\pi}{2}\right]\right)$に従うとする.
分岐した際にも娘枝の長さと角度の設定が必要となる.ここでは長さは末端分節の伸長と同じ正規分布に従うとする.また,分岐角度は2つの娘枝について一様分布$U(0, \beta_1),\ U(-\beta_2, 0)\ \left(\beta_1, \beta_2\in \left[0, \frac{\pi}{2}\right]\right)$にそれぞれ従うとする.
以上をまとめてシミュレーションを実装する.
\begin{lstlisting}[language=julia]
function neurite_growth_model(
        tree_info_init, seg_vec_init, nt, dt, B∞, E, S, τ, μₑ, σₑ,
        turn_rate=5, max_branch_angle=0.1π, max_turn_angle=5e-3π,
        history_num=3)
    tree_info = copy(tree_info_init)
    seg_vec = copy(seg_vec_init)
    num_branching = 0
    
    if history_num > 1
        tree_info_history = []
        seg_vec_history = []
        history_timing = floor.(Int, collect(range(1, nt, length=history_num+1)))[2:end] 
    end
    
    @showprogress for tt in 1:nt
        t = tt*dt # Current time
        
        n, C = 0.0, 0.0
        for j in 1:size(tree_info)[1]
            if tree_info[j][3] == 1
                n += 1
                C += 2 ^(-S*tree_info[j][2])
            end
        end
        C /= n
        
        for j in 1:size(tree_info)[1]
            if tree_info[j][3] == 1
                γ = tree_info[j][2]
                p_branch = branching_prob(t, dt, γ, n, C, B∞, E, S, τ)
                if p_branch > rand()
                    # Neurite branching
                    num_branching += 1
                    branch_lens = (μₑ .+ randn(2) * σₑ)*dt # branch length
                    branch_angles = [1, -1] .* rand(Uniform(0, max_branch_angle), 2)
                    for k in 1:2 # add two daughter branches
                        push!(tree_info, [j, γ+1, 1]) 
                        push!(seg_vec, [branch_lens[k], seg_vec[j][2]+branch_angles[k]])
                    end
                    tree_info[j][3] = 0 # reset centrifugal order of original branch
                else
                    # Neurite elongation
                    Δlen = (μₑ .+ randn() * σₑ)*dt
                    p_turn = turn_rate*Δlen
                    if p_turn > rand()
                        push!(tree_info, [j, γ, 1]) # add segment
                        seg_angle = seg_vec[j][2] + rand(Uniform(-max_turn_angle, max_turn_angle))
                        push!(seg_vec, [Δlen, seg_angle])
                        tree_info[j][3] = 0 # reset centrifugal order of original branch
                    else
                        seg_vec[j][1] += Δlen
                    end
                end
            end
        end
        if history_num > 1
            if tt in history_timing
                push!(tree_info_history, copy(tree_info))
                push!(seg_vec_history, copy(seg_vec))
            end
        end
    end
    println("Num. branching: ", num_branching)
    if history_num > 1
        return tree_info_history, seg_vec_history
    else
        return tree_info, seg_vec
    end
end;
\end{lstlisting}
パラメータを設定する.このパラメータは\citep{Koene2009-hv}, \citep{Van_Ooyen2014-fb}に基づいている.Van Peltモデルにおいて錐体細胞のような複雑な形態を作成するには,各部位に分割してシミュレーションし結合することが必要となる.今回は軸索のパラメータを用いる.
\begin{lstlisting}[language=julia]
B∞ = 13.2; E = 0.319; S = -0.205; τ = 1681541; 
μₑ = 2.14e-4; σₑ = 3.98e-4;

turn_rate = 3
max_branch_angle = 0.25π
max_turn_angle = 1e-2π

T = 18*24*60*60 # duration of growth; convert 18 days to sec
dt = 200 # sec
nt = round(Int, T/dt);
\end{lstlisting}
シミュレーションを実行する.
\begin{lstlisting}[language=julia]
history_num = 3 # 記録する状態の数; 等間隔で記録.
init_branch_num = 2 # 初めの神経突起枝の数

# 細胞体のパラメータ
tree_info_init = [[1, 0, 0]]
seg_vec_init = [[0.0, 0.0]]

# 初期神経突起のパラメータ (神経突起が放射状に出るように設定)
Random.seed!(0)
for i in 1:init_branch_num
    push!(tree_info_init, [1, 0, 1])
    push!(seg_vec_init, [(μₑ + randn()*σₑ)*dt, (i-1)/init_branch_num*2π+1e-2*randn()])
end

@time tree_info_history, seg_vec_history = neurite_growth_model(
    tree_info_init, seg_vec_init, nt, dt, B∞, E, S, τ, μₑ, σₑ,
    turn_rate, max_branch_angle, max_turn_angle, history_num);
\end{lstlisting}
結果を表示する.初めの神経突起の数を2にしていてもそれ以上出ているように見えるのは最初期に分岐しているためである (細胞体を描画しなければ確認できる).
\begin{lstlisting}[language=julia]
maxwidth, minwidth = 1.5, 0.5 # 描画する神経突起の最大/最小の太さ
days_range = floor.(Int, collect(range(1, 18, history_num+1)))[2:end]

fig, ax = subplots(1, history_num, sharex=true, sharey=true)
for i in 1:history_num
    lines, _ = segments_lines(tree_info_history[i], seg_vec_history[i]);
    linewidths = range(maxwidth, minwidth, length=length(lines));

    line_segments = matplotlib.collections.LineCollection(lines, linewidths=linewidths, color="k")
    ax[i].add_collection(line_segments)  # dendrite
    ax[i].scatter(0, 0, s=50, color="k") # soma
    ax[i].set_xlabel(L"$x\ (\mu m)$"); ax[i].set_ylabel(L"$y\ (\mu m)$")
    ax[i].set_aspect("equal")
    ax[i].set_title(string(days_range[i])*" days")
    ax[i].label_outer()
end
\end{lstlisting}
\begin{figure}[ht]
	\centering
	\includegraphics[scale=0.8, max width=\linewidth]{./fig/neuron-model/neurite-growth-model/cell019.png}
	\caption{cell019.png}
	\label{cell019.png}
\end{figure}
対称性の破れを考慮していないので,円系に成長している.
ToDo: 神経細胞極性についての記述.
 % ok

\chapter{シナプス伝達のモデル}
% \section{予測符号化}
\subsection{観測世界の階層的予測}
\textbf{階層的予測符号化(hierarchical predictive coding; HPC)} は\cite{Rao1999-zv}により導入された.構築するネットワークは入力層を含め,3層のネットワークとする.LGNへの入力として画像 $\mathbf{x} \in \mathbb{R}^{n_0}$を考える.画像 $\mathbf{x}$ の観測世界における隠れ変数,すなわち\textbf{潜在変数} (latent variable)を$\mathbf{r} \in \mathbb{R}^{n_1}$とし,ニューロン群によって発火率で表現されているとする (真の変数と $\mathbf{r}$は異なるので文字を分けるべきだが簡単のためにこう表す).このとき,


\mathbf{x} = f(\mathbf{U}\mathbf{r}) + \boldsymbol{\epsilon}


が成立しているとする.ただし,$f(\cdot)$は活性化関数 (activation function),$\mathbf{U} \in \mathbb{R}^{n_0 \times n_1}$は重み行列である.
$\boldsymbol{\epsilon} \in \mathbb{R}^{n_0}$ は $\mathcal{N}(\mathbf{0}, \sigma^2 \mathbf{I})$ からサンプリングされるとする.

潜在変数 $\mathbf{r}$はさらに高次 (higher-level)の潜在変数 $\mathbf{r}^h$により,次式で表現される.


\mathbf{r} = \mathbf{r}^{td}+\boldsymbol{\epsilon}^{td}=f(\mathbf{U}^h \mathbf{r}^h)+\boldsymbol{\epsilon}^{td}


ただし,Top-downの予測信号を $\mathbf{r}^{td}:=f(\mathbf{U}^h \mathbf{r}^h)$とした.また,$\mathbf{r}^{td} \in \mathbb{R}^{n_1}$, $\mathbf{r}^{h} \in \mathbb{R}^{n_2}$, $\mathbf{U}^h \in \mathbb{R}^{n_1 \times n_2}$ である.
$\boldsymbol{\epsilon}^{td} \in \mathbb{R}^{n_1}$は$\mathcal{N}(\mathbf{0}$, $\sigma_{td}^2 \mathbf{I}$) からサンプリングされるとする.

話は飛ぶが,Predictive codingのネットワークの特徴は
\begin{itemize}
\item 階層的な構造
\item 高次による低次の予測 (Feedback or Top-down信号)
\item 低次から高次への誤差信号の伝搬 (Feedforward or Bottom-up 信号)
\end{itemize}

である.ここまでは高次表現による低次表現の予測,というFeedback信号について説明してきたが,この部分はSparse codingでも同じである.それではPredictive codingのもう一つの要となる,低次から高次への予測誤差の伝搬というFeedforward信号はどのように導かれるのだろうか.結論から言えば,これは\textbf{復元誤差 (reconstruction error)の最小化を行う再帰的ネットワーク (recurrent network)を考慮することで自然に導かれる}.

 % ok
% \section{予測符号化}
\subsection{観測世界の階層的予測}
\textbf{階層的予測符号化(hierarchical predictive coding; HPC)} は\cite{Rao1999-zv}により導入された.構築するネットワークは入力層を含め,3層のネットワークとする.LGNへの入力として画像 $\mathbf{x} \in \mathbb{R}^{n_0}$を考える.画像 $\mathbf{x}$ の観測世界における隠れ変数,すなわち\textbf{潜在変数} (latent variable)を$\mathbf{r} \in \mathbb{R}^{n_1}$とし,ニューロン群によって発火率で表現されているとする (真の変数と $\mathbf{r}$は異なるので文字を分けるべきだが簡単のためにこう表す).このとき,


\mathbf{x} = f(\mathbf{U}\mathbf{r}) + \boldsymbol{\epsilon}


が成立しているとする.ただし,$f(\cdot)$は活性化関数 (activation function),$\mathbf{U} \in \mathbb{R}^{n_0 \times n_1}$は重み行列である.
$\boldsymbol{\epsilon} \in \mathbb{R}^{n_0}$ は $\mathcal{N}(\mathbf{0}, \sigma^2 \mathbf{I})$ からサンプリングされるとする.

潜在変数 $\mathbf{r}$はさらに高次 (higher-level)の潜在変数 $\mathbf{r}^h$により,次式で表現される.


\mathbf{r} = \mathbf{r}^{td}+\boldsymbol{\epsilon}^{td}=f(\mathbf{U}^h \mathbf{r}^h)+\boldsymbol{\epsilon}^{td}


ただし,Top-downの予測信号を $\mathbf{r}^{td}:=f(\mathbf{U}^h \mathbf{r}^h)$とした.また,$\mathbf{r}^{td} \in \mathbb{R}^{n_1}$, $\mathbf{r}^{h} \in \mathbb{R}^{n_2}$, $\mathbf{U}^h \in \mathbb{R}^{n_1 \times n_2}$ である.
$\boldsymbol{\epsilon}^{td} \in \mathbb{R}^{n_1}$は$\mathcal{N}(\mathbf{0}$, $\sigma_{td}^2 \mathbf{I}$) からサンプリングされるとする.

話は飛ぶが,Predictive codingのネットワークの特徴は
\begin{itemize}
\item 階層的な構造
\item 高次による低次の予測 (Feedback or Top-down信号)
\item 低次から高次への誤差信号の伝搬 (Feedforward or Bottom-up 信号)
\end{itemize}

である.ここまでは高次表現による低次表現の予測,というFeedback信号について説明してきたが,この部分はSparse codingでも同じである.それではPredictive codingのもう一つの要となる,低次から高次への予測誤差の伝搬というFeedforward信号はどのように導かれるのだろうか.結論から言えば,これは\textbf{復元誤差 (reconstruction error)の最小化を行う再帰的ネットワーク (recurrent network)を考慮することで自然に導かれる}.


% \section{予測符号化}
\subsection{観測世界の階層的予測}
\textbf{階層的予測符号化(hierarchical predictive coding; HPC)} は\cite{Rao1999-zv}により導入された.構築するネットワークは入力層を含め,3層のネットワークとする.LGNへの入力として画像 $\mathbf{x} \in \mathbb{R}^{n_0}$を考える.画像 $\mathbf{x}$ の観測世界における隠れ変数,すなわち\textbf{潜在変数} (latent variable)を$\mathbf{r} \in \mathbb{R}^{n_1}$とし,ニューロン群によって発火率で表現されているとする (真の変数と $\mathbf{r}$は異なるので文字を分けるべきだが簡単のためにこう表す).このとき,


\mathbf{x} = f(\mathbf{U}\mathbf{r}) + \boldsymbol{\epsilon}


が成立しているとする.ただし,$f(\cdot)$は活性化関数 (activation function),$\mathbf{U} \in \mathbb{R}^{n_0 \times n_1}$は重み行列である.
$\boldsymbol{\epsilon} \in \mathbb{R}^{n_0}$ は $\mathcal{N}(\mathbf{0}, \sigma^2 \mathbf{I})$ からサンプリングされるとする.

潜在変数 $\mathbf{r}$はさらに高次 (higher-level)の潜在変数 $\mathbf{r}^h$により,次式で表現される.


\mathbf{r} = \mathbf{r}^{td}+\boldsymbol{\epsilon}^{td}=f(\mathbf{U}^h \mathbf{r}^h)+\boldsymbol{\epsilon}^{td}


ただし,Top-downの予測信号を $\mathbf{r}^{td}:=f(\mathbf{U}^h \mathbf{r}^h)$とした.また,$\mathbf{r}^{td} \in \mathbb{R}^{n_1}$, $\mathbf{r}^{h} \in \mathbb{R}^{n_2}$, $\mathbf{U}^h \in \mathbb{R}^{n_1 \times n_2}$ である.
$\boldsymbol{\epsilon}^{td} \in \mathbb{R}^{n_1}$は$\mathcal{N}(\mathbf{0}$, $\sigma_{td}^2 \mathbf{I}$) からサンプリングされるとする.

話は飛ぶが,Predictive codingのネットワークの特徴は
\begin{itemize}
\item 階層的な構造
\item 高次による低次の予測 (Feedback or Top-down信号)
\item 低次から高次への誤差信号の伝搬 (Feedforward or Bottom-up 信号)
\end{itemize}

である.ここまでは高次表現による低次表現の予測,というFeedback信号について説明してきたが,この部分はSparse codingでも同じである.それではPredictive codingのもう一つの要となる,低次から高次への予測誤差の伝搬というFeedforward信号はどのように導かれるのだろうか.結論から言えば,これは\textbf{復元誤差 (reconstruction error)の最小化を行う再帰的ネットワーク (recurrent network)を考慮することで自然に導かれる}.

\lstinputlisting[language=julia]{./text/synapse-model/expo-synapse/001.jl}
\lstinputlisting[language=julia]{./text/synapse-model/expo-synapse/002.jl}
\lstinputlisting[language=julia]{./text/synapse-model/expo-synapse/003.jl}
\begin{figure}[ht]
	\centering
	\includegraphics[scale=0.8, max width=\linewidth]{./fig/synapse-model/expo-synapse/cell003.png}
	\caption{cell003.png}
	\label{cell003.png}
\end{figure}
2種類の指数関数型シナプスの動態.破線は単一指数関数型シナプスで, 実線は二重指数関数型シナプスである.
変更しない定数を保持する \jl{struct} の \jl{FHNParameter} と, 変数を保持する \jl{mutable struct} の \jl{FHN} を作成する.
いくつかの処理について解説しておく.まず,一番目のforループ内の\jl{v[i]}の\jl{((dt*tcount) > (tlast[i] + tref))}は最後にニューロンが発火した時刻\jl{tlast[i]}に不応期\jl{tref}を足した時刻よりも現在の時刻\jl{dt*tcount[1]}が大きければ膜電位の更新を許可し,小さければ更新しない.二番目のforループにおける\jl{fire[i]}はニューロンの膜電位が閾値電位\jl{vthr}を超えたら\jl{True}となる.\jl{v[i]}などの更新式にある\jl{ifelse(a, b, c)}はaが\jl{True}の時はbを返し,\jl{False}の時はcを返す関数であり,\jl{v[i] = ifelse(fire[i], vreset, v[i])}は\jl{fire[i]}が\jl{True}なら\jl{v[i]}をリセット電位\jl{vreset}とし,そうでなければそのままの値を返すという処理である.同様にして\jl{tlast[i]}は発火したときにその時刻を記録する変数となっている.なお,\jl{v_[i] = ifelse(fire[i], vpeak, v[i])}は実際のシミュレーションにおいて意味をなさない.単に発火時の電位\jl{vpeak}を含めて記録すると描画時の見栄えが良いというだけである.

これらの\jl{struct}と関数を用いてシミュレーションを実行する.\jl{I} はHHモデルのときと同じように矩形波を入力する.実は\jl{I}は入力電流ではなく入力電流に比例する量となっているが,これは膜抵抗を乗じた後の値であると考えるとよい.


% \section{動力学モデル}
\subsection{チャネル動態の動力学的表現}
指数関数型シナプスとモデルの振る舞いはほぼ同一だが, 式の構成が少し異なるモデルとして\textbf{動力学モデル} (Kinetic model, またはMarkov kinetic model)がある ([Destexhe et al., 1994](https://link.springer.com/article/10.1007/BF00961734); [Destexhe et al., 2002](http://cns.iaf.cnrs-gif.fr/files/handbook98.pdf)).動力学モデルはHHモデルのゲート変数の式と類似した式で表される.このモデルではチャネルが開いた状態(Open)と閉じた状態(Close), および神経伝達物質(neurotransmitter)の放出状態(T)の2つの要素に関する状態がある.また, 閉$\to$開の反応速度を$\alpha$, 開$\to$閉の反応速度を$\beta$とする.このとき,これらを表す状態遷移の式は次のようになる.


\begin{equation}
\text{Close}+\text{T}  \underset{\beta}{\overset{\alpha}{\rightleftharpoons}}\text{Open}    
\end{equation}


ここで, シナプス動態を$r$とすると


\begin{equation}
\frac{dr}{dt}=\alpha T (1-r) - \beta r
\end{equation}


となる.ただし, Tはシナプス前細胞が発火したときにインパルス的に1だけ増加するとする.また, $\alpha, \beta$は速度なので, 時定数の逆数であることに注意しよう. $\alpha=2000 \text{ms}^{-1}$, $\beta=200 \text{ms}^{-1}$とすると, シナプス動態は次のようになる.
\lstinputlisting[language=julia]{./text/synapse-model/kinetic-synapse/001.jl}
\lstinputlisting[language=julia]{./text/synapse-model/kinetic-synapse/002.jl}
\begin{figure}[ht]
	\centering
	\includegraphics[scale=0.8, max width=\linewidth]{./fig/synapse-model/kinetic-synapse/cell002.png}
	\caption{cell002.png}
	\label{cell002.png}
\end{figure}
\subsection{Hodgkin-Huxleyモデルにおけるシナプスモデル}
これまで明示的にスパイクの発生が表現されたモデルを用いてきたが,HHモデルでは単なる膜電位の変数があるのみである.ここでは前述した動力学的モデルを用いてHHモデルにおけるシナプス動態の記述を行う ([Destexhe et al., 1994](https://www.mitpressjournals.org/doi/10.1162/neco.1994.6.1.14); [Batista et al., 2014](https://www.sciencedirect.com/science/article/pii/S0378437114004592)).

$r_{j}$を$j$番目のニューロンのpre-synaptic dynamicsとすると,$r_{j}$は次式に従う.


\frac{\mathrm{d} r_{j}}{\mathrm{d} t}=\left(\frac{1}{\tau_{r}}-\frac{1}{\tau_{d}}\right) \frac{1-r_{j}}{1+\exp \left(-V_{j}+V_{0}\right)}-\frac{r_{j}}{\tau_{d}}


ただし,時定数 $\tau_r=0.5, \tau_d = 8$ (ms), 反転電位 $V_0 = -20$ (mV)とする.前節で既に$r$の描画は行ったが,パルス波を印加した場合の挙動を確認する.
\lstinputlisting[language=julia]{./text/synapse-model/kinetic-synapse/004.jl}
シミュレーションを実行する.
\lstinputlisting[language=julia]{./text/synapse-model/kinetic-synapse/006.jl}
描画してみる.
\lstinputlisting[language=julia]{./text/synapse-model/kinetic-synapse/008.jl}
\begin{figure}[ht]
	\centering
	\includegraphics[scale=0.8, max width=\linewidth]{./fig/appendix/slow-feature-analysis/cell008.png}
	\caption{cell008.png}
	\label{cell008.png}
\end{figure}

% \section{予測符号化}
\subsection{観測世界の階層的予測}
\textbf{階層的予測符号化(hierarchical predictive coding; HPC)} は\cite{Rao1999-zv}により導入された.構築するネットワークは入力層を含め,3層のネットワークとする.LGNへの入力として画像 $\mathbf{x} \in \mathbb{R}^{n_0}$を考える.画像 $\mathbf{x}$ の観測世界における隠れ変数,すなわち\textbf{潜在変数} (latent variable)を$\mathbf{r} \in \mathbb{R}^{n_1}$とし,ニューロン群によって発火率で表現されているとする (真の変数と $\mathbf{r}$は異なるので文字を分けるべきだが簡単のためにこう表す).このとき,


\mathbf{x} = f(\mathbf{U}\mathbf{r}) + \boldsymbol{\epsilon}


が成立しているとする.ただし,$f(\cdot)$は活性化関数 (activation function),$\mathbf{U} \in \mathbb{R}^{n_0 \times n_1}$は重み行列である.
$\boldsymbol{\epsilon} \in \mathbb{R}^{n_0}$ は $\mathcal{N}(\mathbf{0}, \sigma^2 \mathbf{I})$ からサンプリングされるとする.

潜在変数 $\mathbf{r}$はさらに高次 (higher-level)の潜在変数 $\mathbf{r}^h$により,次式で表現される.


\mathbf{r} = \mathbf{r}^{td}+\boldsymbol{\epsilon}^{td}=f(\mathbf{U}^h \mathbf{r}^h)+\boldsymbol{\epsilon}^{td}


ただし,Top-downの予測信号を $\mathbf{r}^{td}:=f(\mathbf{U}^h \mathbf{r}^h)$とした.また,$\mathbf{r}^{td} \in \mathbb{R}^{n_1}$, $\mathbf{r}^{h} \in \mathbb{R}^{n_2}$, $\mathbf{U}^h \in \mathbb{R}^{n_1 \times n_2}$ である.
$\boldsymbol{\epsilon}^{td} \in \mathbb{R}^{n_1}$は$\mathcal{N}(\mathbf{0}$, $\sigma_{td}^2 \mathbf{I}$) からサンプリングされるとする.

話は飛ぶが,Predictive codingのネットワークの特徴は
\begin{itemize}
\item 階層的な構造
\item 高次による低次の予測 (Feedback or Top-down信号)
\item 低次から高次への誤差信号の伝搬 (Feedforward or Bottom-up 信号)
\end{itemize}

である.ここまでは高次表現による低次表現の予測,というFeedback信号について説明してきたが,この部分はSparse codingでも同じである.それではPredictive codingのもう一つの要となる,低次から高次への予測誤差の伝搬というFeedforward信号はどのように導かれるのだろうか.結論から言えば,これは\textbf{復元誤差 (reconstruction error)の最小化を行う再帰的ネットワーク (recurrent network)を考慮することで自然に導かれる}.

 % ok
% \section{動的シナプス}
シナプス前活動に応じて\textbf{シナプス伝達効率}\index{しなぷすでんたつこうりつ@シナプス伝達効率} (synaptic efficacy)が動的に変化する性質を\textbf{短期的シナプス可塑性}\index{たんきてきしなぷすかそせい@短期的シナプス可塑性} (Short-term synaptic plasticity; STSP)といい,このような性質を持つシナプスを\textbf{動的シナプス}\index{どうてきしなぷす@動的シナプス} (dynamical synapses)と呼ぶ.シナプス伝達効率が減衰する現象を短期抑圧 (short-term depression; STD),増強する現象を短期促通(short-term facilitation; STF)という.さらにそれぞれに対応するシナプスを減衰シナプス,増強シナプスという.ここでは\citep{Mongillo2008-kk}および\citep{Orhan2019-rq}で用いられている定式化を使用する.
\begin{align}
\frac{\mathrm{d} x(t)}{\mathrm{d} t}=\frac{1-x(t)}{\tau_{x}}-u(t) x(t) r(t) \Delta t \\
\frac{\mathrm{d} u(t)}{\mathrm{d} t}=\frac{U-u(t)}{\tau_{u}}+U(1-u(t)) r(t) \Delta t
\end{align}
ただし,$x$を利用可能な神経伝達物質の量, $u$を利用されている神経伝達物質の量(the neurotransmitter utilization), $\tau_x$は神経伝達物質の時定数 , $\tau_u$はutilization, $U$はincrement , $\Delta t$を時間幅とする.ここでは$\tau_x=$(200 ms/1,500 ms; facilitating/depressing),  $\tau_u=$(1,500 ms/200 ms; facilitating/depressing), $U=$(0.15/0.45; facilitating/depressing), $\Delta t=$10msとする.
\begin{lstlisting}[language=julia]
using PyPlot
rc("axes.spines", top=false, right=false)
\end{lstlisting}
\begin{lstlisting}[language=julia]
# ms
Uf, τfₓ, τfᵤ = 0.15, 200, 1500
Ud, τdₓ, τdᵤ = 0.45, 1500, 200
dt = 1
T = 4000
tarray = 1:dt:T;
nt = Int(T/dt);
s = zeros(nt) # stimuli
s[500:150:2000] .= 1;
s[2500:200:3000] .= 1;
\end{lstlisting}
\begin{lstlisting}[language=julia]
# short-term synaptic plasticity
function stsp(dt, T, s, U, τₓ, τᵤ, τₛ=30)
    nt = Int(T/dt);
    αₓ, αᵤ, αₛ = dt/τₓ, dt/τᵤ, dt/τₛ # 時定数を減衰率に変換

    u, x, r = zeros(nt), zeros(nt), zeros(nt)
    u[1], x[1] = U, 1
    
    for t in 1:nt-1
        x[t+1] = x[t] + αₓ*(1-x[t]) - u[t]*x[t]*s[t]
        u[t+1] = u[t] + αᵤ*(U-u[t]) + U*(1-u[t])*s[t]
        x[t+1], u[t+1] = clamp.([x[t+1], u[t+1]], 0, 1) # for numerical stability
        r[t+1] = (1-αₛ)*r[t] + u[t]*x[t]*s[t]/U
    end
    return u, x, r
end;
\end{lstlisting}
\begin{lstlisting}[language=julia]
# simulation
uf, xf, rf = stsp(dt, T, s, Uf, τfₓ, τfᵤ)
ud, xd, rd = stsp(dt, T, s, Ud, τdₓ, τdᵤ);

# compute synaptic efficacy
xuf = uf .* xf / Uf 
xud = ud .* xd / Ud; 
\end{lstlisting}
\begin{lstlisting}[language=julia]
fig, axes = subplots(4, 2, figsize=(6, 5), sharex="col", height_ratios=[1, 2, 2, 2])
for i in 1:2
    axes[1,i].plot(tarray, s, "k");
    axes[2,i].set_ylim(0, 1.1); 
end
axes[1,1].set_title("Facilitating synapse"); axes[1,1].set_ylabel("Presynaptic\n spike");
axes[1,2].set_title("Depressing synapse"); 
axes[2,1].plot(tarray, uf); axes[2,1].plot(tarray, xf, "tab:red"); 
axes[2,1].set_ylabel("Synaptic variables"); 
axes[2,2].plot(tarray, ud, label=L"$u$"); axes[2,2].plot(tarray, xd, "tab:red", label=L"$x$"); axes[2,2].legend(ncol=2)
axes[3,1].plot(tarray, xuf, "k"); axes[3,1].set_ylabel("Synaptic\n efficacy")
axes[3,2].plot(tarray, xud, "k"); 
axes[4,1].plot(tarray, rf, "k"); axes[4,1].set_xlabel("Time (ms)"); axes[4,1].set_ylabel("Postsynaptic current")
axes[4,2].plot(tarray, rd, "k"); axes[4,2].set_xlabel("Time (ms)")
fig.align_labels()
fig.tight_layout()
\end{lstlisting}
\begin{figure}[ht]
	\centering
	\includegraphics[scale=0.8, max width=\linewidth]{./fig/synapse-model/dynamical-synapses/cell005.png}
	\caption{cell005.png}
	\label{cell005.png}
\end{figure}


\chapter{神経回路網の演算処理}
% \section{予測符号化}
\subsection{観測世界の階層的予測}
\textbf{階層的予測符号化(hierarchical predictive coding; HPC)} は\cite{Rao1999-zv}により導入された.構築するネットワークは入力層を含め,3層のネットワークとする.LGNへの入力として画像 $\mathbf{x} \in \mathbb{R}^{n_0}$を考える.画像 $\mathbf{x}$ の観測世界における隠れ変数,すなわち\textbf{潜在変数} (latent variable)を$\mathbf{r} \in \mathbb{R}^{n_1}$とし,ニューロン群によって発火率で表現されているとする (真の変数と $\mathbf{r}$は異なるので文字を分けるべきだが簡単のためにこう表す).このとき,


\mathbf{x} = f(\mathbf{U}\mathbf{r}) + \boldsymbol{\epsilon}


が成立しているとする.ただし,$f(\cdot)$は活性化関数 (activation function),$\mathbf{U} \in \mathbb{R}^{n_0 \times n_1}$は重み行列である.
$\boldsymbol{\epsilon} \in \mathbb{R}^{n_0}$ は $\mathcal{N}(\mathbf{0}, \sigma^2 \mathbf{I})$ からサンプリングされるとする.

潜在変数 $\mathbf{r}$はさらに高次 (higher-level)の潜在変数 $\mathbf{r}^h$により,次式で表現される.


\mathbf{r} = \mathbf{r}^{td}+\boldsymbol{\epsilon}^{td}=f(\mathbf{U}^h \mathbf{r}^h)+\boldsymbol{\epsilon}^{td}


ただし,Top-downの予測信号を $\mathbf{r}^{td}:=f(\mathbf{U}^h \mathbf{r}^h)$とした.また,$\mathbf{r}^{td} \in \mathbb{R}^{n_1}$, $\mathbf{r}^{h} \in \mathbb{R}^{n_2}$, $\mathbf{U}^h \in \mathbb{R}^{n_1 \times n_2}$ である.
$\boldsymbol{\epsilon}^{td} \in \mathbb{R}^{n_1}$は$\mathcal{N}(\mathbf{0}$, $\sigma_{td}^2 \mathbf{I}$) からサンプリングされるとする.

話は飛ぶが,Predictive codingのネットワークの特徴は
\begin{itemize}
\item 階層的な構造
\item 高次による低次の予測 (Feedback or Top-down信号)
\item 低次から高次への誤差信号の伝搬 (Feedforward or Bottom-up 信号)
\end{itemize}

である.ここまでは高次表現による低次表現の予測,というFeedback信号について説明してきたが,この部分はSparse codingでも同じである.それではPredictive codingのもう一つの要となる,低次から高次への予測誤差の伝搬というFeedforward信号はどのように導かれるのだろうか.結論から言えば,これは\textbf{復元誤差 (reconstruction error)の最小化を行う再帰的ネットワーク (recurrent network)を考慮することで自然に導かれる}.

\lstinputlisting[language=julia]{./text/neuronal-computation/neuronal-arithmetic/001.jl}
\lstinputlisting[language=julia]{./text/neuronal-computation/neuronal-arithmetic/002.jl}
\lstinputlisting[language=julia]{./text/neuronal-computation/neuronal-arithmetic/003.jl}
\lstinputlisting[language=julia]{./text/neuronal-computation/neuronal-arithmetic/004.jl}
\lstinputlisting[language=julia]{./text/neuronal-computation/neuronal-arithmetic/005.jl}
\lstinputlisting[language=julia]{./text/neuronal-computation/neuronal-arithmetic/006.jl}
\lstinputlisting[language=julia]{./text/neuronal-computation/neuronal-arithmetic/007.jl}
\lstinputlisting[language=julia]{./text/neuronal-computation/neuronal-arithmetic/008.jl}
\lstinputlisting[language=julia]{./text/neuronal-computation/neuronal-arithmetic/009.jl}
\begin{figure}[ht]
	\centering
	\includegraphics[scale=0.8, max width=\linewidth]{./fig/bayesian-brain/mcmc/cell009.png}
	\caption{cell009.png}
	\label{cell009.png}
\end{figure}
 % ok

\chapter{局所学習則}
% \section{Hebb則と教師なし学習}
\subsection{Hebb則}
神経回路はどのようにして自己組織化するのだろうか.1940年代にカナダの心理学者Donald O. Hebbにより著書"The Organization of Behavior"\citep{Hebb1949-iv} で提案された学習則は「細胞Aが反復的または持続的に細胞Bの発火に関与すると,細胞Aが細胞Bを発火させる効率が向上するような成長過程または代謝変化が一方または両方の細胞に起こる」というものであった.すなわち,発火に時間的相関のある細胞間のシナプス結合を強化するという学習則である.これを\textbf{Hebbの学習則 (Hebbian learning rule)}\index{Hebbのがくしゅうそく (Hebbian learning rule)@Hebbの学習則 (Hebbian learning rule)} あるいは\textbf{Hebb則(Hebb's rule)}\index{Hebbのり(Hebb's rule)@Hebb則(Hebb's rule)} という.Hebb則は (Hebb自身ではなく) Shatzにより"cells that fire together wire together" (共に活動する細胞は共に結合する)と韻を踏みながら短く言い換えられている \citep{Shatz1992-he}.

\subsubsection{Hebb則の導出}
数式でHebb則を表してみよう.$n$個のシナプス前細胞と$m$個の後細胞の発火率をそれぞれ$\mathbf{x}\in \mathbb{R}^n, \mathbf{y}\in \mathbb{R}^m$ とする.前細胞と後細胞間のシナプス結合強度を表す行列を$\mathbf{W}\in \mathbb{R}^{m\times n}$とし,$\mathbf{y}=\mathbf{W}\mathbf{x}$が成り立つとする.このようなモデルを線形ニューロンモデル (Linear neuron model) という.このとき,Hebb則は


\begin{equation}
\tau\frac{d\mathbf{W}}{dt}=\phi(\mathbf{y})\varphi(\mathbf{x})^\top
\end{equation}


として表される.ただし,$\tau$は時定数であり,$\eta\triangleq1/\tau$ は\textbf{学習率 (learning rate)}\index{がくしゅうりつ (learning rate)@学習率 (learning rate)} と呼ばれる学習の速さを決定するパラメータとなる.$\varphi(\cdot)$および$\phi(\cdot)$は,それぞれシナプス前細胞および後細胞の活動量に応じて重みの変化量を決定する関数である.ただし,$\varphi(\cdot), \phi(\cdot)$は基本的に恒等関数に設定される場合が多い.この場合,Hebb則は$
\tau\dfrac{d\mathbf{W}}{dt}=\mathbf{y}\mathbf{x}^\top=(\text{post})\cdot (\text{pre})^\top
$と簡潔に表現される.

このHebb則は数学的に導出されたものではないが,特定の目的関数を神経活動及び重みを変化させて最適化するようなネットワークを構築すれば自然に出現する.このようなネットワークを\textbf{エネルギーベースモデル (energy-based models)}\index{えねるぎーべーすもでる (energy-based models)@エネルギーベースモデル (energy-based models)} といい,次章で扱う.エネルギーベースモデルでは,先にエネルギー関数 (あるいはコスト関数) $\mathcal{E}$ を定義し,その目的関数を最小化するような神経活動 $\mathbf{z}$ および重み行列 $\mathbf{W}$ のダイナミクスをそれぞれ,


\begin{equation}
\frac{d \mathbf{z}}{dt}\propto-\frac{\partial \mathcal{E}}{\partial \mathbf{z}},\ \frac{d \mathbf{W}}{dt}\propto-\frac{\partial \mathcal{E}}{\partial \mathbf{W}}
\end{equation}


として導出する.この手順の逆を行う,すなわち先に神経細胞の活動ダイナミクスを定義し,神経活動で積分することで神経回路のエネルギー関数$\mathcal{E}$を導出し,さらに $\mathcal{E}$ を重み行列で微分することでHebb則が導出できる \citep{Isomura2020-sn}.Hebb則の導出を連続時間線形ニューロンモデル $\dfrac{d\mathbf{y}}{dt}=\mathbf{W}\mathbf{x}$ を例にして考えよう.ここで$\dfrac{\partial\mathcal{E}}{\partial\mathbf{y}}\triangleq-\dfrac{d\mathbf{y}}{dt}$となるようなエネルギー関数 $\mathcal{E}(\mathbf{x}, \mathbf{y}, \mathbf{W})$を仮定すると,


\begin{equation}
\mathcal{E}(\mathbf{x}, \mathbf{y}, \mathbf{W})=-\int \mathbf{W}\mathbf{x}\ d\mathbf{y}=-\mathbf{y}^\top \mathbf{W}\mathbf{x} \in \mathbb{R}
\end{equation}


となる.これをさらに$\mathbf{W}$で微分すると,


\begin{equation}
\dfrac{\partial\mathcal{E}}{\partial\mathbf{W}}=-\mathbf{y}\mathbf{x}^\top\Rightarrow
\frac{d\mathbf{W}}{dt}=-\dfrac{\partial\mathcal{E}}{\partial\mathbf{W}}=\mathbf{y}\mathbf{x}^\top
\end{equation}


となり,Hebb則が導出できる (簡単のため時定数は1とした).
\subsection{Hebb則の安定化とLTP/LTD}
\subsubsection{BCM則}
Hebb則には問題点があり,シナプス結合強度が際限なく増大するか,0に近づくこととなってしまう.これを数式で確認しておこう.前細胞と後細胞がそれぞれ1つの場合を考える.2細胞間の結合強度を$w\ (>0)$ とし,$y=wx$が成り立つとすると,Hebb則は$\dfrac{dw}{dt}=\eta yx=\eta x^2w$となる.この場合,$\eta x^2>1$ なら $\lim_{t\to\infty} w= \infty$, $\eta x^2<1$ なら $\lim_{t\to\infty} w= 0$ となる.当然,生理的にシナプス結合強度が無限大となることはあり得ないが,不安定なほど大きくなってしまう可能性があることに違いはない.このため,Hebb則を安定化させるための修正が必要とされた.

Cooper, Liberman, Ojaらにより頭文字をとって\textbf{CLO則}\index{CLOのり@CLO則} (CLO rule) が提案された \citep{Cooper1979-wz}.その後,Bienenstock, Cooper, Munroらにより提案された学習則は同様に頭文字をとって\textbf{BCM則}\index{BCMのり@BCM則} (BCM rule) と呼ばれている\citep{Bienenstock1982-km} \citep{Cooper2012-ec}.

$\mathbf{x}\in \mathbb{R}^d, \mathbf{w}\in \mathbb{R}^d, y\in \mathbb{R}$とし,単一の出力$y = \mathbf{w}^\top \mathbf{x}=\mathbf{x}^\top \mathbf{w}$を持つ線形ニューロンを仮定する.重みの更新則は次のようにする.


\begin{equation}
\frac{d\mathbf{w}}{dt} = \eta_w \mathbf{x} \phi(y, \theta_m)
\end{equation}


ここで関数$\phi$は$\phi(y, \theta_m)=y(y-\theta_m)$などとする.また$\theta_m\triangleq\mathbb{E}[y^2]$は閾値を決定するパラメータ,\textbf{修正閾値(modification threshold)}\index{しゅうせいいきち(modification threshold)@修正閾値(modification threshold)} であり,


\begin{equation}
\frac{d\theta_m}{dt} = \eta_{\theta} \left(y^2-\theta_m\right)
\end{equation}


として更新される.

ToDo: 詳細
\lstinputlisting[language=julia]{./text/local-learning-rule/pca-hebbian-learning/002.jl}
\lstinputlisting[language=julia]{./text/local-learning-rule/pca-hebbian-learning/003.jl}
\lstinputlisting[language=julia]{./text/local-learning-rule/pca-hebbian-learning/004.jl}
\begin{figure}[ht]
	\centering
	\includegraphics[scale=0.8, max width=\linewidth]{./fig/bayesian-brain/mcmc/cell004.png}
	\caption{cell004.png}
	\label{cell004.png}
\end{figure}
\subsubsection{Hebb則の生理的機序}
ここでHebb則およびBCM則の生理的基盤について触れておこう.
LTPの実験的発見 \citep{Bliss1973-vj} \citep{Dudek1992-nz}

ToDo:実験的発見のsurvey
\subsubsection{Oja則}
Hebb則を安定化させる別のアプローチとして,結合強度を正規化するという手法が考えられる.BCM則と同様に$\mathbf{x}\in \mathbb{R}^d, \mathbf{w}\in \mathbb{R}^d, y\in \mathbb{R}$とし,単一の出力$y = \mathbf{w}^\top \mathbf{x}=\mathbf{x}^\top \mathbf{w}$を持つ線形ニューロンを仮定する.$\eta$を学習率とすると,$\mathbf{w}\leftarrow\dfrac{\mathbf{w}+\eta \mathbf{x}y}{\|\mathbf{w}+\eta \mathbf{x}y\|}$とすれば正規化できる.ここで,$f(\eta)\triangleq\dfrac{\mathbf{w}+\eta \mathbf{x}y}{\|\mathbf{w}+\eta \mathbf{x}y\|}$とし,$\eta=0$においてTaylor展開を行うと,


\begin{align}
f(\eta)&\approx f(0) + \eta \left.\frac{df(\eta^*)}{d\eta^*}\right|_{\eta^*=0} + \mathcal{O}(\eta^2)\\
&=\frac{\mathbf{w}}{\|\mathbf{w}\|} + \eta \left(\frac{\mathbf{x}y}{\|\mathbf{w}\|}-\frac{y^2\mathbf{w}}{\|\mathbf{w}\|^3}\right)+ \mathcal{O}(\eta^2)
\end{align}


ここで$\|\mathbf{w}\|=1$として,1次近似すれば$f(\eta)\approx \mathbf{w} + \eta \left(\mathbf{x}y-y^2 \mathbf{w}\right)$となる.重みの変化が連続的であるとすると,


\begin{equation}
\frac{d\mathbf{w}}{dt} = \eta \left(\mathbf{x}y-y^2 \mathbf{w}\right)
\end{equation}


として重みの更新則が得られる.これを\textbf{Oja則 (Oja's rule)}\index{Ojaのり (Oja's rule)@Oja則 (Oja's rule)} と呼ぶ \citep{Oja1982-yd}.こうして得られた学習則において$\|\mathbf{w}\|\to 1$となることを確認しよう.


\begin{equation}
\frac{d\|\mathbf{w}\|^2}{dt}=2\mathbf{w}^\top\frac{d\mathbf{w}}{dt}= 2\eta y^2\left(1-\|\mathbf{w}\|^2\right)
\end{equation}


より,$\dfrac{d\|\mathbf{w}\|^2}{dt}=0$のとき,$\|\mathbf{w}\|= 1$となる.
\subsubsection{恒常的可塑性}
Oja則は更新時の即時的な正規化から導出されたものであるが,恒常的可塑性 (synaptic scaling)により安定化しているという説がある\citep{Turrigiano2008-lm}\citep{Yee2017-fb}.しかし,この過程は遅すぎるため,Hebb則の不安定化を安定化するに至らない\citep{Zenke2017-el}

ToDo:恒常的可塑性の詳細

Johansen, Joshua P., Lorenzo Diaz-Mataix, Hiroki Hamanaka, Takaaki Ozawa, Edgar Ycu, Jenny Koivumaa, Ashwani Kumar, et al. 2014. “Hebbian and Neuromodulatory Mechanisms Interact to Trigger Associative Memory Formation.” Proceedings of the National Academy of Sciences 111 (51): E5584–92.
\subsection{Hebb則と主成分分析}
Oja則を用いることで\textbf{主成分分析(Principal component analysis; PCA)}\index{しゅせいぶんぶんせき(Principal component analysis; PCA)@主成分分析(Principal component analysis; PCA)} という処理をニューラルネットワークにおいて実現できる.主成分分析とは-

ToDo:主成分分析の説明
\lstinputlisting[language=julia]{./text/local-learning-rule/pca-hebbian-learning/009.jl}
\lstinputlisting[language=julia]{./text/local-learning-rule/pca-hebbian-learning/010.jl}
\lstinputlisting[language=julia]{./text/local-learning-rule/pca-hebbian-learning/011.jl}
\begin{figure}[ht]
	\centering
	\includegraphics[scale=0.8, max width=\linewidth]{./fig/local-learning-rule/pca-hebbian-learning/cell011.png}
	\caption{cell011.png}
	\label{cell011.png}
\end{figure}
\subsubsection{Oja則によるPCAの実行}
ここでOja則が主成分分析を実行できることを示す.重みの変化量の期待値を取る.


\begin{align}
\frac{d\mathbf{w}}{dt} &= \eta \left(\mathbf{x}y - y^2 \mathbf{w}\right)=\eta \left(\mathbf{x}\mathbf{x}^\top \mathbf{w} - \left[\mathbf{w}^\top \mathbf{x}\mathbf{x}^\top \mathbf{w}\right] \mathbf{w}\right)\\
\mathbb{E}\left[\frac{d\mathbf{w}}{dt}\right] &= \eta \left(\mathbf{C} \mathbf{w} - \left[\mathbf{w}^\top \mathbf{C} \mathbf{w}\right] \mathbf{w}\right)
\end{align}


$\mathbf{C}\triangleq\mathbb{E}[\mathbf{x}\mathbf{x}^\top]\in \mathbb{R}^{d\times d}$とする.$\mathbf{x}$の平均が0の場合,$\mathbf{C}$は分散共分散行列である.$\mathbb{E}\left[\dfrac{d\mathbf{w}}{dt}\right]=0$となる$\mathbf{w}$が収束する固定点(fixed point)では次の式が成り立つ.


\begin{equation}
\mathbf{C}\mathbf{w} = \lambda \mathbf{w}
\end{equation}


これは固有値問題であり,$\lambda\triangleq\mathbf{w}^\top \mathbf{C} \mathbf{w}$は固有値,$\mathbf{w}$は固有ベクトル(eigen vector)になる.

ここでサンプルサイズを$n$とし,$\mathbf{X} \in \mathbb{R}^{d\times n}, \mathbf{y}=\mathbf{X}^\top\mathbf{w} \in \mathbb{R}^n$とする.標本平均で近似して$\mathbf{C}\simeq \mathbf{X}\mathbf{X}^\top$とする.この場合,


\begin{align}
\mathbb{E}\left[\frac{d\mathbf{w}}{dt}\right] &\simeq \eta \left(\mathbf{X}\mathbf{X}^\top \mathbf{w} - \left[\mathbf{w}^\top \mathbf{X}\mathbf{X}^\top \mathbf{w}\right] \mathbf{w}\right)\\
&=\eta \left(\mathbf{X}\mathbf{y} - \left[\mathbf{y}^\top\mathbf{y}\right] \mathbf{w}\right)
\end{align}


となる.
\lstinputlisting[language=julia]{./text/local-learning-rule/pca-hebbian-learning/013.jl}
\lstinputlisting[language=julia]{./text/local-learning-rule/pca-hebbian-learning/014.jl}
\begin{figure}[ht]
	\centering
	\includegraphics[scale=0.8, max width=\linewidth]{./fig/bayesian-brain/mcmc/cell014.png}
	\caption{cell014.png}
	\label{cell014.png}
\end{figure}
後のためにOja則においてネットワークが$q$個の複数出力を持つ場合を考えよう.重み行列を$\mathbf{W} \in \mathbb{R}^{q\times d}$, 出力を$\mathbf{y}=\mathbf{W}\mathbf{x} \in \mathbb{R}^{q}, \mathbf{Y}=\mathbf{W}\mathbf{X} \in \mathbb{R}^{q\times n}$とする.この場合の更新則は


\begin{equation}
\frac{d\mathbf{W}}{dt} = \eta \left(\mathbf{y}\mathbf{x}^\top - \mathrm{Diag}\left[\mathbf{y}\mathbf{y}^\top\right] \mathbf{W}\right)
\end{equation}


となる.ただし,$\mathrm{Diag}(\cdot)$は行列の対角成分からなる対角行列を生み出す作用素である.
\subsubsection{Sanger則}
Oja則に複数の出力を持たせた場合であっても,出力が直交しないため,PCAの第1主成分しか求めることができない.\textbf{Sanger則 (Sanger's rule)}\index{Sangerのり (Sanger's rule)@Sanger則 (Sanger's rule)},あるいは\textbf{一般化Hebb則 (generalized Hebbian algorithm; GHA)}\index{いっぱんかHebbのり (generalized Hebbian algorithm; GHA)@一般化Hebb則 (generalized Hebbian algorithm; GHA)} は,Oja則に\textbf{Gram–Schmidtの正規直交化法(Gram–Schmidt orthonormalization)}\index{Gram–Schmidtのせいきちょっこうかほう(Gram–Schmidt orthonormalization)@Gram–Schmidtの正規直交化法(Gram–Schmidt orthonormalization)} を組み合わせた学習則であり,次式で表される.


\begin{equation}
\frac{d\mathbf{W}}{dt} = \eta \left(\mathbf{y}\mathbf{x}^\top - \mathrm{LT}\left[\mathbf{y}\mathbf{y}^\top\right] \mathbf{W}\right)
\end{equation}


$\mathrm{LT}(\cdot)$は行列の対角成分より上側の要素を0にした下三角行列(lower triangular matrix)を作り出す作用素である.Sanger則を用いればPCAの第2主成分以降も求めることができる.
\lstinputlisting[language=julia]{./text/local-learning-rule/pca-hebbian-learning/017.jl}
\lstinputlisting[language=julia]{./text/local-learning-rule/pca-hebbian-learning/018.jl}
\begin{figure}[ht]
	\centering
	\includegraphics[scale=0.8, max width=\linewidth]{./fig/motor-learning/infinite-horizon-ofc/cell018.png}
	\caption{cell018.png}
	\label{cell018.png}
\end{figure}
Oja則,Sanger則をまとめて一つの関数にしておこう.\jl{identity()}は恒等関数である.
\lstinputlisting[language=julia]{./text/local-learning-rule/pca-hebbian-learning/020.jl}
\subsection{非線形Hebb学習}
出力$\mathbf{y}$に非線形関数$g(\cdot)$を適用し,$\mathbf{y}\to g(\mathbf{y})$として置き換えることで非線形Hebb学習となる\citep{Oja1997-hr}\citep{Brito2016-mx}. 関数\jl{HebbianPCA}の\jl{func}引数に非線形関数を渡すことで実現できる.

ToDo: 詳細
\subsubsection{非負主成分分析によるグリッドパターンの創発}
内側嗅内皮質(MEC)にある\textbf{グリッド細胞 (grid cells)}\index{ぐりっどさいぼう (grid cells)@グリッド細胞 (grid cells)} は六角形格子状の発火パターンにより自己位置等を符号化するのに貢献している.この発火パターンを生み出すモデルは多数あるが,\textbf{場所細胞(place cells)}\index{ばしょさいぼう(place cells)@場所細胞(place cells)} の発火パターンを\textbf{非負主成分分析(nonnegative principal component analysis)}\index{ひふしゅせいぶんぶんせき(nonnegative principal component analysis)@非負主成分分析(nonnegative principal component analysis)} で次元削減するとグリッド細胞のパターンが生まれるというモデルがある \citep{Dordek2016-ff}.非線形Hebb学習を用いてこのモデルを実装しよう.なお,同様のことは\textbf{非負値行列因子分解 (NMF: nonnegative matrix factorization)}\index{ひふあたいぎょうれついんしぶんかい (NMF: nonnegative matrix factorization)@非負値行列因子分解 (NMF: nonnegative matrix factorization)} でも可能である.
\paragraph{場所細胞の発火パターン}
まず,訓練データとなる場所細胞の発火パターンを人工的に作成する.場所細胞の発火パターンは\textbf{Difference of Gaussians (DoG)}\index{Difference of Gaussians (DoG)} で近似する.DoGは大きさの異なる2つのガウス関数の差分を取った関数であり,画像に適応すればband-passフィルタとして機能する.また,DoGは網膜神経節細胞等の受容野のON中心OFF周辺型受容野のモデルとしても用いられる.受容野中央では活動が大きく,その周辺では活動が抑制される,という特性を持つ.2次元のガウス関数とDoG関数を実装する.
\lstinputlisting[language=julia]{./text/local-learning-rule/pca-hebbian-learning/024.jl}
モデルのパラメータを設定する.
\lstinputlisting[language=julia]{./text/local-learning-rule/pca-hebbian-learning/026.jl}
先にガウス関数とDoG関数がどのような見た目になるか確認しよう.
\lstinputlisting[language=julia]{./text/local-learning-rule/pca-hebbian-learning/028.jl}
\lstinputlisting[language=julia]{./text/local-learning-rule/pca-hebbian-learning/029.jl}
\begin{figure}[ht]
	\centering
	\includegraphics[scale=0.8, max width=\linewidth]{./fig/local-learning-rule/pca-hebbian-learning/cell029.png}
	\caption{cell029.png}
	\label{cell029.png}
\end{figure}
場所細胞の活動パターンを生み出す.それぞれの場所受容野の中心は空間を均等に覆うように作成する (一様分布で生み出してもよい).
\lstinputlisting[language=julia]{./text/local-learning-rule/pca-hebbian-learning/031.jl}
線形PCAの場合
\lstinputlisting[language=julia]{./text/local-learning-rule/pca-hebbian-learning/033.jl}
\lstinputlisting[language=julia]{./text/local-learning-rule/pca-hebbian-learning/034.jl}
\begin{figure}[ht]
	\centering
	\includegraphics[scale=0.8, max width=\linewidth]{./fig/local-learning-rule/pca-hebbian-learning/cell034.png}
	\caption{cell034.png}
	\label{cell034.png}
\end{figure}
自己相関マップ(autocorrelation map)を確認する.

ToDo: 相関の計算の説明
\lstinputlisting[language=julia]{./text/local-learning-rule/pca-hebbian-learning/036.jl}
\lstinputlisting[language=julia]{./text/local-learning-rule/pca-hebbian-learning/037.jl}
\lstinputlisting[language=julia]{./text/local-learning-rule/pca-hebbian-learning/038.jl}
\begin{figure}[ht]
	\centering
	\includegraphics[scale=0.8, max width=\linewidth]{./fig/energy-based-model/sparse-coding/cell038.png}
	\caption{cell038.png}
	\label{cell038.png}
\end{figure}
非負PCAの場合
\lstinputlisting[language=julia]{./text/local-learning-rule/pca-hebbian-learning/040.jl}
\lstinputlisting[language=julia]{./text/local-learning-rule/pca-hebbian-learning/041.jl}
\lstinputlisting[language=julia]{./text/local-learning-rule/pca-hebbian-learning/042.jl}
\begin{figure}[ht]
	\centering
	\includegraphics[scale=0.8, max width=\linewidth]{./fig/bayesian-brain/neural-sampling/cell042.png}
	\caption{cell042.png}
	\label{cell042.png}
\end{figure}
\lstinputlisting[language=julia]{./text/local-learning-rule/pca-hebbian-learning/043.jl}
\lstinputlisting[language=julia]{./text/local-learning-rule/pca-hebbian-learning/044.jl}
\begin{figure}[ht]
	\centering
	\includegraphics[scale=0.8, max width=\linewidth]{./fig/local-learning-rule/pca-hebbian-learning/cell044.png}
	\caption{cell044.png}
	\label{cell044.png}
\end{figure}
Place cellの受容野をDoGに設定したが,これが無いと格子状の受容野は出現しない.path integrationをRNNで実行する場合も同様.一方で,DoGは場所細胞の受容野としては不適切である.

No Free Lunch from Deep Learning in Neuroscience: A Case Study through Models of the Entorhinal-Hippocampal Circuit 
\url{https://openreview.net/forum?id=mxi1xKzNFrb}

ToDo: 他のgrid cellsのモデルについて
 % ok2
% \section{ヘブ/反ヘブ則とMDS}
Hebbian/anti-hebbian ruleとMDSの関係.

% \section{予測符号化}
\subsection{観測世界の階層的予測}
\textbf{階層的予測符号化(hierarchical predictive coding; HPC)} は\cite{Rao1999-zv}により導入された.構築するネットワークは入力層を含め,3層のネットワークとする.LGNへの入力として画像 $\mathbf{x} \in \mathbb{R}^{n_0}$を考える.画像 $\mathbf{x}$ の観測世界における隠れ変数,すなわち\textbf{潜在変数} (latent variable)を$\mathbf{r} \in \mathbb{R}^{n_1}$とし,ニューロン群によって発火率で表現されているとする (真の変数と $\mathbf{r}$は異なるので文字を分けるべきだが簡単のためにこう表す).このとき,


\mathbf{x} = f(\mathbf{U}\mathbf{r}) + \boldsymbol{\epsilon}


が成立しているとする.ただし,$f(\cdot)$は活性化関数 (activation function),$\mathbf{U} \in \mathbb{R}^{n_0 \times n_1}$は重み行列である.
$\boldsymbol{\epsilon} \in \mathbb{R}^{n_0}$ は $\mathcal{N}(\mathbf{0}, \sigma^2 \mathbf{I})$ からサンプリングされるとする.

潜在変数 $\mathbf{r}$はさらに高次 (higher-level)の潜在変数 $\mathbf{r}^h$により,次式で表現される.


\mathbf{r} = \mathbf{r}^{td}+\boldsymbol{\epsilon}^{td}=f(\mathbf{U}^h \mathbf{r}^h)+\boldsymbol{\epsilon}^{td}


ただし,Top-downの予測信号を $\mathbf{r}^{td}:=f(\mathbf{U}^h \mathbf{r}^h)$とした.また,$\mathbf{r}^{td} \in \mathbb{R}^{n_1}$, $\mathbf{r}^{h} \in \mathbb{R}^{n_2}$, $\mathbf{U}^h \in \mathbb{R}^{n_1 \times n_2}$ である.
$\boldsymbol{\epsilon}^{td} \in \mathbb{R}^{n_1}$は$\mathcal{N}(\mathbf{0}$, $\sigma_{td}^2 \mathbf{I}$) からサンプリングされるとする.

話は飛ぶが,Predictive codingのネットワークの特徴は
\begin{itemize}
\item 階層的な構造
\item 高次による低次の予測 (Feedback or Top-down信号)
\item 低次から高次への誤差信号の伝搬 (Feedforward or Bottom-up 信号)
\end{itemize}

である.ここまでは高次表現による低次表現の予測,というFeedback信号について説明してきたが,この部分はSparse codingでも同じである.それではPredictive codingのもう一つの要となる,低次から高次への予測誤差の伝搬というFeedforward信号はどのように導かれるのだろうか.結論から言えば,これは\textbf{復元誤差 (reconstruction error)の最小化を行う再帰的ネットワーク (recurrent network)を考慮することで自然に導かれる}.

\lstinputlisting[language=julia]{./text/local-learning-rule/slow-feature-analysis/001.jl}
\subsubsection{事前分布の設定}
事前分布$p(\mathbf{r})$としては,0においてピークがあり,裾の重い(heavy tail)を持つsparse distributionあるいは \textbf{super-Gaussian distribution} (Laplace 分布やCauchy分布などGaussian分布よりもkurtoticな分布)を用いるのが良い.このような分布では,$\mathbf{r}$の各要素$r_i$はほとんど0に等しく,ある入力に対しては大きな値を取る.$p(\mathbf{r})$は一般化して式(4), (5)のように表記する.


\begin{aligned}
p(\mathbf{r})&=\prod_j p(r_j)\\
p(r_j)&=\frac{1}{Z_{\beta}}\exp \left[-\beta S(r_j)\right]
\end{aligned}


ただし,$\beta$は逆温度(inverse temperature), $Z_{\beta}$は規格化定数 (分配関数) である.これらの用語は統計力学における正準分布 (ボルツマン分布)から来ている.$S(x)$と分布の関係をまとめた表が以下となる (cf. \url{https://pdfs.semanticscholar.org/be08/da912362bf40fe3ded78bdadc644f921b4e7.pdf}).

\lstinputlisting[language=julia]{./text/local-learning-rule/slow-feature-analysis/003.jl}
2種類の指数関数型シナプスの動態.破線は単一指数関数型シナプスで, 実線は二重指数関数型シナプスである.
\lstinputlisting[language=julia]{./text/local-learning-rule/slow-feature-analysis/005.jl}
いくつかの処理について解説しておく.まず,一番目のforループ内の\jl{v[i]}の\jl{((dt*tcount) > (tlast[i] + tref))}は最後にニューロンが発火した時刻\jl{tlast[i]}に不応期\jl{tref}を足した時刻よりも現在の時刻\jl{dt*tcount[1]}が大きければ膜電位の更新を許可し,小さければ更新しない.二番目のforループにおける\jl{fire[i]}はニューロンの膜電位が閾値電位\jl{vthr}を超えたら\jl{True}となる.\jl{v[i]}などの更新式にある\jl{ifelse(a, b, c)}はaが\jl{True}の時はbを返し,\jl{False}の時はcを返す関数であり,\jl{v[i] = ifelse(fire[i], vreset, v[i])}は\jl{fire[i]}が\jl{True}なら\jl{v[i]}をリセット電位\jl{vreset}とし,そうでなければそのままの値を返すという処理である.同様にして\jl{tlast[i]}は発火したときにその時刻を記録する変数となっている.なお,\jl{v_[i] = ifelse(fire[i], vpeak, v[i])}は実際のシミュレーションにおいて意味をなさない.単に発火時の電位\jl{vpeak}を含めて記録すると描画時の見栄えが良いというだけである.

これらの\jl{struct}と関数を用いてシミュレーションを実行する.\jl{I} はHHモデルのときと同じように矩形波を入力する.実は\jl{I}は入力電流ではなく入力電流に比例する量となっているが,これは膜抵抗を乗じた後の値であると考えるとよい.

\lstinputlisting[language=julia]{./text/local-learning-rule/slow-feature-analysis/007.jl}
\lstinputlisting[language=julia]{./text/local-learning-rule/slow-feature-analysis/008.jl}
\begin{figure}[ht]
	\centering
	\includegraphics[scale=0.8, max width=\linewidth]{./fig/appendix/slow-feature-analysis/cell008.png}
	\caption{cell008.png}
	\label{cell008.png}
\end{figure}
\subsubsection{正規方程式を用いた推定}条件に基づいて目的関数$L(\mathbf{\theta})$を微分すると次のような方程式が得られる.
$$
\mathbf{X}^\top\mathbf{X}\mathbf{\hat\theta}=\mathbf{X}^\top\mathbf{y}
$$
これを\textbf{正規方程式} (normal equation)と呼ぶ.この正規方程式より、係数の推定値は$\mathbf{\hat\theta}={(\mathbf{X}^\top\mathbf{X})}^{-1}X^\top\mathbf{y}$という式で得られる.なお,正規方程式自体は$\mathbf{y}=\mathbf{X}\mathbf{\theta}$の左から$\mathbf{X}^\top$をかける,と覚えると良い.
\lstinputlisting[language=julia]{./text/local-learning-rule/slow-feature-analysis/010.jl}
## 画像の復元
\lstinputlisting[language=julia]{./text/local-learning-rule/slow-feature-analysis/012.jl}
\lstinputlisting[language=julia]{./text/local-learning-rule/slow-feature-analysis/013.jl}
\begin{figure}[ht]
	\centering
	\includegraphics[scale=0.8, max width=\linewidth]{./fig/neuron-model/hodgkin-huxley/cell013.png}
	\caption{cell013.png}
	\label{cell013.png}
\end{figure}
\subsection{参考文献}- \url{https://towardsdatascience.com/a-brief-introduction-to-slow-feature-analysis-18c901bc2a58}
- \url{https://github.com/flatironinstitute/bio-sfa}
- \url{https://github.com/fulviadelduca/slow-feature-analysis}
- [Deep Slow Feature Analysis Network](https://github.com/rulixiang/DSFANet)
- \url{https://nbviewer.jupyter.org/github/pierrelux/notebooks/blob/master/Slow%20Feature%20Analysis.ipynb}

% # STDP則による教師なし学習
\subsection{STDP(spike-timing-dependent plasticity)則}
\subsubsection{Pair-based STDP則}
\textbf{Spike-timing-dependent plasticity}\index{Spike-timing-dependent plasticity} (STDP)はシナプス前細胞と後細胞の発火時刻の差によってシナプス強度が変化するという現象です(Markram et al. 1997; Bi and Poo 1998).典型的なSTDP側は \textbf{Pair-based STDP則}\index{Pair-based STDPのり@Pair-based STDP則} と呼ばれ,シナプス前細胞と後細胞の2つのスパイクのペアの発火時刻によってLTPやLTDが起こります.この節ではこのPair-based STDP則について説明します.
シナプス後細胞におけるスパイク(postsynaptic spike)の発生時刻$t_\text{post}$とシナプス前細胞におけるスパイク(presynaptic spike)の発生時刻$t_\text{pre}$の差を$\Delta t_{\text{spike}}=t_\text{post}-t_\text{pre}$とします\footnote{$\Delta t_{\text{spike}}$の定義は元々は逆になっており,(Song et al,, 2000)では$\Delta t_{\text{spike}}=t_\text{pre}-t_\text{post}$としています.また,添え字は離散時のタイムステップと混同しないために付けています.}.$\Delta t_{\text{spike}}$はシナプス前細胞,後細胞の順で発火すれば正,逆なら負となります.Pair-based STDP則では,シナプス前細胞から後細胞へのシナプス強度($w$)\footnote{シナプス強度$w$に添え字をつけていませんが,この場合はシナプス前細胞と後細胞の2つの細胞しかないと仮定して考えています.}の変化$\Delta w$は$\Delta t_{\text{spike}}$に依存的に以下の式に従って変化します(Song et al., 2000).
\begin{equation}
\Delta w = \begin{cases}
A_{+} \exp\left(-\dfrac{\Delta t_{\text{spike}}}{\tau_{+}}\right) &(\Delta t_{\text{spike}}> 0) \\
-A_{-} \exp\left(-\dfrac{|\Delta t_{\text{spike}}|}{\tau_{-}}\right) &(\Delta t_{\text{spike}}< 0)
\end{cases}
\end{equation}
$A_+, A_-$は正の定数,または重み依存的な関数 (後述) です.$\Delta t_{\text{spike}}>0$のときはLTPが起こり,$\Delta t_{\text{spike}}<0$のときはLTDが起こります.このタイプのSTDP則は\textbf{Hebbian STDP}\index{Hebbian STDP} と呼ばれ,Hebb則に従うシナプス強度の変化が起こります\footnote{Hebb則に従わないSTDPもあり,例えばLTPとLTDの挙動が逆のものを\textbf{Anti-Hebbian STDP}\index{Anti-Hebbian STDP}と呼びます(Bell et al., 1997など).}.
$A_+=0.01$, $A_-/A_+=1.05$, $\tau_{+}=\tau_{-}=20$ msとしたときの$\Delta t_{\text{spike}}$に対する$\Delta w$は図\ref{fig:stdp}のようになります.
\begin{lstlisting}[language=julia]
tau_p = tau_m = 20 #ms
A_p = 0.01
A_m = 1.05*A_p
dt = np.arange(-50, 50, 1) #ms
dw = A_p*np.exp(-dt/tau_p)*(dt>0) - A_m*np.exp(dt/tau_p)*(dt<0) 

plt.figure(figsize=(5, 4))
plt.plot(dt, dw)
plt.hlines(0, -50, 50); plt.xlim(-50, 50)
plt.xlabel("$\Delta t$ (ms)"); plt.ylabel("$\Delta w$")
plt.show()
\end{lstlisting}
\subsection{オンライン STDP則}
単に2つのニューロンを考えるなら上で紹介した式でも良いのですが,ネットワーク全体を考えると実装は複雑になり効率的ではありません。また,スパイク発生時刻を記憶しておくことは生物学的に妥当ではありません。そこで,スパイク活動のトレース(trace)というローカル変数を用いてSTDPを記述してみましょう。
\begin{align}
\frac{dx_\text{pre}}{dt}&=-\frac{x_\text{pre}}{\tau_+}+\sum_{t_{\text{pre}}^{(i)} <t} \delta \left(t-t_{\text{pre}}^{(i)}\right)\\
\frac{dx_\text{post}}{dt}&=-\frac{x_\text{post}}{\tau_-}+\sum_{t_{\text{post}}^{(j)}<t} \delta \left(t-t_{\text{post}}^{(j)}\right)
\end{align}
とします。ただし,$t_{\text{pre}}^{(i)}$はシナプス前細胞の$i$番目のスパイク,$t_{\text{post}}^{(j)}$はシナプス後細胞の$i$番目のスパイクを意味します。また,$x_\text{pre}$, $x_\text{post}$はそれぞれシナプス前細胞,後細胞のスパイクのトレースです。トレースはそれぞれの細胞においてスパイクが発生したときに1増加し\footnote{トレースの値域を$0\leq x\leq1$に制限するため,スパイクが発生したとき1にリセットするという場合もあります(Morrison et al., 2008)。その場合はx(t+\Delta t)=\left(1-\frac{\Delta t}{\tau}\right)x(t)\cdot(1-\delta_{t,t'})+\delta_{t,t'}のようにします ($t'$はスパイクの発生時刻) 。},それ以外では時定数$\tau_+, \tau_-$で指数関数的に減少します。これは既に1章で説明した単一指数関数型シナプスと同じです。生理学的解釈ですが,$x_\text{pre}$はNMDA受容体のイオンチャネルの開口割合,$x_\text{post}$は逆伝搬活動電位 (back-propagating action potential; bAP) \footnote{誤差逆伝搬法(Back-propagation)とは異なります。}やbAPによるカルシウムの流入と捉えることができます(cf. 『標準生理学』)。
そして,$x_\text{pre}, x_\text{post}$を用いて重みの更新式は
\begin{equation}
\frac{dw}{dt}=A_+ x_\text{pre} \cdot \underbrace{\sum_{t_{\text{post}}^{(j)}<t} \delta \left(t-t_{\text{post}}^{(j)}\right)}_{\text{シナプス後細胞のスパイク}} - A_- x_{\text{post}} \cdot \underbrace{\sum_{t_{\text{pre}}^{(i)} <t} \delta \left(t-t_{\text{pre}}^{(i)}\right)}_{\text{シナプス前細胞のスパイク}}
\end{equation}
と表せます。
これらの式をEuler法によりタイムステップ$\Delta t$で離散化すると,
\begin{align}
x_{\text{pre}}(t+\Delta t)&=\left(1-\frac{\Delta t}{\tau_{+}}\right)\cdot x_{\text{pre}}(t)+
\delta_{t,t_{\text{pre}}^{(i)}}\\
x_{\text{post}}(t+\Delta t)&=\left(1-\frac{\Delta t}{\tau_{-}}\right)\cdot x_{\text{post}}(t)+\delta_{t,t_{\text{post}}^{(j)}}\\
w(t+\Delta t)&=w(t)+A_+ x_{\text{pre}}\cdot \delta_{t,t_{\text{post}}^{(j)}} - A_-x_{\text{post}}\cdot \delta_{t,t_{\text{pre}}^{(i)}}
\end{align}
となります。ただし,$\delta_{t,t'}$はKroneckerのdelta関数で,$t=t'$のときに1, それ以外は0となります。$\delta_{t,t'}$は実装時においてスパイクが起こったときに1, その他は0を取る変数を用いると良いでしょう。
それでは,Online STDPを実装してみましょう\footnote{コードは\texttt{./SingleFileSimulations/STDP/stdp2.py}です。}。
\begin{lstlisting}[language=julia]
#定数
dt = 1e-3 #sec
T = 0.5 #sec
nt = round(T/dt)
tau_p = tau_m = 2e-2 #ms
A_p = 0.01; A_m = 1.05*A_p

#pre/postsynaptic spikes
spike_pre = np.zeros(nt); spike_pre[[50, 200, 225, 300, 425]] = 1
spike_post = np.zeros(nt); spike_post[[100, 150, 250, 350, 400]] = 1

#記録用配列
x_pre_arr = np.zeros(nt); x_post_arr = np.zeros(nt)
w_arr = np.zeros(nt)

#初期化
x_pre = x_post = 0 # pre/post synaptic trace
w = 0 # synaptic weight

#Online STDP
for t in range(nt):
    x_pre = x_pre*(1-dt/tau_p) + spike_pre[t]
    x_post = x_post*(1-dt/tau_m) + spike_post[t]
    dw = A_p*x_pre*spike_post[t] - A_m*x_post*spike_pre[t]
    w += dw #重みの更新
    
    x_pre_arr[t] = x_pre
    x_post_arr[t] = x_post
    w_arr[t] = w

# 描画
time = np.arange(nt)*dt*1e3
def hide_ticks(): #上と右の軸を表示しないための関数
    plt.gca().spines['right'].set_visible(False)
    plt.gca().spines['top'].set_visible(False)
    plt.gca().yaxis.set_ticks_position('left')
    plt.gca().xaxis.set_ticks_position('bottom')
plt.figure(figsize=(6, 6))
plt.subplot(5,1,1)
plt.plot(time, x_pre_arr, color="k")
plt.ylabel("$x_{pre}$"); hide_ticks(); plt.xticks([])
plt.subplot(5,1,2)
plt.plot(time, spike_pre, color="k")
plt.ylabel("pre- spikes"); hide_ticks(); plt.xticks([])
plt.subplot(5,1,3)
plt.plot(time, spike_post, color="k")
plt.ylabel("post- spikes"); hide_ticks(); plt.xticks([])
plt.subplot(5,1,4)
plt.plot(time, x_post_arr, color="k")
plt.ylabel("$x_{post}$"); hide_ticks(); plt.xticks([])
plt.subplot(5,1,5)
plt.plot(time, w_arr, color="k")
plt.xlabel("$t$ (ms)"); plt.ylabel("$w$"); hide_ticks()
plt.show()
\end{lstlisting}
結果は図\ref{fig:online_stdp}のようになります。
\begin{figure}[htbp]
    \centering
    \includegraphics[scale=0.5]{figs/online_stdp.pdf}
    \caption{オンライン STDP則:(1段目)シナプス前細胞のスパイクトレース (2段目)シナプス前細胞のスパイク (3段目)シナプス後細胞のスパイク (4段目)シナプス後細胞のスパイクトレース (5段目)重みの変化(初期値0)}
    \label{fig:online_stdp}
\end{figure}
\subsubsection{行列を用いたオンラインSTDP則の実装}
この節ではシナプス前細胞と後細胞の数を一般化し,今まで2つのニューロン間で考えていたSTDP則を行列計算で実装する方法について説明します。
まず,シナプス前細胞,後細胞がそれぞれ$N_\text{pre}$, $N_\text{post}$あるとします。また,Kroneckerのdelta関数の代わりに,スパイクが起こったときに1, その他は0の値を取る明示的な変数$\boldsymbol{s}(t)$を用いることにします。ここで,シナプス前細胞,後細胞についてスパイク変数は$\boldsymbol{s}_{\text{pre}} \in \mathbb{R}^{N_\text{pre}}, \boldsymbol{s}_{\text{post}} \in \mathbb{R}^{N_\text{post}}$であり,スパイクのトレースは$\boldsymbol{x}_{\text{pre}} \in \mathbb{R}^{N_\text{pre}}, \boldsymbol{x}_{\text{post}} \in \mathbb{R}^{N_\text{post}}$です。さらにシナプスから後細胞へのシナプス強度を$N_\text{post} \times N_\text{pre}$行列の$W$とします。このとき,Online STDP則は
\begin{align}
\boldsymbol{x}_{\text{pre}}(t+\Delta t)&=\left(1-\frac{\Delta t}{\tau_{+}}\right)\cdot \boldsymbol{x}_{\text{pre}}(t)+
\boldsymbol{s}_{\text{pre}}(t)\\
\boldsymbol{x}_{\text{post}}(t+\Delta t)&=\left(1-\frac{\Delta t}{\tau_{-}}\right)\cdot \boldsymbol{x}_{\text{post}}(t)+\boldsymbol{s}_{\text{post}}(t)\\
W(t+\Delta t)&=W(t)+A_+ \boldsymbol{s}_{\text{post}}(t)(\boldsymbol{x}_{\text{pre}}(t))^\top - A_-\boldsymbol{x}_{\text{post}}(t)(\boldsymbol{s}_{\text{pre}}(t))^\top
\end{align}
と書けます。ただし,$^\top$を転置記号とし,$\boldsymbol{x}$を列ベクトル,$\boldsymbol{x}^\top$を行ベクトルとしています。
これらを用いてOnline STDP則と元のSTDPの式が一致しているかの確認をしてみましょう。タイムステップ\jl{dt}を1ms, シミュレーション時間\texttt{T}は50msとし,シミュレーションのタイムステップ数\jl{nt}と同数のシナプス前細胞,2つのシナプス後細胞があるとします。それぞれのシナプス前細胞は\jl{dt}だけずれて発火し\footnote{この場合,シナプス前細胞のスパイクを表す \jl{spike_pre} 行列として \jl{nt} 次単位行列を与えればよいです。}(つまり発火時刻の範囲は$[0\text{ms}, 50\text{ms}]$),2つの後細胞は$t=0, 50$msに発火します。こうすることで,発火の時間差として$[-50\text{ms}, 50\text{ms}]$が生まれます。
\begin{lstlisting}[language=julia]
dt = 1e-3; T = 5e-2 #sec
nt = round(T/dt)
N_pre = nt; N_post = 2
tau_p = tau_m = 2e-2 #ms
A_p = 0.01; A_m = 1.05*A_p

# pre/postsynaptic spikes
spike_pre = np.eye(N_pre) #単位行列でdtごとに発火するニューロンをN個作成
spike_post = np.zeros((N_post, nt))
spike_post[0, -1] = spike_post[1, 0] = 1

# 初期化
x_pre = np.zeros(N_pre)
x_post = np.zeros(N_post)
W = np.zeros((N_post, N_pre))

for t in range(nt):
    # 1次元配列 -> 縦ベクトル or 横ベクトル
    spike_pre_ = np.expand_dims(spike_pre[:, t], 0) # (1, N)
    spike_post_ = np.expand_dims(spike_post[:, t], 1) # (2, 1)
    x_pre_ = np.expand_dims(x_pre, 0) # (1, N)
    x_post_ = np.expand_dims(x_post, 1) # (2, 1)
    
    # Online STDP
    dW = A_p*np.matmul(spike_post_, x_pre_)
    dW -= A_m*np.matmul(x_post_, spike_pre_)
    W += dW
    
    # Update
    x_pre = x_pre*(1-dt/tau_p) + spike_pre[:, t]
    x_post = x_post*(1-dt/tau_m) + spike_post[:, t]

# 結果
delta_w = np.zeros(nt*2-1) # スパイク時間差 = 0msが重複
delta_w[:nt] = W[0, :]; delta_w[nt:] = W[1, 1:]

# 描画
time = np.arange(-T, T-dt, dt)*1e3
plt.figure(figsize=(5, 4))
plt.plot(time, delta_w[::-1])
plt.hlines(0, -50, 50)
plt.xlabel("$\Delta t$ (ms)")
plt.ylabel("$\Delta w$")
plt.xlim(-50, 50)
plt.show()
\end{lstlisting}
このコードを実行すると図\ref{fig:online_stdp2}のようになります。
\begin{figure}[htbp]
    \centering
    \includegraphics[scale=0.5]{figs/online_stdp2.pdf}
    \caption{Online STDP}
    \label{fig:online_stdp2}
\end{figure}
\subsection{重み依存的なSTDP}
生理学的にはシナプス強度$w$には$w_{\min} < w < w_{\max}$というような制限(bound)が存在すると考えられます\footnote{受容体の数が際限なく増加したり減少したりすることはないと考えられるためです。もしLTPによりシナプス強度が限りなく増大した場合,シナプス後細胞の発火頻度が高くなり,実際には発火を誘発していないシナプス前細胞とのシナプス強度も大きくなってしまいます。LTPの暴走を防ぐための機構の1つとして\textbf{恒常性可塑性}(homeostatic scaling)または}\index{こうじょうせいかそせい}(homeostatic scaling)または@恒常性可塑性}(homeostatic scaling)または}シナプススケーリング}(synaptic scaling)と呼ばれる現象があります。}。多くの場合では$w_{\min}=0$となっているので,以下では$w\in [0, w_{\max}]$となる場合を考えます。また,前節までは正の定数としていた$A_+, A_-$を重み依存的な関数であるとします(つまり$A_\pm=A_\pm(w)$とします)。
重みの制限には\textbf{ソフト制限(soft bound)}と}\index{そふとせいげん(soft bound)}と@ソフト制限(soft bound)}と}ハード制限(hard bound)}があります(Gerstner and Kistler, 2002, Chapter 11)。ソフト制限は,重みが上界 (または下界) に近づくにつれ重みの変化が小さくなるというものです。
\begin{align}
A_+(w) &= \eta_+\cdot (w_{\max}-w) \\
A_-(w) &= \eta_-\cdot w
\end{align}
ここで$\eta_+, \eta_-$は正の値で,**学習率(learning rate)}を意味します。
次に,ハード制限は重みが上限 (または下限) に達した際に,重みが増加 (または減少) しないというものです。Heavisideの階段関数$\Theta(x)$ (ただし$x<0$で$\Theta(x)=0$, $x\geq 0$で$\Theta(x)=1$)を用いて
\begin{align}
A_+(w) &= \eta_+\cdot \Theta(w_{\max}-w) \\
A_-(w) &= \eta_-\cdot \Theta(-w)
\end{align}
となります。
\subsubsection{STDP則と2層WTAネットワークによる教師なし学習}
この節ではSTDP則と\textbf{Winner-take-all (WTA)}の機構を用いた}\index{Winner-take-all (WTA)}のきこうをもちいた@Winner-take-all (WTA)}の機構を用いた}自己組織化マップ}(self-organizing map, SOM)による教師なし学習の研究について紹介します(Diehl \& Cook, 2015)\footnote{Brianを用いた実装は\url{https://github.com/peter-u-diehl/stdp-mnist}, Brian2を用いた実装は\url{https://github.com/zxzhijia/Brian2STDPMNIST}で公開されています(ただしPython2). }. Diehlらの提案するモデルにより,MNISTデータセットにおいて教師なし学習でテストデータに対し, 95\%の正解率を出しています. 現在でもDiehlらの研究を発展させて, Spiking CNNの学習に応用する研究が進んでいます. 
WTA (Winner-take-all)というのはネットワーク内のニューロンが互いに抑制しあう\footnote{これを\textbf{側抑制}(lateral inhibition)と言います. }ことで, 最も発火率の高いニューロン以外は抑制されて発火しないようになる, という機構です\footnote{例えばANNで用いられるSoftmaxやMax-poolingの操作はWTAのメカニズムのモデル化の1つといえます. }. WTAの機構による学習は}\index{そくよくせい}(lateral inhibition)といいます. }ことで, もっともはっかりつのたかいにゅーろんいがいはよくせいされてはっかしないようになる, というきこうです\footnote{たとえばANNでもちいられるSoftmaxやMax-poolingのそうさはWTAのめかにずむのもでるかの1つといえます. }. WTAのきこうによるがくしゅうは@側抑制}(lateral inhibition)と言います. }ことで, 最も発火率の高いニューロン以外は抑制されて発火しないようになる, という機構です\footnote{例えばANNで用いられるSoftmaxやMax-poolingの操作はWTAのメカニズムのモデル化の1つといえます. }. WTAの機構による学習は}競合学習}(Competitive learning)と呼んだりします。Diehlらが提案したネットワークは2層から成り立っており, 1層目は入力層($28\times 28=$784ニューロン)で, MNISTデータセットの画像1画素に1つのニューロンが対応します. このとき, 画像はポアソン過程モデルでスパイク列に符号化(encoding)されます. 2層目は興奮性ニューロンおよび同数の抑制性ニューロンから成ります(図\ref{fig:diehl}). 1つの興奮性ニューロンは1つの抑制性ニューロンに投射し, 抑制性ニューロンは自分に入力したニューロン以外の興奮性ニューロンを抑制します. こうすることで側抑制の仕組みが実装されています. 
\begin{figure}[htbp]
    \centering
    \includegraphics[scale=1.5]{figs/diehl.pdf}
    \caption{(Diehl \& Cook, 2015)で提案されたネットワーク.  1層目が入力層,2層目が興奮性ニューロン層と抑制性ニューロン層から構成されます。興奮性ニューロンと抑制性ニューロンの間の結合重みは固定で,入力層から興奮性ニューロン層へ投射される結合重みをSTDP則で学習します。}
    \label{fig:diehl}
\end{figure}
\subsection{ニューロンとシナプスのモデル}
ニューロンとシナプスのモデルとしてはConductance-basedモデルを用いています. 
\subsubsection{ニューロンのモデル}
興奮性ニューロンと抑制性ニューロンの膜電位$V$は次の式に従います. 
\begin{equation}
\tau \frac{d V}{d t}=\left(E_{\text {rest}}-V\right)+g_{e}\left(E_{\text {exc}}-V\right)+g_{i}\left(E_{\text {inh}}-V\right)    
\end{equation}
これは通常のConductance-basedモデルと同じ式ですが, ネットワークの発火率を一定に保つため, 興奮性ニューロンの膜電位の発火閾値は発火のたびに$\theta=0.05 $mV上昇するとします. 発火の無い場合は時定数$10^7$ msで減衰します. 著者実装では350 ms間, 画像をエンコードしたスパイク列を入力し, 変数をリセットするために150 ms間, 何も入力しないブランク期間を設定しています. この間に膜電位等は全てリセットされますが, 発火閾値のみ, 少し減衰するだけに留まります. さらに, 興奮性ニューロンの発火数が少ない場合には, 入力のポアソンスパイクの発火率を上げ, 再度同じ画像を提示します. 
また, 興奮性ニューロンの膜電位の時定数としては生理学的に逸脱した100 msという値を用いています. これは時定数を大きくすることで, 多くのスパイク入力を積分することができ, ノイズによる影響を減らすことができると説明されています. 
まず,準備としてこのようなニューロンを実装しておきます\footnote{コードは\texttt{./TrainingSNN/Models/Neurons.py}に含まれます. }. コード自体は2章で紹介したConductance-based LIFに修正を加えたものです。
\begin{minted}[frame=lines, framesep=2mm, baselinestretch=1.2, bgcolor=shadecolor,fontsize=\small]{python}
class DiehlAndCook2015LIF:
    def __init__(self, N, dt=1e-3, tref=5e-3, tc_m=1e-1,
                 vrest=-65, vreset=-65, init_vthr=-52, vpeak=20,
                 theta_plus=0.05, theta_max=35, tc_theta=1e4,
                 e_exc=0, e_inh=-100):
        self.N = N
        self.dt = dt
        self.tref = tref
        self.tc_m = tc_m 
        self.vreset = vreset
        self.vrest = vrest
        self.init_vthr = init_vthr
        self.theta = np.zeros(N)
        self.theta_plus = theta_plus
        self.theta_max = theta_max
        self.tc_theta = tc_theta
        self.vpeak = vpeak
        self.e_exc = e_exc \section{興奮性シナプスの平衡電位}
        self.e_inh = e_inh \section{抑制性シナプスの平衡電位}
        
        self.v = self.vreset*np.ones(N)
        self.vthr = self.init_vthr
        self.v_ = None
        self.tlast = 0
        self.tcount = 0
    
    def initialize_states(self, random_state=False):
        if random_state:
            self.v = self.vreset + \ 
                     np.random.rand(self.N)*(self.vthr-self.vreset) 
        else:
            self.v = self.vreset*np.ones(self.N)
        self.vthr = self.init_vthr
        self.theta = np.zeros(self.N)
        self.tlast = 0
        self.tcount = 0
        
    def __call__(self, g_exc, g_inh):
        I_synExc = g_exc*(self.e_exc - self.v) 
        I_synInh = g_inh*(self.e_inh - self.v)
        dv = (self.vrest - self.v + I_synExc + I_synInh) / self.tc_m
        v = self.v+((self.dt*self.tcount)>(self.tlast+self.tref))*dv*self.dt
        
        s = 1*(v>=self.vthr) #発火時は1, その他は0の出力
        
        \section{閾値の更新}
        theta = (1-self.dt/self.tc_theta)*self.theta + self.theta_plus*s
        self.theta = np.clip(theta, 0, self.theta_max)
        self.vthr = self.theta + self.init_vthr
        
        self.tlast = self.tlast*(1-s) + self.dt*self.tcount*s
        v = v*(1-s) + self.vpeak*s #閾値を超えると膜電位をvpeakにする
        self.v_ = v #発火時の電位も含めて記録するための変数
        self.v = v*(1-s) + self.vreset*s  #発火時に膜電位をリセット
        self.tcount += 1
        
        return s 
\end{minted}
追加した部分は閾値の更新のところです。最終的な閾値は\texttt{self.theta}と\texttt{self.init\_vthr}を加算した\texttt{self.vthr}を用います。変動するのは\texttt{self.theta}ですが,これはスパイクトレースと同様の実装をしています。ただし,閾値が上がりすぎることを防ぐため,上限を\texttt{self.theta\_max}として\texttt{np.clip()}により制限しています。
\subsubsection{シナプスのモデル}
シナプスにはConductance-basedの単一指数関数型シナプス(single exponential synapse)を用いています. スパイクが入るとコンダクタンス$g$は$w$だけあがり, その他では以下のように減衰します
\begin{equation}
\tau_{x} \frac{dg_x}{d t}=-g_x    
\end{equation}
ただし,$x=e, i$で,それぞれ興奮性,抑制性を意味する添え字です. なお,この実装自体は第2章で述べたものと変わりません. 
\subsection{興奮性ニューロンのラベリング}
このネットワークには出力層がなく, 通常とは異なる形式で画像を分類しています. まず, 訓練時において興奮性ニューロンの活動を全て記録します. 次に, 画像に元々されていたラベルを用いて, 興奮性の各ニューロンの各ラベルの画像への平均発火数を計算します. このとき, 各ニューロンにおいて最も発火数が多かったラベルを求め, そのラベルを各ニューロンに割り当てます. そして, 割り当てたラベルを用い, 入力画像に対する各ラベルに割り当てられたニューロンの平均発火数を求めます. このとき平均発火数が最も高いニューロン群のラベルが, 入力画像の予測ラベルとなります. また, 推論時には訓練時においてニューロンに割り当てたラベルを用います.
それではニューロンへラベルを割り当てる関数(\texttt{assign\_labels})を実装してみましょう\footnote{以降のコードは特に明記が無い限り\texttt{./TrainingSNN/LIF\_WTA\_STDP\_MNIST.py}に含まれます。また一部,BindsNETの実装を参考にしています。}。\texttt{assign\_labels}は5つの変数を引数に取ります。\texttt{spikes}は, サイズが\texttt{(n\_samples, n\_neurons)}の2次元配列で,各サンプルにおいて各興奮性ニューロンが何回発火したかを記録したものです。 \texttt{labels}はサイズが\texttt{(n\_samples, )}の配列で,各サンプルの教師ラベルを表します。\texttt{n\_labels}はラベルの数です。今回はMNISTなので10個です。 \texttt{rates}はサイズが\texttt{(n\_samples, n\_neurons)}の配列で,各興奮性ニューロンの各ラベルに対する発火率を表します。\texttt{rates}は2度目の計算以降に引数として渡すと,過去の\texttt{rates}と\texttt{alpha}によって調整される割合で指数平均を取ります。
\begin{minted}[frame=lines, framesep=2mm, baselinestretch=1.2, bgcolor=shadecolor,fontsize=\small]{python}
def assign_labels(spikes, labels, n_labels, rates=None, alpha=1.0):
    n_neurons = spikes.shape[1] 
    
    if rates is None:        
        rates = np.zeros((n_neurons, n_labels)).astype(np.float32)
    
    \section{時間の軸でスパイク数の和を取る}
    for i in range(n_labels):
        \section{サンプル内の同じラベルの数を求める}
        n_labeled = np.sum(labels == i).astype(np.int16)
    
        if n_labeled > 0:
            \section{label == iのサンプルのインデックスを取得}
            indices = np.where(labels == i)[0]
            
            \section{label == iに対する各ニューロンごとの平均発火率を計算}
            #(前回の発火率との移動平均)
            rates[:, i] = alpha*rates[:, i] + \ 
                          (np.sum(spikes[indices], axis=0)/n_labeled)
    
    sum_rate = np.sum(rates, axis=1)
    sum_rate[sum_rate==0] = 1
    \section{クラスごとの発火頻度の割合を計算する}
    proportions = rates / np.expand_dims(sum_rate, 1) \section{(n_neurons, n_labels)}
    proportions[proportions != proportions] = 0  \section{Set NaNs to 0}
    
    \section{最も発火率が高いラベルを各ニューロンに割り当てる (n_neurons,)}
    assignments = np.argmax(proportions, axis=1).astype(np.uint8) 
    return assignments, proportions, rates
\end{minted}
ここで\texttt{assignments}はサイズが\texttt{(n\_neurons,)}の配列で,各ニューロンに割り当てられたラベルを表します。これを用いて,各サンプルに対してラベルを予測する関数\texttt{prediction}を実装しましょう。
\begin{minted}[frame=lines, framesep=2mm, baselinestretch=1.2, bgcolor=shadecolor,fontsize=\small]{python}
def prediction(spikes, assignments, n_labels):    
    n_samples = spikes.shape[0]
    
    \section{各サンプルについて各ラベルの発火率を見る}
    rates = np.zeros((n_samples, n_labels)).astype(np.float32)
    
    for i in range(n_labels):
        \section{各ラベルが振り分けられたニューロンの数}
        n_assigns = np.sum(assignments == i).astype(np.uint8)
    
        if n_assigns > 0:
            \section{各ラベルのニューロンのインデックスを取得}
            indices = np.where(assignments == i)[0]
    
            \section{各ラベルのニューロンのレイヤー全体における平均発火数を求める}
            rates[:, i] = np.sum(spikes[:, indices], axis=1) / n_assigns
    
    \section{レイヤーの平均発火率が最も高いラベルを出力}
    return np.argmax(rates, axis=1).astype(np.uint8) \section{(n_samples, )}
\end{minted}
\texttt{prediction}は3つの引数, \texttt{spikes}, \texttt{assignments}, \texttt{n\_labels}を取ります。これらはそれぞれ先ほど説明した同名の配列と同じです。そして,各ラベルに紐づけられたニューロンごとに平均発火数を計算し,最も平均発火数の多かったラベルを入力サンプルのラベルとします。
\subsection{MNISTデータセットのスパイク列への変換}
入力として用いるために,MNISTデータセットをスパイク列へ変換する関数を書きましょう。MNISTデータセットを読み込む関数を別に書いても良いですが,煩雑となるため,ANNのライブラリに付属するデータ読み込み関数を利用してみます。ここでは例としてChainerの\texttt{chainer.datasets.get\_mnist()}を用いますが,他のライブラリでも大きな修正をすることなく実装できると思います。
\begin{minted}[frame=lines, framesep=2mm, baselinestretch=1.2, bgcolor=shadecolor,fontsize=\small]{python}
def online_load_and_encoding_dataset(dataset, i, dt, n_time, max_fr=32,
                                     norm=196):
    fr_tmp = max_fr*norm/np.sum(dataset[i][0])
    fr = fr_tmp*np.repeat(np.expand_dims(dataset[i][0],
                                         axis=0), n_time, axis=0)
    input_spikes = np.where(np.random.rand(n_time, 784) < fr*dt, 1, 0)
    input_spikes = input_spikes.astype(np.uint8)
    return input_spikes
\end{minted}
この関数の使用例と, スパイク列への変換が正しく行われているかを確認するコードは次のようになります。
\begin{minted}[frame=lines, framesep=2mm, baselinestretch=1.2, bgcolor=shadecolor,fontsize=\small]{python}
import chainer
dt = 1e-3; t_inj = 0.350; nt_inj = round(t_inj/dt)
train, _ = chainer.datasets.get_mnist() \section{ChainerによるMNISTデータの読み込み}
input_spikes = online_load_and_encoding_dataset(dataset=train, i=0,
                                                dt=dt, n_time=nt_inj)
\section{描画}
plt.imshow(np.reshape(np.sum(input_spikes, axis=0), (28, 28)),
           cmap="gray")
plt.show()
\end{minted}
ここでスパイク列を時間的に加算し,28$\times$28の配列に変換した後に描画をしています。結果は図\ref{fig:jitteredMNIST}のようになります。
\begin{figure}[htbp]
    \centering
    \includegraphics[scale=0.4]{figs/five.pdf}
    \caption{スパイク列に変換したMNISTデータの例(5).}
    \label{fig:jitteredMNIST}
\end{figure}
\subsection{ネットワークの実装}
それではネットワークを実装してみましょう。長いですが,それぞれの部分はこれまでの実装の組み合わせです。ただし,著者実装におけるハイパーパラメータでは正常に学習が進まなかったので,各ハイパーパラメータの値は変更しています\footnote{そもそも論文にハイパーパラメータが書いておらず,著者実装のコードを読むしかありません。}。
\begin{minted}[frame=lines, framesep=2mm, baselinestretch=1.2, bgcolor=shadecolor,fontsize=\small]{python}
class DiehlAndCook2015Network:
    def __init__(self, n_in=784, n_neurons=100, wexc=2.25, winh=0.875,
                 dt=1e-3, wmin=0.0, wmax=5e-2, lr=(1e-2, 1e-4),
                 update_nt=100):
        self.dt = dt
        self.lr_p, self.lr_m = lr
        self.wmax = wmax
        self.wmin = wmin
        self.n_neurons = n_neurons
        self.n_in = n_in
        self.norm = 0.1
        self.update_nt = update_nt
        \section{Neurons}
        self.exc_neurons = DiehlAndCook2015LIF(n_neurons, dt=dt, tref=5e-3,
                                               tc_m=1e-1,
                                               vrest=-65, vreset=-65, 
                                               init_vthr=-52,
                                               vpeak=20, theta_plus=0.05,
                                               theta_max=35,
                                               tc_theta=1e4,
                                               e_exc=0, e_inh=-100)
        self.inh_neurons = ConductanceBasedLIF(n_neurons, dt=dt, tref=2e-3,
                                               tc_m=1e-2,
                                               vrest=-60, vreset=-45,
                                               vthr=-40, vpeak=20,
                                               e_exc=0, e_inh=-85)
        \section{Synapses}
        self.input_synapse = SingleExponentialSynapse(n_in, dt=dt, td=1e-3)
        self.exc_synapse = SingleExponentialSynapse(n_neurons, dt=dt, td=1e-3)
        self.inh_synapse = SingleExponentialSynapse(n_neurons, dt=dt, td=2e-3)
        
        self.input_synaptictrace = SingleExponentialSynapse(n_in, dt=dt,
                                                            td=2e-2)
        self.exc_synaptictrace = SingleExponentialSynapse(n_neurons, dt=dt,
                                                          td=2e-2)
        
        \section{Connections (重みの初期化)}
        initW = 1e-3*np.random.rand(n_neurons, n_in)
        self.input_conn = FullConnection(n_in, n_neurons,
                                         initW=initW)
        self.exc2inh_W = wexc*np.eye(n_neurons)
        self.inh2exc_W = (winh/n_neurons)*(np.ones((n_neurons, n_neurons)) \
                          - np.eye(n_neurons))
        
        self.delay_input = DelayConnection(N=n_neurons, delay=5e-3, dt=dt)
        self.delay_exc2inh = DelayConnection(N=n_neurons, delay=2e-3, dt=dt)
        self.g_inh = np.zeros(n_neurons)
        
        self.tcount = 0
        self.s_in_ = np.zeros((self.update_nt, n_in)) 
        self.s_exc_ = np.zeros((n_neurons, self.update_nt))
        self.x_in_ = np.zeros((self.update_nt, n_in)) 
        self.x_exc_ = np.zeros((n_neurons, self.update_nt))
        
    \section{スパイクトレースのリセット}
    def reset_trace(self):
        self.s_in_ = np.zeros((self.update_nt, self.n_in)) 
        self.s_exc_ = np.zeros((self.n_neurons, self.update_nt))
        self.x_in_ = np.zeros((self.update_nt, self.n_in)) 
        self.x_exc_ = np.zeros((self.n_neurons, self.update_nt))
        self.tcount = 0
    
    \section{状態の初期化}
    def initialize_states(self):
        self.exc_neurons.initialize_states()
        self.inh_neurons.initialize_states()
        self.delay_input.initialize_states()
        self.delay_exc2inh.initialize_states()
        self.input_synapse.initialize_states()
        self.exc_synapse.initialize_states()
        self.inh_synapse.initialize_states()
        
    def __call__(self, s_in, stdp=True):
        \section{入力層}
        c_in = self.input_synapse(s_in)
        x_in = self.input_synaptictrace(s_in)
        g_in = self.input_conn(c_in)
        \section{興奮性ニューロン層}
        s_exc = self.exc_neurons(self.delay_input(g_in), self.g_inh)
        c_exc = self.exc_synapse(s_exc)
        g_exc = np.dot(self.exc2inh_W, c_exc)
        x_exc = self.exc_synaptictrace(s_exc)
        \section{抑制性ニューロン層        }
        s_inh = self.inh_neurons(self.delay_exc2inh(g_exc), 0)
        c_inh = self.inh_synapse(s_inh)
        self.g_inh = np.dot(self.inh2exc_W, c_inh)
        if stdp:
            \section{スパイク列とスパイクトレースを記録}
            self.s_in_[self.tcount] = s_in
            self.s_exc_[:, self.tcount] = s_exc
            self.x_in_[self.tcount] = x_in 
            self.x_exc_[:, self.tcount] = x_exc
            self.tcount += 1
            \section{Online STDP}
            if self.tcount == self.update_nt:
                W = np.copy(self.input_conn.W)
                
                \section{postに投射される重みが均一になるようにする}
                W_abs_sum = np.expand_dims(np.sum(np.abs(W), axis=1), 1)
                W_abs_sum[W_abs_sum == 0] = 1.0
                W *= self.norm / W_abs_sum
                
                \section{STDP則}
                dW = self.lr_p*(self.wmax-W)*np.dot(self.s_exc_, self.x_in_)
                dW -= self.lr_m*W*np.dot(self.x_exc_, self.s_in_)
                clipped_dW = np.clip(dW / self.update_nt, -5e-2, 5e-2)
                self.input_conn.W = np.clip(W + clipped_dW,
                                            self.wmin, self.wmax)
                self.reset_trace() \section{スパイク列とスパイクトレースをリセット}
        
        return s_exc
\end{minted}
ここで\texttt{\_\_call\_\_()}関数は入力スパイク列\texttt{s\_in}とSTDP則による学習をするかどうかのBoolean変数である\texttt{stdp}の2つの値を引数とします。\texttt{stdp}が\textttt{True}のとき,入力層と興奮性ニューロン層のスパイク列,スパイクトレースが記録され,\texttt{self.tcount}が\textttt{self.update\_nt}となったときにSTDP則による重みの更新が行われます。重みの更新の前に,興奮性ニューロンへ投射される重みの合計が\texttt{self.norm}となるように正規化をしています。また,ここでのSTDP則は重み依存的なものとしています。なお,重みの更新時に重みが
\texttt{self.wmin}と\texttt{self.wmax}範囲となるように\texttt{np.clip()}で制限をしています。
\subsection{学習イテレーションと結果の表示}
実際にシミュレーションをする部分を書いていきます。タイムステップは1 msとし,350 msの間は画像を変換したスパイク列を入力,150 msの間何も入力しない,というようにします。また,興奮性ニューロンの数は100個とし,訓練データのサンプル数は10000, エポック数は20とします。
なお,学習後にネットワークの評価をするために,このファイル内の関数を呼び出します。そのため,この部分は外部から呼び出されないように\texttt{if \_\_name\_\_ == '\_\_main\_\_':}と記述した中に書いておきます。
\begin{minted}[frame=lines, framesep=2mm, baselinestretch=1.2, bgcolor=shadecolor,fontsize=\small]{python}
if __name__ == '__main__':
    \section{350ms画像入力,150ms入力なしでリセットさせる(膜電位の閾値以外)}
    dt = 1e-3 \section{タイムステップ(sec)}
    t_inj = 0.350 \section{刺激入力時間(sec)}
    t_blank = 0.150 \section{ブランク時間(sec)}
    nt_inj = round(t_inj/dt)
    nt_blank = round(t_blank/dt)
    
    n_neurons = 100 \section{興奮性/抑制性ニューロンの数}
    n_labels = 10 \section{ラベル数}
    n_epoch = 20 \section{エポック数}
    
    n_train = 10000 \section{訓練データの数}
    update_nt = nt_inj \section{STDP則による重みの更新間隔}
    
    \section{ChainerによるMNISTデータの読み込み}
    train, _ = chainer.datasets.get_mnist()
    labels = np.array([train[i][1] for i in range(n_train)]) \section{ラベルの配列}
    
    \section{ネットワークの定義}
    network = DiehlAndCook2015Network(n_in=784, n_neurons=n_neurons,
                                      wexc=2.25, winh=0.875,
                                      dt=dt, wmin=0.0, wmax=0.1,
                                      lr=(1e-2, 1e-4),
                                      update_nt=update_nt)
    
    network.initialize_states() \section{ネットワークの初期化}
    spikes = np.zeros((n_train, n_neurons)).astype(np.uint8) \section{スパイク記録変数}
    accuracy_all = np.zeros(n_epoch) \section{訓練精度を記録する変数}
    blank_input = np.zeros(784) \section{ブランク入力}
    init_max_fr = 32 \section{初期のポアソンスパイクの最大発火率}
    
    \section{結果を保存するディレクトリ}
    results_save_dir = "./LIF_WTA_STDP_MNIST_results/"
    os.makedirs(results_save_dir, exist_ok=True) \section{ディレクトリが無ければ作成}
    
    \section{Simulation}
    for epoch in range(n_epoch):
        for i in tqdm(range(n_train)):
            max_fr = init_max_fr
            while(True):
                \section{入力スパイクをオンラインで生成}
                input_spikes = online_load_and_encoding_dataset(train, i, dt,
                                                                nt_inj,
                                                                max_fr)
                spike_list = [] \section{サンプルごとにスパイクを記録するリスト}
                \section{画像刺激の入力}
                for t in range(nt_inj):
                    s_exc = network(input_spikes[t], stdp=True)
                    spike_list.append(s_exc)
                
                \section{スパイク数を記録}
                spikes[i] = np.sum(np.array(spike_list), axis=0)
                
                \section{ブランク刺激の入力}
                for _ in range(nt_blank):
                    _ = network(blank_input, stdp=False)
    
                \section{スパイク数を計算}
                num_spikes_exc = np.sum(np.array(spike_list))
                \section{スパイク数が5より大きければ次のサンプルへ}
                if num_spikes_exc >= 5:
                    break
                else: \section{スパイク数が5より小さければ入力発火率を上げて再度刺激}
                    max_fr += 16
        
        \section{ニューロンを各ラベルに割り当てる}
        if epoch == 0:
            assignments, proportions, rates = assign_labels(spikes, labels,
                                                            n_labels)
        else:
            assignments, proportions, rates = assign_labels(spikes, labels,
                                                            n_labels, rates)
        print("Assignments:\n", assignments)
        
        \section{スパイク数の確認(正常に発火しているか確認)}
        sum_nspikes = np.sum(spikes, axis=1)
        mean_nspikes = np.mean(sum_nspikes).astype(np.float16)
        print("Ave. spikes:", mean_nspikes)
        print("Min. spikes:", sum_nspikes.min())
        print("Max. spikes:", sum_nspikes.max())
    
        \section{入力サンプルのラベルを予測する}
        predicted_labels = prediction(spikes, assignments, n_labels)
        
        \section{訓練精度を計算}
        accuracy = np.mean(np.where(labels==predicted_labels, 1, 0))
        print("epoch :", epoch, " accuracy :", accuracy)
        accuracy_all[epoch] = accuracy
        
        \section{学習率の減衰}
        network.lr_p *= 0.85
        network.lr_m *= 0.85
        
        \section{重みの保存(エポック毎)}
        np.save(results_save_dir+"weight_epoch"+str(epoch)+".npy",
                network.input_conn.W)
        
    \section{結果}
    plt.figure(figsize=(5,4))
    plt.plot(np.arange(1, n_epoch+1), accuracy_all*100,
             color="k")
    plt.xlabel("Epoch")
    plt.ylabel("Train accuracy (%)")
    plt.savefig(results_save_dir+"accuracy.png")
    
    np.save(results_save_dir+"assignments.npy", assignments)
    np.save(results_save_dir+"weight.npy", network.input_conn.W)
    np.save(results_save_dir+"exc_neurons_theta.npy",
            network.exc_neurons.theta)
\end{minted}
シミュレーション部は入れ子状のループとなっています。\texttt{while}によるループは, サンプルを入力した時に興奮性ニューロン全体の発火数が5を超えない限り, 入力の発火率を増加させて再度スパイク列を入力する,ということを行います。
各エポック終了時には刺激時の興奮性ニューロンの発火情報から,興奮性ニューロンの各ラベルへの割り当てと各サンプルのラベルの予測,および訓練精度の計算を行います。最後にSTDP則における学習率を減衰させます。
学習が終了した後は訓練精度の変化と各ニューロンに割り当てたラベル(\texttt{assignments}),入力層から興奮性ニューロンへの結合重み,興奮性ニューロンの閾値を保存しておきます。なお,学習時における訓練精度の変化は図\ref{fig:MNISTaccuracy}のようになりました。
\begin{figure}[htbp]
    \centering
    \includegraphics[scale=0.5]{figs/accuracy.pdf}
    \caption{訓練時の精度の変化. 最終的な訓練精度は69.9\%となりました。}
    \label{fig:MNISTaccuracy}
\end{figure}
\subsection{ネットワークの評価}
学習後のネットワークを学習に用いなかったテストデータで評価してみましょう。
\footnote{コードは\texttt{./TrainingSNN/LIF\_WTA\_STDP\_MNIST\_evaluation.py}です。}。ほぼネットワークの学習の際に記述したコードを流用するだけです。まず,学習後の重みを読み込んで,テストデータ(1000サンプル)を入力し,それに対するスパイクを記録します。次に訓練時に行った各ニューロンのラベルへの割り当てを用いて各サンプルのラベルを予測し,実際のラベルと比較し,推論精度を計算します。
\begin{minted}[frame=lines, framesep=2mm, baselinestretch=1.2, bgcolor=shadecolor,fontsize=\small]{python}
import chainer
from LIF_WTA_STDP_MNIST import online_load_and_encoding_dataset, prediction
from LIF_WTA_STDP_MNIST import DiehlAndCook2015Network
\section{350ms画像入力,150ms入力なしでリセットさせる(膜電位の閾値以外)}
dt = 1e-3 \section{タイムステップ(sec)}
t_inj = 0.350 \section{刺激入力時間(sec)}
t_blank = 0.150 \section{ブランク時間(sec)}
nt_inj = round(t_inj/dt)
nt_blank = round(t_blank/dt)
n_neurons = 100 #興奮性/抑制性ニューロンの数
n_labels = 10 #ラベル数
n_train = 1000 \section{訓練データの数}
update_nt = nt_inj \section{STDP則による重みの更新間隔}
_, test = chainer.datasets.get_mnist() \section{ChainerによるMNISTデータの読み込み}
labels = np.array([test[i][1] for i in range(n_train)]) \section{ラベルの配列}
\section{結果が保存されているディレクトリ}
results_save_dir = "./LIF_WTA_STDP_MNIST_results/"
\section{ネットワークの定義}
network = DiehlAndCook2015Network(n_in=784, n_neurons=n_neurons,
                                  wexc=2.25, winh=0.875,
                                  dt=dt, wmin=0.0, wmax=0.1,
                                  lr=(1e-2, 1e-4),
                                  update_nt=update_nt)
network.initialize_states() \section{ネットワークの初期化}
\section{重みと閾値のload}
network.input_conn.W = np.load(results_save_dir+"weight.npy")
network.exc_neurons.theta = np.load(results_save_dir+"exc_neurons_theta.npy")
network.exc_neurons.theta_plus = 0 \section{閾値が上昇しないようにする}
#スパイクを記録する変数
spikes = np.zeros((n_train, n_neurons)).astype(np.uint8)
blank_input = np.zeros(784) \section{ブランク入力}
init_max_fr = 32 \section{初期のポアソンスパイクの最大発火率}
for i in tqdm(range(n_train)):
    max_fr = init_max_fr
    while(True):
        \section{入力スパイクをオンラインで生成}
        input_spikes = online_load_and_encoding_dataset(test, i, dt,
                                                        nt_inj, max_fr)
        spike_list = [] \section{サンプルごとにスパイクを記録するリスト}
        \section{画像刺激の入力}
        for t in range(nt_inj):
            s_exc = network(input_spikes[t], stdp=False)
            spike_list.append(s_exc)
        
        spikes[i] = np.sum(np.array(spike_list), axis=0) \section{スパイク数を記録}
        
        \section{ブランク刺激の入力}
        for _ in range(nt_blank):
            _ = network(blank_input, stdp=False)
        num_spikes_exc = np.sum(np.array(spike_list)) \section{スパイク数を計算}
        if num_spikes_exc >= 5: \section{スパイク数が5より大きければ次のサンプルへ}
            break
        else: \section{スパイク数が5より小さければ入力発火率を上げて再度刺激}
            max_fr += 16
            
\section{入力サンプルのラベルを予測する}
assignments = np.load(results_save_dir+"assignments.npy")
predicted_labels = prediction(spikes, assignments, n_labels)
\section{訓練精度を計算}
accuracy = np.mean(np.where(labels==predicted_labels, 1, 0))
print("Test accuracy :", accuracy)
\end{minted}
実行した結果,\colorbox{shadecolor}{\texttt{Test accuracy : 0.653}}となり,テストデータ(1000サンプル)での精度は65.3\%となりました。100個のニューロンを学習させた際の論文における精度が約85\%であるので,それに比べるとかなり低い値です。これには学習データを少なくしている,ハイパーパラメータを著者実装とは異なるものにしているということなどが原因であると考えられます。
\subsection{興奮性ニューロンの受容野の描画}
最後に,学習後の入力層から興奮性ニューロン層へのシナプス重みを描画してみましょう\footnote{コードは\texttt{./TrainingSNN/LIF\_WTA\_STDP\_MNIST\_visualize\_weights.py}です。}。各興奮性ニューロンへ投射する784個のシナプス重みを28$\times$28に\texttt{reshape}して描画するだけです。
\begin{minted}[frame=lines, framesep=2mm, baselinestretch=1.2, bgcolor=shadecolor,fontsize=\small]{python}
epoch = 19
n_neurons = 100
\section{結果が保存されているディレクトリ}
results_save_dir = "./LIF_WTA_STDP_MNIST_results/"
input_conn_W = np.load(results_save_dir+"weight_epoch"+str(epoch)+".npy")
reshaped_W = np.reshape(input_conn_W, (n_neurons, 28, 28))
\section{描画}
fig = plt.figure(figsize=(6,6))
fig.subplots_adjust(left=0, right=1, bottom=0, top=1,
                    hspace=0.05, wspace=0.05)
row = col = np.sqrt(n_neurons)
for i in tqdm(range(n_neurons)):
  ax = fig.add_subplot(row, col, i+1, xticks=[], yticks=[])
  ax.imshow(reshaped_W[i], cmap="gray")
plt.savefig("weights_"+str(epoch)+".png")
plt.show()
\end{minted}
結果は図\ref{fig:Diehl_weights}のようになります。
\begin{figure}[htbp]
 \begin{minipage}{0.5\hsize}
  \begin{center}
   \includegraphics[width=60mm]{figs/weights_0.pdf}
  \end{center}
 \end{minipage}
 \begin{minipage}{0.5\hsize}
  \begin{center}
   \includegraphics[width=60mm]{figs/weights_19.pdf}
  \end{center}
 \end{minipage}
\caption{学習後の100個の興奮性ニューロンの重みを描画したもの。各数字に対応するフィルターが生まれていることが分かります。(左) 1エポック終了時の重み。受容野が生じつつあるのが分かります。(右) 20エポック終了時。一番上の段の10個のニューロンをラベルに割り当てた結果は``5, 2, 6, 3, 3, 9, 4, 3, 0, 2"となっており,受容野の見た目と一致しています。}
\label{fig:Diehl_weights}
\end{figure}
\end{document}

% \section{ロジスティック回帰とパーセプトロン}
ロジスティック回帰 (logistic regression) 
, perceptron
1層パーセプトロン
\url{https://www.cs.utexas.edu/~gdurrett/courses/fa2022/perc-lr-connections.pdf}
\url{https://en.wikipedia.org/wiki/Perceptron}
\url{https://arxiv.org/abs/2012.03642}
perceptronは0/1 or -1/1のどちらか
UNDERSTANDING STRAIGHT-THROUGH ESTIMATOR IN TRAINING ACTIVATION QUANTIZED NEURAL NETS
Yoshua Bengio, Nicholas L´eonard, and Aaron Courville. Estimating or propagating gradients through stochastic neurons for conditional computation. arXiv preprint arXiv:1308.3432, 2013.
Hinton (2012) in his lecture 15b
G. Hinton. Neural networks for machine learning, 2012.
\url{https://www.cs.toronto.edu/~hinton/coursera_lectures.html}
delta rule
\begin{lstlisting}[language=julia]
using Random, PyPlot, ProgressMeter
rc("axes.spines", top=false, right=false)
\end{lstlisting}
\begin{lstlisting}[language=julia]
N = 400 # num of inputs
dims = 2  # dims of inputs 
Random.seed!(1234);
\end{lstlisting}
\begin{lstlisting}[language=julia]
X = [randn(Int(N/2), dims);  3.0 .+ randn(Int(N/2), dims)];
y = [zeros(Int(N/2)); ones(Int(N/2))];
\end{lstlisting}
\begin{lstlisting}[language=julia]
figure(figsize=(4, 4))
scatter(X[y.==0, 1], X[y.==0, 2])
scatter(X[y.==1, 1], X[y.==1, 2])
xlabel(L"$x_1$"); ylabel(L"$x_2$"); 
tight_layout()
\end{lstlisting}
\begin{figure}[ht]
	\centering
	\includegraphics[scale=0.8, max width=\linewidth]{./fig/local-learning-rule/logistic-regression-perceptron/cell005.png}
	\caption{cell005.png}
	\label{cell005.png}
\end{figure}
\begin{lstlisting}[language=julia]
m, n = 2, 1
\end{lstlisting}
\begin{lstlisting}[language=julia]
step(x) = 1.0(x > 0)
\end{lstlisting}
\begin{lstlisting}[language=julia]
x = -1:0.01:1
figure(figsize=(3, 2))
plot(x, step.(x))
tight_layout()
\end{lstlisting}
\begin{figure}[ht]
	\centering
	\includegraphics[scale=0.8, max width=\linewidth]{./fig/local-learning-rule/logistic-regression-perceptron/cell008.png}
	\caption{cell008.png}
	\label{cell008.png}
\end{figure}
Here σ denotes the (point-wise) activation function, $W \in R^{m\times n}$
is the weight-matrix and $b \in R^n$
is
the bias-vector. The vector $x \in R^m$ and the vector $y \in R^n$ denote the input, respectively the output
\begin{equation}
y=\sigma(W^\top x + b)
\end{equation}
\begin{align}
& \text { Initialize } W^0, b^0 \text {; } \\
& \text { for } k=1,2, \ldots \text { do } \\
& \qquad \begin{array}{|l}
\text { for } i=1, \ldots, s \text { do } \\
e_i=y_i-\sigma\left(\left(W^k\right)^{\top} x_i+b^k\right) \\
W^{k+1}=W^k+e_i x_i^{\top} \\
b^{k+1}=b^k+e_i
\end{array} \\
& \text { end }
\end{align}
\begin{lstlisting}[language=julia]
n_iter = 100
loss = zeros(n_iter);
W = randn(n, m)
b = randn(n)
for t in 1:n_iter
    ŷ = step.(W * X' .+ b)'
    e = y - ŷ
    loss[t] = sum(e.^2) 
    W[:, :] += 0.1*(1/200) * e' * X
    b[:] .+= 0.1*(1/200) * sum(e)
end
\end{lstlisting}
\begin{lstlisting}[language=julia]
figure(figsize=(3,2))
plot(loss)
xlabel("Iteration"); ylabel("Loss")
tight_layout()
\end{lstlisting}
\begin{figure}[ht]
	\centering
	\includegraphics[scale=0.8, max width=\linewidth]{./fig/local-learning-rule/logistic-regression-perceptron/cell011.png}
	\caption{cell011.png}
	\label{cell011.png}
\end{figure}
\begin{lstlisting}[language=julia]
ŷ = step.(W * X' .+ b)'; # prediction
\end{lstlisting}
\begin{lstlisting}[language=julia]
p1 = X[ŷ[:, 1] .== 0, :]
p2 = X[ŷ[:, 1] .== 1, :];
\end{lstlisting}
ax + by + c = 0  
y = -a/b x - c/b
\begin{lstlisting}[language=julia]
xx = -3:0.01:6
yy = -W[1]/W[2]*xx .- b / W[2];
\end{lstlisting}
\begin{lstlisting}[language=julia]
figure(figsize=(4, 4))
scatter(p1[:, 1], p1[:, 2])
scatter(p2[:, 1], p2[:, 2])
plot(xx, yy, color="k")
xlabel(L"$x_1$"); ylabel(L"$x_2$"); 
tight_layout()
\end{lstlisting}
\begin{figure}[ht]
	\centering
	\includegraphics[scale=0.8, max width=\linewidth]{./fig/local-learning-rule/logistic-regression-perceptron/cell016.png}
	\caption{cell016.png}
	\label{cell016.png}
\end{figure}
 % ok2
% \section{予測符号化}
\subsection{観測世界の階層的予測}
\textbf{階層的予測符号化(hierarchical predictive coding; HPC)} は\cite{Rao1999-zv}により導入された.構築するネットワークは入力層を含め,3層のネットワークとする.LGNへの入力として画像 $\mathbf{x} \in \mathbb{R}^{n_0}$を考える.画像 $\mathbf{x}$ の観測世界における隠れ変数,すなわち\textbf{潜在変数} (latent variable)を$\mathbf{r} \in \mathbb{R}^{n_1}$とし,ニューロン群によって発火率で表現されているとする (真の変数と $\mathbf{r}$は異なるので文字を分けるべきだが簡単のためにこう表す).このとき,


\mathbf{x} = f(\mathbf{U}\mathbf{r}) + \boldsymbol{\epsilon}


が成立しているとする.ただし,$f(\cdot)$は活性化関数 (activation function),$\mathbf{U} \in \mathbb{R}^{n_0 \times n_1}$は重み行列である.
$\boldsymbol{\epsilon} \in \mathbb{R}^{n_0}$ は $\mathcal{N}(\mathbf{0}, \sigma^2 \mathbf{I})$ からサンプリングされるとする.

潜在変数 $\mathbf{r}$はさらに高次 (higher-level)の潜在変数 $\mathbf{r}^h$により,次式で表現される.


\mathbf{r} = \mathbf{r}^{td}+\boldsymbol{\epsilon}^{td}=f(\mathbf{U}^h \mathbf{r}^h)+\boldsymbol{\epsilon}^{td}


ただし,Top-downの予測信号を $\mathbf{r}^{td}:=f(\mathbf{U}^h \mathbf{r}^h)$とした.また,$\mathbf{r}^{td} \in \mathbb{R}^{n_1}$, $\mathbf{r}^{h} \in \mathbb{R}^{n_2}$, $\mathbf{U}^h \in \mathbb{R}^{n_1 \times n_2}$ である.
$\boldsymbol{\epsilon}^{td} \in \mathbb{R}^{n_1}$は$\mathcal{N}(\mathbf{0}$, $\sigma_{td}^2 \mathbf{I}$) からサンプリングされるとする.

話は飛ぶが,Predictive codingのネットワークの特徴は
\begin{itemize}
\item 階層的な構造
\item 高次による低次の予測 (Feedback or Top-down信号)
\item 低次から高次への誤差信号の伝搬 (Feedforward or Bottom-up 信号)
\end{itemize}

である.ここまでは高次表現による低次表現の予測,というFeedback信号について説明してきたが,この部分はSparse codingでも同じである.それではPredictive codingのもう一つの要となる,低次から高次への予測誤差の伝搬というFeedforward信号はどのように導かれるのだろうか.結論から言えば,これは\textbf{復元誤差 (reconstruction error)の最小化を行う再帰的ネットワーク (recurrent network)を考慮することで自然に導かれる}.

\subsubsection{重み行列$\mathbf{A}$の作成}

\lstinputlisting[language=julia]{./text/local-learning-rule/self-organizing-map/002.jl}
UNDERSTANDING STRAIGHT-THROUGH ESTIMATOR IN TRAINING ACTIVATION QUANTIZED NEURAL NETS

Yoshua Bengio, Nicholas L´eonard, and Aaron Courville. Estimating or propagating gradients through stochastic neurons for conditional computation. arXiv preprint arXiv:1308.3432, 2013.

\lstinputlisting[language=julia]{./text/local-learning-rule/self-organizing-map/004.jl}
\lstinputlisting[language=julia]{./text/local-learning-rule/self-organizing-map/005.jl}
\lstinputlisting[language=julia]{./text/local-learning-rule/self-organizing-map/006.jl}
\lstinputlisting[language=julia]{./text/local-learning-rule/self-organizing-map/007.jl}
## 7.9.2 更新関数の定義

\subsubsection{正規方程式を用いた推定}条件に基づいて目的関数$L(\mathbf{\theta})$を微分すると次のような方程式が得られる.
$$
\mathbf{X}^\top\mathbf{X}\mathbf{\hat\theta}=\mathbf{X}^\top\mathbf{y}
$$
これを\textbf{正規方程式} (normal equation)と呼ぶ.この正規方程式より、係数の推定値は$\mathbf{\hat\theta}={(\mathbf{X}^\top\mathbf{X})}^{-1}X^\top\mathbf{y}$という式で得られる.なお,正規方程式自体は$\mathbf{y}=\mathbf{X}\mathbf{\theta}$の左から$\mathbf{X}^\top$をかける,と覚えると良い.
\lstinputlisting[language=julia]{./text/local-learning-rule/self-organizing-map/010.jl}
\lstinputlisting[language=julia]{./text/local-learning-rule/self-organizing-map/011.jl}
### 訓練データで学習

\lstinputlisting[language=julia]{./text/local-learning-rule/self-organizing-map/013.jl}
\lstinputlisting[language=julia]{./text/local-learning-rule/self-organizing-map/014.jl}
\begin{figure}[ht]
	\centering
	\includegraphics[scale=0.8, max width=\linewidth]{./fig/bayesian-brain/mcmc/cell014.png}
	\caption{cell014.png}
	\label{cell014.png}
\end{figure}
## 相図の描画

\lstinputlisting[language=julia]{./text/local-learning-rule/self-organizing-map/016.jl}
\lstinputlisting[language=julia]{./text/local-learning-rule/self-organizing-map/017.jl}
\lstinputlisting[language=julia]{./text/local-learning-rule/self-organizing-map/018.jl}
\lstinputlisting[language=julia]{./text/local-learning-rule/self-organizing-map/019.jl}
対称性の破れを考慮していないので,円系に成長している.
*ToDo*:
神経細胞極性についての記述.
\lstinputlisting[language=julia]{./text/local-learning-rule/self-organizing-map/021.jl}
\begin{figure}[ht]
	\centering
	\includegraphics[scale=0.8, max width=\linewidth]{./fig/motor-learning/infinite-horizon-ofc/cell021.png}
	\caption{cell021.png}
	\label{cell021.png}
\end{figure}
\subsubsection{非負主成分分析によるグリッドパターンの創発}
内側嗅内皮質(MEC)にある\textbf{グリッド細胞 (grid cells)} は六角形格子状の発火パターンにより自己位置等を符号化するのに貢献している.この発火パターンを生み出すモデルは多数あるが,\textbf{場所細胞(place cells)} の発火パターンを\textbf{非負主成分分析(nonnegative principal component analysis)} で次元削減するとグリッド細胞のパターンが生まれるというモデルがある \cite{Dordek2016-ff}.非線形Hebb学習を用いてこのモデルを実装しよう.なお,同様のことは\textbf{非負値行列因子分解 (NMF: nonnegative matrix factorization)} でも可能である.

\lstinputlisting[language=julia]{./text/local-learning-rule/self-organizing-map/023.jl}
パラメータを更新する関数を定義する.今回はより生理学的に妥当にするため,軟判定非負閾値関数を用いる.

\lstinputlisting[language=julia]{./text/local-learning-rule/self-organizing-map/025.jl}
\lstinputlisting[language=julia]{./text/local-learning-rule/self-organizing-map/026.jl}
\lstinputlisting[language=julia]{./text/local-learning-rule/self-organizing-map/027.jl}
\subsubsection{対角行列}

aaa

\lstinputlisting[language=julia]{./text/local-learning-rule/self-organizing-map/029.jl}
\lstinputlisting[language=julia]{./text/local-learning-rule/self-organizing-map/030.jl}
\begin{figure}[ht]
	\centering
	\includegraphics[scale=0.8, max width=\linewidth]{./fig/bayesian-brain/neural-sampling/cell030.png}
	\caption{cell030.png}
	\label{cell030.png}
\end{figure}
 % ok2

\chapter{生成モデルとエネルギーベースモデル}
% \section{エネルギーベースモデル}
本章では\textbf{エネルギーベースモデル (energy-based models; EBMs)}\index{えねるぎーべーすもでる (energy-based models; EBMs)@エネルギーベースモデル (energy-based models; EBMs)} という枠組みに含まれるモデルを紹介する.エネルギーベースモデルではネットワークの状態をスカラー値に変換するエネルギー関数 (あるいはコスト関数) を定義し,推論時と学習時の双方においてエネルギーを最小化するようにネットワークの状態を更新する.\citep{LeCun2006-dt}
入力 $\mathbf{x}\in \mathbb{R}^d$, エネルギー関数 $E_\theta: \mathbb{R}^d\to \mathbb{R}$を考える.
\begin{align}
p_\theta(\mathbf{x})&=\frac{\exp(-E_\theta(\mathbf{x}))}{Z_\theta}\\
Z_\theta &= \int \exp(-E_\theta(\mathbf{x})) d\mathbf{x}
\end{align}
$Z_\theta$は分配関数.
 % ok
% \section{予測符号化}
\subsection{観測世界の階層的予測}
\textbf{階層的予測符号化(hierarchical predictive coding; HPC)} は\cite{Rao1999-zv}により導入された.構築するネットワークは入力層を含め,3層のネットワークとする.LGNへの入力として画像 $\mathbf{x} \in \mathbb{R}^{n_0}$を考える.画像 $\mathbf{x}$ の観測世界における隠れ変数,すなわち\textbf{潜在変数} (latent variable)を$\mathbf{r} \in \mathbb{R}^{n_1}$とし,ニューロン群によって発火率で表現されているとする (真の変数と $\mathbf{r}$は異なるので文字を分けるべきだが簡単のためにこう表す).このとき,


\mathbf{x} = f(\mathbf{U}\mathbf{r}) + \boldsymbol{\epsilon}


が成立しているとする.ただし,$f(\cdot)$は活性化関数 (activation function),$\mathbf{U} \in \mathbb{R}^{n_0 \times n_1}$は重み行列である.
$\boldsymbol{\epsilon} \in \mathbb{R}^{n_0}$ は $\mathcal{N}(\mathbf{0}, \sigma^2 \mathbf{I})$ からサンプリングされるとする.

潜在変数 $\mathbf{r}$はさらに高次 (higher-level)の潜在変数 $\mathbf{r}^h$により,次式で表現される.


\mathbf{r} = \mathbf{r}^{td}+\boldsymbol{\epsilon}^{td}=f(\mathbf{U}^h \mathbf{r}^h)+\boldsymbol{\epsilon}^{td}


ただし,Top-downの予測信号を $\mathbf{r}^{td}:=f(\mathbf{U}^h \mathbf{r}^h)$とした.また,$\mathbf{r}^{td} \in \mathbb{R}^{n_1}$, $\mathbf{r}^{h} \in \mathbb{R}^{n_2}$, $\mathbf{U}^h \in \mathbb{R}^{n_1 \times n_2}$ である.
$\boldsymbol{\epsilon}^{td} \in \mathbb{R}^{n_1}$は$\mathcal{N}(\mathbf{0}$, $\sigma_{td}^2 \mathbf{I}$) からサンプリングされるとする.

話は飛ぶが,Predictive codingのネットワークの特徴は
\begin{itemize}
\item 階層的な構造
\item 高次による低次の予測 (Feedback or Top-down信号)
\item 低次から高次への誤差信号の伝搬 (Feedforward or Bottom-up 信号)
\end{itemize}

である.ここまでは高次表現による低次表現の予測,というFeedback信号について説明してきたが,この部分はSparse codingでも同じである.それではPredictive codingのもう一つの要となる,低次から高次への予測誤差の伝搬というFeedforward信号はどのように導かれるのだろうか.結論から言えば,これは\textbf{復元誤差 (reconstruction error)の最小化を行う再帰的ネットワーク (recurrent network)を考慮することで自然に導かれる}.

\subsubsection{重み行列$\mathbf{A}$の作成}

\lstinputlisting[language=julia]{./text/energy-based-model/hopfield-model/002.jl}
\lstinputlisting[language=julia]{./text/energy-based-model/hopfield-model/003.jl}
\lstinputlisting[language=julia]{./text/energy-based-model/hopfield-model/004.jl}
変更しない定数を保持する \jl{struct} の \jl{FHNParameter} と, 変数を保持する \jl{mutable struct} の \jl{FHN} を作成する.
\lstinputlisting[language=julia]{./text/energy-based-model/hopfield-model/006.jl}
\lstinputlisting[language=julia]{./text/energy-based-model/hopfield-model/007.jl}
\begin{figure}[ht]
	\centering
	\includegraphics[scale=0.8, max width=\linewidth]{./fig/bayesian-brain/neural-sampling/cell007.png}
	\caption{cell007.png}
	\label{cell007.png}
\end{figure}
## 7.9.2 更新関数の定義

\lstinputlisting[language=julia]{./text/energy-based-model/hopfield-model/009.jl}
\lstinputlisting[language=julia]{./text/energy-based-model/hopfield-model/010.jl}
\begin{figure}[ht]
	\centering
	\includegraphics[scale=0.8, max width=\linewidth]{./fig/bayesian-brain/neural-sampling/cell010.png}
	\caption{cell010.png}
	\label{cell010.png}
\end{figure}
## 画像の復元
\lstinputlisting[language=julia]{./text/energy-based-model/hopfield-model/012.jl}
\lstinputlisting[language=julia]{./text/energy-based-model/hopfield-model/013.jl}
\lstinputlisting[language=julia]{./text/energy-based-model/hopfield-model/014.jl}
\begin{figure}[ht]
	\centering
	\includegraphics[scale=0.8, max width=\linewidth]{./fig/bayesian-brain/mcmc/cell014.png}
	\caption{cell014.png}
	\label{cell014.png}
\end{figure}
## 相図の描画

 % ok
% \section{予測符号化}
\subsection{観測世界の階層的予測}
\textbf{階層的予測符号化(hierarchical predictive coding; HPC)} は\cite{Rao1999-zv}により導入された.構築するネットワークは入力層を含め,3層のネットワークとする.LGNへの入力として画像 $\mathbf{x} \in \mathbb{R}^{n_0}$を考える.画像 $\mathbf{x}$ の観測世界における隠れ変数,すなわち\textbf{潜在変数} (latent variable)を$\mathbf{r} \in \mathbb{R}^{n_1}$とし,ニューロン群によって発火率で表現されているとする (真の変数と $\mathbf{r}$は異なるので文字を分けるべきだが簡単のためにこう表す).このとき,


\mathbf{x} = f(\mathbf{U}\mathbf{r}) + \boldsymbol{\epsilon}


が成立しているとする.ただし,$f(\cdot)$は活性化関数 (activation function),$\mathbf{U} \in \mathbb{R}^{n_0 \times n_1}$は重み行列である.
$\boldsymbol{\epsilon} \in \mathbb{R}^{n_0}$ は $\mathcal{N}(\mathbf{0}, \sigma^2 \mathbf{I})$ からサンプリングされるとする.

潜在変数 $\mathbf{r}$はさらに高次 (higher-level)の潜在変数 $\mathbf{r}^h$により,次式で表現される.


\mathbf{r} = \mathbf{r}^{td}+\boldsymbol{\epsilon}^{td}=f(\mathbf{U}^h \mathbf{r}^h)+\boldsymbol{\epsilon}^{td}


ただし,Top-downの予測信号を $\mathbf{r}^{td}:=f(\mathbf{U}^h \mathbf{r}^h)$とした.また,$\mathbf{r}^{td} \in \mathbb{R}^{n_1}$, $\mathbf{r}^{h} \in \mathbb{R}^{n_2}$, $\mathbf{U}^h \in \mathbb{R}^{n_1 \times n_2}$ である.
$\boldsymbol{\epsilon}^{td} \in \mathbb{R}^{n_1}$は$\mathcal{N}(\mathbf{0}$, $\sigma_{td}^2 \mathbf{I}$) からサンプリングされるとする.

話は飛ぶが,Predictive codingのネットワークの特徴は
\begin{itemize}
\item 階層的な構造
\item 高次による低次の予測 (Feedback or Top-down信号)
\item 低次から高次への誤差信号の伝搬 (Feedforward or Bottom-up 信号)
\end{itemize}

である.ここまでは高次表現による低次表現の予測,というFeedback信号について説明してきたが,この部分はSparse codingでも同じである.それではPredictive codingのもう一つの要となる,低次から高次への予測誤差の伝搬というFeedforward信号はどのように導かれるのだろうか.結論から言えば,これは\textbf{復元誤差 (reconstruction error)の最小化を行う再帰的ネットワーク (recurrent network)を考慮することで自然に導かれる}.

\subsubsection{重み行列$\mathbf{A}$の作成}

\subsubsection{事前分布の設定}
事前分布$p(\mathbf{r})$としては,0においてピークがあり,裾の重い(heavy tail)を持つsparse distributionあるいは \textbf{super-Gaussian distribution} (Laplace 分布やCauchy分布などGaussian分布よりもkurtoticな分布)を用いるのが良い.このような分布では,$\mathbf{r}$の各要素$r_i$はほとんど0に等しく,ある入力に対しては大きな値を取る.$p(\mathbf{r})$は一般化して式(4), (5)のように表記する.


\begin{aligned}
p(\mathbf{r})&=\prod_j p(r_j)\\
p(r_j)&=\frac{1}{Z_{\beta}}\exp \left[-\beta S(r_j)\right]
\end{aligned}


ただし,$\beta$は逆温度(inverse temperature), $Z_{\beta}$は規格化定数 (分配関数) である.これらの用語は統計力学における正準分布 (ボルツマン分布)から来ている.$S(x)$と分布の関係をまとめた表が以下となる (cf. \url{https://pdfs.semanticscholar.org/be08/da912362bf40fe3ded78bdadc644f921b4e7.pdf}).

UNDERSTANDING STRAIGHT-THROUGH ESTIMATOR IN TRAINING ACTIVATION QUANTIZED NEURAL NETS

Yoshua Bengio, Nicholas L´eonard, and Aaron Courville. Estimating or propagating gradients through stochastic neurons for conditional computation. arXiv preprint arXiv:1308.3432, 2013.

\lstinputlisting[language=julia]{./text/energy-based-model/boltzmann-machine/004.jl}
\lstinputlisting[language=julia]{./text/energy-based-model/boltzmann-machine/005.jl}
\lstinputlisting[language=julia]{./text/energy-based-model/boltzmann-machine/006.jl}
\begin{figure}[ht]
	\centering
	\includegraphics[scale=0.8, max width=\linewidth]{./fig/bayesian-brain/bayesian-linear-regression/cell006.png}
	\caption{cell006.png}
	\label{cell006.png}
\end{figure}
\lstinputlisting[language=julia]{./text/energy-based-model/boltzmann-machine/007.jl}
\lstinputlisting[language=julia]{./text/energy-based-model/boltzmann-machine/008.jl}
\subsubsection{正規方程式を用いた推定}条件に基づいて目的関数$L(\mathbf{\theta})$を微分すると次のような方程式が得られる.
$$
\mathbf{X}^\top\mathbf{X}\mathbf{\hat\theta}=\mathbf{X}^\top\mathbf{y}
$$
これを\textbf{正規方程式} (normal equation)と呼ぶ.この正規方程式より、係数の推定値は$\mathbf{\hat\theta}={(\mathbf{X}^\top\mathbf{X})}^{-1}X^\top\mathbf{y}$という式で得られる.なお,正規方程式自体は$\mathbf{y}=\mathbf{X}\mathbf{\theta}$の左から$\mathbf{X}^\top$をかける,と覚えると良い.
\lstinputlisting[language=julia]{./text/energy-based-model/boltzmann-machine/010.jl}
## 画像の復元
### 訓練データで学習

損失関数を定義する.

\lstinputlisting[language=julia]{./text/energy-based-model/boltzmann-machine/014.jl}
## 相図の描画

50msから200msまでで11回, 250msから400msまでで16回発火しているので発火回数は計27回であり,この結果は正しい.
\lstinputlisting[language=julia]{./text/energy-based-model/boltzmann-machine/017.jl}
\lstinputlisting[language=julia]{./text/energy-based-model/boltzmann-machine/018.jl}
\begin{figure}[ht]
	\centering
	\includegraphics[scale=0.8, max width=\linewidth]{./fig/motor-learning/infinite-horizon-ofc/cell018.png}
	\caption{cell018.png}
	\label{cell018.png}
\end{figure}
\lstinputlisting[language=julia]{./text/energy-based-model/boltzmann-machine/019.jl}
\lstinputlisting[language=julia]{./text/energy-based-model/boltzmann-machine/020.jl}
\begin{figure}[ht]
	\centering
	\includegraphics[scale=0.8, max width=\linewidth]{./fig/energy-based-model/predictive-coding/cell020.png}
	\caption{cell020.png}
	\label{cell020.png}
\end{figure}
通常のPoisson spikeと差はあまり感じられないが,高頻度発火の場合に通常のモデルとの違いが明瞭となる.
\lstinputlisting[language=julia]{./text/energy-based-model/boltzmann-machine/022.jl}
\begin{figure}[ht]
	\centering
	\includegraphics[scale=0.8, max width=\linewidth]{./fig/energy-based-model/predictive-coding/cell022.png}
	\caption{cell022.png}
	\label{cell022.png}
\end{figure}
複数の点が同じ位置に重なっていることに注意.

\lstinputlisting[language=julia]{./text/energy-based-model/boltzmann-machine/024.jl}
\begin{figure}[ht]
	\centering
	\includegraphics[scale=0.8, max width=\linewidth]{./fig/neuron-model/hodgkin-huxley/cell024.png}
	\caption{cell024.png}
	\label{cell024.png}
\end{figure}
 % ok
% \section{スパース符号化}
\subsection{Sparse codingと生成モデル}
\textbf{Sparse codingモデル}\index{Sparse codingもでる@Sparse codingモデル} \citep{Olshausen1996-xe} \citep{Olshausen1997-qu}はV1のニューロンの応答特性を説明する\textbf{線形生成モデル}\index{せんけいせいせいもでる@線形生成モデル} (linear generative model)である.まず,画像パッチ $\mathbf{x}$ が基底関数(basis function) $\mathbf{\Phi} = [\phi_j]$ のノイズを含む線形和で表されるとする (係数は $\mathbf{r}=[r_j]$ とする).
\begin{equation}
\mathbf{x} = \sum_j r_j \phi_j +\boldsymbol{\epsilon}= \mathbf{\Phi} \mathbf{r}+ \boldsymbol{\epsilon}
\end{equation}
ただし,$\boldsymbol{\epsilon} \sim \mathcal{N}(\mathbf{0}, \sigma^2 \mathbf{I})$ である.このモデルを神経ネットワークのモデルと考えると, $\mathbf{\Phi}$ は重み行列,係数 $\mathbf{r}$ は入力よりも高次の神経細胞の活動度を表していると解釈できる.ただし,$r_j$ は負の値も取るので単純に発火率と捉えられないのはこのモデルの欠点である.
Sparse codingでは神経活動 $\mathbf{r}$ が潜在変数の推定量を表現しているという仮定の下,少数の基底で画像 (や目的変数)を表すことを目的とする.要は上式において,ほとんどが0で,一部だけ0以外の値を取るという疎 (=sparse)な係数$\mathbf{r}$を求めたい.
\subsubsection{確率的モデルの記述}
入力される画像パッチ $\mathbf{x}_i\ (i=1, \ldots, N)$ の真の分布を $p_{data}(\mathbf{x})$ とする.また,$\mathbf{x}$ の生成モデルを $p(\mathbf{x}|\mathbf{\Phi})$ とする.さらに潜在変数 $\mathbf{r}$ の事前分布 (prior)を $p(\mathbf{r})$, 画像パッチ $\mathbf{x}$ の尤度 (likelihood)を $p(\mathbf{x}|\mathbf{r}, \mathbf{\Phi})$ とする.このとき,
\begin{equation}
p(\mathbf{x}|\mathbf{\Phi})=\int p(\mathbf{x}|\mathbf{r}, \mathbf{\Phi})p(\mathbf{r})d\mathbf{r}
\end{equation}
が成り立つ.$p(\mathbf{x}|\mathbf{r}, \mathbf{\Phi})$は,(1)式においてノイズ項を$\boldsymbol{\epsilon} \sim\mathcal{N}(\mathbf{0}, \sigma^2 \mathbf{I})$としたことから,
\begin{equation}
p(\mathbf{x}|\ \mathbf{r}, \mathbf{\Phi})=\mathcal{N}\left(\mathbf{x}|\ \mathbf{\Phi} \mathbf{r}, \sigma^2 \mathbf{I} \right)=\frac{1}{Z_{\sigma}} \exp\left(-\frac{\|\mathbf{x} - \mathbf{\Phi} \mathbf{r}\|^2}{2\sigma^2}\right)
\end{equation}
と表せる.ただし,$Z_{\sigma}$は規格化定数である.
\subsubsection{事前分布の設定}
事前分布$p(\mathbf{r})$としては,0においてピークがあり,裾の重い(heavy tail)を持つsparse distributionあるいは \textbf{super-Gaussian distribution}\index{super-Gaussian distribution} (Laplace 分布やCauchy分布などGaussian分布よりもkurtoticな分布)を用いるのが良い.このような分布では,$\mathbf{r}$の各要素$r_i$はほとんど0に等しく,ある入力に対しては大きな値を取る.$p(\mathbf{r})$は一般化して式(4), (5)のように表記する.
\begin{align}
p(\mathbf{r})&=\prod_j p(r_j)\\
p(r_j)&=\frac{1}{Z_{\beta}}\exp \left[-\beta S(r_j)\right]
\end{align}
ただし,$\beta$は逆温度(inverse temperature), $Z_{\beta}$は規格化定数 (分配関数) である.これらの用語は統計力学における正準分布 (ボルツマン分布)から来ている.$S(x)$と分布の関係をまとめた表が以下となる (cf. \url{https://pdfs.semanticscholar.org/be08/da912362bf40fe3ded78bdadc644f921b4e7.pdf}).
|$S(r)$|$\dfrac{dS(r)}{dr}$|$p(r)$|分布名|尖度(kurtosis)|
|:-:|:-:|:-:|:-:|:-:|
|$r^2$|$2r$|$\dfrac{1}{\alpha \sqrt{2\pi}}\exp\left(-\dfrac{r^2}{2\alpha^2}\right)$|Gaussian 分布|0|
|$\vert r\vert$|$\text{sign}(r)$|$\dfrac{1}{2\alpha}\exp\left(-\dfrac{\vert r\vert}{\alpha}\right)$|Laplace 分布|3.0|
|$\ln (\alpha^2+r^2)$|$\dfrac{2r}{\alpha^2+r^2}$|$\dfrac{\alpha}{\pi}\dfrac{1}{\alpha^2+r^2}=\dfrac{\alpha}{\pi}\exp[-\ln (\alpha^2+r^2)]$|Cauchy 分布|-|
分布$p(r)$や$S(r)$を描画すると次のようになる.
\begin{lstlisting}[language=julia]
using PyPlot

x = range(-5, 5, length=300)
figure(figsize=(7,3))
subplot(1,2,1)
title(L"$p(x)$")
plot(x, 1/sqrt(2pi)*exp.(-(x.^2)/2), color="black", linestyle="--",label="Gaussian")
plot(x, 1/2*exp.(-abs.(x)), label="Laplace")
plot(x, 1 ./ (pi*(1 .+ x.^2)), label="Cauchy")
xlim(-5, 5); 
xlabel(L"$x$")
legend()

subplot(1,2,2)
title(L"S(x)")
plot(x, x.^2, color="black", linestyle="--",label="Gaussian")
plot(x, abs.(x), label="Laplace")
plot(x, log.(1 .+ x.^2), label="Cauchy")
xlim(-5, 5); ylim(0, 5)
xlabel(L"$x$")

tight_layout()
\end{lstlisting}
\begin{figure}[ht]
	\centering
	\includegraphics[scale=0.8, max width=\linewidth]{./fig/energy-based-model/sparse-coding/cell005.png}
	\caption{cell005.png}
	\label{cell005.png}
\end{figure}
\subsection{目的関数の設定と最適化}
最適な生成モデルを得るために,入力される画像パッチの真の分布 $p_{data}(\mathbf{x})$と$\mathbf{x}$の生成モデル $p(\mathbf{x}|\mathbf{\Phi})$を近づける.このために,2つの分布のKullback-Leibler ダイバージェンス $D_{\text{KL}}\left(p_{data}(\mathbf{x}) \Vert\ p(\mathbf{x}|\mathbf{\Phi})\right)$を最小化したい.しかし,真の分布は得られないので,経験分布 
\begin{equation}
\hat{p}_{data}(\mathbf{x})\triangleq\frac{1}{N}\sum_{i=1}^N \delta(\mathbf{x}-\mathbf{x}_i)
\end{equation}
を近似として用いる ($\delta(\cdot)$ はDiracのデルタ関数である).ゆえに$D_{\text{KL}}\left(\hat{p}_{data}(\mathbf{x}) \Vert\ p(\mathbf{x}|\mathbf{\Phi})\right)$を最小化する.
\begin{align}
D_{\text{KL}}\left(\hat{p}_{data}(\mathbf{x}) \Vert\ p(\mathbf{x}|\mathbf{\Phi})\right)&=\int \hat{p}_{data}(\mathbf{x}) \log \frac{\hat{p}_{data}(\mathbf{x})}{p(\mathbf{x}|\mathbf{\Phi})} d\mathbf{x}\\
&=\mathbb{E}_{\hat{p}_{data}} \left[\ln \frac{\hat{p}_{data}(\mathbf{x})}{p(\mathbf{x}|\mathbf{\Phi})}\right]\\
&=\mathbb{E}_{\hat{p}_{data}} \left[\ln \hat{p}_{data}(\mathbf{x})\right]-\mathbb{E}_{\hat{p}_{data}} \left[\ln p(\mathbf{x}|\mathbf{\Phi})\right]
\end{align}
が成り立つ.(7)式の1番目の項は一定なので,$D_{\text{KL}}\left(\hat{p}_{data}(\mathbf{x}) \Vert\ p(\mathbf{x}|\mathbf{\Phi})\right)$ を最小化するには$\mathbb{E}_{\hat{p}_{data}} \left[\ln p(\mathbf{x}|\mathbf{\Phi})\right]$を最大化すればよい.ここで,
\begin{equation}
\mathbb{E}_{\hat{p}_{data}} \left[\ln p(\mathbf{x}|\mathbf{\Phi})\right]=\sum_{i=1}^N \hat{p}_{data}(\mathbf{x}_i)\ln p(\mathbf{x}_i|\mathbf{\Phi})=\frac{1}{N}\sum_{i=1}^N \ln p(\mathbf{x}_i|\mathbf{\Phi})
\end{equation}
が成り立つ.また,(2)式より
\begin{equation}
\ln p(\mathbf{x}|\mathbf{\Phi})=\ln \int p(\mathbf{x}|\mathbf{r}, \mathbf{\Phi})p(\mathbf{r})d\mathbf{r}
\end{equation}
が成り立つので,近似として $\displaystyle \int p(\mathbf{x}|\mathbf{r}, \mathbf{\Phi})p(\mathbf{r})d\mathbf{r}$ を $p(\mathbf{x}|\mathbf{r}, \mathbf{\Phi})p(\mathbf{r}) \left(=p(\mathbf{x}, \mathbf{r}| \mathbf{\Phi})\right)$ で評価する.これらの近似の下,最適な$\mathbf{\Phi}=\mathbf{\Phi}^*$は次のようにして求められる.
\begin{align}
\mathbf{\Phi}^*&=\text{arg} \min_{\mathbf{\Phi}} \min_{\mathbf{r}} D_{\text{KL}}\left(\hat{p}_{data}(\mathbf{x}) \| p(\mathbf{x}|\mathbf{\Phi})\right)\\
&=\text{arg} \max_{\mathbf{\Phi}} \max_{\mathbf{r}} \mathbb{E}_{\hat{p}_{data}} \left[\ln p(\mathbf{x}|\mathbf{\Phi})\right]\\
&= \text{arg} \max_{\mathbf{\Phi}}\sum_{i=1}^N \max_{\mathbf{r}_i} \ln p(\mathbf{x}_i|\mathbf{\Phi})\\
&\approx \text{arg} \max_{\mathbf{\Phi}}\sum_{i=1}^N \max_{\mathbf{r}_i} \ln p(\mathbf{x}_i|\mathbf{r}_i, \mathbf{\Phi})p(\mathbf{r}_i)\\
&=\text{arg}\min_{\mathbf{\Phi}} \sum_{i=1}^N \min_{\mathbf{r}_i}\ E(\mathbf{x}_i, \mathbf{r}_i|\mathbf{\Phi})
\end{align}
ただし,$\mathbf{x}_i$に対する神経活動を $\mathbf{r}_i$とした.また,$E(\mathbf{x}, \mathbf{r}|\mathbf{\Phi})$はコスト関数であり,次式のように表される.
\begin{align}
E(\mathbf{x}, \mathbf{r}|\mathbf{\Phi})\triangleq&-\ln p(\mathbf{x}|\mathbf{r}, \mathbf{\Phi})p(\mathbf{r})\\
=&\underbrace{\left\|\mathbf{x}-\mathbf{\Phi} \mathbf{r}\right\|^2}_{\text{preserve information}} + \lambda \underbrace{\sum_j S\left(r_j\right)}_{\text{sparseness of}\ r_j}
\end{align}
ただし,$\lambda=2\sigma^2\beta$は正則化係数(この式から逆温度$\beta$が正則化の度合いを調整するパラメータであることがわかる.)であり,1行目から2行目へは式(3), (4), (5)を用いた.ここで,第1項が復元損失,第2項が罰則項 (正則化項)となっている.
式(9)で表される最適化手順を最適な$\mathbf{r}$と$\mathbf{\Phi}$を求める過程に分割しよう.まず, $\mathbf{\Phi}$を固定した下で$E(\mathbf{x}_n, \mathbf{r}_i|\mathbf{\Phi})$を最小化する$\mathbf{r}_i=\hat{\mathbf{r}}_i$を求める.
\begin{equation}
\hat{\mathbf{r}}_i=\text{arg}\min_{\mathbf{r}_i}E(\mathbf{x}_i, \mathbf{r}_i|\mathbf{\Phi})\ \left(= \text{arg}\max_{\mathbf{r}_i}p(\mathbf{r}_i|\mathbf{x}_i)\right)
\end{equation}
これは $\mathbf{r}$ について \textbf{MAP推定}\index{MAPすいてい@MAP推定} (maximum a posteriori estimation)を行うことに等しい.次に$\hat{\mathbf{r}}$を用いて
\begin{equation}
\mathbf{\Phi}^*=\text{arg}\min_{\mathbf{\Phi}} \sum_{i=1}^N E(\mathbf{x}_i, \hat{\mathbf{r}}_i|\mathbf{\Phi})\ \left(= \text{arg}\max_{\mathbf{\Phi}} \prod_{i=1}^N p(\mathbf{x}_i|\hat{\mathbf{r}}_i, \mathbf{\Phi})\right)
\end{equation}
とすることにより,$\mathbf{\Phi}$を最適化する.こちらは $\mathbf{\Phi}$ について \textbf{最尤推定}\index{さいゆうすいてい@最尤推定} (maximum likelihood estimation)を行うことに等しい.
\subsection{ Locally competitive algorithm (LCA) }
$\mathbf{r}$の勾配法による更新則は,$E$の微分により次のように得られる.
\begin{equation}
\frac{d \mathbf{r}}{dt}= -\frac{\eta_\mathbf{r}}{2}\frac{\partial E}{\partial \mathbf{r}}=\eta_\mathbf{r} \cdot\left[\mathbf{\Phi}^\top (\mathbf{x}-\mathbf{\Phi}\mathbf{r})- \frac{\lambda}{2}S'\left(\mathbf{r}\right)\right]
\end{equation}
ただし,$\eta_{\mathbf{r}}$は学習率である.この式により$\mathbf{r}$が収束するまで最適化するが,単なる勾配法ではなく,\citep{Olshausen1996-xe}では\textbf{共役勾配法}\index{きょうやくこうばいほう@共役勾配法} (conjugate gradient method)を用いている.しかし,共役勾配法は実装が煩雑で非効率であるため,より効率的かつ生理学的な妥当性の高い学習法として,\textbf{LCA}\index{LCA}  (locally competitive algorithm)が提案されている \citep{Rozell2008-wp}.LCAは\textbf{側抑制}\index{そくよくせい@側抑制} (local competition, lateral inhibition)と\textbf{閾値関数}\index{いきちかんすう@閾値関数} (thresholding function)を用いる更新則である.LCAによる更新を行うRNNは通常のRNNとは異なり,コスト関数(またはエネルギー関数)を最小化する動的システムである.このような機構はHopfield networkで用いられているために,Olshausenは\textbf{Hopfield trick}\index{Hopfield trick}と呼んでいる.
\subsubsection{軟判定閾値関数を用いる場合 (ISTA)}
$S(x)=|x|$とした場合の閾値関数を用いる手法として\textbf{ISTA}\index{ISTA}(Iterative Shrinkage Thresholding Algorithm)がある.ISTAはL1-norm正則化項に対する近接勾配法で,要はLasso回帰に用いる勾配法である.
解くべき問題は次式で表される.
\begin{equation}
\mathbf{r} = \mathop{\rm arg~min}\limits_{\mathbf{r}}\left\{\|\mathbf{x}-\mathbf{\Phi}\mathbf{r}\|^2_2+\lambda\|\mathbf{r}\|_1\right\}
\end{equation}
詳細は後述するが,次のように更新することで解が得られる.
\begin{itemize}
\item $\mathbf{r}(0)$を要素が全て0のベクトルで初期化:$\mathbf{r}(0)=\mathbf{0}$
\item $\mathbf{r}_*(t+1)=\mathbf{r}(t)+\eta_\mathbf{r}\cdot \mathbf{\Phi}^\top(\mathbf{x}-\mathbf{\Phi}\mathbf{r}(t))$
\item $\mathbf{r}(t+1) = \Theta_\lambda(\mathbf{r}_*(t+1))$
\item $\mathbf{r}$が収束するまで2と3を繰り返す
\end{itemize}
ここで$\Theta_\lambda(\cdot)$は\textbf{軟判定閾値関数}\index{なんはんていいきちかんすう@軟判定閾値関数} (Soft thresholding function)と呼ばれ,次式で表される.
\begin{equation}
\Theta_\lambda(y)= 
\begin{cases} 
y-\lambda & (y>\lambda)\\ 
0 & (-\lambda\leq y\leq\lambda)\\ 
 y+\lambda & (y<-\lambda) 
\end{cases}
\end{equation}
$\Theta_\lambda(\cdot)$を関数として定義すると次のようになる.また,ReLU (ランプ関数)は\jl{max(x, 0)}で実装できる.この点から考えればReLUを軟判定非負閾値関数 (soft nonnegative thresholding function)と捉えることもできる \citep{Papyan2018-yr}.
\begin{lstlisting}[language=julia]
# thresholding function of S(x)=|x|
soft_thres(x, λ) = max(x - λ, 0) - max(-x - λ, 0)
soft_nonneg_thres(x, λ) = max(x - λ, 0) # relu(x-λ)
\end{lstlisting}
次に$\Theta_\lambda(\cdot)$を描画すると次のようになる.
\begin{lstlisting}[language=julia]
xmin, xmax = -5, 5
x = range(xmin, xmax, length=100)
y = soft_thres.(x, 1)

figure(figsize=(4,4.5))
subplot(2,2,1)
title(L"$S(x)=|x|$")
plot(x, abs.(x))
xlim(xmin, xmax); ylim(0, 10)
hlines(y=xmax, xmin=xmin, xmax=xmax, color="k", alpha=0.2)
vlines(x=0, ymin=0, ymax=xmax*2, color="k", alpha=0.2)

subplot(2,2,2)
title(L"$\frac{\partial S(x)}{\partial x}$")
plot(x, x, "k--")
plot(x, sign.(x))
xlim(xmin, xmax); ylim(xmin, xmax)
hlines(y=0, xmin=xmin, xmax=xmax, color="k", alpha=0.2)
vlines(x=0, ymin=xmin, ymax=xmax, color="k", alpha=0.2)

subplot(2,2,3)
title(L"$f_\lambda(x)=x+\lambda\cdot\frac{\partial S(x)}{\partial x}$")
plot(x, x, "k--")
plot(x, x + 1*sign.(x))
xlabel(L"$x$")
xlim(-5, 5); ylim(-5, 5)
hlines(y=0, xmin=xmin, xmax=xmax, color="k", alpha=0.2)
vlines(x=0, ymin=xmin, ymax=xmax, color="k", alpha=0.2)

subplot(2,2,4)
title(L"$\Theta_\lambda(x)$")
plot(x, x, "k--")
plot(x, y)
xlabel(L"$x$")
xlim(-5, 5); ylim(-5, 5)
hlines(y=0, xmin=xmin, xmax=xmax, color="k", alpha=0.2)
vlines(x=0, ymin=xmin, ymax=xmax, color="k", alpha=0.2)

tight_layout()
\end{lstlisting}
\begin{figure}[ht]
	\centering
	\includegraphics[scale=0.8, max width=\linewidth]{./fig/energy-based-model/sparse-coding/cell011.png}
	\caption{cell011.png}
	\label{cell011.png}
\end{figure}
なお,軟判定閾値関数は次の目的関数$C$を最小化する$x$を求めることで導出できる.
\begin{equation}
C=\frac{1}{2}(y-x)^2+\lambda |x|
\end{equation}
ただし,$x, y, \lambda$はスカラー値とする.$|x|$が微分できないが,これは場合分けを考えることで解決する.$x\geq 0$を考えると,(6)式は
\begin{equation}
C=\frac{1}{2}(y-x)^2+\lambda x = \{x-(y-\lambda)\}^2+\lambda(y-\lambda)
\end{equation}
となる.(7)式の最小値を与える$x$は場合分けをして考えると,$y-\lambda\geq0$のとき二次関数の頂点を考えて$x=y-\lambda$となる. 一方で$y-\lambda<0$のときは$x\geq0$において単調増加な関数となるので,最小となるのは$x=0$のときである.同様の議論を$x\leq0$に対しても行うことで (5)式が得られる.
なお,閾値関数としては軟判定閾値関数だけではなく,硬判定閾値関数や$y=x - \text{tanh}(x)$ (Tanh-shrink)など様々な関数を用いることができる.
\subsection{重み行列の更新則}
$\mathbf{r}$が収束したら勾配法により$\mathbf{\Phi}$を更新する.
\begin{equation}
\Delta \phi_i(\boldsymbol{x}) = -\eta \frac{\partial E}{\partial \mathbf{\Phi}}=\eta\cdot\left[\left([\mathbf{x}-\mathbf{\Phi}\mathbf{r}\right)\mathbf{r}^\top\right]
\end{equation}
\subsection{Sparse coding networkの実装}
ネットワークは入力層を含め2層の単純な構造である.今回は,入力はランダムに切り出した16×16 (=256)の画像パッチとし,これを入力層の256個のニューロンが受け取るとする.入力層のニューロンは次層の100個のニューロンに投射するとする.100個のニューロンが入力をSparseに符号化するようにその活動および重み行列を最適化する.
\subsubsection{画像データの読み込み}
データは\url{http://www.rctn.org/bruno/sparsenet/}からダウンロードできる.これはアメリカ北西部で撮影された自然画像であり,van Hateren's Natural Image Dataset \url{http://bethgelab.org/datasets/vanhateren/} から取得されたものである.\jl{IMAGES_RAW.mat}は10枚の自然画像で,\jl{IMAGES.mat}はそれを白色化したものである.\jl{mat}ファイルの読み込みには MAT.jl \url{https://github.com/JuliaIO/MAT.jl}を用いる.
\begin{lstlisting}[language=julia]
using MAT
#using PyPlot
\end{lstlisting}
\begin{lstlisting}[language=julia]
# datasets from http://www.rctn.org/bruno/sparsenet/
mat_images_raw = matopen("../_static/datasets/IMAGES_RAW.mat")
imgs_raw = read(mat_images_raw, "IMAGESr")

mat_images = matopen("../_static/datasets/IMAGES.mat")
imgs = read(mat_images, "IMAGES")

close(mat_images_raw)
close(mat_images)
\end{lstlisting}
画像データを描画する.
\begin{lstlisting}[language=julia]
figure(figsize=(8, 3))
subplots_adjust(hspace=0.1, wspace=0.1)
for i=1:10
    subplot(2, 5, i)
    imshow(imgs_raw[:,:,i], cmap="gray")
    axis("off")
end
suptitle("Natural Images", fontsize=12)
subplots_adjust(top=0.9)  
\end{lstlisting}
\begin{figure}[ht]
	\centering
	\includegraphics[scale=0.8, max width=\linewidth]{./fig/energy-based-model/sparse-coding/cell019.png}
	\caption{cell019.png}
	\label{cell019.png}
\end{figure}
\subsubsection{モデルの定義}
必要なパッケージを読み込む.
\begin{lstlisting}[language=julia]
using Parameters: @unpack # or using UnPack
using LinearAlgebra, Random, Statistics, ProgressMeter
Random.seed!(0)
rc("axes.spines", top=false, right=false)
\end{lstlisting}
モデルを定義する.
\begin{lstlisting}[language=julia]
@kwdef struct OFParameter{FT}
    lr_r::FT = 1e-2 # learning rate of r
    lr_Phi::FT = 1e-2 # learning rate of Phi
    λ::FT = 5e-3 # regularization parameter
end

@kwdef mutable struct OlshausenField1996Model{FT}
    param::OFParameter = OFParameter{FT}()
    num_inputs::Int32
    num_units::Int32
    batch_size::Int32
    r::Array{FT} = zeros(batch_size, num_units) # activity of neurons
    Phi::Array{FT} = randn(num_inputs, num_units) .* sqrt(1/num_units)
end
\end{lstlisting}
パラメータを更新する関数を定義する.今回はより生理学的に妥当にするため,軟判定非負閾値関数を用いる.
\begin{lstlisting}[language=julia]
function updateOF!(variable::OlshausenField1996Model, param::OFParameter, inputs::Array, training::Bool)
    @unpack num_inputs, num_units, batch_size, r, Phi = variable
    @unpack lr_r, lr_Phi, λ = param

    # Updates                
    error = inputs .- r * Phi'
    r_ = r +lr_r .* error * Phi

    #r[:, :] = soft_thres.(r_, λ)
    r[:, :] = soft_nonneg_thres.(r_, λ)

    if training 
        error = inputs - r * Phi'
        dPhi = error' * r
        Phi[:, :] += lr_Phi * dPhi
    end
    
    return error
end
\end{lstlisting}
行ごとに正規化する関数を定義する.
\begin{lstlisting}[language=julia]
function normalize_rows(A::Array)
    return A ./ sqrt.(sum(A.^2, dims=1) .+ 1e-8)
end
\end{lstlisting}
損失関数を定義する.
\begin{lstlisting}[language=julia]
function calculate_total_error(error, r, λ)
    recon_error = mean(error.^2)
    sparsity_r = λ*mean(abs.(r)) 
    return recon_error + sparsity_r
end
\end{lstlisting}
シミュレーションを実行する関数を定義する.外側の\jl{for loop}では画像パッチの作成と\jl{r}の初期化を行う.内側の\jl{for loop}では\jl{r}が収束するまで更新を行い,収束したときに重み行列\jl{Phi}を更新する.
\begin{lstlisting}[language=julia]
function run_simulation(imgs, num_iter, nt_max, batch_size, sz, num_units, eps)
    H, W, num_images = size(imgs)
    num_inputs = sz^2

    model = OlshausenField1996Model{Float32}(num_inputs=num_inputs, num_units=num_units, batch_size=batch_size)
    errorarr = zeros(num_iter) # Vector to save errors    
    
    # Run simulation
    @showprogress "Computing..." for iter in 1:num_iter
        # Get the coordinates of the upper left corner of clopping image randomly.
        beginx = rand(1:W-sz, batch_size)
        beginy = rand(1:H-sz, batch_size)

        inputs = zeros(batch_size, num_inputs)  # Input image patches

        # Get images randomly
        for i in 1:batch_size        
            idx = rand(1:num_images)
            img = imgs[:, :, idx]
            clop = img[beginy[i]:beginy[i]+sz-1, beginx[i]:beginx[i]+sz-1][:]
            inputs[i, :] = clop .- mean(clop)
        end

        model.r = zeros(batch_size, num_units) # Reset r states
        model.Phi = normalize_rows(model.Phi) # Normalize weights
        # Input image patches until latent variables are converged 
        r_tm1 = zeros(batch_size, num_units)  # set previous r (t minus 1)

        for t in 1:nt_max
            # Update r without update weights 
            error = updateOF!(model, model.param, inputs, false)

            dr = model.r - r_tm1 

            # Compute norm of r
            dr_norm = sqrt(sum(dr.^2)) / sqrt(sum(r_tm1.^2) + 1e-8)
            r_tm1 .= model.r # update r_tm1

            # Check convergence of r, then update weights
            if dr_norm < eps
                error = updateOF!(model, model.param, inputs, true)
                errorarr[iter] = calculate_total_error(error, model.r, model.param.λ) # Append errors
                break
            end

            # If failure to convergence, break and print error
            if t >= nt_max-1
                print("Error at patch:", iter_, dr_norm)
                errorarr[iter] = calculate_total_error(error, model.r, model.param.λ) # Append errors
                break
            end
        end
        # Print moving average error
        if iter % 100 == 0
            moving_average_error = mean(errorarr[iter-99:iter])
            println("iter: ", iter, "/", num_iter, ", Moving average error:", moving_average_error)
        end
    end
    return model, errorarr
end
\end{lstlisting}
\jl{r_tm1 .= model.r}の部分は,要素ごとのコピーを実行している.\jl{r_tm1 = copy(model.r)}でもよいが,新たなメモリ割り当てが生じるので避けている.\jl{@. r_tm1 = model.r}としてもよい.シミュレーションの実行をする.
\begin{lstlisting}[language=julia]
# Simulation constants
num_iter = 500 # number of iterations
nt_max = 1000 # Maximum number of simulation time
batch_size = 250 # Batch size

sz = 16 # image patch size
num_units = 100 # number of neurons (units)
eps = 1e-2 # small value which determines convergence

model, errorarr = run_simulation(imgs, num_iter, nt_max, batch_size, sz, num_units, eps);
\end{lstlisting}
\subsubsection{訓練中の損失の描画}
訓練中の損失の変化を描画してみよう.損失が低下し,学習が進行したことが分かる.
\begin{lstlisting}[language=julia]
# Plot error
figure(figsize=(4, 2))
ylabel("Error")
xlabel("Iterations")
plot(1:num_iter, errorarr)
tight_layout()
\end{lstlisting}
\begin{figure}[ht]
	\centering
	\includegraphics[scale=0.8, max width=\linewidth]{./fig/energy-based-model/sparse-coding/cell035.png}
	\caption{cell035.png}
	\label{cell035.png}
\end{figure}
\subsubsection{重み行列 (受容野)の描画}
学習後の重み行列 \jl{Phi} ($\mathbf{\Phi}$)を可視化してみよう.
\begin{lstlisting}[language=julia]
# Plot Receptive fields
figure(figsize=(4.2, 4))
subplots_adjust(hspace=0.1, wspace=0.1)
for i in 1:num_units
    subplot(10, 10, i)
    imshow(reshape(model.Phi[:, i], (sz, sz)), cmap="gray")
    axis("off")
end
suptitle("Receptive fields", fontsize=14)
subplots_adjust(top=0.925)
\end{lstlisting}
\begin{figure}[ht]
	\centering
	\includegraphics[scale=0.8, max width=\linewidth]{./fig/energy-based-model/sparse-coding/cell037.png}
	\caption{cell037.png}
	\label{cell037.png}
\end{figure}
白色が\textbf{ON領域}\index{ONりょういき@ON領域}(興奮),黒色が\textbf{OFF領域}\index{OFFりょういき@OFF領域}(抑制)を表す.Gaborフィルタ様の局所受容野が得られており,これは一次視覚野(V1)における単純型細胞(simple cells)の受容野に類似している.
\subsubsection{画像の再構成}
学習したモデルを用いて入力画像が再構成されるか確認しよう.
\begin{lstlisting}[language=julia]
H, W, num_images = size(imgs)
num_inputs = sz^2

# Get the coordinates of the upper left corner of clopping image randomly.
beginx = rand(1:W-sz, batch_size)
beginy = rand(1:H-sz, batch_size)

inputs = zeros(batch_size, num_inputs)  # Input image patches

# Get images randomly
for i in 1:batch_size        
    idx = rand(1:num_images)
    img = imgs[:, :, idx]
    clop = img[beginy[i]:beginy[i]+sz-1, beginx[i]:beginx[i]+sz-1][:]
    inputs[i, :] = clop .- mean(clop)
end

model.r = zeros(batch_size, num_units) # Reset r states

# Input image patches until latent variables are converged 
r_tm1 = zeros(batch_size, num_units)  # set previous r (t minus 1)

for t in 1:nt_max
    # Update r without update weights 
    error = updateOF!(model, model.param, inputs, false)

    dr = model.r - r_tm1 

    # Compute norm of r
    dr_norm = sqrt(sum(dr.^2)) / sqrt(sum(r_tm1.^2) + 1e-8)
    r_tm1 .= model.r # update r_tm1

    # Check convergence of r, then update weights
    if dr_norm < eps
        break
    end
end;
\end{lstlisting}
神経活動 $\mathbf{r}$がスパースになっているか確認しよう.
\begin{lstlisting}[language=julia]
figure(figsize=(3, 2))
hist(model.r[:], bins=50)
xlim(0, 0.5)
tight_layout()
\end{lstlisting}
\begin{figure}[ht]
	\centering
	\includegraphics[scale=0.8, max width=\linewidth]{./fig/energy-based-model/sparse-coding/cell042.png}
	\caption{cell042.png}
	\label{cell042.png}
\end{figure}
要素がほとんど0のスパースなベクトルになっていることがわかる.次に画像を再構成する.
\begin{lstlisting}[language=julia]
reconst = model.r * model.Phi'
println(size(reconst))
\end{lstlisting}
再構成した結果を描画する.
\begin{lstlisting}[language=julia]
figure(figsize=(7.5, 3))
subplots_adjust(hspace=0.1, wspace=0.1)
num_show = 5
for i in 1:num_show
    subplot(2, num_show, i)
    imshow(reshape(inputs[i, :], (sz, sz)), cmap="gray")
    xticks([]); yticks([]); 
    if i == 1
        ylabel("Input\n images")
    end

    subplot(2, num_show, num_show+i)
    imshow(reshape(reconst[i, :], (sz, sz)), cmap="gray")
    xticks([]); yticks([]); 
    if i == 1
        ylabel("Reconstructed\n images")
    end
end
\end{lstlisting}
\begin{figure}[ht]
	\centering
	\includegraphics[scale=0.8, max width=\linewidth]{./fig/energy-based-model/sparse-coding/cell046.png}
	\caption{cell046.png}
	\label{cell046.png}
\end{figure}
上段が入力画像,下段が再構成された画像である.差異はあるものの,概ね再構成されていることがわかる.
論文以外の参考資料
\begin{itemize}
\item \url{http://www.scholarpedia.org/article/Sparse_coding}
\item Bruno Olshausen: “Sparse coding in brains and machines”(\url{https://talks.stanford.edu/bruno-olshausen-sparse-coding-in-brains-and-machines/}), \url{http://www.rctn.org/bruno/public/Simons-sparse-coding.pdf}
\item \url{https://redwood.berkeley.edu/wp-content/uploads/2018/08/sparse-coding-ICA.pdf}
\item \url{https://redwood.berkeley.edu/wp-content/uploads/2018/08/sparse-coding-LCA.pdf}
\item \url{https://redwood.berkeley.edu/wp-content/uploads/2018/08/Dylan-lca_overcompleteness_09-27-2018.pdf}
\end{itemize}
 % ok
% \section{予測符号化}
\subsection{観測世界の階層的予測}
\textbf{階層的予測符号化(hierarchical predictive coding; HPC)}\index{かいそうてきよそくふごうか(hierarchical predictive coding; HPC)@階層的予測符号化(hierarchical predictive coding; HPC)} は\citep{Rao1999-zv}により導入された.構築するネットワークは入力層を含め,3層のネットワークとする.LGNへの入力として画像 $\mathbf{x} \in \mathbb{R}^{n_0}$を考える.画像 $\mathbf{x}$ の観測世界における隠れ変数,すなわち\textbf{潜在変数}\index{せんざいへんすう@潜在変数} (latent variable)を$\mathbf{r} \in \mathbb{R}^{n_1}$とし,ニューロン群によって発火率で表現されているとする (真の変数と $\mathbf{r}$は異なるので文字を分けるべきだが簡単のためにこう表す).このとき,
\begin{equation}
\mathbf{x} = f(\mathbf{U}\mathbf{r}) + \boldsymbol{\epsilon}
\end{equation}
が成立しているとする.ただし,$f(\cdot)$は活性化関数 (activation function),$\mathbf{U} \in \mathbb{R}^{n_0 \times n_1}$は重み行列である.
$\boldsymbol{\epsilon} \in \mathbb{R}^{n_0}$ は $\mathcal{N}(\mathbf{0}, \sigma^2 \mathbf{I})$ からサンプリングされるとする.
潜在変数 $\mathbf{r}$はさらに高次 (higher-level)の潜在変数 $\mathbf{r}^h$により,次式で表現される.
\begin{equation}
\mathbf{r} = \mathbf{r}^{td}+\boldsymbol{\epsilon}^{td}=f(\mathbf{U}^h \mathbf{r}^h)+\boldsymbol{\epsilon}^{td}
\end{equation}
ただし,Top-downの予測信号を $\mathbf{r}^{td}\coloneqqf(\mathbf{U}^h \mathbf{r}^h)$とした.また,$\mathbf{r}^{td} \in \mathbb{R}^{n_1}$, $\mathbf{r}^{h} \in \mathbb{R}^{n_2}$, $\mathbf{U}^h \in \mathbb{R}^{n_1 \times n_2}$ である.
$\boldsymbol{\epsilon}^{td} \in \mathbb{R}^{n_1}$は$\mathcal{N}(\mathbf{0}$, $\sigma_{td}^2 \mathbf{I}$) からサンプリングされるとする.
話は飛ぶが,Predictive codingのネットワークの特徴は
\begin{itemize}
\item 階層的な構造
\item 高次による低次の予測 (Feedback or Top-down信号)
\item 低次から高次への誤差信号の伝搬 (Feedforward or Bottom-up 信号)
\end{itemize}
である.ここまでは高次表現による低次表現の予測,というFeedback信号について説明してきたが,この部分はSparse codingでも同じである.それではPredictive codingのもう一つの要となる,低次から高次への予測誤差の伝搬というFeedforward信号はどのように導かれるのだろうか.結論から言えば,これは\textbf{復元誤差 (reconstruction error)の最小化を行う再帰的ネットワーク (recurrent network)を考慮することで自然に導かれる}\index{ふくげんごさ (reconstruction error)のさいしょうかをおこなうさいきてきねっとわーく (recurrent network)をこうりょすることでしぜんにみちびかれる@復元誤差 (reconstruction error)の最小化を行う再帰的ネットワーク (recurrent network)を考慮することで自然に導かれる}.
\subsection{損失関数と学習則}
\subsubsection{事前分布の設定}
$\mathbf{r}$の事前分布$p(\mathbf{r})$はCauchy分布を用いる.$p(\mathbf{r})$の負の対数事前分布を$g(\mathbf{r})\coloneqq-\log p(\mathbf{r})$としておく.
\begin{align}
p(\mathbf{r})&=\prod_i p(r_i)=\prod_i \exp\left[-\alpha \ln(1+r_i^2)\right]\\
g(\mathbf{r})&=-\ln p(\mathbf{r})=\alpha \sum_i \ln(1+r_i^2)\\
g'(\mathbf{r})&=\frac{\partial g(\mathbf{r})}{\partial \mathbf{r}}=\left[\frac{2\alpha r_i}{1+r_i^2}\right]_i
\end{align}
次に重み行列$\mathbf{U}$の事前分布 $p(\mathbf{U})$はGaussian分布とする.$p(\mathbf{U})$の負の対数事前分布を$h(\mathbf{U})\coloneqq-\ln p(\mathbf{U})$とすると,次のように表される.
\begin{align}
p(\mathbf{U})&=\exp(-\lambda\|\mathbf{U}\|^2_F)\\
h(\mathbf{U})&=-\ln p(\mathbf{U})=\lambda\|\mathbf{U}\|^2_F\\
h'(\mathbf{U})&=\frac{\partial h(\mathbf{U})}{\partial \mathbf{U}}=2\lambda \mathbf{U}
\end{align}
ただし,$\|\cdot \| _ F^2$はフロベニウスノルムを意味する.
\subsubsection{損失関数の設定}
Sparse codingと同様に考えることにより,損失関数 $E$を次のように定義する.
\begin{align}
E=\underbrace{\frac{1}{\sigma^{2}}\|\mathbf{x}-f(\mathbf{U} \mathbf{r})\|^2+\frac{1}{\sigma_{t d}^{2}}\left\|\mathbf{r}-f(\mathbf{U}^h \mathbf{r}^h)\right\|^2}_{\text{reconstruction error}}+\underbrace{g(\mathbf{r})+g(\mathbf{r}^{h})+h(\mathbf{U})+h(\mathbf{U}^h)}_{\text{sparsity penalty}}
\end{align}
潜在変数 $\mathbf{r}, \mathbf{r}^h$ と 重み行列 $\mathbf{U}, \mathbf{U}^h$ のそれぞれに事前分布を仮定しているため,これらについてのMAP推定を行うことに相当する.
\subsubsection{再帰ネットワークの更新則}
簡単のために$\mathbf{z}\coloneqq\mathbf{U}\mathbf{r}, \mathbf{z}^h\coloneqq\mathbf{U}^h\mathbf{r}^h$とする.
\begin{align}
\frac{d \mathbf{r}}{d t}&=-\frac{k_{1}}{2} \frac{\partial E}{\partial \mathbf{r}}=k_{1}\cdot\Bigg(\frac{1}{\sigma^{2}} \mathbf{U}^{T}\bigg[\frac{\partial f(\mathbf{z})}{\partial \mathbf{z}}\odot\underbrace{(\mathbf{x}-f(\mathbf{z}))}_{\text{bottom-up error}}\bigg]-\frac{1}{\sigma_{t d}^{2}}\underbrace{\left(\mathbf{r}-f(\mathbf{z}^h)\right)}_{\text{top-down error}}-\frac{1}{2}g'(\mathbf{r})\Bigg)\\
\frac{d \mathbf{r}^h}{d t}&=-\frac{k_{1}}{2} \frac{\partial E}{\partial \mathbf{r}^h}=k_{1}\cdot\Bigg(\frac{1}{\sigma_{t d}^{2}}(\mathbf{U}^h)^\top\bigg[\frac{\partial f(\mathbf{z}^h)}{\partial \mathbf{z}^h}\odot\underbrace{\left(\mathbf{r}-f(\mathbf{z}^h)\right)}_{\text{bottom-up error}}\bigg]-\frac{1}{2}g'(\mathbf{r}^h)\Bigg)
\end{align}
ただし,$k_1$は更新率 (updating rate)である.または,発火率の時定数を$\tau\coloneqq1/k_1$として,$k_1$は発火率の時定数$\tau$の逆数であると捉えることもできる.ここで1番目の式において,中間表現 $\mathbf{r}$ のダイナミクスはbottom-up errorとtop-down errorで記述されている.このようにbottom-up errorが $\mathbf{r}$ への入力となることは自然に導出される.なお,top-down errorに関しては高次からの予測 (prediction)の項 $f(\mathbf{x}^h)$とleaky-integratorとしての項 $-\mathbf{r}$に分割することができる.また$\mathbf{U}^\top, (\mathbf{U}^h)^\top$は重み行列の転置となっており,bottom-upとtop-downの投射において対称な重み行列を用いることを意味している.$-g'(\mathbf{r})$は発火率を抑制してスパースにすることを目的とする項だが,無理やり解釈をすると自己再帰的な抑制と言える.
\subsubsection{画像データの読み込み}
「スパース符号化」と同様にデータは\url{http://www.rctn.org/bruno/sparsenet/}からダウンロードできるファイルを用いる.
\begin{lstlisting}[language=julia]
using MAT

# datasets from http://www.rctn.org/bruno/sparsenet/
mat_images = matopen("../_static/datasets/IMAGES.mat")
imgs = read(mat_images, "IMAGES")

close(mat_images)
\end{lstlisting}
\subsubsection{モデルの定義}
必要なパッケージを読み込む.
\begin{lstlisting}[language=julia]
using Parameters: @unpack # or using UnPack
using LinearAlgebra, Random, Statistics, PyPlot, ProgressMeter
\end{lstlisting}
モデルを定義する.
\begin{lstlisting}[language=julia]
@kwdef struct RBParameter{FT}
    α::FT = 1.0
    αh::FT = 0.05
    σ²::FT = 1.0
    σ²td::FT = 10
    σ⁻²::FT = 1/σ²       
    σ⁻²td::FT = 1/σ²td
    k₁::FT = 0.3 # k_1: update rate
    λ::FT = 0.02 # regularization parameter
end

@kwdef mutable struct RaoBallard1999Model{FT}
    param::RBParameter = RBParameter{FT}()
    num_units_lv0::UInt16 = 256 # number of units of level0
    num_units_lv1::UInt16 = 32
    num_units_lv2::UInt16 = 128
    num_lv1::UInt16 = 3
    k₂::FT = 0.2 # k_2: learning rate
    r::Array{FT} = zeros(num_lv1, num_units_lv1) # activity of neurons
    rh::Array{FT} = zeros(num_units_lv2) # activity of neurons
    U::Array{FT} = randn(num_units_lv0, num_units_lv1) .* sqrt(2.0 / (num_units_lv0+num_units_lv1))
    Uh::Array{FT} = randn(num_lv1*num_units_lv1, num_units_lv2) .* sqrt(2.0 / (num_lv1*num_units_lv1+num_units_lv2))
end
\end{lstlisting}
パラメータを更新する関数を定義する.
\begin{lstlisting}[language=julia]
function update!(variable::RaoBallard1999Model, param::RBParameter, inputs::Array, training::Bool)
    @unpack num_units_lv0, num_units_lv1, num_units_lv2, num_lv1, k₂, r, rh, U, Uh = variable
    @unpack α, αh, σ⁻², σ⁻²td, k₁, λ = param

    r_reshaped = r[:] # (96)

    fx = r * U' # (3, 256)
    fxh = Uh * rh # (96, )

    # Calculate errors
    error = inputs - fx # (3, 256)
    errorh = r_reshaped - fxh # (96, ) 
    errorh_reshaped = reshape(errorh, (num_lv1, num_units_lv1)) # (3, 32)

    g_r = α * r ./ (1.0 .+ r .^ 2) # (3, 32)
    g_rh = αh * rh ./ (1.0 .+ rh .^ 2) # (64, )

    # Update r and rh
    dr = k₁ * (σ⁻² * error * U - σ⁻²td * errorh_reshaped - g_r)
    drh = k₁ * (σ⁻²td * Uh' * errorh - g_rh)
    
    r[:, :] += dr
    rh[:] += drh
    
    if training 
        U[:, :] += k₂ * (σ⁻² * error' * r - num_lv1 * λ * U)
        Uh[:, :] += k₂ * (σ⁻²td * errorh * rh' - λ * Uh)
    end

    return error, errorh, dr, drh
end
\end{lstlisting}
入力に乗じるGaussianフィルタを定義する.
\begin{lstlisting}[language=julia]
# Gaussian mask for inputs
function gaussian_2d(sizex=16, sizey=16, sigma=5)
    x, y = 0:sizex-1, 0:sizey-1
    x0, y0 = (sizex-1)/2, (sizey-1)/2
    f(x,y) = exp(-((x-x0)^2 + (y-y0)^2) / (2.0*(sigma^2)))
    gau = f.(x', y)
    return gau ./ sum(gau)
end
\end{lstlisting}
\begin{lstlisting}[language=julia]
gau = gaussian_2d()
figure(figsize=(2,2))
title("Gaussian mask")
imshow(gau)
tight_layout()
\end{lstlisting}
\begin{figure}[ht]
	\centering
	\includegraphics[scale=0.8, max width=\linewidth]{./fig/energy-based-model/predictive-coding/cell012.png}
	\caption{cell012.png}
	\label{cell012.png}
\end{figure}
損失関数を定義する.
\begin{lstlisting}[language=julia]
function calculate_total_error(error, errorh, variable::RaoBallard1999Model, param::RBParameter)
    @unpack r, rh, U, Uh = variable
    @unpack α, αh, σ⁻², σ⁻²td, k₁, λ = param
    recon_error = σ⁻² * sum(error.^2) + σ⁻²td * sum(errorh.^2)
    sparsity_r = α * sum(r.^2) + αh * sum(rh.^2)
    sparsity_U = λ * (sum(U.^2) + sum(Uh.^2))
    return recon_error + sparsity_r + sparsity_U
end;
\end{lstlisting}
シミュレーションを実行する関数を定義する.外側の\jl{for loop}では画像パッチの作成と\jl{r}の初期化を行う.内側の\jl{for loop}では\jl{r}が収束するまで更新を行い,収束したときに重み行列\jl{Phi}を更新する.
\begin{lstlisting}[language=julia]
function run_simulation(imgs, num_iter, nt_max, eps)
    # Define model
    model = RaoBallard1999Model{Float32}()
    
    # Simulation constants
    H, W, num_images = size(imgs)
    input_scale = 40 # scale factor of inputs
    gmask = gaussian_2d() # Gaussian mask
    errorarr = zeros(num_iter) # Vector to save errors    
    
    # Run simulation
    @showprogress "Computing..." for iter in 1:num_iter
        # Get images randomly
        idx = rand(1:num_images)
        img = imgs[:, :, idx]

        # Get the coordinates of the upper left corner of clopping image randomly.
        beginx = rand(1:W-27)
        beginy = rand(1:H-17)
        img_clopped = img[beginy:beginy+15, beginx:beginx+25]

        # Clop three patches
        inputs = stack([(gmask .* img_clopped[:, 1+i*5:i*5+16])[:] for i = 0:2])'
        inputs = (inputs .- mean(inputs)) .* input_scale

        # Reset states
        model.r = inputs * model.U 
        model.rh = model.Uh' * model.r[:]

        # Input an image patch until latent variables are converged 
        for i in 1:nt_max
            # Update r and rh without update weights 
            error, errorh, dr, drh = update!(model, model.param, inputs, false)

            # Compute norm of r and rh
            dr_norm = sqrt(sum(dr.^2))
            drh_norm = sqrt(sum(drh.^2))

            # Check convergence of r and rh, then update weights
            if dr_norm < eps && drh_norm < eps
                error, errorh, dr, drh = update!(model, model.param, inputs, true)
                errorarr[iter] = calculate_total_error(error, errorh, model, model.param) # Append errors
                break
            end

            # If failure to convergence, break and print error
            if i >= nt_max-2
                println("Error at patch:", iter)
                println(dr_norm, drh_norm)
                break
            end
        end


        # Decay learning rate         
        if iter % 40 == 39
            model.k₂ /= 1.015
        end

        # Print moving average error
        if iter % 1000 == 0
            moving_average_error = mean(errorarr[iter-999:iter])
            println("[", iter, "/", num_iter, "] Moving average error:", moving_average_error)
        end
    end
    return model, errorarr
end
\end{lstlisting}
シミュレーションの実行をする
\begin{lstlisting}[language=julia]
# Simulation constants
num_iter = 5000 # number of iterations
nt_max = 1000 # Maximum number of simulation time
eps = 1e-3 # small value which determines convergence

model, errorarr = run_simulation(imgs, num_iter, nt_max, eps);
\end{lstlisting}
\subsubsection{訓練中の損失の描画}
訓練中の損失の変化を描画してみよう.損失が低下し,学習が進行したことが分かる.
\begin{lstlisting}[language=julia]
function moving_average(x, n=100)
    ret = cumsum(x)
    ret[n:end] = ret[n:end] - ret[1:end-n+1]
    return ret[n - 1:end] / n
end

# Plot error
moving_average_error = moving_average(errorarr)
figure(figsize=(4, 2))
ylabel("Moving error")
xlabel("Iterations")
plot(1:size(moving_average_error)[1], moving_average_error)
tight_layout()
\end{lstlisting}
\begin{figure}[ht]
	\centering
	\includegraphics[scale=0.8, max width=\linewidth]{./fig/energy-based-model/predictive-coding/cell020.png}
	\caption{cell020.png}
	\label{cell020.png}
\end{figure}
\subsubsection{重み行列 (受容野)の描画}
学習後の重み行列 ($\mathbf{U}$)を可視化してみよう.
\begin{lstlisting}[language=julia]
# Plot Receptive fields
figure(figsize=(6, 3))
subplots_adjust(hspace=0.1, wspace=0.1)
for i in 1:32
    subplot(4, 8, i)
    imshow(reshape(model.U[:, i], (16, 16)), cmap="gray")
    axis("off")
end
suptitle("Receptive fields of level 1", fontsize=14)
subplots_adjust(top=0.9)
\end{lstlisting}
\begin{figure}[ht]
	\centering
	\includegraphics[scale=0.8, max width=\linewidth]{./fig/energy-based-model/predictive-coding/cell022.png}
	\caption{cell022.png}
	\label{cell022.png}
\end{figure}
白色が\textbf{ON領域}\index{ONりょういき@ON領域}(興奮),黒色が\textbf{OFF領域}\index{OFFりょういき@OFF領域}(抑制)を表す.Gaborフィルタ様の局所受容野が得られている.次に,Level2のニューロンの受容野は$\mathbf{U}$と$\mathbf{U}^h$の積を計算することで描画できる.
\begin{lstlisting}[language=julia]
# Plot Receptive fields of level 2
zero_padding = zeros(80, 32)
U0 = [model.U; zero_padding; zero_padding]
U1 = [zero_padding; model.U; zero_padding]
U2 = [zero_padding; zero_padding; model.U]
U_ = [U0 U1 U2]
Uh_ = U_ * model.Uh 

figure(figsize=(7, 3))
subplots_adjust(hspace=0.1, wspace=0.1)
for i in 1:24
    subplot(4, 6, i)
    imshow(reshape(Uh_[:, i], (16, 26)), cmap="gray")
    axis("off")
end

suptitle("Receptive fields of level 2", fontsize=14)
subplots_adjust(top=0.9)
\end{lstlisting}
\begin{figure}[ht]
	\centering
	\includegraphics[scale=0.8, max width=\linewidth]{./fig/energy-based-model/predictive-coding/cell024.png}
	\caption{cell024.png}
	\label{cell024.png}
\end{figure}
 % ok

\chapter{貢献度分配問題の解決策}
% \section{予測符号化}
\subsection{観測世界の階層的予測}
\textbf{階層的予測符号化(hierarchical predictive coding; HPC)} は\cite{Rao1999-zv}により導入された.構築するネットワークは入力層を含め,3層のネットワークとする.LGNへの入力として画像 $\mathbf{x} \in \mathbb{R}^{n_0}$を考える.画像 $\mathbf{x}$ の観測世界における隠れ変数,すなわち\textbf{潜在変数} (latent variable)を$\mathbf{r} \in \mathbb{R}^{n_1}$とし,ニューロン群によって発火率で表現されているとする (真の変数と $\mathbf{r}$は異なるので文字を分けるべきだが簡単のためにこう表す).このとき,


\mathbf{x} = f(\mathbf{U}\mathbf{r}) + \boldsymbol{\epsilon}


が成立しているとする.ただし,$f(\cdot)$は活性化関数 (activation function),$\mathbf{U} \in \mathbb{R}^{n_0 \times n_1}$は重み行列である.
$\boldsymbol{\epsilon} \in \mathbb{R}^{n_0}$ は $\mathcal{N}(\mathbf{0}, \sigma^2 \mathbf{I})$ からサンプリングされるとする.

潜在変数 $\mathbf{r}$はさらに高次 (higher-level)の潜在変数 $\mathbf{r}^h$により,次式で表現される.


\mathbf{r} = \mathbf{r}^{td}+\boldsymbol{\epsilon}^{td}=f(\mathbf{U}^h \mathbf{r}^h)+\boldsymbol{\epsilon}^{td}


ただし,Top-downの予測信号を $\mathbf{r}^{td}:=f(\mathbf{U}^h \mathbf{r}^h)$とした.また,$\mathbf{r}^{td} \in \mathbb{R}^{n_1}$, $\mathbf{r}^{h} \in \mathbb{R}^{n_2}$, $\mathbf{U}^h \in \mathbb{R}^{n_1 \times n_2}$ である.
$\boldsymbol{\epsilon}^{td} \in \mathbb{R}^{n_1}$は$\mathcal{N}(\mathbf{0}$, $\sigma_{td}^2 \mathbf{I}$) からサンプリングされるとする.

話は飛ぶが,Predictive codingのネットワークの特徴は
\begin{itemize}
\item 階層的な構造
\item 高次による低次の予測 (Feedback or Top-down信号)
\item 低次から高次への誤差信号の伝搬 (Feedforward or Bottom-up 信号)
\end{itemize}

である.ここまでは高次表現による低次表現の予測,というFeedback信号について説明してきたが,この部分はSparse codingでも同じである.それではPredictive codingのもう一つの要となる,低次から高次への予測誤差の伝搬というFeedforward信号はどのように導かれるのだろうか.結論から言えば,これは\textbf{復元誤差 (reconstruction error)の最小化を行う再帰的ネットワーク (recurrent network)を考慮することで自然に導かれる}.

\lstinputlisting[language=julia]{./text/solve-credit-assignment-problem/backpropagation/001.jl}
\lstinputlisting[language=julia]{./text/solve-credit-assignment-problem/backpropagation/002.jl}
UNDERSTANDING STRAIGHT-THROUGH ESTIMATOR IN TRAINING ACTIVATION QUANTIZED NEURAL NETS

Yoshua Bengio, Nicholas L´eonard, and Aaron Courville. Estimating or propagating gradients through stochastic neurons for conditional computation. arXiv preprint arXiv:1308.3432, 2013.

2種類の指数関数型シナプスの動態.破線は単一指数関数型シナプスで, 実線は二重指数関数型シナプスである.
\lstinputlisting[language=julia]{./text/solve-credit-assignment-problem/backpropagation/005.jl}
いくつかの処理について解説しておく.まず,一番目のforループ内の\jl{v[i]}の\jl{((dt*tcount) > (tlast[i] + tref))}は最後にニューロンが発火した時刻\jl{tlast[i]}に不応期\jl{tref}を足した時刻よりも現在の時刻\jl{dt*tcount[1]}が大きければ膜電位の更新を許可し,小さければ更新しない.二番目のforループにおける\jl{fire[i]}はニューロンの膜電位が閾値電位\jl{vthr}を超えたら\jl{True}となる.\jl{v[i]}などの更新式にある\jl{ifelse(a, b, c)}はaが\jl{True}の時はbを返し,\jl{False}の時はcを返す関数であり,\jl{v[i] = ifelse(fire[i], vreset, v[i])}は\jl{fire[i]}が\jl{True}なら\jl{v[i]}をリセット電位\jl{vreset}とし,そうでなければそのままの値を返すという処理である.同様にして\jl{tlast[i]}は発火したときにその時刻を記録する変数となっている.なお,\jl{v_[i] = ifelse(fire[i], vpeak, v[i])}は実際のシミュレーションにおいて意味をなさない.単に発火時の電位\jl{vpeak}を含めて記録すると描画時の見栄えが良いというだけである.

これらの\jl{struct}と関数を用いてシミュレーションを実行する.\jl{I} はHHモデルのときと同じように矩形波を入力する.実は\jl{I}は入力電流ではなく入力電流に比例する量となっているが,これは膜抵抗を乗じた後の値であると考えるとよい.

\lstinputlisting[language=julia]{./text/solve-credit-assignment-problem/backpropagation/007.jl}
## 7.9.2 更新関数の定義

\lstinputlisting[language=julia]{./text/solve-credit-assignment-problem/backpropagation/009.jl}
\subsection{ランジュバン・モンテカルロ法 (LMC)}拡散過程
$$
{\frac{d\theta}{dt}}=\nabla \log p (\theta)+{\sqrt 2}{d{W}}
$$
Euler–Maruyama法により,
\lstinputlisting[language=julia]{./text/solve-credit-assignment-problem/backpropagation/011.jl}
### 訓練データで学習

\lstinputlisting[language=julia]{./text/solve-credit-assignment-problem/backpropagation/013.jl}
\subsection{参考文献}- \url{https://towardsdatascience.com/a-brief-introduction-to-slow-feature-analysis-18c901bc2a58}
- \url{https://github.com/flatironinstitute/bio-sfa}
- \url{https://github.com/fulviadelduca/slow-feature-analysis}
- [Deep Slow Feature Analysis Network](https://github.com/rulixiang/DSFANet)
- \url{https://nbviewer.jupyter.org/github/pierrelux/notebooks/blob/master/Slow%20Feature%20Analysis.ipynb}
## 相図の描画

50msから200msまでで11回, 250msから400msまでで16回発火しているので発火回数は計27回であり,この結果は正しい.
\lstinputlisting[language=julia]{./text/solve-credit-assignment-problem/backpropagation/017.jl}
入力は64(網膜座標系での位置)+2(眼球位置信号)=66とする.眼球位置信号は原著ではmonotonic形式による32(=8ユニット×2(x, y方向)×2 (傾き正負))ユニットで構成されるが,簡単のために眼球位置信号も$x, y$の2次元とする.視覚刺激は-40度から40度までの範囲であり,10度で離散化する.よって,網膜座標系での位置は$8\times 8$の行列で表現される.位置は2次元のGaussianで表現する.ただし,1/e幅 (ピークから1/eに減弱する幅) は15度である.$1/e$の代わりに$1/2$とすれば半値全幅(FWHM)となる.スポットサイズを$w$,Gaussianを$G(x)$とすると.$G(x+w/2)=G/e$より,$\sigma=\frac{\sqrt{2}w}{4}$と求まる.

\lstinputlisting[language=julia]{./text/solve-credit-assignment-problem/backpropagation/019.jl}
\lstinputlisting[language=julia]{./text/solve-credit-assignment-problem/backpropagation/020.jl}
通常のPoisson spikeと差はあまり感じられないが,高頻度発火の場合に通常のモデルとの違いが明瞭となる.
\lstinputlisting[language=julia]{./text/solve-credit-assignment-problem/backpropagation/022.jl}
複数の点が同じ位置に重なっていることに注意.

\lstinputlisting[language=julia]{./text/solve-credit-assignment-problem/backpropagation/024.jl}
損失の変化を描画する.

\lstinputlisting[language=julia]{./text/solve-credit-assignment-problem/backpropagation/026.jl}
\begin{figure}[ht]
	\centering
	\includegraphics[scale=0.8, max width=\linewidth]{./fig/motor-learning/infinite-horizon-ofc/cell026.png}
	\caption{cell026.png}
	\label{cell026.png}
\end{figure}
テストデータを用いて,出力を確認する.

\lstinputlisting[language=julia]{./text/solve-credit-assignment-problem/backpropagation/028.jl}
\begin{figure}[ht]
	\centering
	\includegraphics[scale=0.8, max width=\linewidth]{./fig/solve-credit-assignment-problem/backpropagation/cell028.png}
	\caption{cell028.png}
	\label{cell028.png}
\end{figure}
\subsubsection{線形行列方程式}


\mathbf{A}\mathbf{x}=\mathbf{b}


$\mathbf{A}$が正則の場合,逆行列が存在し,


\mathbf{x}=\mathbf{A}^{-1}\mathbf{b}

\lstinputlisting[language=julia]{./text/solve-credit-assignment-problem/backpropagation/030.jl}
\begin{figure}[ht]
	\centering
	\includegraphics[scale=0.8, max width=\linewidth]{./fig/bayesian-brain/neural-sampling/cell030.png}
	\caption{cell030.png}
	\label{cell030.png}
\end{figure}
なお,前述したようにガンマ過程モデルの方がポアソン過程モデルよりも皮質ニューロンのモデルとしては優れているが,入力画像のエンコーディングをガンマ過程モデルにすることでSNNの認識精度が向上するかどうかはまだ十分に研究されていない.また,(Deger, et al., 2012\url{https://pubmed.ncbi.nlm.nih.gov/21964584/})ではPPDやガンマ過程の重ね合わせによるスパイク列を生成するアルゴリズムを考案している.

\lstinputlisting[language=julia]{./text/solve-credit-assignment-problem/backpropagation/032.jl}
\begin{figure}[ht]
	\centering
	\includegraphics[scale=0.8, max width=\linewidth]{./fig/solve-credit-assignment-problem/backpropagation/cell032.png}
	\caption{cell032.png}
	\label{cell032.png}
\end{figure}

% # 線形多層ニューラルネットワークの学習ダイナミクス
\subsection{線形多層ニューラルネットワークにおける学習ダイナミクスと知識の獲得}
> A. M. Saxe, J. L. McClelland, S. Ganguli. "\textbf{A mathematical theory of semantic development in deep neural networks}\index{A mathematical theory of semantic development in deep neural networks}". *PNAS.* (2019). ([arXiv](https://arxiv.org/abs/1810.10531)). ([PNAS](https://www.pnas.org/content/early/2019/05/16/1820226116))
\subsubsection{モデルと学習}
入力 $\mathbf{x}$ は「もの」の項目(例えばカナリア,犬,サーモン,樫など),出力 $\mathbf{y}$はそれぞれの項目の性質・特性となっている.例えばカナリア(Canary)は成長し(Grow),動き(Move),空を飛べる(Fly)ので,Canaryという入力に対し,ネットワークが出力するのはGrow, Move, Flyとなる.モデルは3層の全結合線形ネットワークである.
\hat{\mathbf{y}}=\mathbf{W}_2 \mathbf{W}_1\mathbf{x} 
ただし非線形な活性化関数が無いことに注意しよう.このようなネットワークを線形ニューラルネットワーク (linear neural network)と呼ぶ.当然, $\mathbf{W}_s=\mathbf{W}_2 \mathbf{W}_1$として, 上のネットワークは
\hat{\mathbf{y}}=\mathbf{W}_s\mathbf{x}
とまとめることができる.このため,線形な活性化関数で深いニューラルネットワークを構築しても意味がなく,それゆえ非線形な活性化関数が必要となる.しかし,\textbf{勾配降下法で学習させると}\index{こうばいこうかほうでがくしゅうさせると@勾配降下法で学習させると}3層と2層のネットワークの学習ダイナミクスはそれぞれ異なるものとなり,得られる解にも違いが生まれる.加えて,深い(3層の)ネットワークである場合のみ,幼児の発達における非線形な現象が説明できる.
3層ネットワークの学習(重みの更新)は誤差逆伝搬から導かれる次の2式により行う.
\begin{aligned} \tau \frac{d\mathbf{W}_1}{dt} &=\mathbf{W}_2^\top \left(\mathbf{\Sigma}^{yx} - \mathbf{W}_2 \mathbf{W}_1 \mathbf{\Sigma}^{x}\right)\\
\tau \frac{d\mathbf{W}_2}{dt} &=\left(\mathbf{\Sigma}^{yx} - \mathbf{W}_2 \mathbf{W}_1 \mathbf{\Sigma}^{x}\right) \mathbf{W}_1^\top
\end{aligned}
ただし,$ \mathbf{\Sigma}^{x}$は入力間の関係を表す行列,$\mathbf{\Sigma}^{yx}$は入出力の関係を表す行列である.
\subsubsection{特異値分解(SVD)による学習ダイナミクスの解析}
学習ダイナミクスは$ \mathbf{\Sigma}^{yx}$に対する特異値分解(singular value decomposition; SVD)を用いて説明できる.
\mathbf{\Sigma}^{yx}=\mathbf{USV}^\top
行列$ \mathbf{ S}$の対角成分の非ゼロ要素が特異値である.次に学習途中の時刻$(t)$における$\hat{\mathbf{\Sigma}}^{yx}(t)=\mathbf{W}_2 (t) \mathbf{W}_1(t) \mathbf{\Sigma}^{x}$に対してSVDを実行し,特異値$\mathbf{A}(t)=[a_{\alpha}(t)]$を得る.この $a _ {\alpha}(t)$だが,3層のネットワークでは大きな特異値から先に学習されるのに対し,2層のネットワークでは全ての特異値が同時に学習される.このダイナミクスだが,\textbf{低ランク近似}\index{ていらんくきんじ@低ランク近似} (low-rank approximation)が生じていて,特異値の大きな要素から学習されていると捉えることができる.学習が進むとランクが大きくなっていく,ということである.低ランク近似の例として,SVDによる画像の圧縮と復元を見てみよう.カメラマンの画像に対し,低ランク近似を行い,ランクを上げていく.するとランクが上がるにつれて,画像が鮮明になる.
\begin{lstlisting}[language=julia]
using PyPlot, LinearAlgebra, TestImages
rc("axes.spines", top=false, right=false)
\end{lstlisting}
\begin{lstlisting}[language=julia]
# Low-rank approximation with SVD
function LowRankApprox(U, s, V; rank=1)
    Ur, sr, Vr = U[:, 1:rank], s[1:rank], V[:, 1:rank]
    return Ur * diagm(sr) * Vr'
end;
\end{lstlisting}
\begin{lstlisting}[language=julia]
img = convert(Array{Float64}, testimage("cameraman"));
U, s, V = svd(img); # PythonとJuliaでVに転置がかかっているか否かの違いあり.
ranklist = [5, 15, 25]
nr = length(ranklist)
img_approx = [LowRankApprox(U, s, V, rank=r) for r in ranklist];
\end{lstlisting}
\begin{lstlisting}[language=julia]
figure(figsize=(10, 3))
subplot(1, nr+1, 1); imshow(img, cmap="gray"); title("Original image"); axis("off")
for i in 1:nr
    subplot(1, nr+1, i+1); imshow(img_approx[i], cmap="gray"); title("rank="*string(ranklist[i])); axis("off")
end
tight_layout()
\end{lstlisting}
\begin{figure}[ht]
	\centering
	\includegraphics[scale=0.8, max width=\linewidth]{./fig/solve-credit-assignment-problem/linear-network-learning-dynamics/cell005.png}
	\caption{cell005.png}
	\label{cell005.png}
\end{figure}
これと同じことが知識の獲得において生じていると見なすことができる.
\begin{lstlisting}[language=julia]
# Set initial values
N₁, N₂, N₃ = 4, 16, 7
Σyx = [ones(4)'; ones(2)' zeros(2)'; zeros(2)' ones(2)'; I(4)];
Σx = I(N₁)
_, s, _ = svd(Σyx);
eps = 1e-2
W₁, W₂ = eps*rand(N₂,N₁), eps*rand(N₃,N₂) # weight for deep
Ws = eps*rand(N₃,N₁) # weight for shallow

#Simulation & training
dt = 0.005
Nt = 1500

# Singular values for shallow, deep
A, B = zeros(Nt, N₁), zeros(Nt, N₁); 
\end{lstlisting}
\begin{lstlisting}[language=julia]
# Shallow network
for t in 1:Nt
    # Update weights
    Ws += (Σyx - Ws * Σx) * dt   
    # SVD & save results
    Σ̂yx = Ws * Σx
    _, a, _ = svd(Σ̂yx)
    A[t, :] += a
end
\end{lstlisting}
\begin{lstlisting}[language=julia]
# Deep network
for t in 1:Nt
    # Update weights
    δ = Σyx - W₂ * W₁ * Σx
    W₁ += (W₂' * δ) * dt
    W₂ += (δ * W₁') * dt
    # SVD & save results
    Σ̂yx = W₂ * W₁ * Σx
    _, b, _ = svd(Σ̂yx)
    B[t, :] += b
end
\end{lstlisting}
\begin{lstlisting}[language=julia]
# Plot results
T = range(0, 1, length=Nt)
figure(figsize=(8, 3), dpi=100)
subplot(1,2,1); title("Two-layer network")
for i in 1:4
    plot(T, A[:,i], label="a"*string(i))
    axhline(s[i], color="gray", linestyle="dashed")
end
xlim(0, 1); xlabel("t"); ylabel("A(t)"); legend(ncol=2)

subplot(1,2,2); title("Three-layer network")
for i in 1:4
    plot(T, B[:,i], label="b"*string(i))
    axhline(s[i], color="gray", linestyle="dashed")
end
xlim(0, 1); xlabel("t"); ylabel("B(t)"); legend(ncol=2)
tight_layout()
\end{lstlisting}
\begin{figure}[ht]
	\centering
	\includegraphics[scale=0.8, max width=\linewidth]{./fig/solve-credit-assignment-problem/linear-network-learning-dynamics/cell010.png}
	\caption{cell010.png}
	\label{cell010.png}
\end{figure}
3層線形ネットワーク (deep)では大きな特異値から学習が始まっているのが分かる.また,それぞれの特異値の学習においてはシグモイド関数様の急速な学習段階が見られる.一方で2層線形ネットワーク (shallow)では全ての特異値の学習が初めから起こっていることがわかる.パラメータが少ないため,収束はこちらの方が速い.
このモデルが面白い理由の一つとして,知識の混同 (例えば『芋虫には骨がある』) の仕組みを提供することがある.発達において,大きい特異値から先に学習されるため,「動く」,「成長する」などの動物の要素が先に獲得される.身の回りの動物のほとんどが「骨を持つ」ので,\textbf{低ランク近似により,『芋虫にも骨がある』と錯覚してしまう}\index{ていらんくきんじにより,『いもむしにもほねがある』とさっかくしてしまう@低ランク近似により,『芋虫にも骨がある』と錯覚してしまう}のではないか,という仮説が立てられている.
\subsection{線形多層ニューラルネットワークにおける勾配降下法による低ランク解の獲得}
> Jing, L., Zbontar, J. & LeCun, Y. \textbf{Implicit Rank-Minimizing Autoencoder}\index{Implicit Rank-Minimizing Autoencoder}. *NeurIPS' 20*, 2020. \url{https://arxiv.org/abs/2010.00679}
([Arora et al., *NeurIPS' 19*. 2019](https://arxiv.org/abs/1905.13655))は深層線形ニューラルネットワークが低ランクの解を導出できることを理論的及び実験的に実証した.([Gunasekar et al., *NeurIPS' 18*. 2018](https://arxiv.org/abs/1806.00468))は,線形畳み込みニューラルネットワークにおいて勾配降下が正則化作用を持つことを示した.
証明は省略するが,([Arora et al., *NeurIPS' 19*. 2019](https://arxiv.org/abs/1905.13655))におけるTheorem 3.を紹介する.まず,$N$層の線形多層ニューラルネットワークを考え,$W_j \in \mathbb{R}^{d_j \times d_{j−1}}$を$j$層の重みとする.$t$を学習のタイムステップとし,$W(t) \in \mathbb{R}^{d \times d^\prime}$を重み行列を全て乗じた行列とする (ただし$d \coloneqq  d_N, d^\prime \coloneqq  d_0$).つまり$W(t)\coloneqq \prod_{j=1}^N W_j(t)$である.
ここで$W(t)$を特異値分解し,$W(t) = U(t)S(t)V^\top(t)$と表現する.$S(t)$は対角行列で,その要素を$\sigma_1(t), \ldots , \sigma_{\min\{d, d^\prime\}}(t),$とする.これが$W(t)$の特異値となる.さらに$U(t), V (t)$の列ベクトルをそれぞれ $\mathbf{u}_1(t), \ldots, \mathbf{u}_{\min\{d, d^\prime\}}(t)$, および $\mathbf{v}_1(t), \ldots, \mathbf{v}_{\min\{d,d^\prime \}}(t)$とする.このとき,特異値$ \sigma_r(t)\ (r=1, \ldots, \min\{d,d^\prime \})$の損失関数$\mathcal{L}(W(t))$に対する勾配降下法による変化は
\frac{d \sigma_r(t)}{dt} = - N \cdot \left[\sigma_r(t)\right]^{1 - \frac{1}{N}} \cdot \left\langle \nabla \mathcal{L}(W(t)) , \mathbf{u}_r(t) \mathbf{v}_r^\top(t) \right\rangle
と表される (Arora et al., 2019; Theorem 3).(1)式で重要なのは$\left[\sigma_r(t)\right]^{1 - \frac{1}{N}}$の項である.これは$N\geq 2$のときに\textbf{特異値$\sigma_r(t)\ (\geq 0)$を小さくするような正則化作用が生じる}\index{とくいち$\sigma_r(t)\ (\geq 0)$をちーさくするようなせいそくかさようがしょうじる@特異値$\sigma_r(t)\ (\geq 0)$を小さくするような正則化作用が生じる}ことを意味している.一方で,隠れ層が1つのニューラルネットワーク ($N=1$)の場合 (1)式は
\frac{d \sigma_r(t)}{dt} = - \left\langle \nabla \mathcal{L}(W(t)) , \mathbf{u}_r(t) \mathbf{v}_r^\top(t) \right\rangle
となり,正則化作用は消失する.
このように線形多層ニューラルネットワークを勾配降下法で学習させると\textbf{陰的正則化(implicit regularization)}\index{いんてきせいそくか(implicit regularization)@陰的正則化(implicit regularization)} により低ランクの解が得られるということが複数の研究により明らかとなっている (線形多層ニューラルネットワークの陰的正則化に関して日本語で書かれた資料としては鈴木大慈先生の[深層学習の数理](https://www.slideshare.net/trinmu/ss-161240890)のスライドp.64, 65がある).Jingらはこの性質を用い,\textbf{Autoencoderに線形層を複数追加}\index{Autoencoderにせんけいそうをふくすうついか@Autoencoderに線形層を複数追加}するという簡便な方法で低次元表現を学習する決定論的Autoencoder (\textbf{Implicit Rank-Minimizing Autoencoder; IRMAE)}\index{Implicit Rank-Minimizing Autoencoder; IRMAE)} を考案した.

% \section{BPTT (backpropagation through time)}
通時的誤差逆伝播法
## モデルの定義
ライブラリの読み込み.
\lstinputlisting[language=julia]{./text/solve-credit-assignment-problem/bptt/003.jl}
\lstinputlisting[language=julia]{./text/solve-credit-assignment-problem/bptt/004.jl}
\lstinputlisting[language=julia]{./text/solve-credit-assignment-problem/bptt/005.jl}
\jl{w_in}は入力層から再帰層への重み,\jl{w_rec}は再帰重み,\jl{w_out}は出力重みである.
\lstinputlisting[language=julia]{./text/solve-credit-assignment-problem/bptt/007.jl}
## 7.9.2 更新関数の定義
\lstinputlisting[language=julia]{./text/solve-credit-assignment-problem/bptt/009.jl}
\subsection{正弦波の学習}
例として正弦波を出力するRNNを考える.入力1,中間64, 出力2のRNNである.
\lstinputlisting[language=julia]{./text/solve-credit-assignment-problem/bptt/011.jl}
入力と訓練データの確認をする.
\lstinputlisting[language=julia]{./text/solve-credit-assignment-problem/bptt/013.jl}
\begin{figure}[ht]
	\centering
	\includegraphics[scale=0.8, max width=\linewidth]{./fig/neuron-model/hodgkin-huxley/cell013.png}
	\caption{cell013.png}
	\label{cell013.png}
\end{figure}
モデルの定義をする.
\lstinputlisting[language=julia]{./text/solve-credit-assignment-problem/bptt/015.jl}
学習を実行する.
\lstinputlisting[language=julia]{./text/solve-credit-assignment-problem/bptt/017.jl}
損失の推移を確認する.
\lstinputlisting[language=julia]{./text/solve-credit-assignment-problem/bptt/019.jl}
\begin{figure}[ht]
	\centering
	\includegraphics[scale=0.8, max width=\linewidth]{./fig/energy-based-model/sparse-coding/cell019.png}
	\caption{cell019.png}
	\label{cell019.png}
\end{figure}
## 学習後の出力の確認
\lstinputlisting[language=julia]{./text/solve-credit-assignment-problem/bptt/021.jl}
見やすいように出力のピークに応じて中間層のユニットをソートする.
\lstinputlisting[language=julia]{./text/solve-credit-assignment-problem/bptt/023.jl}
出力層,中間層の出力を描画する.
\lstinputlisting[language=julia]{./text/solve-credit-assignment-problem/bptt/025.jl}
\begin{figure}[ht]
	\centering
	\includegraphics[scale=0.8, max width=\linewidth]{./fig/neuron-model/hodgkin-huxley/cell025.png}
	\caption{cell025.png}
	\label{cell025.png}
\end{figure}

% \section{予測符号化}
\subsection{観測世界の階層的予測}
\textbf{階層的予測符号化(hierarchical predictive coding; HPC)} は\cite{Rao1999-zv}により導入された.構築するネットワークは入力層を含め,3層のネットワークとする.LGNへの入力として画像 $\mathbf{x} \in \mathbb{R}^{n_0}$を考える.画像 $\mathbf{x}$ の観測世界における隠れ変数,すなわち\textbf{潜在変数} (latent variable)を$\mathbf{r} \in \mathbb{R}^{n_1}$とし,ニューロン群によって発火率で表現されているとする (真の変数と $\mathbf{r}$は異なるので文字を分けるべきだが簡単のためにこう表す).このとき,


\mathbf{x} = f(\mathbf{U}\mathbf{r}) + \boldsymbol{\epsilon}


が成立しているとする.ただし,$f(\cdot)$は活性化関数 (activation function),$\mathbf{U} \in \mathbb{R}^{n_0 \times n_1}$は重み行列である.
$\boldsymbol{\epsilon} \in \mathbb{R}^{n_0}$ は $\mathcal{N}(\mathbf{0}, \sigma^2 \mathbf{I})$ からサンプリングされるとする.

潜在変数 $\mathbf{r}$はさらに高次 (higher-level)の潜在変数 $\mathbf{r}^h$により,次式で表現される.


\mathbf{r} = \mathbf{r}^{td}+\boldsymbol{\epsilon}^{td}=f(\mathbf{U}^h \mathbf{r}^h)+\boldsymbol{\epsilon}^{td}


ただし,Top-downの予測信号を $\mathbf{r}^{td}:=f(\mathbf{U}^h \mathbf{r}^h)$とした.また,$\mathbf{r}^{td} \in \mathbb{R}^{n_1}$, $\mathbf{r}^{h} \in \mathbb{R}^{n_2}$, $\mathbf{U}^h \in \mathbb{R}^{n_1 \times n_2}$ である.
$\boldsymbol{\epsilon}^{td} \in \mathbb{R}^{n_1}$は$\mathcal{N}(\mathbf{0}$, $\sigma_{td}^2 \mathbf{I}$) からサンプリングされるとする.

話は飛ぶが,Predictive codingのネットワークの特徴は
\begin{itemize}
\item 階層的な構造
\item 高次による低次の予測 (Feedback or Top-down信号)
\item 低次から高次への誤差信号の伝搬 (Feedforward or Bottom-up 信号)
\end{itemize}

である.ここまでは高次表現による低次表現の予測,というFeedback信号について説明してきたが,この部分はSparse codingでも同じである.それではPredictive codingのもう一つの要となる,低次から高次への予測誤差の伝搬というFeedforward信号はどのように導かれるのだろうか.結論から言えば,これは\textbf{復元誤差 (reconstruction error)の最小化を行う再帰的ネットワーク (recurrent network)を考慮することで自然に導かれる}.

\subsubsection{重み行列$\mathbf{A}$の作成}

\subsubsection{事前分布の設定}
事前分布$p(\mathbf{r})$としては,0においてピークがあり,裾の重い(heavy tail)を持つsparse distributionあるいは \textbf{super-Gaussian distribution} (Laplace 分布やCauchy分布などGaussian分布よりもkurtoticな分布)を用いるのが良い.このような分布では,$\mathbf{r}$の各要素$r_i$はほとんど0に等しく,ある入力に対しては大きな値を取る.$p(\mathbf{r})$は一般化して式(4), (5)のように表記する.


\begin{aligned}
p(\mathbf{r})&=\prod_j p(r_j)\\
p(r_j)&=\frac{1}{Z_{\beta}}\exp \left[-\beta S(r_j)\right]
\end{aligned}


ただし,$\beta$は逆温度(inverse temperature), $Z_{\beta}$は規格化定数 (分配関数) である.これらの用語は統計力学における正準分布 (ボルツマン分布)から来ている.$S(x)$と分布の関係をまとめた表が以下となる (cf. \url{https://pdfs.semanticscholar.org/be08/da912362bf40fe3ded78bdadc644f921b4e7.pdf}).

UNDERSTANDING STRAIGHT-THROUGH ESTIMATOR IN TRAINING ACTIVATION QUANTIZED NEURAL NETS

Yoshua Bengio, Nicholas L´eonard, and Aaron Courville. Estimating or propagating gradients through stochastic neurons for conditional computation. arXiv preprint arXiv:1308.3432, 2013.

2種類の指数関数型シナプスの動態.破線は単一指数関数型シナプスで, 実線は二重指数関数型シナプスである.
変更しない定数を保持する \jl{struct} の \jl{FHNParameter} と, 変数を保持する \jl{mutable struct} の \jl{FHN} を作成する.
\lstinputlisting[language=julia]{./text/solve-credit-assignment-problem/surrogate-gradient-snn/006.jl}
\lstinputlisting[language=julia]{./text/solve-credit-assignment-problem/surrogate-gradient-snn/007.jl}
## 7.9.2 更新関数の定義


% \section{Reservoir ComputingとしてのRecurrent SNNの教師あり学習}
この章ではReservoir ComputingとしてのRecurrent SNNと、それを学習するためのFORCE法について解説します。
\section{Reservoir Computing}
\textbf{Reservoir Computing}は、RNN\footnote{ここでは発火率モデルについてのRNNについて述べています。}のモデルの一種です。一般のRNNが全ての結合重みを学習するのに対し、Reservoir ComputingではRNNのユニット間の結合重みはランダムに初期化して固定し、\textbf{出力の結合重みだけを学習}します。そのため、Reservoir Computingは学習するパラメータが少なく、学習も高速に行えます(もちろん関数の表現力は一般のRNNの方が優れています)。\par
Reservoirというのは溜め池(貯水池)を意味します。Reservoir Computingでは、まず入力信号をランダムな固定重みにより高次空間の信号に変換し、Reservoir RNN(信号の溜め池)に保持します。そして、Reservoir RNNのユニットの活動として保持された情報を学習可能な重みにより線形変換し、出力とします。このとき、ネットワークの出力が教師信号と一致するように出力重みを更新します。

%\begin{comment}
\begin{figure}[htbp]
    \centering
    \includegraphics[scale=1.2]{figs/reservoir_computing.pdf}
    \caption{(A)Reservoir Computingの一般的なモデル。入力と中間にはランダムに固定された重みを用い、出力のみ学習可能となっています。 (B)FORCE法で用いるモデルの1つ。7.3節以降でこのモデルの実装を行います。}
    \label{fig:RC}
\end{figure}
%\end{comment}
\subsection{FORCE法とRecurrent SNNへの適用}
Reservoir Computingにおける教師あり学習の手法の1つとして、\textbf{FORCE法}と呼ばれるものがあります。\textbf{FORCE} (First-Order Reduced and Controlled Error)法は(Sussillo \& Abbott, 2009)で提案された学習法で、元々は発火率ベースのRNNに対するオンラインの学習法です (具体的な方法については次節で解説します)。さらに(Nicola \& Clopath, 2017)はFORCE法がRecurrent SNNの学習に直接的に使用できる、ということを示しました。この章では(Nicola \& Clopath, 2017)の手法を用いてReservoir ComputingとしてのRecurrent SNNの教師あり学習を行います。
\subsubsection{Recurrent SNNに正弦波を学習させる}
今回はRecurrent SNNのニューロンの活動をデコードしたものが正弦波となるように(すなわち正弦波を教師信号として)訓練することを目標とします。先になりますが、結果は図のようになります。

\subsubsection{ネットワークの構造と教師信号}
ネットワークの構造は図のようになっています。ネットワークには特別な入力があるわけではなく、再帰的な入力によって活動が持続しています(膜電位の初期値をランダムにしているため開始時に発火するニューロン\footnote{ここでの「ニューロン」はこれ以後も含め、Reservoirのユニットを指します。}があり、またバイアス電流もあります)。\par
まず、Reservoirニューロンの数を$N$とし、出力の数を$N_\text{out}$とします。$i$番目のニューロンの入力はバイアス電流を$I_{\text{Bias}}$として、


\begin{equation}
I_i=s_i+I_{\text{Bias}}    
\end{equation}


と表されます。ただし、$s_i$は 


\begin{equation}
s_{i}=\sum_{j=1}^{N} \omega_{i j} r_{j}    
\end{equation}


として計算されます。$r_j$が$j$番目のニューロンの出力(シナプスフィルターをかけられたスパイク列), $\omega_{i j}$は$j$番目のニューロンから$i$番目のニューロンへの結合重みを意味します。\par
次にニューロンの活動$r_j$を重み$\phi\in \mathbb{R}^{N\times N_\text{out}}$で線形にデコードし、その出力$\hat{\boldsymbol{x}}(t)$を教師信号$\boldsymbol{x}(t)$に近づけます。すなわち、


\begin{equation}
\hat{\boldsymbol{x}}(t)=\sum_{j=1}^{N} \boldsymbol{\phi}_j r_{j}=\phi^\intercal\boldsymbol{r}
\end{equation}


とします。ただし、$^\intercal$を転置記号とし、$\boldsymbol{x}$を列ベクトル、$\boldsymbol{x}^\intercal$を行ベクトルとします。また、$\boldsymbol{\phi}_j\in \mathbb{R}^{N_\text{out}}$です。\par
ここから少しややこしいのですが、ネットワークの重み$\Omega=[\omega_{ij}]\in \mathbb{R}^{N\times N}$は 


\begin{equation}
\omega_{i j}=G \omega_{i j}^{0}+Q \boldsymbol{\eta}_{i}^\intercal \boldsymbol{\phi}_j 
\end{equation}


となっています。$\omega_{i j}^{0}$は固定された再帰重みです。$G, Q$は定数で、$\eta=[\boldsymbol{\eta}_{i}^\intercal]\in \mathbb{R}^{N\times N_\text{out}}$は$-1$か1に等確率に決められた行列です。よって学習するパラメータは$\phi$のみです。よってバイアスを抜いた入力電流$s_{i}$は次のように分割できます。


\begin{align}
s_{i}&=\sum_{j=1}^{N} \omega_{i j} r_{j}\\
&=\sum_{j=1}^{N} \left(G \omega_{i j}^{0}+Q \boldsymbol{\eta}_{i}^\intercal \boldsymbol{\phi}_j \right)r_{j}\\
&=Q\boldsymbol{\eta}_{i}^\intercal \hat{\boldsymbol{x}}(t)+\sum_{j=1}^{N} G \omega_{i j}^{0}r_{j}
\end{align}


\subsubsection{固定重みの初期化}
固定された結合重み$\omega_{i j}^{0}$は$\mathcal{N}(0, (Np)^{-1})$の正規分布からランダムサンプリングした値です($N$はニューロンの数、$p$は定数)。ただし、各ニューロンが投射される重みの平均が0になるようにスケーリングします。
\subsection{RLS法による重みの更新}
\footnote{ModelDBにおいて公開されているMATLABのコード(\url{https://senselab.med.yale.edu/ModelDB/ShowModel.cshtml?model=190565})を参考にしました。}
FORCE法は\textbf{RLSフィルタ}(recursive least squares filter, 再帰的最小二乗法フィルタ)という\textbf{適応フィルタ}(adaptive filter)の一種を学習するアルゴリズムを、RNNの学習に適応したものです\footnote{なお、(Sussillo \& Abbott, 2009)ではDelta則を用いることで、RLS法を用いない重みの更新則も紹介されています。}。
誤差を 


\begin{equation}
\boldsymbol{e}(t)=\hat{\boldsymbol{x}}(t)-\boldsymbol{x}(t)=\phi(t-\Delta t)^\intercal \boldsymbol{r}(t)-\boldsymbol{x}(t)    
\end{equation}


とした場合\footnote{実際にはこれは真の誤差ではなく、事前誤差(apriori error)と呼ばれるものです。真の誤差は$\phi(t)^\intercal \boldsymbol{r}(t)-\boldsymbol{x}(t)$と表されます。}、出力重み$\phi$を次の式で更新します。


\begin{align}
\phi(t)&=\phi(t-\Delta t)-P(t) \boldsymbol{r}(t)\boldsymbol{e}(t)^\intercal\\
P(t)&=P(t-\Delta t)-\frac{P(t-\Delta t) \boldsymbol{r}(t) \boldsymbol{r}(t)^\intercal P(t-\Delta t)}{1+\boldsymbol{r}(t)^\intercal P(t-\Delta
t) \boldsymbol{r}(t)} 
\end{align}


また、初期値は$\phi(0)=0,
P(0)=I_{N}\lambda^{-1}$です。$I_{N}$は$N$次の単位行列を意味します。$\lambda$は正則化のための定数です。

\subsubsection{FORCE法の実装}
それではFORCE法の実装をしてみましょう\footnote{コードは\texttt{./TrainingSNN/LIF\_FORCE\_sinewave.py}です。}。Reservoirネットワークは2000個のLIFニューロンで構成されているとします。また出力ユニットの個数は1です。まず、各種定数と教師信号を定義します。
\lstinputlisting[language=julia]{./text/solve-credit-assignment-problem/reservoir-computing/002.jl}
次にニューロンとシナプスを定義します。
\lstinputlisting[language=julia]{./text/solve-credit-assignment-problem/reservoir-computing/004.jl}
シナプスのインスタンスとして\texttt{synapses\_out, synapses\_rec}があります。実は\texttt{synapses\_out}だけでも良いのですが、高速化のために2つ用意しています。また、\texttt{OMEGA}はランダムに生成した後にスケーリングをしています。次に各種変数の初期化と、記録用変数を定義します。
\lstinputlisting[language=julia]{./text/solve-credit-assignment-problem/reservoir-computing/006.jl}
それではシミュレーションのメインの部分を書いていきましょう。
\lstinputlisting[language=julia]{./text/solve-credit-assignment-problem/reservoir-computing/008.jl}
途中で少し不思議に思われるようなことをしています。\\
\colorbox{shadecolor}{\texttt{PSC = synapses\_rec(JD*(len\_idx>0))}}の部分(とその少し上)ですが、これはデコードに用いる\texttt{r}を行列変換するよりも発火した結合重みの和を取り、再帰入力のシナプス後細胞のモデルに入力した方が速いという理由によります。\texttt{t}が一定のステップの範囲にある場合はFORCE法により学習を実行します。最後に各種変数を記録しています。\par
それでは学習後の結果を表示しましょう。初めに発火数と発火率を表示し、次に学習前と学習後の5つのニューロンの膜電位、最後に学習前/中間と学習後のデコード結果を描画します(なお、この本に記載はしていないですがコードには重みの固有値の描画も付けています)。
\lstinputlisting[language=julia]{./text/solve-credit-assignment-problem/reservoir-computing/010.jl}
結果は図\ref{fig:LIF_FORCE_1}, \ref{fig:LIF_FORCE_2}のようになります。また、同様のシミュレーションをIzhikevichニューロンで行った\footnote{コードは\texttt{./TrainingSNN/Izhikevich\_FORCE\_sinewave.py}です。}結果も示しています。
\begin{figure}[htbp]
    \centering
    \includegraphics[scale=0.4]{figs/LIF_FORCE_prepost.pdf}
    \includegraphics[scale=0.4]{figs/Iz_FORCE_prepost.pdf}
    \caption{FORCE法による学習前(左)と学習後(右)の発火率の変化。(上)LIFニューロン、(下)をIzhikevichニューロン}
    \label{fig:LIF_FORCE_1}
\end{figure}
\begin{figure}[H]
    \centering
    \includegraphics[scale=0.4]{figs/LIF_FORCE_decoded.pdf}
    \caption{FORCE法による学習前(左)と学習後(右)のデコード結果の変化(LIFニューロンの場合)。教師信号は実線、デコード結果は破線で示している。}
    \label{fig:LIF_FORCE_2}
\end{figure}
\subsection{鳥の鳴き声の再現と海馬の記憶と再生}
(Nicola \& Clopath, 2017)では教師信号として正弦波以外にもVan der Pol方程式やLorenz方程式の軌道を用いて実験しています。さらに教師信号としてベートーヴェンの歓喜の歌(Ode to joy)や鳥の鳴き声を用いても学習可能であったようです。\par
話は少しずれますが、小鳥の運動前野である\textbf{HVC}には連鎖的に結合したニューロン群が存在します。これはリズムを生み出すための計時に関わっているといわれています。カナリアのHVCニューロンを実験的に損傷(ablation)させると歌が歌えなくなるという実験がありますが、同様にSNNのHVCパターンをablationすると学習した歌が再生できなくなったようです。このような計時に関わるパターンを\textbf{HDTS}(high-dimentional temporal signal)とNicolaらは呼んでいます。HDTSを学習させた後に歓喜の歌を学習させると、HDTSがない場合よりも短い時間かつ高精度で学習できたようです。\par
さらにHDTSを外部入力とし、同時に映像を学習させる、という実験もしています(HDTSを内的に学習させる場合も行っています)。ネットワークは記録した映像を実時間で再生することができましたが、外部信号のHDTSを加速させることで圧縮再生が可能だったそうです。さらにHDTSを逆にすると、逆再生もできたそうです。\par
ニューロンの発火のタスク依存的な圧縮は実験的に観察されています(例えばEuston, et al., 2007)。空間的な課題(箱の中に入れて探索させるなど)をラットにさせると、課題中に記憶された場所細胞の順序だった活動は、ラットの睡眠中に圧縮再生されるという実験結果があります。その圧縮比は5.4〜8.1だったそうですが、この比率はSNNが映像を大きな損失なく再生できる圧縮比とほぼ同じであったようです。Nicolaらはさらに進んでSNNを用いて海馬における急速圧縮学習の機構における介在細胞の働きについての研究も行っています(Nicola \& Clopath, 2019)。
\section{RLS法の導出}
ここからはRLS法の導出を行います(cf. Haykin, 2002)。RLS法では次の損失関数$C\in \mathbb{R}^{N_\text{out}}$を最小化するような重み$\phi=[\boldsymbol{\phi}_j]\in \mathbb{R}^{N\times N_\text{out}}$を求めます。シミュレーション時間を$T$とすると、$C$は
\begin{equation}
C=\int_{0}^T(\hat{\boldsymbol{x}}(t)-\boldsymbol{x}(t))^{2} \mathrm{d} t+\lambda \phi^\intercal \phi
\end{equation}
です。ただし、$\hat{\boldsymbol{x}}(t), \boldsymbol{x}(t) \in \mathbb{R}^{N_\text{out}}$です。\par
さて、式の$C$を最小化するような$\phi$を数値的に求めるためには、損失関数の近似が必要です。まず、
時間幅$\Delta t$で$C$を離散化します。さらに$n$ステップ目における重み$\phi(n)$により、$\hat{\boldsymbol{x}}(i)\simeq \phi(n)^\intercal \boldsymbol{r}(i)$と近似します。このとき、$n$ステップ目の損失関数$C(n)$は
\begin{align}
C(n)&\simeq \sum_{i=0}^{n}(\hat{\boldsymbol{x}}(i)-\boldsymbol{x}(i))^{2}+\lambda \phi(n)^\intercal \phi(n)\\     
&\simeq \sum_{i=0}^{n}(\phi(n)^\intercal \boldsymbol{r}(i)-\boldsymbol{x}(i))^{2}+\lambda \phi(n)^\intercal \phi(n)
\end{align}
となります。ここでL2正則化(ridge)付きの(通常の)最小二乗法の\textbf{正規方程式}(normal equation)により、$C(n)$を最小化する$\phi(n)$は
\begin{align}
\phi(n) &= \left[\sum_{i=0}^{n}(\boldsymbol{r}(i)\boldsymbol{r}(i)^\intercal+\lambda I_N)\right]^{-1}\left[\sum_{i=0}^{n}\boldsymbol{r}(i)\boldsymbol{x}(i)^\intercal\right]\\
&=P(n)\psi(n)
\end{align}
となります\footnote{重み$\phi$で$C$を微分し、勾配が0となるときの方程式の解です。}。ただし、
\begin{align}
P(n)^{-1}&= \sum_{i=0}^{n}(\boldsymbol{r}(i)\boldsymbol{r}(i)^\intercal+\lambda I_N)\ \left(=\int_{0}^T \boldsymbol{r}(t) \boldsymbol{r}(t)^\intercal \mathrm{d} t+\lambda I_{N}\right)\\
\psi(n)&=\sum_{i=0}^{n}\boldsymbol{r}(i)\boldsymbol{x}(i)^\intercal
\end{align}
です。$P(n)$は$\boldsymbol{r}(n)$の相関行列の時間積分と係数倍した単位行列の和の逆行列となっています。また、
\begin{equation}
P(n)^{-1}=P(n-1)^{-1}+\boldsymbol{r}(n) \boldsymbol{r}(n)^\intercal
\end{equation}
となります。ここで、\textbf{逆行列の補助定理}(Matrix Inversion Lemma, またはSherman-Morrison-Woodbury Identity)より、
\begin{align}
X&=A+BCD\\
\Rightarrow X^{-1}&=A^{-1} - A^{-1}B(C^{-1}+DA^{-1}B)^{-1}DA^{-1}
\end{align}
となるので、$X={P}(n)^{-1}, A=P(n-1)^{-1}, B= \boldsymbol{r}(n), C=I_{N}, D=\boldsymbol{r}(n)^\intercal$とすると、
\begin{align}
P(n)&=P(n-1)-\frac{P(n-1) \boldsymbol{r}(n) \boldsymbol{r}(n)^\intercal P(n-1)}{1+\boldsymbol{r}(n)^\intercal P(n-1) \boldsymbol{r}(n)} 
\end{align}
が成り立ちます(右辺2項目の分母はスカラーとなります)。
さらに
\begin{align}
\psi(n)&=\psi(n-1)+\boldsymbol{r}(n)\boldsymbol{x}(n)^\intercal\\
&=P(n-1)^{-1}\phi(n-1)+\boldsymbol{r}(n)\boldsymbol{x}(n)^\intercal\\
&=\left\{P(n)^{-1}-\boldsymbol{r}(n) \boldsymbol{r}(n)^\intercal\right\}\phi(n-1)+\boldsymbol{r}(n)\boldsymbol{x}(n)^\intercal
\end{align}
となります。式(6.22)から式(6.23)へは
\begin{equation}
\phi(n)=P(n)\psi(n) \Rightarrow \psi(n)=P(n)^{-1}\phi(n)
\end{equation}
であること、式(6.23)から式(6.24)へは式(6.18)により、
\begin{equation}
P(n-1)^{-1}=P(n)^{-1}-\boldsymbol{r}(n) \boldsymbol{r}(n)^\intercal
\end{equation}
であることを用いています。よって、
\begin{align}
\phi(n)&=P(n)\psi(n)\notag\\
&=P(n)\left[\left\{P(n)^{-1}-\boldsymbol{r}(n) \boldsymbol{r}(n)^\intercal\right\}\phi(n-1)+\boldsymbol{r}(n)\boldsymbol{x}(n)^\intercal\right]\notag\\
&=\phi(n-1)-P(n)\boldsymbol{r}(n)\boldsymbol{r}(n)^\intercal\phi(n-1)+P(n)\boldsymbol{r}(n)\boldsymbol{x}(n)^\intercal\notag\\
&=\phi(n-1)-P(n)\boldsymbol{r}(n)\left[\boldsymbol{r}(n)^\intercal\phi(n-1)-\boldsymbol{x}(n)^\intercal\right]\notag\\
&=\phi(n-1)-P(n)\boldsymbol{r}(n)\boldsymbol{e}(n)^\intercal
\end{align}
となります。式(6.22)と式(6.27)を連続時間での表記法にすると、式(6. 9,10)の更新式となります。
\subsection{RLSフィルタのアルゴリズム}
\footnote{ModelDBにおいて公開されているMATLABのコード(\url{https://senselab.med.yale.edu/ModelDB/ShowModel.cshtml?model=190565})を参考にしました。}
FORCE法は\textbf{RLSフィルタ}(recursive least squares filter, 再帰的最小二乗法フィルタ)という\textbf{適応フィルタ}(adaptive filter)の一種を学習するアルゴリズムを、RNNの学習に適応したものです。
誤差を 
\begin{equation}
\boldsymbol{e}(t)=\hat{\boldsymbol{x}}(t)-\boldsymbol{x}(t)=\phi(t-\Delta t)^\intercal \boldsymbol{r}(t)-\boldsymbol{x}(t)    
\end{equation}
とした場合\footnote{実際にはこれは真の誤差ではなく、事前誤差(apriori error)と呼ばれるものです。真の誤差は$\phi(t)^\intercal \boldsymbol{r}(t)-\boldsymbol{x}(t)$と表されます。}、出力重み$\phi$を次の式で更新します。
\begin{align}
\phi(t)&=\phi(t-\Delta t)-\boldsymbol{P}(t) \boldsymbol{r}(t)\boldsymbol{e}(t)^\intercal\\
\boldsymbol{P}(t)&=\boldsymbol{P}(t-\Delta t)-\frac{\boldsymbol{P}(t-\Delta t) \boldsymbol{r}(t) \boldsymbol{r}(t)^\intercal \boldsymbol{P}(t-\Delta t)}{1+\boldsymbol{r}(t)^\intercal \boldsymbol{P}(t-\Delta
t) \boldsymbol{r}(t)} 
\end{align}
ここで$^\intercal$を転置記号とし、$\boldsymbol{x}$を列ベクトル、$\boldsymbol{x}^\intercal$を行ベクトルとします。また、初期値は$\phi(0)=0,
\boldsymbol{P}(0)=I_{N}\lambda^{-1}$です。$I_{N}$は$N$次の単位行列を意味します。$\lambda$は正則化のための定数です。
\section{RLSフィルタの導出}
ここからはRLSフィルタの導出を行います。まずReservoirニューロンの数を$N$とし、出力の数を$N_\text{out}$とします。RLSフィルタでは次の損失関数$C\in \mathbb{R}^{N_\text{out}}$を最小化するような重み$\phi=[\phi_j]\in \mathbb{R}^{N\times N_\text{out}}$を求めます。シミュレーション時間を$T$とすると、$C$は
\begin{equation}
C=\int_{0}^T(\hat{\boldsymbol{x}}(t)-\boldsymbol{x}(t))^{2} \mathrm{d} t+\lambda \phi^\intercal \phi
\end{equation}
です。ただし、$\hat{\boldsymbol{x}}(t), \boldsymbol{x}(t) \in \mathbb{R}^{N_\text{out}}$です。\par
さて、式の$C$を最小化するような$\phi$を数値的に求めるためには、損失関数の近似が必要です。まず、
時間幅$\Delta t$で離散化したステップ数を$n=T/\Delta t$とし、$C$を離散化します。さらに$n$ステップ目における重み$\phi(n)$により、$\hat{\boldsymbol{x}}(i)\simeq \phi(n)^\intercal \boldsymbol{r}(i)$と近似します。このとき、$n$ステップ目の損失関数$C(n)$は
\begin{align}
C(n)&\simeq \sum_{i=0}^{n}(\hat{\boldsymbol{x}}(i)-\boldsymbol{x}(i))^{2}+\lambda \phi(n)^\intercal \phi(n)\\     
&\simeq \sum_{i=0}^{n}(\phi(n)^\intercal \boldsymbol{r}(i)-\boldsymbol{x}(i))^{2}+\lambda \phi(n)^\intercal \phi(n)
\end{align}
となります。ここでL2正則化(ridge)付きの(通常の)最小二乗法の\textbf{正規方程式}(normal equation)により、$C(n)$を最小化する$\phi(n)$は
\begin{align}
\phi(n) &= \left[\sum_{i=0}^{n}(\boldsymbol{r}(i)\boldsymbol{r}(i)^\intercal+\lambda I_N)\right]^{-1}\left[\sum_{i=0}^{n}\boldsymbol{r}(i)\boldsymbol{x}(i)^\intercal\right]\\
&=P(n)\psi(n)
\end{align}
となります\footnote{重み$\phi$で$C$を微分し、勾配が0となるときの方程式の解です。}。ただし、
\begin{align}
P(n)^{-1}&= \sum_{i=0}^{n}(\boldsymbol{r}(i)\boldsymbol{r}(i)^\intercal+\lambda I_N)\ \left(=\int_{0}^T \boldsymbol{r}(t) \boldsymbol{r}(t)^\intercal \mathrm{d} t+\lambda I_{N}\right)\\
\psi(n)&=\sum_{i=0}^{n}\boldsymbol{r}(i)\boldsymbol{x}(i)^\intercal
\end{align}
です。$\boldsymbol{P}(n)$は$\boldsymbol{r}(n)$の相関行列の時間積分と係数倍した単位行列の和の逆行列となっています。また、
\begin{equation}
P(n)^{-1}=P(n-1)^{-1}+\boldsymbol{r}(n) \boldsymbol{r}(n)^\intercal
\end{equation}
となります。ここで、\textbf{逆行列の補助定理}(Matrix Inversion Lemma, またはSherman-Morrison-Woodbury Identity)より、
\begin{align}
X&=A+BCD\\
\Rightarrow X^{-1}&=A^{-1} - A^{-1}B(C^{-1}+DA^{-1}B)^{-1}DA^{-1}
\end{align}
となるので、$X={P}(n)^{-1}, A=\boldsymbol{P}(n-1)^{-1}, B= \boldsymbol{r}(n), C=I_{N}, D=\boldsymbol{r}(n)^\intercal$とすると、
\begin{align}
\boldsymbol{P}(n)&=\boldsymbol{P}(n-1)-\frac{\boldsymbol{P}(n-1) \boldsymbol{r}(n) \boldsymbol{r}(n)^\intercal \boldsymbol{P}(n-1)}{1+\boldsymbol{r}(n)^\intercal \boldsymbol{P}(n-1) \boldsymbol{r}(n)} 
\end{align}
が成り立ちます(右辺2項目の分母はスカラーとなります)。
さらに
\begin{align}
\psi(n)&=\psi(n-1)+\boldsymbol{r}(n)\boldsymbol{x}(n)^\intercal\\
&=P(n-1)^{-1}\phi(n-1)+\boldsymbol{r}(n)\boldsymbol{x}(n)^\intercal\\
&=\left\{P(n)^{-1}-\boldsymbol{r}(n) \boldsymbol{r}(n)^\intercal\right\}\phi(n-1)+\boldsymbol{r}(n)\boldsymbol{x}(n)^\intercal
\end{align}
となります。式から式へは
%%ここ式の番号入れる
\begin{equation}
\phi(n)=P(n)\psi(n) \Rightarrow \psi(n)=P(n)^{-1}\phi(n)
\end{equation}
であること、式から式へは式により、
\begin{equation}
P(n-1)^{-1}=P(n)^{-1}-\boldsymbol{r}(n) \boldsymbol{r}(n)^\intercal
\end{equation}
であることを用いています。よって、
\begin{align}
\phi(n)&=P(n)\psi(n)\\
&=P(n)\left[\left\{P(n)^{-1}-\boldsymbol{r}(n) \boldsymbol{r}(n)^\intercal\right\}\phi(n-1)+\boldsymbol{r}(n)\boldsymbol{x}(n)^\intercal\right]\\
&=\phi(n-1)-P(n)\boldsymbol{r}(n)\boldsymbol{r}(n)^\intercal\phi(n-1)+P(n)\boldsymbol{r}(n)\boldsymbol{x}(n)^\intercal\\
&=\phi(n-1)-P(n)\boldsymbol{r}(n)\left[\boldsymbol{r}(n)^\intercal\phi(n-1)-\boldsymbol{x}(n)^\intercal\right]\\
&=\phi(n-1)-P(n)\boldsymbol{r}(n)\boldsymbol{e}(n)^\intercal
\end{align}
となります。式と式を連続時間での表記法にすると、前節における式と式の更新式となります。


\chapter{運動制御}
\section{躍度最小モデル}
躍度最小モデル (minimum-jerk model; \cite{Flash1985-vj})を実装する.解析的に求まるが以下では二次計画法を用いて数値的に求める.

\subsection{等式制約下の二次計画法 (Equality Constrained Quadratic Programming)}

$n$個の変数があり,$m$個の制約条件がある等式制約二次計画問題を考える.$\mathbf {x}\in \mathbb{R}^n$, 対称行列$\mathbf{P}\in \mathbb{R}^{n\times n}$,  $\mathbf {q}\in \mathbb{R}^{n}$, $\mathbf{A}\in \mathbb{R}^{m\times n}$, $\mathbf {b}\in \mathbb{R}^m$.このとき,問題は次のようになる.


\begin{align}
&{\text{Minimize}}\quad {\frac {1}{2}}\mathbf {x}^\top \mathbf{P}\mathbf {x} +\mathbf {q} ^{\top}\mathbf {x}\\
&{\text{subject to}}\quad \mathbf{A}\mathbf {x} =\mathbf {b}
\end{align}


Lagrangeの未定乗数法を用いると解は


\begin{equation}
{\begin{bmatrix}\mathbf{P}&\mathbf{A}^\top\\\mathbf{A}&0\end{bmatrix}}{\begin{bmatrix}\mathbf {x} \\
\lambda \end{bmatrix}}={\begin{bmatrix}-\mathbf {q} \\\mathbf {b} \end{bmatrix}}
\end{equation}


の解として与えられる.ここで $\lambda \in \mathbb{R}^{m}$  はLagrange乗数のベクトルである.
\lstinputlisting[language=julia]{./text/motor-learning/minimum-jerk/001.jl}
\lstinputlisting[language=julia]{./text/motor-learning/minimum-jerk/002.jl}
ちなみにjuliaでは $Ax=b$の解を出すとき,\jl{x=A^-1 * b}よりも\jl{x=A \ b}とした方がよい.
\lstinputlisting[language=julia]{./text/motor-learning/minimum-jerk/004.jl}
\subsection{躍度最小モデルの実装}
1次元における運動を考えよう.この仮定ではサッカードするときの眼球運動などが当てはまる.以下では \cite{Yazdani2012-sx} での問題設定を用いる.Toeplitz行列を用いた実装はYazdaniらのPythonでcvxoptを用いた実装を参考にして作成した.

問題設定は以下のようにする.


\begin{align}
&\underset{u(t)}{\operatorname{minimize}}\quad \|u(t)\|_2 \\
&\text{subject to} \quad \dot{\mathbf{x}}(t)=A \mathbf{x}(t)+B u(t)
\end{align}


ただし,$\|\cdot\|_{2}$は$L_{2}$ノルムを意味し,$A=\left[\begin{array}{lll}0 & 1 & 0 \\ 0 & 0 & 1 \\ 0 & 0 & 0\end{array}\right], B=\left[\begin{array}{l}0 \\ 0 \\ 1\end{array}\right], \mathbf{x}(t)=\left[\begin{array}{l}x(t) \\ \dot{x}(t) \\ \ddot{x}(t)\end{array}\right], u(t)=\dddot x(t)$とする.すなわち,制御信号$u(t)$は躍度$\dddot x(t)$と等しいとする.
\lstinputlisting[language=julia]{./text/motor-learning/minimum-jerk/006.jl}
\lstinputlisting[language=julia]{./text/motor-learning/minimum-jerk/007.jl}
実際には\jl{D_jerk}には\jl{(1/dt)^3}を乗じるべきであるが,二次計画法の数値的な安定性のために結果の描画の際にのみ乗じる.
\lstinputlisting[language=julia]{./text/motor-learning/minimum-jerk/009.jl}
二次計画法を解く.
\lstinputlisting[language=julia]{./text/motor-learning/minimum-jerk/011.jl}
位置解を速度,加速度,躍度に変換する.
\lstinputlisting[language=julia]{./text/motor-learning/minimum-jerk/013.jl}
結果を描画する.
\lstinputlisting[language=julia]{./text/motor-learning/minimum-jerk/015.jl}
\begin{figure}[ht]
	\centering
	\includegraphics[scale=0.8, max width=\linewidth]{./fig/neuron-model/isi/cell015.png}
	\caption{cell015.png}
	\label{cell015.png}
\end{figure}
\subsection{経由点を通る場合}
経由点問題(via-point problem)を考える.
\lstinputlisting[language=julia]{./text/motor-learning/minimum-jerk/017.jl}
\lstinputlisting[language=julia]{./text/motor-learning/minimum-jerk/018.jl}
\lstinputlisting[language=julia]{./text/motor-learning/minimum-jerk/019.jl}
\begin{figure}[ht]
	\centering
	\includegraphics[scale=0.8, max width=\linewidth]{./fig/energy-based-model/sparse-coding/cell019.png}
	\caption{cell019.png}
	\label{cell019.png}
\end{figure}
 % ok
\section{終点誤差分散最小モデル}
\textbf{終点誤差分散最小モデル} (minimum-variance model; \cite{Harris1998-gj})を実装する.

$\mathbf{A}\in \mathbb{R}^{n\times n}$, $\mathbf{B}\in \mathbb{R}^{n}$とする.$\dot{x}=\mathbf{A}_{c}\mathbf{x}+\mathbf{B}_{c}(u + w)$について,差分化すると


\begin{align}
\mathbf{x}(t+dt)&=\mathbf{x}(t)+\dot{\mathbf{x}}dt\\
\mathbf{x}_{t+1}&=\mathbf{I}\mathbf{x}_t+(\mathbf{A}_{c}dt)\mathbf{x}_t+(\mathbf{B}_{c}dt)(u + w)
\end{align}


となる(ここで$\mathbf{I}$は単位行列)ので,$\mathbf{A}=\mathbf{I}+\mathbf{A}_{c}dt, \mathbf{B}=\mathbf{B}_cdt$として


\begin{equation}
\mathbf{x}_{t+1} = \mathbf{A} \mathbf{x}_t + \mathbf{B}(u_t + w_t)
\end{equation}


と表せる. $\mathbf{x}_t$の平均は


\begin{equation}
\mathbb{E}\left[\mathbf{x}_{t}\right]=\mathbf{A}^{t} \mathbf{x}_{0}+\sum_{i=0}^{t-1} \mathbf{A}^{t-1-i} \mathbf{B} u_{i}
\end{equation}


$\mathbf{x}_t$の分散は


\begin{equation}
\operatorname{Cov}\left[\mathbf{x}_{t}\right]=k \sum_{i=0}^{t-1}\left(\mathbf{A}^{t-1-i} \mathbf{B}\right)\left(\mathbf{A}^{t-1-i} \mathbf{B}\right)^{\top} u_{i}^{2}
\end{equation}


となる.
\subsection{終点誤差分散最小モデルの実装}
以下では田中先生の\url{https://www.motorcontrol.jp/archives/?MC13}のコードを参考に作成した.
\lstinputlisting[language=julia]{./text/motor-learning/minimum-variance/002.jl}
\lstinputlisting[language=julia]{./text/motor-learning/minimum-variance/003.jl}
\lstinputlisting[language=julia]{./text/motor-learning/minimum-variance/004.jl}
\lstinputlisting[language=julia]{./text/motor-learning/minimum-variance/005.jl}
\lstinputlisting[language=julia]{./text/motor-learning/minimum-variance/006.jl}
結果の描画.
\lstinputlisting[language=julia]{./text/motor-learning/minimum-variance/008.jl}
\begin{figure}[ht]
	\centering
	\includegraphics[scale=0.8, max width=\linewidth]{./fig/appendix/slow-feature-analysis/cell008.png}
	\caption{cell008.png}
	\label{cell008.png}
\end{figure}
 % ok
\section{最適フィードバック制御モデル}
ToDo: infiniteOFCと数式の統一を行う.
\subsection{最適フィードバック制御モデルの構造}
\textbf{最適フィードバック制御モデル (optimal feedback control; OFC)}\index{さいてきふぃーどばっくせいぎょもでる (optimal feedback control; OFC)@最適フィードバック制御モデル (optimal feedback control; OFC)} の特徴として目標軌道を必要としないことが挙げられる.\textbf{Kalman フィルタ}\index{Kalman ふぃるた@Kalman フィルタ}による状態推定と\textbf{線形2次レギュレーター(LQR: linear-quadratic regurator)}\index{せんけい2つぎれぎゅれーたー(LQR: linear-quadratic regurator)@線形2次レギュレーター(LQR: linear-quadratic regurator)} により推定された状態に基づいて運動指令を生成という2つの流れが基本となる.
\subsubsection{系の状態変化}
\begin{align}
&\text {Dynamics} \quad \mathbf{x}_{t+1}=A \mathbf{x}_{t}+B \mathbf{u}_{t}+\boldsymbol{\xi}_{t}+\sum_{i=1}^{c} \varepsilon_{t}^{i} C_{i} \mathbf{u}_{t}\\
&\text {Feedback} \quad \mathbf{y}_{t}=H \mathbf{x}_{t}+\omega_{t}+\sum_{i=1}^{d} \epsilon_{t}^{i} D_{i} \mathbf{x}_{t}\\
&\text{Cost per step}\quad \mathbf{x}_{t}^\top Q_{t} \mathbf{x}_{t}+\mathbf{u}_{t}^\top R \mathbf{u}_{t}
\end{align}
\subsubsection{LQG}
加法ノイズしかない場合($C=D=0$),制御問題は\textbf{線形2次ガウシアン(LQG: linear-quadratic-Gaussian)制御}\index{せんけい2つぎがうしあん(LQG: linear-quadratic-Gaussian)せいぎょ@線形2次ガウシアン(LQG: linear-quadratic-Gaussian)制御}と呼ばれる.
\paragraph{運動制御 (Linear-Quadratic Regulator)}
\begin{align}
\mathbf{u}_{t}&=-L_{t} \widehat{\mathbf{x}}_{t}\\
L_{t}&=\left(R+B^{\top} S_{t+1} B\right)^{-1} B^{\top} S_{t+1} A\\
S_{t}&=Q_{t}+A^{\top} S_{t+1}\left(A-B L_{t}\right)\\
s_t &= \mathrm{tr}(S_{t+1}\Omega^\xi) + s_{t+1}; s_T=0
\end{align}
$\boldsymbol{S}_{T}=Q$
\paragraph{状態推定 (Kalman Filter)}
\begin{align}
\widehat{\mathbf{x}}_{t+1}&=A \widehat{\mathbf{x}}_{t}+B \mathbf{u}_{t}+K_{t}\left(\mathbf{y}_{t}-H \widehat{\mathbf{x}}_{t}\right)+\boldsymbol{\eta}_{t} \\ 
K_{t}&=A \Sigma_{t} H^{\top}\left(H \Sigma_{t} H^{\top}+\Omega^{\omega}\right)^{-1} \\ 
\Sigma_{t+1}&=\Omega^{\xi}+\left(A-K_{t} H\right) \Sigma_{t} A^{\top}
\end{align}
この場合に限り,運動制御と状態推定を独立させることができる.
\subsubsection{一般化LQG}
状態および制御依存ノイズがある場合,
\subsection{実装}
ライブラリの読み込みと関数の定義.
\begin{lstlisting}[language=julia]
using Base: @kwdef
using Parameters: @unpack
using LinearAlgebra, Kronecker, Random, BlockDiagonals, PyPlot
rc("axes.spines", top=false, right=false)
rc("font", family="Arial") 
\end{lstlisting}
ToDo: struct 修正 (nが両方に入っている) 
\begin{lstlisting}[language=julia]
@kwdef struct Reaching1DModelParameter
    n = 4 # number of dims
    p = 3 # 
    i = 0.25 # kgm^2, 
    b = 0.2 # kgm^2/s
    ta = 0.03 # s
    te = 0.04 # s
    L0 = 0.35 # m

    bu = 1 / (ta * te * i)
    α1 = bu * b
    α2 = 1/(ta * te) + (1/ta + 1/te) * b/i
    α3 = b/i + 1/ta + 1/te

    A = [zeros(p) I(p); -[0, α1, α2, α3]']
    B = [zeros(p); bu]
    C = [I(p) zeros(p)]
    D = Diagonal([1e-3, 1e-2, 5e-2])

    Y = 0.02 * B
    G = 0.03 * I(n)
end

@kwdef struct Reaching1DModelCostParameter
    n = 4
    dt = 1e-2 # sec
    T = 0.5 # sec
    nt = round(Int, T/dt) # num time steps
    Q = [zeros(nt-1, n, n); reshape(Diagonal([1.0, 0.1, 1e-3, 1e-4]), (1, n, n))]
    R = 1e-4 / nt
    
    init_pos = -0.5
    x₁ = [init_pos; zeros(n-1)]#zeros(n)
    Σ₁ = zeros(n, n)
end
\end{lstlisting}
Qの値は各時刻において一般座標 (位置,速度,加速度,躍度)のそれぞれを0にするコストに対する重みづけである.例えば,速度も0にすることを重視すれば2番目の係数を上げる.
$S$と$\Sigma$は各時点での値を一時的にしか必要としないので更新する.
\begin{lstlisting}[language=julia]
function LQG(param::Reaching1DModelParameter, cost_param::Reaching1DModelCostParameter; discrete=true)
    @unpack n, p, A, B, C, D, G = param
    @unpack Q, R, x₁, Σ₁, dt, nt = cost_param

    if discrete
        A = I + A * dt
        B = B * dt
        C = C * dt
        D = sqrt(dt) * D
        G = sqrt(dt) * G
    end
    
    L = zeros(nt-1, n) # Feedback gains
    K = zeros(nt-1, n, p) # Kalman gains
    S = copy(Q[end, :, :]) # S_T = Q
    Σ = copy(Σ₁);

    for t in 1:nt-1
        K[t, :, :] = A * Σ * C' / (C * Σ * C' + D) # update K
        Σ = G + (A - K[t, :, :] * C) * Σ * A'      # update Σ
    end 

    cost = 0
    for t in nt-1:-1:1
        cost += tr(S * G)
        L[t, :] = (R + B' * S * B) \ B' * S * A # update L
        S = Q[t, :, :] + A' * S * (A - B * L[t, :]')     # update S
    end
    
    # adjust cost
    cost += x₁' * S * x₁
    return L, K, cost
end
\end{lstlisting}
\subsubsection{シミュレーション}
信号依存ノイズ Yが入っている場合はLQGとは異なってくる.
\begin{align}
&\mathbf{u}_{t}=-L_{t} \hat{\mathbf{x}}_{t} \\
&L_{t}=\left(B^\top S_{t+1}^{\mathbf{x}} B+R+\sum_{n} C_{n}^\top\left(S_{t+1}^{\mathbf{x}}+S_{t+1}^{\mathrm{e}}\right) C_{n}\right)^{-1} B^\top S_{t+1}^{\mathbf{x}} A \\
&S_{t}^{\mathbf{x}}=Q_{t}+A^\top S_{t+1}^{\mathbf{x}}\left(A-B L_{t}\right) ; \quad S_{T}^{\mathbf{x}}=Q_{T} \\
&S_{t}^{\mathrm{e}}=A^\top S_{t+1}^{\mathbf{x}} B L_t+\left(A-K_{t} H\right)^\top S_{t+1}^{\mathrm{e}}\left(A-K_{t} H\right) ; \quad S_{T}^{\mathrm{e}}=0\\
&s_{t}=\operatorname{tr}\left(S_{t+1}^{\mathrm{x}}\Omega^{\xi}+S_{t+1}^{\mathrm{e}}\left(\Omega^{\xi}+\Omega^{\eta}+K_{t} \Omega^{\omega} K_{t}^{\top}\right)\right)+s_{t+1} ; \quad s_{n}=0 .
\end{align}
\begin{align}
\hat{\mathbf{x}}_{t+1} &=A \hat{\mathbf{x}}_{t}+B \mathbf{u}_{t}+K_{t}\left(\mathbf{y}_{t}-H \hat{\mathbf{x}}_{t}\right) \\
K_{t} &=A \Sigma_{t}^{\mathrm{e}} H^\top\left(H \Sigma_{t}^{\mathrm{e}} H^\top+\Omega^{\omega}\right)^{-1} \\
\Sigma_{t+1}^{\mathrm{e}} &=\left(A-K_{t} H\right) \Sigma_{t}^{\mathrm{e}} A^\top+\sum_{n} C_{n} L_{t} \Sigma_{t}^{\hat{x}} L_{t}^\top C_{n}^\top ; \quad \Sigma_{1}^{\mathrm{e}}=\Sigma_{1} \\
\Sigma_{t+1}^{\hat{\mathbf{x}}} &=K_{t} H \Sigma_{t}^{\mathrm{e}} A^\top+\left(A-B L_{t}\right) \Sigma_{t}^{\hat{\mathbf{x}}}\left(A-B L_{t}\right)^\top ; \quad \Sigma_{1}^{\hat{\mathbf{x}}}=\hat{\mathbf{x}}_{1} \hat{\mathbf{x}}_{1}^\top
\end{align}
\begin{lstlisting}[language=julia]
function gLQG(param::Reaching1DModelParameter, cost_param::Reaching1DModelCostParameter, maxiter=200, ϵ=1e-8)
    @unpack n, p, A, B, C, D, Y, G = param
    @unpack Q, R, x₁, Σ₁, dt, nt = cost_param

    A = I + A * dt
    B = B * dt
    C = C * dt
    D = sqrt(dt) * D
    G = sqrt(dt) * G
    Y = sqrt(dt) * Y
    
    L = zeros(nt-1, n) # Feedback gains
    K = zeros(nt-1, n, p) # Kalman gains
    
    cost = zeros(maxiter)
    for i in 1:maxiter
        Sˣ = copy(Q[end, :, :])
        Sᵉ = zeros(n, n)
        Σˣ̂ = x₁ * x₁' # \Sigma TAB \^x TAB \hat TAB
        Σᵉ = copy(Σ₁)
        
        for t in 1:nt-1
            K[t, :, :] = A * Σᵉ * C' / (C * Σᵉ * C' + D)

            AmBL = A - B * L[t, :]'
            LΣˣ̂L = L[t, :]' * Σˣ̂ * L[t, :]

            Σˣ̂ = K[t, :, :] * C * Σᵉ * A' + AmBL * Σˣ̂ * AmBL'
            Σᵉ = G + (A - K[t, :, :] * C) * Σᵉ * A' + Y * LΣˣ̂L * Y'
        end
        
        for t in nt-1:-1:1
            cost[i] += tr(Sˣ * G + Sᵉ * (G + K[t, :, :] * D * K[t, :, :]'))
            
            L[t, :] = (R + B' * Sˣ * B + Y' * (Sˣ + Sᵉ) * Y) \ B' * Sˣ * A

            AmKC = A - K[t, :, :] * C
            Sᵉ = A' * Sˣ * B * L[t, :]' + AmKC' * Sᵉ * AmKC
            Sˣ = Q[t, :, :] + A' * Sˣ * (A - B * L[t, :]')
        end
        
        # adjust cost
        cost[i] += x₁' * Sˣ * x₁ + tr((Sˣ + Sᵉ) * Σ₁)
        if i > 1 && abs(cost[i] - cost[i-1]) < ϵ
            cost = cost[1:i]
            break
        end
    end
    return L, K, cost
end
\end{lstlisting}
状態ノイズがある場合に関してはTodorovのMATLABコード \url{https://homes.cs.washington.edu/~todorov/software/gLQG.zip}を参照.
位置は目標位置を基準とする座標で表現し,位置が0になるように運動を行う.状態の中に標的位置を含めコストパラメータを修正することで初期位置を基準とする座標系での運動を記述できる.モデルに関してはTodorov2005を参照.
\begin{lstlisting}[language=julia]
function simulation(param::Reaching1DModelParameter, cost_param::Reaching1DModelCostParameter, 
                    L, K; noisy=false)
    @unpack n, p, A, B, C, D, Y, G = param
    @unpack Q, R, x₁, dt, nt = cost_param
    
    X = zeros(n, nt)
    u = zeros(nt)
    X[:, 1] = x₁ # m; initial position (target position is zero)

    if noisy
        sqrtdt = √dt
        X̂ = zeros(n, nt)
        X̂[1, 1] = X[1, 1]
        for t in 1:nt-1
            u[t] = -L[t, :]' * X̂[:, t]
            X[:, t+1] = X[:,t] + (A * X[:,t] + B * u[t]) * dt + sqrtdt * (Y * u[t] * randn() + G * randn(n))
            dy = C * X[:,t] * dt + D * sqrtdt * randn(n-1)
            X̂[:, t+1] = X̂[:,t] + (A * X̂[:,t] + B * u[t]) * dt + K[t, :, :] * (dy - C * X̂[:,t] * dt)
        end
    else
        for t in 1:nt-1
            u[t] = -L[t, :]' * X[:, t]
            X[:, t+1] = X[:, t] + (A * X[:, t] + B * u[t]) * dt
        end
    end
    return X, u
end
\end{lstlisting}
\begin{lstlisting}[language=julia]
function simulation_all(param, cost_param, L, K)
    Xa, ua = simulation(param, cost_param, L, K, noisy=false);
    
    # noisy
    nsim = 10
    XSimAll = []
    uSimAll = []
    for i in 1:nsim
        XSim, u = simulation(param, cost_param, L, K, noisy=true);
        push!(XSimAll, XSim)
        push!(uSimAll, u)
    end
    
    # visualization
    @unpack dt, T = cost_param
    tarray = collect(dt:dt:T)
    label = [L"Position ($m$)", L"Velocity ($m/s$)", L"Acceleration ($m/s^2$)", L"Jerk ($m/s^3$)"]

    fig, ax = subplots(1, 3, figsize=(10, 3))
    for i in 1:2
        for j in 1:nsim
            ax[i].plot(tarray, XSimAll[j][i,:]', "tab:gray", alpha=0.5)
        end

        ax[i].plot(tarray, Xa[i,:], "tab:red")
        ax[i].set_ylabel(label[i]); ax[i].set_xlabel(L"Time ($s$)"); 
        ax[i].set_xlim(0, T); ax[i].grid()
    end

    for j in 1:nsim
        ax[3].plot(tarray, uSimAll[j], "tab:gray", alpha=0.5)
    end
    ax[3].plot(tarray, ua, "tab:red")
    ax[3].set_ylabel(L"Control signal ($N\cdot m$)"); ax[3].set_xlabel(L"Time ($s$)"); 
    ax[3].set_xlim(0, T); ax[3].grid()

    tight_layout()
end
\end{lstlisting}
\begin{lstlisting}[language=julia]
param = Reaching1DModelParameter()
cost_param = Reaching1DModelCostParameter();
\end{lstlisting}
\begin{lstlisting}[language=julia]
L, K, cost = LQG(param, cost_param);
simulation_all(param, cost_param, L, K)
\end{lstlisting}
\begin{figure}[ht]
	\centering
	\includegraphics[scale=0.8, max width=\linewidth]{./fig/motor-learning/optimal-feedback-control/cell016.png}
	\caption{cell016.png}
	\label{cell016.png}
\end{figure}
\begin{lstlisting}[language=julia]
L, K, cost = gLQG(param, cost_param);
simulation_all(param, cost_param, L, K)
\end{lstlisting}
\begin{figure}[ht]
	\centering
	\includegraphics[scale=0.8, max width=\linewidth]{./fig/motor-learning/optimal-feedback-control/cell017.png}
	\caption{cell017.png}
	\label{cell017.png}
\end{figure}
 % ok
\section{無限時間最適フィードバック制御モデル}
\subsection{モデルの構造}
\textbf{無限時間最適フィードバック制御モデル}\index{むげんじかんさいてきふぃーどばっくせいぎょもでる@無限時間最適フィードバック制御モデル} (\textbf{infinite-horizon optimal feedback control model}\index{infinite-horizon optimal feedback control model}) \citep{Qian2013-zy}
\begin{align}
d x&=(\mathbf{A} x+\mathbf{B} u) dt +\mathbf{Y} u d \gamma+\mathbf{G} d \omega \\
d y&=\mathbf{C} x dt+\mathbf{D} d \xi\\
d \hat{x}&=(\mathbf{A} \hat{x}+\mathbf{B} u) dt+\mathbf{K}(dy-\mathbf{C} \hat{x} dt)
\end{align}
\subsection{実装}
ライブラリの読み込みと関数の定義.
\begin{lstlisting}[language=julia]
using Parameters: @unpack
using LinearAlgebra, Kronecker, Random, BlockDiagonals, PyPlot
rc("axes.spines", top=false, right=false)
rc("font", family="Arial") 
\end{lstlisting}
定数の定義
\begin{align}
\alpha_{1}&=\frac{b}{t_{a} t_{e} I},\quad \alpha_{2}=\frac{1}{t_{a} t_{e}}+\left(\frac{1}{t_{a}}+\frac{1}{t_{e}}\right) \frac{b}{I} \\
\alpha_{3}&=\frac{b}{I}+\frac{1}{t_{a}}+\frac{1}{t_{e}},\quad b_{u}=\frac{1}{t_{a} t_{e} I}
\end{align}
\begin{lstlisting}[language=julia]
@kwdef struct SaccadeModelParameter
    n = 4 # number of dims
    i = 0.25 # kgm^2, 
    b = 0.2 # kgm^2/s
    ta = 0.03 # s
    te = 0.04 # s
    L0 = 0.35 # m

    bu = 1 / (ta * te * i)
    α1 = bu * b
    α2 = 1/(ta * te) + (1/ta + 1/te) * b/i
    α3 = b/i + 1/ta + 1/te

    A = [zeros(3) I(3); -[0, α1, α2, α3]']
    B = [zeros(3); bu]
    C = [I(3) zeros(3)]
    D = Diagonal([1e-3, 1e-2, 5e-2])

    Y = 0.02 * B
    G = 0.03 * I(n)

    Q = Diagonal([1.0, 0.01, 0, 0]) 
    R = 0.0001
    U = Diagonal([1.0, 0.1, 0.01, 0])
end
\end{lstlisting}
\begin{align}
\mathbf{X}\triangleq\begin{bmatrix}
x \\
\tilde{x}
\end{bmatrix}, d \bar{\omega} \triangleq\begin{bmatrix}
d \omega \\
d \xi
\end{bmatrix}, \bar{\mathbf{A}} \triangleq\begin{bmatrix}
\mathbf{A}-\mathbf{B} \mathbf{L} & \mathbf{B} \mathbf{L} \\
\mathbf{0} & \mathbf{A}-\mathbf{K} \mathbf{C}
\end{bmatrix}\\
\bar{\mathbf{Y}} \triangleq\begin{bmatrix}
-\mathbf{Y} \mathbf{L} & \mathbf{Y} \mathbf{L} \\
-\mathbf{Y} \mathbf{L} & \mathbf{Y} \mathbf{L}
\end{bmatrix}, \bar{G} \triangleq\begin{bmatrix}
\mathbf{G} & \mathbf{0} \\
\mathbf{G} & -\mathbf{K} \mathbf{D}
\end{bmatrix}
\end{align}
とする.元論文では$F, \bar{F}$が定義されていたが,$F=0$とするため,以後の式から削除した.
\begin{align}
\mathbf{P} &\triangleq\begin{bmatrix}
\mathbf{P}_{11} & \mathbf{P}_{12} \\
\mathbf{P}_{12} & \mathbf{P}_{22}
\end{bmatrix} = \mathbb{E}\left[\mathbf{X} \mathbf{X}^\top\right] \\
\mathbf{V} &\triangleq\begin{bmatrix}
\mathbf{Q}+\mathbf{L}^\top R \mathbf{L} & -\mathbf{L}^\top R \mathbf{L} \\
-\mathbf{L}^\top R \mathbf{L} & \mathbf{L}^\top R \mathbf{L}+\mathbf{U}
\end{bmatrix}
\end{align}
aaa
\begin{align}
&K=\mathbf{P}_{22} \mathbf{C}^\top\left(\mathbf{D} \mathbf{D}^\top\right)^{-1} \\
&\mathbf{L}=\left(R+\mathbf{Y}^\top\left(\mathbf{S}_{11}+\mathbf{S}_{22}\right) \mathbf{Y}\right)^{-1} \mathbf{B}^\top \mathbf{S}_{11} \\
&\bar{\mathbf{A}}^\top \mathbf{S}+\mathbf{S} \bar{\mathbf{A}}+\bar{\mathbf{Y}}^\top \mathbf{S} \bar{\mathbf{Y}}+\mathbf{V}=0 \\
&\bar{\mathbf{A}} \mathbf{P}+\mathbf{P} \bar{\mathbf{A}}^\top+\bar{\mathbf{Y}} \mathbf{P} \bar{\mathbf{Y}}^\top+\bar{\mathbf{G}} \bar{\mathbf{G}}^\top=0
\end{align}
$\mathbf{A} = (a_{ij})$ を $m \times n$ 行列,$\mathbf{B} = (b_{kl})$ を $p \times q$ 行列とすると、それらのクロネッカー積 $\mathbf{A} \otimes \mathbf{B}$ は
\begin{equation}
\mathbf{A}\otimes \mathbf{B}={\begin{bmatrix}a_{11}\mathbf{B}&\cdots &a_{1n}\mathbf{B}\\\vdots &\ddots &\vdots \\a_{m1}\mathbf{B}&\cdots &a_{mn}\mathbf{B}\end{bmatrix}}
\end{equation}
で与えられる $mp \times nq$ 区分行列である.
Roth's column lemma (vec-trick) 
\begin{equation}
(\mathbf{B}^\top \otimes \mathbf{A})\text{vec}(\mathbf{X}) = \text{vec}(\mathbf{A}\mathbf{X}\mathbf{B})=\text{vec}(\mathbf{C})
\end{equation}
によりこれを解くと,
\begin{align}
\mathbf{S} &= -\text{vec}^{-1}\left(\left(\mathbf{I} \otimes \bar{\mathbf{A}}^\top + \bar{\mathbf{A}}^\top \otimes \mathbf{I} + \bar{\mathbf{Y}}^\top \otimes \bar{\mathbf{Y}}^\top\right)^{-1}\text{vec}(\mathbf{V})\right)\\
\mathbf{P} &= -\text{vec}^{-1}\left(\left(\mathbf{I} \otimes \bar{\mathbf{A}} + \bar{\mathbf{A}} \otimes \mathbf{I} + \bar{\mathbf{Y}} \otimes \bar{\mathbf{Y}}\right)^{-1}\text{vec}(\bar{\mathbf{G}}\bar{\mathbf{G}}^\top)\right)
\end{align}
となる.ここで$\mathbf{I}=\mathbf{I}^\top$を用いた.
\subsubsection{K, L, S, Pの計算}
K, L, S, Pの計算は次のようにする.
\begin{enumerate}
\item LとKをランダムに初期化
\item SとPを計算
\item LとKを更新
\item 収束するまで2と3を繰り返す.
\end{enumerate}
収束スピードはかなり速い.
\begin{lstlisting}[language=julia]
function infinite_horizon_ofc(param::SaccadeModelParameter, maxiter=1000, ϵ=1e-8)
    @unpack n, A, B, C, D, Y, G, Q, R, U = param
    
    # initialize
    L = rand(n)' # Feedback gains
    K = rand(n, 3) # Kalman gains
    I₂ₙ = I(2n)

    for _ in 1:maxiter
        Ā = [A-B*L B*L; zeros(size(A)) (A-K*C)]
        Ȳ = [-ones(2) ones(2)] ⊗ (Y*L) 
        Ḡ = [G zeros(size(K)); G (-K*D)]
        V = BlockDiagonal([Q, U]) + [1 -1; -1 1] ⊗ (L'* R * L)

        # update S, P
        S = -reshape((I₂ₙ ⊗ (Ā)' +  (Ā)' ⊗ I₂ₙ + (Ȳ)' ⊗ (Ȳ)') \ vec(V), (2n, 2n))
        P = -reshape((I₂ₙ ⊗ Ā +  Ā ⊗ I₂ₙ + Ȳ ⊗  Ȳ) \ vec(Ḡ * (Ḡ)'), (2n, 2n))

        # update K, L
        P₂₂ = P[n+1:2n, n+1:2n]
        S₁₁ = S[1:n, 1:n]
        S₂₂ = S[n+1:2n, n+1:2n]

        Kₜ₋₁ = copy(K)
        Lₜ₋₁ = copy(L)

        K = P₂₂ * C' / (D * D')
        L = (R + Y' * (S₁₁ + S₂₂) * Y) \ B' * S₁₁
        if sum(abs.(K - Kₜ₋₁)) < ϵ && sum(abs.(L - Lₜ₋₁)) < ϵ
            break
        end
    end
    return L, K
end
\end{lstlisting}
\begin{lstlisting}[language=julia]
param = SaccadeModelParameter()
L, K = infinite_horizon_ofc(param);
\end{lstlisting}
\subsubsection{シミュレーション}
関数を書く.
\begin{lstlisting}[language=julia]
function simulation(param::SaccadeModelParameter, L, K, dt=0.001, T=2.0, init_pos=-0.5; noisy=true)
    @unpack n, A, B, C, D, Y, G, Q, R, U = param
    nt = round(Int, T/dt)
    X = zeros(n, nt)
    u = zeros(nt)
    X[1, 1] = init_pos # m; initial position (target position is zero)

    if noisy
        sqrtdt = √dt
        X̂ = zeros(n, nt)
        X̂[1, 1] = X[1, 1]
        for t in 1:nt-1
            u[t] = -L * X̂[:, t]
            X[:, t+1] = X[:,t] + (A * X[:,t] + B * u[t]) * dt + sqrtdt * (Y * u[t] * randn() + G * randn(n))
            dy = C * X[:,t] * dt + D * sqrtdt * randn(n-1)
            X̂[:, t+1] = X̂[:,t] + (A * X̂[:,t] + B * u[t]) * dt + K * (dy - C * X̂[:,t] * dt)
        end
    else
        for t in 1:nt-1
            u[t] = -L * X[:, t]
            X[:, t+1] = X[:, t] + (A * X[:, t] + B * u[t]) * dt
        end
    end
    return X, u
end
\end{lstlisting}
理想状況でのシミュレーション
\begin{lstlisting}[language=julia]
dt = 1e-3
T = 1.0
\end{lstlisting}
\begin{lstlisting}[language=julia]
Xa, ua = simulation(param, L, K, dt, T, noisy=false);
\end{lstlisting}
\subsubsection{ノイズを含むシミュレーション}
ノイズを含む場合.
\begin{lstlisting}[language=julia]
n = 4
nsim = 10
XSimAll = []
uSimAll = []
for i in 1:nsim
    XSim, u = simulation(param, L, K, dt, T, noisy=true);
    push!(XSimAll, XSim)
    push!(uSimAll, u)
end
\end{lstlisting}
結果の描画
\begin{lstlisting}[language=julia]
tarray = collect(dt:dt:T)
label = [L"Position ($m$)", L"Velocity ($m/s$)", L"Acceleration ($m/s^2$)", L"Jerk ($m/s^3$)"]

fig, ax = subplots(1, 3, figsize=(10, 3))
for i in 1:2
    for j in 1:nsim
        ax[i].plot(tarray, XSimAll[j][i,:]', "tab:gray", alpha=0.5)
    end
    
    ax[i].plot(tarray, Xa[i,:], "tab:red")
    ax[i].set_ylabel(label[i]); ax[i].set_xlabel(L"Time ($s$)"); ax[i].set_xlim(0, T); ax[i].grid()
end

for j in 1:nsim
    ax[3].plot(tarray, uSimAll[j], "tab:gray", alpha=0.5)
end
ax[3].plot(tarray, ua, "tab:red")
ax[3].set_ylabel(L"Control signal ($N\cdot m$)"); ax[3].set_xlabel(L"Time ($s$)"); ax[3].set_xlim(0, T); ax[3].grid()

tight_layout()
\end{lstlisting}
\begin{figure}[ht]
	\centering
	\includegraphics[scale=0.8, max width=\linewidth]{./fig/motor-learning/infinite-horizon-ofc/cell017.png}
	\caption{cell017.png}
	\label{cell017.png}
\end{figure}
\subsection{Target jump}
target jumpする場合の最適制御 \citep{Li2018-qt}. 状態にtarget位置も含むモデルであればtarget位置をずらせばよいが,ここでは自己位置をずらしtargetとの相対位置を変化させることでtarget jumpを実現する.
\begin{lstlisting}[language=julia]
function target_jump_simulation(param::SaccadeModelParameter, L, K, dt=0.001, T=2.0, 
        Ttj=0.4, tj_dist=0.1, 
        init_pos=-0.5; noisy=true)
    # Ttj : target jumping timing (sec)
    # tj_dist : target jump distance
    @unpack n, A, B, C, D, Y, G, Q, R, U = param
    nt = round(Int, T/dt)
    ntj = round(Int, Ttj/dt)
    X = zeros(n, nt)
    u = zeros(nt)
    X[1, 1] = init_pos # m; initial position (target position is zero)

    if noisy
        sqrtdt = √dt
        X̂ = zeros(n, nt)
        X̂[1, 1] = X[1, 1]
        for t in 1:nt-1
            if t == ntj
                X[1, t] -= tj_dist # When k == ntj, target jumpさせる(実際には現在の位置をずらす)
                X̂[1, t] -= tj_dist
            end
            u[t] = -L * X̂[:, t]
            X[:, t+1] = X[:,t] + (A * X[:,t] + B * u[t]) * dt + sqrtdt * (Y * u[t] * randn() + G * randn(n))
            dy = C * X[:,t] * dt + D * sqrtdt * randn(n-1)
            X̂[:, t+1] = X̂[:,t] + (A * X̂[:,t] + B * u[t]) * dt + K * (dy - C * X̂[:,t] * dt)
        end
    else
        for t in 1:nt-1
            if t == ntj
                X[1, t] -= tj_dist # When k == ntj, target jumpさせる(実際には現在の位置をずらす)
            end
            u[t] = -L * X[:, t]
            X[:, t+1] = X[:, t] + (A * X[:, t] + B * u[t]) * dt
        end
    end
    X[1, 1:ntj-1] .-= tj_dist;
    return X, u
end
\end{lstlisting}
\begin{lstlisting}[language=julia]
Ttj = 0.4
tj_dist = 0.1
nt = round(Int, T/dt)
ntj = round(Int, Ttj/dt);
\end{lstlisting}
\begin{lstlisting}[language=julia]
Xtj, utj = target_jump_simulation(param, L, K, dt, T, noisy=false);
\end{lstlisting}
\begin{lstlisting}[language=julia]
XtjAll = []
utjAll = []
for i in 1:nsim
    XSim, u = target_jump_simulation(param, L, K, dt, T, noisy=true);
    push!(XtjAll, XSim)
    push!(utjAll, u)
end
\end{lstlisting}
\begin{lstlisting}[language=julia]
target_pos = zeros(nt)
target_pos[1:ntj-1] .-= tj_dist; 

fig, ax = subplots(1, 3, figsize=(10, 3))
for i in 1:2
    ax[1].plot(tarray, target_pos, "tab:green")
    for j in 1:nsim
        ax[i].plot(tarray, XtjAll[j][i,:]', "tab:gray", alpha=0.5)
    end
    ax[i].axvline(x=Ttj, color="gray", linestyle="dashed")
    ax[i].plot(tarray, Xtj[i,:], "tab:red")
    ax[i].set_ylabel(label[i]); ax[i].set_xlabel(L"Time ($s$)"); ax[i].set_xlim(0, T); ax[i].grid()
end
for j in 1:nsim
    ax[3].plot(tarray, utjAll[j], "tab:gray", alpha=0.5)
end
ax[3].axvline(x=Ttj, color="gray", linestyle="dashed")
ax[3].plot(tarray, utj, "tab:red")
ax[3].set_ylabel(L"Control signal ($N\cdot m$)"); ax[3].set_xlabel(L"Time ($s$)"); ax[3].set_xlim(0, T); ax[3].grid()

tight_layout()
\end{lstlisting}
\begin{figure}[ht]
	\centering
	\includegraphics[scale=0.8, max width=\linewidth]{./fig/motor-learning/infinite-horizon-ofc/cell023.png}
	\caption{cell023.png}
	\label{cell023.png}
\end{figure}
 % ok
% \section{予測符号化}
\subsection{観測世界の階層的予測}
\textbf{階層的予測符号化(hierarchical predictive coding; HPC)} は\cite{Rao1999-zv}により導入された.構築するネットワークは入力層を含め,3層のネットワークとする.LGNへの入力として画像 $\mathbf{x} \in \mathbb{R}^{n_0}$を考える.画像 $\mathbf{x}$ の観測世界における隠れ変数,すなわち\textbf{潜在変数} (latent variable)を$\mathbf{r} \in \mathbb{R}^{n_1}$とし,ニューロン群によって発火率で表現されているとする (真の変数と $\mathbf{r}$は異なるので文字を分けるべきだが簡単のためにこう表す).このとき,


\mathbf{x} = f(\mathbf{U}\mathbf{r}) + \boldsymbol{\epsilon}


が成立しているとする.ただし,$f(\cdot)$は活性化関数 (activation function),$\mathbf{U} \in \mathbb{R}^{n_0 \times n_1}$は重み行列である.
$\boldsymbol{\epsilon} \in \mathbb{R}^{n_0}$ は $\mathcal{N}(\mathbf{0}, \sigma^2 \mathbf{I})$ からサンプリングされるとする.

潜在変数 $\mathbf{r}$はさらに高次 (higher-level)の潜在変数 $\mathbf{r}^h$により,次式で表現される.


\mathbf{r} = \mathbf{r}^{td}+\boldsymbol{\epsilon}^{td}=f(\mathbf{U}^h \mathbf{r}^h)+\boldsymbol{\epsilon}^{td}


ただし,Top-downの予測信号を $\mathbf{r}^{td}:=f(\mathbf{U}^h \mathbf{r}^h)$とした.また,$\mathbf{r}^{td} \in \mathbb{R}^{n_1}$, $\mathbf{r}^{h} \in \mathbb{R}^{n_2}$, $\mathbf{U}^h \in \mathbb{R}^{n_1 \times n_2}$ である.
$\boldsymbol{\epsilon}^{td} \in \mathbb{R}^{n_1}$は$\mathcal{N}(\mathbf{0}$, $\sigma_{td}^2 \mathbf{I}$) からサンプリングされるとする.

話は飛ぶが,Predictive codingのネットワークの特徴は
\begin{itemize}
\item 階層的な構造
\item 高次による低次の予測 (Feedback or Top-down信号)
\item 低次から高次への誤差信号の伝搬 (Feedforward or Bottom-up 信号)
\end{itemize}

である.ここまでは高次表現による低次表現の予測,というFeedback信号について説明してきたが,この部分はSparse codingでも同じである.それではPredictive codingのもう一つの要となる,低次から高次への予測誤差の伝搬というFeedforward信号はどのように導かれるのだろうか.結論から言えば,これは\textbf{復元誤差 (reconstruction error)の最小化を行う再帰的ネットワーク (recurrent network)を考慮することで自然に導かれる}.

\subsubsection{重み行列$\mathbf{A}$の作成}

\subsubsection{事前分布の設定}
事前分布$p(\mathbf{r})$としては,0においてピークがあり,裾の重い(heavy tail)を持つsparse distributionあるいは \textbf{super-Gaussian distribution} (Laplace 分布やCauchy分布などGaussian分布よりもkurtoticな分布)を用いるのが良い.このような分布では,$\mathbf{r}$の各要素$r_i$はほとんど0に等しく,ある入力に対しては大きな値を取る.$p(\mathbf{r})$は一般化して式(4), (5)のように表記する.


\begin{aligned}
p(\mathbf{r})&=\prod_j p(r_j)\\
p(r_j)&=\frac{1}{Z_{\beta}}\exp \left[-\beta S(r_j)\right]
\end{aligned}


ただし,$\beta$は逆温度(inverse temperature), $Z_{\beta}$は規格化定数 (分配関数) である.これらの用語は統計力学における正準分布 (ボルツマン分布)から来ている.$S(x)$と分布の関係をまとめた表が以下となる (cf. \url{https://pdfs.semanticscholar.org/be08/da912362bf40fe3ded78bdadc644f921b4e7.pdf}).

\lstinputlisting[language=julia]{./text/motor-learning/biological-ofc/003.jl}
2種類の指数関数型シナプスの動態.破線は単一指数関数型シナプスで, 実線は二重指数関数型シナプスである.
変更しない定数を保持する \jl{struct} の \jl{FHNParameter} と, 変数を保持する \jl{mutable struct} の \jl{FHN} を作成する.
いくつかの処理について解説しておく.まず,一番目のforループ内の\jl{v[i]}の\jl{((dt*tcount) > (tlast[i] + tref))}は最後にニューロンが発火した時刻\jl{tlast[i]}に不応期\jl{tref}を足した時刻よりも現在の時刻\jl{dt*tcount[1]}が大きければ膜電位の更新を許可し,小さければ更新しない.二番目のforループにおける\jl{fire[i]}はニューロンの膜電位が閾値電位\jl{vthr}を超えたら\jl{True}となる.\jl{v[i]}などの更新式にある\jl{ifelse(a, b, c)}はaが\jl{True}の時はbを返し,\jl{False}の時はcを返す関数であり,\jl{v[i] = ifelse(fire[i], vreset, v[i])}は\jl{fire[i]}が\jl{True}なら\jl{v[i]}をリセット電位\jl{vreset}とし,そうでなければそのままの値を返すという処理である.同様にして\jl{tlast[i]}は発火したときにその時刻を記録する変数となっている.なお,\jl{v_[i] = ifelse(fire[i], vpeak, v[i])}は実際のシミュレーションにおいて意味をなさない.単に発火時の電位\jl{vpeak}を含めて記録すると描画時の見栄えが良いというだけである.

これらの\jl{struct}と関数を用いてシミュレーションを実行する.\jl{I} はHHモデルのときと同じように矩形波を入力する.実は\jl{I}は入力電流ではなく入力電流に比例する量となっているが,これは膜抵抗を乗じた後の値であると考えるとよい.

\lstinputlisting[language=julia]{./text/motor-learning/biological-ofc/007.jl}
## 7.9.2 更新関数の定義

\lstinputlisting[language=julia]{./text/motor-learning/biological-ofc/009.jl}
\subsection{ランジュバン・モンテカルロ法 (LMC)}拡散過程
$$
{\frac{d\theta}{dt}}=\nabla \log p (\theta)+{\sqrt 2}{d{W}}
$$
Euler–Maruyama法により,
## 画像の復元
\lstinputlisting[language=julia]{./text/motor-learning/biological-ofc/012.jl}
\lstinputlisting[language=julia]{./text/motor-learning/biological-ofc/013.jl}
\lstinputlisting[language=julia]{./text/motor-learning/biological-ofc/014.jl}
\lstinputlisting[language=julia]{./text/motor-learning/biological-ofc/015.jl}
\lstinputlisting[language=julia]{./text/motor-learning/biological-ofc/016.jl}
\begin{figure}[ht]
	\centering
	\includegraphics[scale=0.8, max width=\linewidth]{./fig/neuron-model/fhn/cell016.png}
	\caption{cell016.png}
	\label{cell016.png}
\end{figure}
\lstinputlisting[language=julia]{./text/motor-learning/biological-ofc/017.jl}
\begin{figure}[ht]
	\centering
	\includegraphics[scale=0.8, max width=\linewidth]{./fig/motor-learning/optimal-feedback-control/cell017.png}
	\caption{cell017.png}
	\label{cell017.png}
\end{figure}
 % 全体修正する
% \section{ラット自由行動下の軌跡のシミュレーション}
ラットが自由行動下において箱の中を探索する際の軌跡をシミュレーションする \citep{Raudies2012-gp}.これまでと異なり,現象論的に運動を生成する.場所細胞・格子細胞等自己位置と神経活動が相関する細胞のシミュレーションにおいて用いられる \citep{George2024-rv}.
\begin{lstlisting}[language=julia]
using PyPlot, LinearAlgebra, Random, Distributions
rc("axes.spines", top=false, right=false)
\end{lstlisting}
\begin{lstlisting}[language=julia]
box_width, box_height = 0.8, 0.8; # Width and height of environment (meters)
perimeter_dist = 0.03  # Perimeter region distance to walls (meters)
σv = 0.13              # Forward velocity Rayleigh distribution scale (m/sec)
μω = 0.0               # Rotation velocity Gaussian distribution mean (rad/sec)
σω = (330 / 360) * 2π  # Rotation velocity Gaussian distribution standard deviation (rad/sec)
dt = 0.02              # Simulation-step time increment (seconds)
decel_rate = 0.25;     # velocity reduction factor when located in the perimeter
\end{lstlisting}
並進速度をレイリー分布,回転速度を正規分布に従うようにランダムサンプリングする.壁の接ベクトルとラットの距離を\jl{dist_wall}, 壁の法線ベクトルとラットの頭方向の角度の差を\jl{angle_wall}とする.なお,壁とはラットの自己速度ベクトルと壁全体との交点である.
\begin{itemize}
\item ラットの自己速度ベクトルと壁全体との交点を求める.
\item 交点の接ベクトルと法線ベクトルを求める.
\item 接ベクトルとの距離を\jl{dist_wall}とする.
\item 法線ベクトルと成す角を\jl{angle_wall}とする.
\end{itemize}
\begin{lstlisting}[language=julia]
function min_dist_angle(pos, head_dir, wall_type="square")
    x, y = pos
    if wall_type == "square"
        dists = [box_width/2-x, box_height/2-y, box_width/2+x, box_height/2+y]
        dist_wall, nearest_wall = findmin(dists)
        angle_wall = mod(head_dir - (nearest_wall-1)*π/2 + π, 2π) - π
    elseif wall_type == "circle"
        dist_wall = box_width/2 - sqrt(x^2 + y^2)
        angle_wall = mod(head_dir - atan(y, x) + π, 2π) - π
    else
        @warn "'wall_type' must be 'square' or 'circle'"
    end 
    return dist_wall, angle_wall
end;
\end{lstlisting}
\begin{lstlisting}[language=julia]
function generate_trajectory(num_steps, wall_type)
    # store arrays
    position, velocity = zeros(num_steps, 2), zeros(num_steps, 2)
    head_dir = zeros(num_steps) # head direction
    speed = rand(Rayleigh(σv), num_steps) # Forward speed
    random_turn = rand(Normal(μω, σω), num_steps) * dt

    # initial values
    head_dir[1] = rand() * 2π
    position[1, :] = (rand(2) .-0.5) .* ([box_width, box_height] .- perimeter_dist)
    
    # iteration of trajectory
    for t in 1:num_steps-1
        turn_angle = random_turn[t]
        dist_wall, angle_wall = min_dist_angle(position[t, :], head_dir[t], wall_type)
        if (dist_wall < perimeter_dist) && (abs(angle_wall) < π/2) # avoid wall
            speed[t] *= decel_rate # deceleration
            turn_angle += sign(angle_wall) * (π/2 - abs(angle_wall))
        end
        velocity[t, :] = speed[t] * [cos(head_dir[t]), sin(head_dir[t])]
        position[t+1, :] = position[t, :] + velocity[t, :] * dt
        head_dir[t+1] = mod(head_dir[t] + turn_angle, 2π) # turn, 
    end
    return position, velocity, speed, head_dir
end;
\end{lstlisting}
5分間のシミュレーションを行う.
\begin{lstlisting}[language=julia]
T = 300 # simulation time (sec)
num_steps = round(Int, T/dt)
wall_types = ["square", "circle"]
positions = zeros(2, num_steps, 2)
for i in 1:2
    positions[i, :, :], _, _, _ =  generate_trajectory(num_steps, wall_types[i]);
end
\end{lstlisting}
黒点から始まり,赤点に終わる.
\begin{lstlisting}[language=julia]
figure(figsize=(8, 4))
for i in 1:2
    subplot(1,2,i)
    title("Wall type: "*wall_types[i])
    xlabel("x (meters)"); ylabel("y (meters)")
    plot(positions[i, 1, 1], positions[i, 1, 2], "ko", label="Start")
    plot(positions[i, end, 1], positions[i, end, 2], "ro", label="Goal")
    plot(positions[i, :, 1], positions[i, :, 2], color="k", alpha=0.3)
end
tight_layout()
\end{lstlisting}
\begin{figure}[ht]
	\centering
	\includegraphics[scale=0.8, max width=\linewidth]{./fig/motor-learning/rat-trajectory/cell009.png}
	\caption{cell009.png}
	\label{cell009.png}
\end{figure}
 % ok

\chapter{強化学習}
% \section{TD学習}
TD (Temporal difference) learningにおいて,\textbf{報酬予測誤差}\index{ほうしゅうよそくごさ@報酬予測誤差}(reward prediction error, \textbf{RPE}\index{RPE}) $\delta_{i}$は次のように計算される. 
 
\begin{equation}
\delta_{i}=r+\gamma V_{j}\left(x^{\prime}\right)-V_{i}(x) 
\end{equation}
 
ただし,現在の状態を$x$, 次の状態を$x'$, 予測価値分布を$V(x)$, 報酬信号を$r$, 時間割引率(time discount)を$\gamma$としました.
また,$V_{j}\left(x^{\prime}\right)$は予測価値分布$V\left(x^{\prime}\right)$からのサンプルです. このRPEは脳内において主に中脳の\textbf{VTA}\index{VTA}(腹側被蓋野)や\textbf{SNc}\index{SNc}(黒質緻密部)における\textbf{ドパミン(dopamine)ニューロン}\index{どぱみん(dopamine)にゅーろん@ドパミン(dopamine)ニューロン}の発火率として表現されています.
ただし,VTAとSNcのドパミンニューロンの役割は同一ではありません.ドパミンニューロンへの入力が異なっています [(Watabe-Uchida et al., _Neuron._ 2012)](https://www.cell.com/neuron/fulltext/S0896-6273(12)00281-4). また,細かいですがドパミンニューロンの発火は報酬量に対して線形ではなく,やや飽和する非線形な応答関数 (Hill functionで近似可能)を持ちます([Eshel et al., _Nat. Neurosci._ 2016](https://www.nature.com/articles/nn.4239)).このため著者実装では報酬 $r$に非線形関数がかかっているものもあります.
先ほどRPEはドパミンニューロンの発火率で表現されている,といいました.RPEが正の場合はドパミンニューロンの発火で表現できますが,単純に考えると負の発火率というものはないため,負のRPEは表現できないように思います.ではどうしているかというと,RPEが0 (予想通りの報酬が得られた場合) でもドパミンニューロンは発火しており,RPEが正の場合にはベースラインよりも発火率が上がるようになっています.逆にRPEが負の場合にはベースラインよりも発火率が減少する(抑制される)ようになっています
    ([Schultz et al., \url{span style="font-style: italic;"}Science.\url{/span} 1997](https://science.sciencemag.org/content/275/5306/1593.long "https://science.sciencemag.org/content/275/5306/1593.long"); [Chang et al., \url{span style="font-style: italic;"}Nat Neurosci\url{/span}. 2016](https://www.nature.com/articles/nn.4191 "https://www.nature.com/articles/nn.4191")).発火率というのを言い換えればISI (inter-spike interval, 発火間隔)の長さによってPREが符号化されている(ISIが短いと正のRPE, ISIが長いと負のRPEを表現)ともいえます ([Bayer et al., \url{span style="font-style: italic;"}J.
    Neurophysiol\url{/span}. 2007](https://www.physiology.org/doi/full/10.1152/jn.01140.2006 "https://www.physiology.org/doi/full/10.1152/jn.01140.2006")).
予測価値(分布) $V(x)$ですが,これは線条体(striatum)のパッチ (SNcに抑制性の投射をする)やVTAのGABAニューロン (VTAのドパミンニューロンに投射して減算抑制をする, ([Eshel, et al., _Nature_. 2015](https://www.nature.com/articles/nature14855 "https://www.nature.com/articles/nature14855")))などにおいて表現されている. この予測価値は通常のTD learningでは次式により更新されます. 
 
\begin{equation}
V_{i}(x) \leftarrow V_{i}(x)+\alpha_{i} f\left(\delta_{i}\right) 
\end{equation}
 
ただし,$\alpha_{i}$は学習率(learning rate), $f(\cdot)$はRPEに対する応答関数である.生理学的には$f(\delta)=\delta$を使うのが妥当である.
TD誤差
\begin{equation}
\delta_{t} = r_{t+1} + \gamma V(s_{t+1}) - V(s_{t})
\end{equation}
価値の更新
\begin{equation}
V(s_{t}) \leftarrow V(s_{t}) + \alpha \delta_{t}
\end{equation}
([Schultz, Dayan & Montague, Science, 1997](https://science.sciencemag.org/content/275/5306/1593))を参考にシミュレーションを行う.60試行する.0秒の時に条件刺激(光刺激)が入る.また,6試行目から40試行目まで条件刺激があってから1.2秒後に報酬が与えられるとする.
\begin{lstlisting}[language=julia]
using PyPlot
rc("axes.spines", top=false, right=false)
\end{lstlisting}
\begin{lstlisting}[language=julia]
num_trial = 60 # 試行回数
T = 3.0f0 # s
dt = 0.1f0 # s
nt = UInt(T/dt) + 1 # number of timesteps
value = zeros(num_trial, nt) 
delta = zeros(num_trial, nt) # TD error

flash_time = UInt(1.1f0/dt)
delay = UInt(1.2f0/dt)
reward_trial = 6:40 # 報酬が貰える試行の区間を設定する
reward = zeros(num_trial, nt)
reward[reward_trial, flash_time+delay] .= 1.0

α = 0.8 # 学習率
γ = 0.99 # 割引率

# simulation
for i in 2:num_trial
    for t in 1:nt-1
        delta[i, t] = reward[i, t] + γ*value[i-1, t+1] - value[i-1, t]
        if t > flash_time
            value[i, t] = value[i-1, t] + α*delta[i, t]
        end
    end
end
\end{lstlisting}
結果を描画する.
\begin{lstlisting}[language=julia]
figure(figsize=(8, 3.5))

subplot2grid((3,3), (0,0), rowspan=3)
imshow(value); colorbar()
title("Value"); ylabel("Trial"); xlabel("Time (s)"); xticks(0:Int(1/dt):nt, -1:1:T-1)

subplot2grid((3,3), (0,1), rowspan=3)
imshow(delta); colorbar()
title("TD error"); ylabel("Trial"); xlabel("Time (s)"); xticks(0:Int(1/dt):nt, -1:1:T-1)

subplot2grid((3,3), (0,2))
plot(-1:0.1:2, delta[6, :]); title("No CS + R (Trial #6)"); xticks([])

subplot2grid((3,3), (1,2))
plot(-1:0.1:2, delta[30, :]); title("CS + R (Trial #30)"); ylabel("TD error"); xticks([])

subplot2grid((3,3), (2,2))
plot(-1:0.1:2, delta[41, :]); title("CS + No R (Trial #41)"); xlabel("Time (s)")

tight_layout()
\end{lstlisting}
\begin{figure}[ht]
	\centering
	\includegraphics[scale=0.8, max width=\linewidth]{./fig/reinforcement-learning/td-learning/cell006.png}
	\caption{cell006.png}
	\label{cell006.png}
\end{figure}
CSは条件刺激(conditioned stimulus), Rは報酬(reward)を意味する.
この際にprediction error信号の時間的変位(temporal shift)が観察されるが,
\begin{itemize}
\item A gradual temporal shift of dopamine responses mirrors the progression of temporal difference error in machine learning
\end{itemize}
https://www.nature.com/articles/s41593-022-01109-2
Rescorla–Wagner model
 % ref修正する

\chapter{神経回路網によるベイズ推論}
% \section{予測符号化}
\subsection{観測世界の階層的予測}
\textbf{階層的予測符号化(hierarchical predictive coding; HPC)} は\cite{Rao1999-zv}により導入された.構築するネットワークは入力層を含め,3層のネットワークとする.LGNへの入力として画像 $\mathbf{x} \in \mathbb{R}^{n_0}$を考える.画像 $\mathbf{x}$ の観測世界における隠れ変数,すなわち\textbf{潜在変数} (latent variable)を$\mathbf{r} \in \mathbb{R}^{n_1}$とし,ニューロン群によって発火率で表現されているとする (真の変数と $\mathbf{r}$は異なるので文字を分けるべきだが簡単のためにこう表す).このとき,


\mathbf{x} = f(\mathbf{U}\mathbf{r}) + \boldsymbol{\epsilon}


が成立しているとする.ただし,$f(\cdot)$は活性化関数 (activation function),$\mathbf{U} \in \mathbb{R}^{n_0 \times n_1}$は重み行列である.
$\boldsymbol{\epsilon} \in \mathbb{R}^{n_0}$ は $\mathcal{N}(\mathbf{0}, \sigma^2 \mathbf{I})$ からサンプリングされるとする.

潜在変数 $\mathbf{r}$はさらに高次 (higher-level)の潜在変数 $\mathbf{r}^h$により,次式で表現される.


\mathbf{r} = \mathbf{r}^{td}+\boldsymbol{\epsilon}^{td}=f(\mathbf{U}^h \mathbf{r}^h)+\boldsymbol{\epsilon}^{td}


ただし,Top-downの予測信号を $\mathbf{r}^{td}:=f(\mathbf{U}^h \mathbf{r}^h)$とした.また,$\mathbf{r}^{td} \in \mathbb{R}^{n_1}$, $\mathbf{r}^{h} \in \mathbb{R}^{n_2}$, $\mathbf{U}^h \in \mathbb{R}^{n_1 \times n_2}$ である.
$\boldsymbol{\epsilon}^{td} \in \mathbb{R}^{n_1}$は$\mathcal{N}(\mathbf{0}$, $\sigma_{td}^2 \mathbf{I}$) からサンプリングされるとする.

話は飛ぶが,Predictive codingのネットワークの特徴は
\begin{itemize}
\item 階層的な構造
\item 高次による低次の予測 (Feedback or Top-down信号)
\item 低次から高次への誤差信号の伝搬 (Feedforward or Bottom-up 信号)
\end{itemize}

である.ここまでは高次表現による低次表現の予測,というFeedback信号について説明してきたが,この部分はSparse codingでも同じである.それではPredictive codingのもう一つの要となる,低次から高次への予測誤差の伝搬というFeedforward信号はどのように導かれるのだろうか.結論から言えば,これは\textbf{復元誤差 (reconstruction error)の最小化を行う再帰的ネットワーク (recurrent network)を考慮することで自然に導かれる}.

\lstinputlisting[language=julia]{./text/bayesian-brain/neural-uncertainty-representation/001.jl}
\begin{figure}[ht]
	\centering
	\includegraphics[scale=0.8, max width=\linewidth]{./fig/bayesian-brain/neural-uncertainty-representation/cell001.png}
	\caption{cell001.png}
	\label{cell001.png}
\end{figure}
\lstinputlisting[language=julia]{./text/bayesian-brain/neural-uncertainty-representation/002.jl}
\lstinputlisting[language=julia]{./text/bayesian-brain/neural-uncertainty-representation/003.jl}
\begin{figure}[ht]
	\centering
	\includegraphics[scale=0.8, max width=\linewidth]{./fig/synapse-model/expo-synapse/cell003.png}
	\caption{cell003.png}
	\label{cell003.png}
\end{figure}

% \section{予測符号化}
\subsection{観測世界の階層的予測}
\textbf{階層的予測符号化(hierarchical predictive coding; HPC)} は\cite{Rao1999-zv}により導入された.構築するネットワークは入力層を含め,3層のネットワークとする.LGNへの入力として画像 $\mathbf{x} \in \mathbb{R}^{n_0}$を考える.画像 $\mathbf{x}$ の観測世界における隠れ変数,すなわち\textbf{潜在変数} (latent variable)を$\mathbf{r} \in \mathbb{R}^{n_1}$とし,ニューロン群によって発火率で表現されているとする (真の変数と $\mathbf{r}$は異なるので文字を分けるべきだが簡単のためにこう表す).このとき,


\mathbf{x} = f(\mathbf{U}\mathbf{r}) + \boldsymbol{\epsilon}


が成立しているとする.ただし,$f(\cdot)$は活性化関数 (activation function),$\mathbf{U} \in \mathbb{R}^{n_0 \times n_1}$は重み行列である.
$\boldsymbol{\epsilon} \in \mathbb{R}^{n_0}$ は $\mathcal{N}(\mathbf{0}, \sigma^2 \mathbf{I})$ からサンプリングされるとする.

潜在変数 $\mathbf{r}$はさらに高次 (higher-level)の潜在変数 $\mathbf{r}^h$により,次式で表現される.


\mathbf{r} = \mathbf{r}^{td}+\boldsymbol{\epsilon}^{td}=f(\mathbf{U}^h \mathbf{r}^h)+\boldsymbol{\epsilon}^{td}


ただし,Top-downの予測信号を $\mathbf{r}^{td}:=f(\mathbf{U}^h \mathbf{r}^h)$とした.また,$\mathbf{r}^{td} \in \mathbb{R}^{n_1}$, $\mathbf{r}^{h} \in \mathbb{R}^{n_2}$, $\mathbf{U}^h \in \mathbb{R}^{n_1 \times n_2}$ である.
$\boldsymbol{\epsilon}^{td} \in \mathbb{R}^{n_1}$は$\mathcal{N}(\mathbf{0}$, $\sigma_{td}^2 \mathbf{I}$) からサンプリングされるとする.

話は飛ぶが,Predictive codingのネットワークの特徴は
\begin{itemize}
\item 階層的な構造
\item 高次による低次の予測 (Feedback or Top-down信号)
\item 低次から高次への誤差信号の伝搬 (Feedforward or Bottom-up 信号)
\end{itemize}

である.ここまでは高次表現による低次表現の予測,というFeedback信号について説明してきたが,この部分はSparse codingでも同じである.それではPredictive codingのもう一つの要となる,低次から高次への予測誤差の伝搬というFeedforward信号はどのように導かれるのだろうか.結論から言えば,これは\textbf{復元誤差 (reconstruction error)の最小化を行う再帰的ネットワーク (recurrent network)を考慮することで自然に導かれる}.

\lstinputlisting[language=julia]{./text/bayesian-brain/bayesian-linear-regression/001.jl}
\lstinputlisting[language=julia]{./text/bayesian-brain/bayesian-linear-regression/002.jl}
UNDERSTANDING STRAIGHT-THROUGH ESTIMATOR IN TRAINING ACTIVATION QUANTIZED NEURAL NETS

Yoshua Bengio, Nicholas L´eonard, and Aaron Courville. Estimating or propagating gradients through stochastic neurons for conditional computation. arXiv preprint arXiv:1308.3432, 2013.

\lstinputlisting[language=julia]{./text/bayesian-brain/bayesian-linear-regression/004.jl}
\lstinputlisting[language=julia]{./text/bayesian-brain/bayesian-linear-regression/005.jl}
\lstinputlisting[language=julia]{./text/bayesian-brain/bayesian-linear-regression/006.jl}
\begin{figure}[ht]
	\centering
	\includegraphics[scale=0.8, max width=\linewidth]{./fig/bayesian-brain/bayesian-linear-regression/cell006.png}
	\caption{cell006.png}
	\label{cell006.png}
\end{figure}

% \section{マルコフ連鎖モンテカルロ法 (MCMC)}
前節では解析的に事後分布の計算をした.事後分布を近似的に推論する方法の1つに\textbf{マルコフ連鎖モンテカルロ法 (Markov chain Monte Carlo methods; MCMC)}\index{まるこふれんさもんてかるろほう (Markov chain Monte Carlo methods; MCMC)@マルコフ連鎖モンテカルロ法 (Markov chain Monte Carlo methods; MCMC)} がある.他の近似推論の手法としてはLaplace近似や変分推論 (variational inference) などがある.MCMCは他の手法に比して,事後分布の推論だけでなく,確率分布を神経活動で表現する方法を提供するという利点がある.

\footnote{
変分推論は入れた方がいいと思うが,紙幅の都合上いれられるか微妙である.
}

データを$X$とし,パラメータを$\theta$とする.


\begin{equation}
p(\theta\mid X)=\frac{p(X\mid \theta)p(\theta)}{\int p(X\mid \theta)p(\theta)d\theta}
\end{equation}


分母の積分計算$\int p(X\mid \theta)p(\theta)d\theta$が求まればよい.

\subsubsection{モンテカルロ法}

\subsubsection{マルコフ連鎖}
\subsection{Metropolis-Hastings法}
\lstinputlisting[language=julia]{./text/bayesian-brain/mcmc/002.jl}
\lstinputlisting[language=julia]{./text/bayesian-brain/mcmc/003.jl}
\lstinputlisting[language=julia]{./text/bayesian-brain/mcmc/004.jl}
\begin{figure}[ht]
	\centering
	\includegraphics[scale=0.8, max width=\linewidth]{./fig/bayesian-brain/mcmc/cell004.png}
	\caption{cell004.png}
	\label{cell004.png}
\end{figure}
\lstinputlisting[language=julia]{./text/bayesian-brain/mcmc/005.jl}
\lstinputlisting[language=julia]{./text/bayesian-brain/mcmc/006.jl}
\lstinputlisting[language=julia]{./text/bayesian-brain/mcmc/007.jl}
\lstinputlisting[language=julia]{./text/bayesian-brain/mcmc/008.jl}
\lstinputlisting[language=julia]{./text/bayesian-brain/mcmc/009.jl}
\begin{figure}[ht]
	\centering
	\includegraphics[scale=0.8, max width=\linewidth]{./fig/bayesian-brain/mcmc/cell009.png}
	\caption{cell009.png}
	\label{cell009.png}
\end{figure}
\subsection{ランジュバン・モンテカルロ法 (LMC)}
拡散過程


\begin{equation}
{\frac{d\theta}{dt}}=\nabla \log p (\theta)+{\sqrt 2}{d{W}}
\end{equation}


Euler–Maruyama法により,
\lstinputlisting[language=julia]{./text/bayesian-brain/mcmc/011.jl}
\lstinputlisting[language=julia]{./text/bayesian-brain/mcmc/012.jl}
\lstinputlisting[language=julia]{./text/bayesian-brain/mcmc/013.jl}
\lstinputlisting[language=julia]{./text/bayesian-brain/mcmc/014.jl}
\begin{figure}[ht]
	\centering
	\includegraphics[scale=0.8, max width=\linewidth]{./fig/bayesian-brain/mcmc/cell014.png}
	\caption{cell014.png}
	\label{cell014.png}
\end{figure}
\subsection{ハミルトニアン・モンテカルロ法 (HMC法)}
ハミルトニアン・モンテカルロ法(Hamiltonian Monte Calro)あるいはハイブリッド・モンテカルロ法(Hybrid Monte Calro)という

一般化座標を$\mathbf{q}$, 一般化運動量を$\mathbf{p}$とする.ポテンシャルエネルギーを$U(\mathbf{q})$としたとき,古典力学 (解析力学) において保存力のみが作用する場合の\textbf{ハミルトニアン (Hamiltonian)}\index{はみるとにあん (Hamiltonian)@ハミルトニアン (Hamiltonian)} $\mathcal{H}(\mathbf{q}, \mathbf{p})$は


\begin{equation}
\mathcal{H}(\mathbf{q}, \mathbf{p})\triangleqU(\mathbf{q})+\frac{1}{2}\|\mathbf{p}\|^2
\end{equation}


となる.このとき,次の2つの方程式が成り立つ.


\begin{equation}
\frac{d\mathbf{q}}{dt}=\frac{\partial \mathcal{H}}{\partial \mathbf{p}}=\mathbf{p},\quad\frac{d\mathbf{p}}{dt}=-\frac{\partial \mathcal{H}}{\partial \mathbf{q}}=-\frac{\partial U}{\partial \mathbf{q}}
\end{equation}


これを\textbf{ハミルトンの運動方程式(hamilton's equations of motion)}\index{はみるとんのうんどうほうていしき(hamilton's equations of motion)@ハミルトンの運動方程式(hamilton's equations of motion)} あるいは\textbf{正準方程式 (canonical equations)}\index{せいじゅんほうていしき (canonical equations)@正準方程式 (canonical equations)} という.


この処理をMetropolis-Hastings法における採用・不採用アルゴリズムという.

リープフロッグ(leap frog)法により離散化する.
\lstinputlisting[language=julia]{./text/bayesian-brain/mcmc/016.jl}
\lstinputlisting[language=julia]{./text/bayesian-brain/mcmc/017.jl}
\lstinputlisting[language=julia]{./text/bayesian-brain/mcmc/018.jl}
\lstinputlisting[language=julia]{./text/bayesian-brain/mcmc/019.jl}
\lstinputlisting[language=julia]{./text/bayesian-brain/mcmc/020.jl}
\begin{figure}[ht]
	\centering
	\includegraphics[scale=0.8, max width=\linewidth]{./fig/energy-based-model/predictive-coding/cell020.png}
	\caption{cell020.png}
	\label{cell020.png}
\end{figure}
*ToDo: 自己相関確認する*
\subsection{線形回帰への適応}
\lstinputlisting[language=julia]{./text/bayesian-brain/mcmc/023.jl}
\lstinputlisting[language=julia]{./text/bayesian-brain/mcmc/024.jl}
\lstinputlisting[language=julia]{./text/bayesian-brain/mcmc/025.jl}
\lstinputlisting[language=julia]{./text/bayesian-brain/mcmc/026.jl}
\lstinputlisting[language=julia]{./text/bayesian-brain/mcmc/027.jl}
\lstinputlisting[language=julia]{./text/bayesian-brain/mcmc/028.jl}
\lstinputlisting[language=julia]{./text/bayesian-brain/mcmc/029.jl}
\lstinputlisting[language=julia]{./text/bayesian-brain/mcmc/030.jl}
\begin{figure}[ht]
	\centering
	\includegraphics[scale=0.8, max width=\linewidth]{./fig/bayesian-brain/neural-sampling/cell030.png}
	\caption{cell030.png}
	\label{cell030.png}
\end{figure}
\lstinputlisting[language=julia]{./text/bayesian-brain/mcmc/031.jl}
\lstinputlisting[language=julia]{./text/bayesian-brain/mcmc/032.jl}
\begin{figure}[ht]
	\centering
	\includegraphics[scale=0.8, max width=\linewidth]{./fig/solve-credit-assignment-problem/backpropagation/cell032.png}
	\caption{cell032.png}
	\label{cell032.png}
\end{figure}
 % ok
% \section{神経サンプリング}
サンプリングに基づく符号化(sampling-based coding; SBC or neural sampling model)をガウス尺度混合モデルを例にとり実装する.
\subsection{ガウス尺度混合モデル}
\textbf{ガウス尺度混合 (Gaussian scale mixture; GSM) モデル}\index{がうすしゃくどこんごう (Gaussian scale mixture; GSM) もでる@ガウス尺度混合 (Gaussian scale mixture; GSM) モデル}は確率的生成モデルの一種である\citep{Wainwright1999-cl}\citep{Orban2016-tm}.GSMモデルでは入力を次式で予測する:
\begin{equation}
\text{入力}={z}\left(\sum \text{神経活動} \times \text{基底} \right) + \text{ノイズ}
\end{equation}
前節までのスパース符号化モデル等と同様に,入力が基底の線形和で表されるとしている.ただし,尺度(scale)パラメータ$z$が基底の線形和に乗じられている点が異なる.\footnote{コードは\citep{Orban2016-tm} \url{https://github.com/gergoorban/sampling_in_gsm}, および\citep{Echeveste2020-sh} \url{https://bitbucket.org/RSE_1987/ssn_inference_numerical_experiments/src/master/}を参考に作成した.}
\subsubsection{事前分布}
$\mathbf{x} \in \mathbb{R}^{N_x}$, $\mathbf{A} \in \mathbb{R}^{N_x\times N_y}$, $\mathbf{y} \in \mathbb{R}^{N_y}$, $\mathbf{z} \in \mathbb{R}$とする.
\begin{equation}
p\left(\mathbf{x}\mid\mathbf{y}, z\right)=\mathcal{N}\left(z \mathbf{A} \mathbf{y}, \sigma_{\mathbf{x}}^{2} \mathbf{I}\right)
\end{equation}
事前分布を
\begin{align}
p\left(\mathbf{y}\right)&=\mathcal{N}\left(\mathbf{0}, \mathbf{C}\right)\\
p\left(z\right)&=\Gamma (k, \vartheta)
\end{align}
とする.$\Gamma(k, \vartheta)$はガンマ分布であり,$k$は形状(shape)パラメータ,$\vartheta$は尺度(scale)パラメータである.$p\left(\mathbf{y}\right)$は$\mathbf{y}$の事前分布であり,刺激がない場合の自発活動の分布を表していると仮定する.
\subsubsection{重み行列$\mathbf{A}$の作成}
\begin{lstlisting}[language=julia]
using PyPlot, LinearAlgebra, Random, Distributions, KernelDensity, StatsBase
using PyPlot: matplotlib
Random.seed!(0)
rc("axes.spines", top=false, right=false)
\end{lstlisting}
\begin{lstlisting}[language=julia]
function gabor(x, y, θ, σ=1, λ=2, ψ=0)
    xθ = x * cos(θ) + y * sin(θ)
    yθ = -x * sin(θ) + y * cos(θ)
    return exp(-.5(xθ^2 + yθ^2)/σ^2) * cos(2π/λ * xθ + ψ)
end;
\end{lstlisting}
\begin{lstlisting}[language=julia]
function get_A(WH, Ny)
    Nx = WH^2
    A = zeros(Nx, Ny) # weight matrix
    p = range(-3, 3, length=WH) # position
    θ = (1:Ny) / Ny * π # theta for gabor
    for i in 1:Ny
        gb = gabor.(p', p, θ[i])
        gb /= norm(gb) + 1e-8 # normalization
        A[:, i] = gb[:] # flatten and save
    end
    return A
end;
\end{lstlisting}
\begin{lstlisting}[language=julia]
WH = 16   # width/height of input image
Nx = WH^2 # dimension of the observed variable x
Ny = 50  # dimension of the hidden variable y

A = get_A(WH, Ny);
\end{lstlisting}
重み行列$\mathbf{A}$の一部を描画してみよう.
\begin{lstlisting}[language=julia]
figure(figsize=(2,2))
plot_idx = [2,4,6,8]
weight_idx = [37,25,50,13]
titles = ["", "0°", "±90°", ""]
for i in 1:4
    subplot(3,3,plot_idx[i])
    title(titles[i])
    imshow(reshape(A[:, weight_idx[i]], WH, WH), cmap="gray")
    axis("off")
end
subplots_adjust(wspace=0.01, hspace=0.01)
\end{lstlisting}
\begin{figure}[ht]
	\centering
	\includegraphics[scale=0.8, max width=\linewidth]{./fig/bayesian-brain/neural-sampling/cell007.png}
	\caption{cell007.png}
	\label{cell007.png}
\end{figure}
\subsubsection{分散共分散行列$\mathbf{C}$の作成}
$\mathbf{C}$は$y$の事前分布の分散共分散行列である.\citep{Orban2016-tm}では自然画像を用いて作成しているが,ここでは簡単のため$\mathbf{A}$と同様に\citep{Echeveste2020-sh}に従って作成する.前項で作成した通り,$\mathbf{A}$の各基底には周期性があるため,類似した基底を持つニューロン同士は類似した出力をすると考えられる.Echevesteらは$\theta\in[-\pi/2, \pi/2)$の範囲においてFourier基底を複数作成し,そのグラム行列(Gram matrix)を係数倍したものを$\mathbf{C}$と設定している.ここではガウス過程(Gaussian process)モデルとの類似性から,周期カーネル(periodic kernel) 
\begin{equation}
K(\theta, \theta')=\exp\left[\phi_1 \cos \left(\dfrac{|\theta-\theta'|}{\phi_2}\right)\right]
\end{equation}
を用いる.ここでは$|\theta-\theta'|=m\pi\ (m=0,1,\ldots)$の際に類似度が最大になればよいので,$\phi_2=0.5$とする.これが正定値行列になるように単位行列の係数倍$\epsilon\mathbf{I}$を加算し,スケーリングした上で,\jl{Symmetric(C)}や\jl{Matrix(Hermitian(C)))}により実対象行列としたものを$\mathbf{C}$とする.$\mathbf{C}$を正定値行列にする理由はJuliaの\jl{MvNormal}がCholesky分解を用いて多変量正規分布の乱数を生成するためである. 事前に\jl{cholesky(C)}が実行できるか確認するのもよい.
\begin{lstlisting}[language=julia]
function get_C(Ny, C_range=[-0.5, 4.0], eps=0.1, ψ₁=2.0, ψ₂=0.5)
    K(x₁, x₂, ψ₁, ψ₂) = exp(ψ₁ * cos(abs(x₁-x₂) / ψ₂)) # periodic kernel
    θ = (1:Ny) / Ny * π # theta for gabor
    C = K.(θ', θ, ψ₁, ψ₂) # create covariance matrix
    C += eps * I # regularization to make C positive definite
    C_min, C_max = minimum(C), maximum(C)
    C = C_range[1] .+ (C_range[2]-C_range[1]) * (C .- C_min) / (C_max - C_min)
    return Symmetric(C); # make symmetric matrix using upper triangular matrix
end;
\end{lstlisting}
\begin{lstlisting}[language=julia]
C = get_C(Ny)

figure(figsize=(3,2))
title(L"$\mathbf{C}$")
ims = imshow(C, origin="lower", cmap="bwr", vmin=-4, vmax=4, extent=(-90, 90, -90, 90))
xticks([-90,0,90]); yticks([-90,0,90]); 
xlabel(L"$\theta$ (Pref. ori)"); ylabel(L"$\theta$ (Pref. ori)")
colorbar(ims);
tight_layout()
\end{lstlisting}
\begin{figure}[ht]
	\centering
	\includegraphics[scale=0.8, max width=\linewidth]{./fig/bayesian-brain/neural-sampling/cell010.png}
	\caption{cell010.png}
	\label{cell010.png}
\end{figure}
ここでPref. oriは最適方位 (preferred orientation)を意味する.
\subsubsection{事後分布の計算}
事後分布は$z$と$\mathbf{y}$のそれぞれについて次のように求められる.
\begin{align}
p(z \mid \mathbf{x}) &\propto p(z) \mathcal{N}\left(0, z^{2} \mathbf{A C A}^{\top}+\sigma_{x}^{2} \mathbf{I}\right)\\
p(\mathbf{y} \mid z, \mathbf{x})& = \mathcal{N}\left(\mu(z, \mathbf{x}), \Sigma(z)\right)
\end{align}
ただし,
\begin{align}
\Sigma(z)&=\left(\mathbf{C}^{-1}+\frac{z^{2}}{\sigma_{x}^{2}} \mathbf{A}^{\top} \mathbf{A}\right)^{-1}\\
\mu(z, \mathbf{x})&=\frac{z}{\sigma_{x}^{2}} \Sigma(z) \mathbf{A}^{\top} \mathbf{x}
\end{align}
である.
最終的な予測において$z$の事後分布は必要でないため,$p(\mathbf{y} \mid z, \mathbf{x})$から$z$を消去することを考えよう.厳密に行う場合,次式のように周辺化(marginalization)により,$z$を (積分) 消去する必要がある.
\begin{equation}
p(\mathbf{y} \mid \mathbf{x}) = \int dz\ p(z\mid \mathbf{x})\cdot p(\mathbf{y} \mid z, \mathbf{x})
\end{equation}
周辺化においては,まず$z$のMAP推定 (最大事後確率推定) 値 $z_{\mathrm{MAP}}$を求める.
\begin{equation}
z_{\mathrm{MAP}} = \underset{z}{\operatorname{argmax}} p(z\mid \mathbf{x})
\end{equation}
次に$z_{\mathrm{MAP}}$の周辺で$p(z\mid \mathbf{x})$を積分し,積分値が一定の閾値を超える$z$の範囲を求め,この範囲で$z$を積分消去してやればよい.しかし,$z$は単一のスカラー値であり,この手法で推定するのは煩雑であるために近似手法が\citep{Echeveste2017-wu}において提案されている.Echevesteらは第一の近似として,$z$の分布を$z_{\mathrm{MAP}}$でのデルタ関数に置き換える,すなわち,$p(z\mid \mathbf{x})\simeq \delta (z-z_{\mathrm{MAP}})$とすることを提案している.この場合,$z$は定数とみなせ,$p(\mathbf{y} \mid \mathbf{x})\simeq p(\mathbf{y} \mid \mathbf{x}, z=z_{\mathrm{MAP}})$となる.第二の近似として,$z_{\mathrm{MAP}}$を真のコントラスト$z^*$で置き換えることが提案されている.GSMへの入力$\mathbf{x}$は元の画像を$\mathbf{\tilde x}$とすると,$\mathbf{x}=z^* \mathbf{\tilde x}$としてスケーリングされる.この入力の前処理の際に用いる$z^*$を用いてしまおうということである.この場合,$p(\mathbf{y} \mid \mathbf{x})\simeq p(\mathbf{y} \mid \mathbf{x}, z=z^*)$となる.しかし,入力を任意の画像とする場合,$z^*$は未知である.簡便さと精度のバランスを取り,ここでは第一の近似,$z=z_{\mathrm{MAP}}$とする手法を用いることにする.
\begin{lstlisting}[language=julia]
# log pdf of p(z)
log_Pz(z, k, θ) = logpdf.(Gamma(k, θ), z)

# pdf of p(z|x)
function Pz_x(z_range, x, ACAᵀ, σₓ², k, θ)
    n_contrasts = length(z_range)
    log_p = zeros(n_contrasts)
    μxz = zeros(size(x))
    dz = z_range[2] - z_range[1]
    for i in 1:n_contrasts
        Cxz = z_range[i]^2 * ACAᵀ + σₓ² * I
        log_p[i] = log_Pz(z_range[i], k, θ) + logpdf(MvNormal(μxz, Symmetric(Cxz)), x)
    end
    p = exp.(log_p .- maximum(log_p)) # for numerical stability
    p /= sum(p) * dz
    return p
end;
\end{lstlisting}
\begin{lstlisting}[language=julia]
# mean and covariance matrix of p(y|x, z)
function post_moments(x, z, σₓ², A, AᵀA, C⁻¹)
    Σz = inv(C⁻¹ + (z^2 / σₓ²) * AᵀA)
    μzx = (z/σₓ²) * Σz * A' * x
    return μzx, Σz
end;
\end{lstlisting}
\subsubsection{シミュレーション}
\begin{lstlisting}[language=julia]
AᵀA = A' * A
ACAᵀ = A * C * A'

σₓ = 1.0 # Noise of the x process
σₓ² = σₓ^2
k, θ = 2.0, 2.0 # Parameter of the gamma dist. for z (Shape, Scale)

C⁻¹ = inv(C); # inverse of C
\end{lstlisting}
入力データの作成
\begin{lstlisting}[language=julia]
Z = [0.0, 0.25, 0.5, 1.0, 2.0] # true contrasts z^*
n_samples = size(Z)[1]
y = rand(MvNormal(zeros(Ny), C), 1) # sampling from P(y)=N(0, C)
X = stack([rand(MvNormal(vec(z*A*y), σₓ*I)) for z in Z])';
\end{lstlisting}
\begin{lstlisting}[language=julia]
x_min, x_max = minimum(X), maximum(X)

figure(figsize=(4,2))
for s in 1:n_samples
    subplot(1, n_samples, s)
    title(L"$z$: "*string(Z[s]))
    imshow(reshape(X[s, :], WH, WH), vmin=x_min, vmax=x_max, cmap="gray")
    axis("off")
end
tight_layout()
\end{lstlisting}
\begin{figure}[ht]
	\centering
	\includegraphics[scale=0.8, max width=\linewidth]{./fig/bayesian-brain/neural-sampling/cell019.png}
	\caption{cell019.png}
	\label{cell019.png}
\end{figure}
事後分布の計算をする.
\begin{lstlisting}[language=julia]
μ_post = zeros(n_samples, Ny)
σ_post = zeros(n_samples, Ny)
Σ_post = zeros(n_samples, Ny, Ny)

z_range = range(0, 5.0, length=100) # range of z for MAP estimation
Z_MAP = zeros(n_samples) 

for s in 1:n_samples
    p_z = Pz_x(z_range, X[s, :], ACAᵀ, σₓ², k, θ)
    Z_MAP[s] = z_range[argmax(p_z)] # MAP estimated z
    μ_post[s, :], Σ_post[s, :, :] = post_moments(X[s, :], Z_MAP[s], σₓ², A, AᵀA, C⁻¹)
    σ_post[s, :] = sqrt.(diag(Σ_post[s, :, :]))
end
\end{lstlisting}
結果
\begin{lstlisting}[language=julia]
θs = range(-90, 90, length=Ny)
cm = get_cmap(:Greens) # get color map
cms = cm.((1:n_samples)/n_samples) # color list

fig, ax = subplots(1, 3, figsize=(7.5, 2))
ax[1].scatter(Z, Z_MAP, c=cms)
ax[1].plot(Z, Z_MAP, color="tab:gray", zorder=0)
ax[1].set_xlabel(L"$z$"); ax[1].set_ylabel(L"$z_{MAP}$"); 
for s in 1:n_samples
    ax[2].plot(θs, μ_post[s, :], color=cms[s])
    ax[3].plot(θs, σ_post[s, :], color=cms[s], label=L"$z$ : "*string(Z[s]))
end
ax[2].set_ylabel(L"$\mu$"); ax[3].set_ylabel(L"$\sigma$")
for i in 2:3
    ax[i].set_xticks([-90,0,90])
    ax[i].set_xlabel(L"$\theta$ (Pref. ori)")
end
ax[3].legend(bbox_to_anchor=(1.05, 1), loc="upper left", borderaxespad=0)
tight_layout()
\end{lstlisting}
\begin{figure}[ht]
	\centering
	\includegraphics[scale=0.8, max width=\linewidth]{./fig/bayesian-brain/neural-sampling/cell023.png}
	\caption{cell023.png}
	\label{cell023.png}
\end{figure}
\begin{lstlisting}[language=julia]
fig, ax = subplots(1, n_samples, figsize=(7.5, 1), sharex="all", sharey="all")
for s in 1:n_samples
    ax[s].set_title(L"$z$ : "*string(Z[s]))
    ims = ax[s].imshow(Σ_post[s, :, :], origin="lower", cmap="bwr", extent=(-90, 90, -90, 90), vmin=-1, vmax=1)
    ax[s].set_xticks([-90,0,90]); ax[s].set_yticks([-90,0,90]);
    if s == 1
        ax[s].set_ylabel(L"$\theta$ (Pref. ori)")
    elseif s == ceil(Int, n_samples/2) 
        ax[s].set_xlabel(L"$\theta$ (Pref. ori)"); 
    end
end
fig.colorbar(ims, ax=ax[n_samples]);
\end{lstlisting}
\begin{figure}[ht]
	\centering
	\includegraphics[scale=0.8, max width=\linewidth]{./fig/bayesian-brain/neural-sampling/cell024.png}
	\caption{cell024.png}
	\label{cell024.png}
\end{figure}
出力のサンプリングを行う.
\begin{lstlisting}[language=julia]
membrane_potential(y, α=2.4, β=1.9, γ=0.6) = α * max(0, y+β)^γ
\end{lstlisting}
事後分布から応答をサンプリングする.
\begin{lstlisting}[language=julia]
nt = 1000
h_gsm = zeros(n_samples, Ny, nt)
for s in 1:n_samples
    μ = μ_post[s, :]
    Σ = Σ_post[s, :, :]
    sample = rand(MvNormal(μ, Symmetric(Σ)), nt)
    h_gsm[s, :, :] = membrane_potential.(sample)
end
\end{lstlisting}
\begin{lstlisting}[language=julia]
# modified from https://matplotlib.org/stable/gallery/statistics/confidence_ellipse.html
function confidence_ellipse(x, y, ax, n_std=3, alpha=1, facecolor="none", edgecolor="tab:gray")
    pearson = cor(x,y)
    rx, ry = sqrt(1 + pearson), sqrt(1 - pearson)
    ellipse = matplotlib.patches.Ellipse((0, 0), width=2*rx, height=2*ry, alpha=alpha, 
        fc=facecolor, ec=edgecolor, lw=2, zorder=0)
    scales = [std(x), std(y)] * n_std
    means = [mean(x), mean(y)]
    transf = matplotlib.transforms.Affine2D().rotate_deg(45).scale(scales...).translate(means...)
    ellipse.set_transform(transf + ax.transData)
    return ax.add_patch(ellipse)
end;
\end{lstlisting}
\begin{lstlisting}[language=julia]
fig, ax = subplots(figsize=(4, 3))
unit_idx = [1, 25]
for s in 1:n_samples
    h₁, h₂ = h_gsm[s, unit_idx[1], :], h_gsm[s, unit_idx[2], :]
    ax.plot(h₁[1:15], h₂[1:15], marker="o", markersize=5, alpha=0.5, color=cms[s], label=L"$z=$"*string(Z[s]))
    confidence_ellipse(h₁, h₂, ax, 3, 1, "none", cms[s])
end
ax.set_xlabel("Neuron #"*string(unit_idx[1])); ax.set_ylabel("Neuron #"*string(unit_idx[2]))
axins = [ax.inset_axes([0.85, -0.25,0.15,0.15]), ax.inset_axes([-0.3,0.85,0.15,0.15])]
for i in 1:2
    axins[i].imshow(reshape(A[:,unit_idx[i]], WH, WH), cmap="gray")
    axins[i].axis("off")
end
ax.set_aspect("equal", "box")
ax.legend(bbox_to_anchor=(1.05, 1), loc="upper left", borderaxespad=0)
tight_layout()
\end{lstlisting}
\begin{figure}[ht]
	\centering
	\includegraphics[scale=0.8, max width=\linewidth]{./fig/bayesian-brain/neural-sampling/cell030.png}
	\caption{cell030.png}
	\label{cell030.png}
\end{figure}
\subsection{興奮性・抑制性神経回路によるサンプリング}
前節で実装したMCMCを\textbf{興奮性・抑制性神経回路 (excitatory-inhibitory (E-I) network)}\index{こうふんせい・よくせいせいしんけいかいろ (excitatory-inhibitory (E-I) network)@興奮性・抑制性神経回路 (excitatory-inhibitory (E-I) network)} で実装する.HMCとLMCの両方を神経回路で実装する.ハミルトニアンを用いる場合,一般化座標$\mathbf{q}$を興奮性神経細胞の活動$\mathbf{u}$, 一般化運動量$\mathbf{p}$を抑制性神経細胞の活動$\mathbf{v}$に対応させる.$\mathbf{u,\ v}$は同じ次元のベクトルとする.$\mathbf{u}, \mathbf{v}$の時間発展はハミルトニアン$\mathcal{H}$を導入して
\begin{equation}
\tau\frac{d\mathbf{u}}{dt} = \frac{\partial \mathcal{H}}{\partial\mathbf{v}},\quad\tau\frac{d\mathbf{v}}{dt} = - \frac{\partial \mathcal{H}}{\partial\mathbf{u}}
\end{equation}
と書ける.一般的には$\mathcal{H}(\mathbf{u}, \mathbf{v}) = E\left( \mathbf{u} \right) + \frac{1}{2}\mathbf{v}^{\top}\mathbf{v}$であり,$p\left( \mathbf{u},\ \mathbf{v} \right) \propto \exp( - \mathcal{H}(\mathbf{u,v}))$である.力学的エネルギーを保つ運動は,対数同時分布における等値線上の運動と同じである.
\citep{Aitchison2016-xu}では
\begin{equation}
\mathcal{H}(\mathbf{u}, \mathbf{v}) = \log p \left(\mathbf{u}, \mathbf{v} \right) + \textrm{Const.} = \log p \left(\mathbf{v} \middle| \mathbf{u} \right) + \log p\left(\mathbf{u} \right) + \textrm{Const.}
\end{equation}
とし,$p\left( \mathbf{v} \middle| \mathbf{u} \right)\mathcal{= N}\left( \mathbf{v};\mathbf{Bu},\ \mathbf{M}^{- 1} \right),\ \ p\left( \mathbf{u} \right) = \mathcal{N\ (}\mathbf{0},\ \mathbf{C}^{- 1})$としている.この場合,
\begin{align}
\frac{d\mathbf{u}}{dt} &= \frac{1}{\tau}\frac{\partial \mathcal{H}}{\partial\mathbf{v}} = \frac{1}{\tau}\frac{\partial\log{p\left( \mathbf{u},\ \mathbf{v} \right)}}{\partial\mathbf{v}} = \ \frac{1}{\tau}\frac{\partial\log{p\left( \mathbf{v} \middle| \mathbf{u} \right)}}{\partial\mathbf{v}}\\
\frac{d\mathbf{v}}{dt} &= - \frac{1}{\tau}\frac{\partial \mathcal{H}}{\partial\mathbf{u}} = - \frac{1}{\tau}\frac{\partial\log{p\left( \mathbf{u},\ \mathbf{v} \right)}}{\partial\mathbf{u}} = \  - \frac{1}{\tau}\frac{\partial\log{p\left( \mathbf{v} \middle| \mathbf{u} \right)}}{\partial\mathbf{u}} - \frac{1}{\tau}\frac{\partial\log{p\left( \mathbf{u} \right)}}{\partial\mathbf{u}}
\end{align}
となる.このままでは等値線上を運動することになるので,Langevinダイナミクスを付け加える.
\begin{align}
\frac{d\mathbf{u}}{dt} &= \frac{1}{\tau}\frac{\partial\log{p\left( \mathbf{v} \middle| \mathbf{u} \right)}}{\partial\mathbf{v}} + \frac{1}{\tau_{L}}\frac{\partial\log{p\left( \mathbf{u},\ \mathbf{v} \right)}}{\partial\mathbf{u}} + \sqrt{\frac{2}{\tau_{L}}}\ d\eta\\
&= \frac{1}{\tau}\frac{\partial\log{p\left( \mathbf{v} \middle| \mathbf{u} \right)}}{\partial\mathbf{v}} + \frac{1}{\tau_{L}}\frac{\partial\log{p\left( \mathbf{v|u} \right)}}{\partial\mathbf{u}} + \frac{1}{\tau_{L}}\frac{\partial\log{p\left( \mathbf{u} \right)}}{\partial\mathbf{u}} + \sqrt{\frac{2}{\tau_{L}}}\ d\eta\\
\frac{d\mathbf{v}}{dt} &= - \frac{1}{\tau}\frac{\partial\log{p\left( \mathbf{v} \middle| \mathbf{u} \right)}}{\partial\mathbf{u}} - \frac{1}{\tau}\frac{\partial\log{p\left( \mathbf{u} \right)}}{\partial\mathbf{u}} + \frac{1}{\tau_{L}}\frac{\partial\log{p\left( \mathbf{u},\mathbf{v} \right)}}{\partial\mathbf{v}} + \sqrt{\frac{2}{\tau_{L}}}\ d\eta\\
&= - \frac{1}{\tau}\frac{\partial\log{p\left( \mathbf{v} \middle| \mathbf{u} \right)}}{\partial\mathbf{u}} + \frac{1}{\tau_{L}}\frac{\partial\log{p\left( \mathbf{v|u} \right)}}{\partial\mathbf{v}} - \frac{1}{\tau}\frac{\partial\log{p\left( \mathbf{u} \right)}}{\partial\mathbf{u}} + \sqrt{\frac{2}{\tau_{L}}}\ d\eta
\end{align}
となる.それぞれの項は
\begin{align}
\frac{\partial\log{p\left( \mathbf{v} \middle| \mathbf{u} \right)}}{\partial\mathbf{v}} &= \mathbf{B}^{\top}\mathbf{M}\left( \mathbf{Bu} - \mathbf{v} \right)\\
\frac{\partial\log{p\left( \mathbf{v} \middle| \mathbf{u} \right)}}{\partial\mathbf{u}} &= - \mathbf{M}\left( \mathbf{Bu} - \mathbf{v} \right)\\
\frac{\partial\log{p\left( \mathbf{u} \right)}}{\partial\mathbf{u}} &= - \mathbf{Cu}
\end{align}
であるので,
\begin{align}
\frac{d\mathbf{u}}{dt} &= \frac{1}{\tau}\mathbf{B}^{\top}\mathbf{M}\left( \mathbf{Bu} - \mathbf{v} \right) - \frac{1}{\tau_{L}}\mathbf{M}\left( \mathbf{Bu} - \mathbf{v} \right) - \frac{1}{\tau_{L}}\mathbf{Cu} + \sqrt{\frac{2}{\tau_{L}}}\ d\eta\\
\frac{d\mathbf{v}}{dt} &= \frac{1}{\tau}\mathbf{M}\left( \mathbf{Bu} - \mathbf{v} \right) + \frac{1}{\tau_{L}}\mathbf{B}^{\top}\mathbf{M}\left( \mathbf{Bu} - \mathbf{v} \right) + \frac{1}{\tau}\mathbf{Cu} + \sqrt{\frac{2}{\tau_{L}}}\ d\eta
\end{align}
となる.$\mathbf{B = I}$ とすると,
\begin{align}
\frac{d\mathbf{u}}{dt} &= \frac{1}{\tau}\mathbf{M}\left( \mathbf{u} - \mathbf{v} \right) - \frac{1}{\tau_{L}}\mathbf{M}\left( \mathbf{u} - \mathbf{v} \right) - \frac{1}{\tau_{L}}\mathbf{Cu} + \sqrt{\frac{2}{\tau_{L}}}\ d\eta\\
&= \left\lbrack \left( \frac{1}{\tau} - \frac{1}{\tau_{L}} \right)\mathbf{M} - \frac{1}{\tau_{L}}\mathbf{C} \right\rbrack\mathbf{u} - \left( \frac{1}{\tau} - \frac{1}{\tau_{L}} \right)\mathbf{Mv} + \sqrt{\frac{2}{\tau_{L}}}\ d\eta\\
\frac{d\mathbf{v}}{dt} &= \frac{1}{\tau}\mathbf{M}\left( \mathbf{u} - \mathbf{v} \right) + \frac{1}{\tau_{L}}\mathbf{M}\left( \mathbf{u} - \mathbf{v} \right) + \frac{1}{\tau}\mathbf{Cu} + \sqrt{\frac{2}{\tau_{L}}}\ d\eta\\
&= \left\lbrack \left( \frac{1}{\tau} + \frac{1}{\tau_{L}} \right)\mathbf{M} + \frac{1}{\tau_{L}}\mathbf{C} \right\rbrack\mathbf{u} - \left( \frac{1}{\tau} + \frac{1}{\tau_{L}} \right)\mathbf{Mv} + \sqrt{\frac{2}{\tau_{L}}}\ d\eta
\end{align}
となり,$\mathbf{u}\mathbf{,\ v}$と定行列およびノイズに依存してサンプリングダイナミクスを記述できる.長々と式変形を書いたが,重要なのは\textbf{興奮性・抑制性という2種類の細胞群の相互作用により生み出された振動を用いてサンプリングにおける自己相関を下げることができる}\index{こうふんせい・よくせいせいという2しゅるいのさいぼうぐんのそうごさようによりうみだされたしんどうをもちいてさんぷりんぐにおけるじこそうかんをさげることができる@興奮性・抑制性という2種類の細胞群の相互作用により生み出された振動を用いてサンプリングにおける自己相関を下げることができる}という点である.
簡単のため,前項で用いた入力刺激のうち,最も$z$が大きいサンプルのみを使用する.
\begin{lstlisting}[language=julia]
dt = 1e-2 # ms
τ, τl = 10.0, 150.0 # ms
α_in = [1/τ - 1/τl, 1/τ + 1/τl]
α_ext = [1/τl, -1/τ]
ρ = sqrt(2*dt/τl);

nt = 50000
M = cat(ones(1,1), C; dims=(1,2));
x_idx = n_samples # get last x
x = X[x_idx, :]
u_init = [1; zeros(Ny)];
\end{lstlisting}
\begin{lstlisting}[language=julia]
function ∇ᵤlogP(u, x, σₓ², A, C⁻¹)
    z, y = abs(u[1]), u[2:end]
    pred_error = A' * (x - z*A*y) / σₓ² # prediction error signal
    du = zeros(size(u))
    du[1] = sign(u[1]) * (y' * pred_error - z)
    du[2:end] = z * pred_error - C⁻¹*y
    return du
end
\end{lstlisting}
\begin{lstlisting}[language=julia]
∇log_p(u) = ∇ᵤlogP(u, x, σₓ², A, C⁻¹);
\end{lstlisting}
\begin{lstlisting}[language=julia]
function neural_lmc(∇log_p::Function, u_init::Vector{Float64}, α::Float64, ρ::Float64, dt::Float64, nt::Int)
    d = length(u_init)
    u = zeros(nt, d)
    u[1, :] = u_init

    for t in 1:nt-1
        I_ext = ∇log_p(u[t, :]) # external input
        u[t+1, :] = u[t, :] + dt * (α * I_ext) + ρ * randn(d) 
    end
    return u
end

function neural_hmc(∇log_p::Function, u_init::Vector{Float64}, M::Matrix{Float64}, 
        α_in::Vector{Float64}, α_ext::Vector{Float64}, ρ::Float64, dt::Float64, nt::Int)
    d = length(u_init)
    u, v = zeros(nt, d), zeros(nt, d)
    u[1, :] = u_init

    for t in 1:nt-1
        I_ext = ∇log_p(u[t, :]) # external input
        I_in = M * (u[t, :] - v[t, :]) # internal input
        u[t+1, :] = u[t, :] + dt * (α_in[1] * I_in + α_ext[1] * I_ext) + ρ * randn(d) 
        v[t+1, :] = v[t, :] + dt * (α_in[2] * I_in + α_ext[2] * I_ext) + ρ * randn(d)
    end
    return u, v
end;
\end{lstlisting}
\begin{lstlisting}[language=julia]
@time u_nlmc = neural_lmc(∇log_p, u_init, α_ext[1], ρ, dt, nt);
\end{lstlisting}
\begin{lstlisting}[language=julia]
@time u_nhmc, v_nhmc = neural_hmc(∇log_p, u_init, M, α_in, α_ext, ρ, dt, nt);
\end{lstlisting}
初めの100msはburn-in期間として除く.またダウンサンプリングする.
\begin{lstlisting}[language=julia]
L = 100
burn_in = 10000
mcmc_time = (burn_in*dt):(L*dt):(nt*dt); # time for plot
\end{lstlisting}
\begin{lstlisting}[language=julia]
mean_nlmc = mean(u_nlmc[burn_in:L:end, 2:end], dims=2); # pseudo-LFP
mean_nhmc = mean(u_nhmc[burn_in:L:end, 2:end], dims=2);

autocorr_nlmc = autocor(mean_nlmc, 1:length(mean_nlmc)-1);
autocorr_nhmc = autocor(mean_nhmc, 1:length(mean_nhmc)-1);
\end{lstlisting}
$z$の推定過程を描画する.また,$z$を除いた$\mathbf{u}$を平均化し,自己相関の度合いを確認する.
\begin{lstlisting}[language=julia]
fig, ax = subplots(1,2,figsize=(6,2.5),sharex="all")
ax[1].plot(mcmc_time, u_nhmc[burn_in:L:end, 1], color="tab:red")
ax[1].plot(mcmc_time, u_nlmc[burn_in:L:end, 1], color="tab:blue")
ax[1].axhline(Z[x_idx], linestyle="dashed", color="tab:gray", alpha=0.5)
ax[1].set_xlabel("Time (ms)"); ax[1].set_ylabel(L"Estimated $z$")
ax[1].set_xlim(mcmc_time[1], mcmc_time[end])

ax[2].plot(mcmc_time[1:end-1], autocorr_nhmc, color="tab:red", label="Hamiltonian")
ax[2].plot(mcmc_time[1:end-1], autocorr_nlmc, color="tab:blue", label="Langevin")
ax[2].set_xlabel("Time (ms)"); ax[2].set_ylabel("Autocorrelation")
ax[2].set_xlim(mcmc_time[1], mcmc_time[end])
ax[2].axhline(0, linestyle="dashed", color="tab:gray", alpha=0.5)
ax[2].legend()
fig.tight_layout()
\end{lstlisting}
\begin{figure}[ht]
	\centering
	\includegraphics[scale=0.8, max width=\linewidth]{./fig/bayesian-brain/neural-sampling/cell045.png}
	\caption{cell045.png}
	\label{cell045.png}
\end{figure}
Hamiltonianネットワークは自己相関を振動により低下させることで,効率の良いサンプリングを実現している.ToDo: 普通にMCMCやる場合も自己相関は確認したほうがいいという話をどこかに書く.
推定された事後分布を特定の神経細胞のペアについて確認する.
\begin{lstlisting}[language=julia]
h_nhmc = membrane_potential.(u_nhmc[burn_in:L:end, :])
h_nlmc = membrane_potential.(u_nlmc[burn_in:L:end, :])

kde_bound = ((-3,5),(0,8)) # ((xlo,xhi),(ylo,yhi))
U_gsm = kde((h_gsm[x_idx, unit_idx[1], :], h_gsm[x_idx, unit_idx[2], :]), boundary=kde_bound)
U_nhmc = kde((h_nhmc[:, unit_idx[1]+1], h_nhmc[:, unit_idx[2]+1]), boundary=kde_bound)
U_nlmc = kde((h_nlmc[:, unit_idx[1]+1], h_nlmc[:, unit_idx[2]+1]), boundary=kde_bound);
\end{lstlisting}
\begin{lstlisting}[language=julia]
fig, ax = plt.subplots(1,3, figsize=(6, 2.5), sharey="all", sharex="all")
ax[1].contourf(U_gsm.x, U_gsm.x, U_gsm.density)
ax[1].set_title("Actual")
ax[2].contourf(U_nhmc.x, U_nhmc.x, U_nhmc.density)
ax[2].set_title("Hamiltonian")
ax[3].contourf(U_nlmc.x, U_nlmc.x, U_nlmc.density)
ax[3].set_title("Langevin")
ax[1].set_ylabel("Neuron #"*string(unit_idx[2]))
ax[2].set_xlabel("Neuron #"*string(unit_idx[1])) 
fig.tight_layout()
\end{lstlisting}
\begin{figure}[ht]
	\centering
	\includegraphics[scale=0.8, max width=\linewidth]{./fig/bayesian-brain/neural-sampling/cell048.png}
	\caption{cell048.png}
	\label{cell048.png}
\end{figure}
Hamiltonianネットワークの方が安定して事後分布を推定することができている.ToDo: 以下の記述.ここでは重みを設定したが, \citep{Echeveste2020-sh}ではRNNにBPTTで重みを学習させている.動的な入力に対するサンプリング \citep{Berkes2011-xj}.burn-inがなくなり効率良くサンプリングできる.
\subsection{Spikingニューラルネットワークにおけるサンプリング}
前項で挙げた例は発火率モデルであったが,SNNにおいてサンプリングを実行する機構自体は考案されている.ToDo: 以下の記述.\citep{Buesing2011-dm}\citep{Masset2022-wh}\citep{Zhang2022-bl}
\subsection{シナプスサンプリング}
ここまでシナプス結合強度は変化せず,神経活動の変動によりサンプリングを行うというモデルについて考えてきた.一方で,シナプス結合強度自体が短時間で変動することによりベイズ推論を実行するというモデルがあり,\textbf{シナプスサンプリング(synaptic sampling)}\index{しなぷすさんぷりんぐ(synaptic sampling)@シナプスサンプリング(synaptic sampling)} と呼ばれる.ToDo: 以下の記述.\citep{Kappel2015-kq}\citep{Aitchison2021-wo}
 % ok2
% \section{予測符号化}
\subsection{観測世界の階層的予測}
\textbf{階層的予測符号化(hierarchical predictive coding; HPC)} は\cite{Rao1999-zv}により導入された.構築するネットワークは入力層を含め,3層のネットワークとする.LGNへの入力として画像 $\mathbf{x} \in \mathbb{R}^{n_0}$を考える.画像 $\mathbf{x}$ の観測世界における隠れ変数,すなわち\textbf{潜在変数} (latent variable)を$\mathbf{r} \in \mathbb{R}^{n_1}$とし,ニューロン群によって発火率で表現されているとする (真の変数と $\mathbf{r}$は異なるので文字を分けるべきだが簡単のためにこう表す).このとき,


\mathbf{x} = f(\mathbf{U}\mathbf{r}) + \boldsymbol{\epsilon}


が成立しているとする.ただし,$f(\cdot)$は活性化関数 (activation function),$\mathbf{U} \in \mathbb{R}^{n_0 \times n_1}$は重み行列である.
$\boldsymbol{\epsilon} \in \mathbb{R}^{n_0}$ は $\mathcal{N}(\mathbf{0}, \sigma^2 \mathbf{I})$ からサンプリングされるとする.

潜在変数 $\mathbf{r}$はさらに高次 (higher-level)の潜在変数 $\mathbf{r}^h$により,次式で表現される.


\mathbf{r} = \mathbf{r}^{td}+\boldsymbol{\epsilon}^{td}=f(\mathbf{U}^h \mathbf{r}^h)+\boldsymbol{\epsilon}^{td}


ただし,Top-downの予測信号を $\mathbf{r}^{td}:=f(\mathbf{U}^h \mathbf{r}^h)$とした.また,$\mathbf{r}^{td} \in \mathbb{R}^{n_1}$, $\mathbf{r}^{h} \in \mathbb{R}^{n_2}$, $\mathbf{U}^h \in \mathbb{R}^{n_1 \times n_2}$ である.
$\boldsymbol{\epsilon}^{td} \in \mathbb{R}^{n_1}$は$\mathcal{N}(\mathbf{0}$, $\sigma_{td}^2 \mathbf{I}$) からサンプリングされるとする.

話は飛ぶが,Predictive codingのネットワークの特徴は
\begin{itemize}
\item 階層的な構造
\item 高次による低次の予測 (Feedback or Top-down信号)
\item 低次から高次への誤差信号の伝搬 (Feedforward or Bottom-up 信号)
\end{itemize}

である.ここまでは高次表現による低次表現の予測,というFeedback信号について説明してきたが,この部分はSparse codingでも同じである.それではPredictive codingのもう一つの要となる,低次から高次への予測誤差の伝搬というFeedforward信号はどのように導かれるのだろうか.結論から言えば,これは\textbf{復元誤差 (reconstruction error)の最小化を行う再帰的ネットワーク (recurrent network)を考慮することで自然に導かれる}.

\lstinputlisting[language=julia]{./text/bayesian-brain/probabilistic-population-coding/001.jl}
\lstinputlisting[language=julia]{./text/bayesian-brain/probabilistic-population-coding/002.jl}
\lstinputlisting[language=julia]{./text/bayesian-brain/probabilistic-population-coding/003.jl}

% \section{分位点・エクスペクタイル回帰による分布符号化}
本章では分位点・エクスペクタイル回帰 (quantile/expectile regression) を用いて

\begin{itemize}
\item Quantileはノンパラ
\item PPCやDPCはパラメトリック
\end{itemize}

Distributional Reinforcement Learning in the Brainに
> Quantile-like codes are non-parametric codes, as they do not a priori assume a specific form of a probability distribution with associated parameters. Previous studies have proposed different population coding schemes. For example, probabilistic population codes (PPCs) [73,74] and distributed distributional codes (DDCs) [75,76] employ population coding schemes from which various statistical parameters of a distribution can be read out, making them parametric codes. As a simple example, a PPC might encode a Gaussian distribution, in which case the mean would be reflected in which specific neurons are most active, and the variance would be reflected in the inverse of the overall activity [73].
\subsection{分位点・エクスペクタイル回帰}
\subsubsection{分位点回帰 (Quantile Regression)}
線形回帰(linear regression)は,誤差が正規分布と仮定したとき(必ずしも正規分布を仮定しなくてもよい)の$X$(説明変数)に対する$Y$(目的変数)の期待値$E[Y]$を求める,というものであった.\textbf{分位点回帰(quantile regression)}\index{ぶんいてんかいき(quantile regression)@分位点回帰(quantile regression)} では,Xに対するYの分布における分位点を通るような直線を引く.

\textbf{分位点}\index{ぶんいてん@分位点}(または分位数)において,代表的なものが\textbf{四分位数}\index{よんぶんいすう@四分位数}である.四分位数は箱ひげ図などで用いるが,例えば第一四分位数は分布を25:75に分ける数,第二四分位数(中央値)は分布を50:50に分ける数である.同様に$q$分位数($q$-quantile)というと分布を$q:1-q$に分ける数となっている.分位点回帰の話に戻る.下図は$x\sim U(0, 5),\quad y=3x+x\cdot \xi,\quad \xi\sim N(0,1)$とした500個の点に対する分位点回帰である.赤い領域はX=1,2,3,4でのYの分布を示している.深緑,緑,黄色の直線はそれぞれ10, 50, 90%tile回帰の結果である.例えば50%tile回帰の結果は,Xが与えられたときのYの中央値(50%tile点)を通るような直線となっている.同様に90%tile回帰の結果は90%tile点を通るような直線となっている.
\lstinputlisting[language=julia]{./text/bayesian-brain/quantile-expectile-regression/002.jl}
\lstinputlisting[language=julia]{./text/bayesian-brain/quantile-expectile-regression/003.jl}
\begin{figure}[ht]
	\centering
	\includegraphics[scale=0.8, max width=\linewidth]{./fig/synapse-model/expo-synapse/cell003.png}
	\caption{cell003.png}
	\label{cell003.png}
\end{figure}
分位点回帰の利点としては,外れ値に対して堅牢(ロバスト)である,Yの分布が非対称である場合にも適応できる,などがある ([Das et al., *Nat Methods*. 2019](https://www.nature.com/articles/s41592-019-0406-y)).
\subsubsection{エクスペクタイル回帰 (Expectile regression)}
エクスペクタイル(expectile)は([Newey and Powell 1987](https://www.jstor.org/stable/1911031?seq=1)) によって導入された統計汎関数 (statistical functional; SF)の一種であり,期待値(expectation)と分位数(quantile)を合わせた概念である.簡単に言えば,中央値(median)の一般化が分位数(quantile)であるのと同様に,期待値(expectation)の一般化がエクスペクタイル(expectile)である.
\subsubsection{勾配法を用いた分位点回帰・エクスペクタイル回帰}
予測誤差$\delta$と$\tau$の関数を


\begin{align}
\text{分位点回帰:}&\quad
\rho_q(\delta; \tau)=\left|\tau-\mathbb{I}_{\delta \leq 0}\right|\cdot |\delta|=\left(\tau-\mathbb{I}_{\delta \leq 0}\right)\cdot \delta\\
\text{エクスペクタイル回帰:}&\quad
\rho_e(\delta; \tau)=\left|\tau-\mathbb{I}_{\delta \leq 0}\right|\cdot \delta^2
\end{align}


と定義する.$\rho_q(\delta; \tau)$のみ,チェック関数 (check function)あるいは非対称絶対損失関数(asymmetric absolute loss function)と呼ぶ.ただし,$\tau$は分位点(quantile),$\mathbb{I}$は指示関数(indicator function)である.この場合,$\mathbb{I}_{\delta \leq 0}$は$\delta \gt 0$なら0, $\delta \leq 0$なら1となる.このとき,目的関数は 


L_{\tau}(\delta)
=\sum_{i=1}^n \rho(\delta_i; \tau)


である.$\rho(\delta; \tau)$を色々な $\tau$についてplotすると次図のようになる.
\lstinputlisting[language=julia]{./text/bayesian-brain/quantile-expectile-regression/007.jl}
\begin{figure}[ht]
	\centering
	\includegraphics[scale=0.8, max width=\linewidth]{./fig/bayesian-brain/neural-sampling/cell007.png}
	\caption{cell007.png}
	\label{cell007.png}
\end{figure}
分位点の場合,$\rho_q(\delta; \tau)$がチェックマーク✓に類似していることからこのような名前が付いている.
$L_\tau$を最小化するような$\theta$の更新式について考える.まず,



\begin{align}
\text{分位点回帰:}&\quad
\frac{\partial \rho_q(\delta; \tau)}{\partial \delta}= \rho_q^{\prime}(\delta; \tau)=\left|\tau-\mathbb{I}_{\delta \leq 0}\right| \cdot
\operatorname{sign}(\delta)\\
\text{エクスペクタイル回帰:}&\quad
\frac{\partial \rho_e(\delta; \tau)}{\partial \delta}= \rho_e^{\prime}(\delta; \tau)=2\left|\tau-\mathbb{I}_{\delta \leq 0}\right| \cdot
\delta
\end{align}


である (ただし$\text{sign}(\cdot)$は符号関数).さらに


\frac{\partial L_{\tau}}{\partial \theta}=\frac{\partial L_{\tau}}{\partial \delta}\frac{\partial \delta(\theta)}{\partial \theta}=-\frac{1}{n} \rho^{\prime}(\delta; \tau) X
 

が成り立つので,$\theta$の更新式は$\theta \leftarrow \theta + \alpha\cdot \dfrac{1}{n} \rho^{\prime}(\delta; \tau) X$と書ける ($\alpha$は学習率である).分位点回帰を単純な勾配法で求める場合,勾配が0となって解が求まらない可能性があるが,目的関数を滑らかにすることで回避できるという研究もある ([Zheng. *IJMLC*. 2011](https://link.springer.com/article/10.1007/s13042-011-0031-2)).この点,Expectileならこの問題を回避できる (?).
\lstinputlisting[language=julia]{./text/bayesian-brain/quantile-expectile-regression/010.jl}
\lstinputlisting[language=julia]{./text/bayesian-brain/quantile-expectile-regression/011.jl}
\lstinputlisting[language=julia]{./text/bayesian-brain/quantile-expectile-regression/012.jl}
\lstinputlisting[language=julia]{./text/bayesian-brain/quantile-expectile-regression/013.jl}
\lstinputlisting[language=julia]{./text/bayesian-brain/quantile-expectile-regression/014.jl}
\lstinputlisting[language=julia]{./text/bayesian-brain/quantile-expectile-regression/015.jl}
\begin{figure}[ht]
	\centering
	\includegraphics[scale=0.8, max width=\linewidth]{./fig/neuron-model/isi/cell015.png}
	\caption{cell015.png}
	\label{cell015.png}
\end{figure}
\subsection{分布型TD学習}
分布型TD学習 (Distributional TD learning) は

Distributional TD learningではRPEの正負に応じて,予測報酬の更新を異なる学習率($\alpha_{i}^{+}, \alpha_{i}^{-}$)を用いて行う. 

 
\begin{cases} V_{i}(x) \leftarrow V_{i}(x)+\alpha_{i}^{+} f\left(\delta_{i}\right) &\text{for }
\delta_{i} \gt 0\\ V_{i}(x) \leftarrow V_{i}(x)+\alpha_{i}^{-} f\left(\delta_{i}\right) &\text{for } \delta_{i} \leq 0 \end{cases} 
 

ここで,シミュレーションにおいては$\alpha_{i}^{+}, \alpha_{i}^{-}\sim U(0,
1)$とする($U$は一様分布).さらにasymmetric scaling factor $\tau_i$を次式により定義する. 

 
\tau_i=\frac{\alpha_{i}^{+}}{\alpha_{i}^{+}+ \alpha_{i}^{-}} 
 

なお,$\alpha_{i}^{+}, \alpha_{i}^{-}\in [0, 1]$より$\tau_i \in
[0,1]$である. 

Classical TD learningとDistributional TD learningにおける各ニューロンのRPEに対する発火率を表現したのが次図となる.
\lstinputlisting[language=julia]{./text/bayesian-brain/quantile-expectile-regression/017.jl}
\lstinputlisting[language=julia]{./text/bayesian-brain/quantile-expectile-regression/018.jl}
\begin{figure}[ht]
	\centering
	\includegraphics[scale=0.8, max width=\linewidth]{./fig/motor-learning/infinite-horizon-ofc/cell018.png}
	\caption{cell018.png}
	\label{cell018.png}
\end{figure}
Classical TD learningではRPEに比例して発火する細胞しかないが,Distributional TD learningではRPEの正負に応じて発火率応答が変化していることがわかる. 特に$\alpha_{i}^{+} \gt \alpha_{i}^{-}$の細胞を\textbf{楽観的細胞 (optimistic cells)}\index{らくかんてきさいぼう (optimistic cells)@楽観的細胞 (optimistic cells)},$\alpha_{i}^{+}\lt
\alpha_{i}^{-}$の細胞を**悲観的細胞 (pessimistic
cells)** と著者らは呼んでいる.実際には2群に分かれているわけではなく,gradientに遷移している.収束する予測価値が細胞ごとに異なることで,$V$には報酬の期待値ではなく複雑な形状の報酬分布が符号化される.その仕組みについて,次項から見ていこう.
\subsubsection{分位数(Quantile)モデルと報酬分布の符号化}

\paragraph{RPEに対する応答がsign関数のモデルと報酬分布の分位点への予測価値の収束}
さて,Distributional RLモデルでどのようにして報酬分布が学習されるかについてみていこう.この項ではRPEに対する応答関数$f(\cdot)$が符合関数(sign function)の場合を考える.結論から言うと,この場合はasymmetric scaling factor $\tau_i$は分位数(quantile)となり,**予測価値
$V_i$は報酬分布の$\tau_i$分位数に収束する**.
    
どういうことかを簡単なシミュレーションで見てみよう.今,報酬分布を平均2, 標準偏差5の正規分布とする (すなわち$r \sim N(2, 5^2)$となります).また,$\tau_i = 0.25, 0.5, 0.75 (i=1,2,3)$とする.このとき,3つの予測価値 $V_i \ (i=1,2,3)$はそれぞれ$N(2, 5^2)$の0.25, 0.5,
0.75分位数に収束する.下図はシミュレーションの結果である.左が$V_i$の変化で,右が報酬分布と0.25, 0.5, 0.75分位数の位置 (黒短線)となっています.対応する分位数に見事に収束していることが分かる.
\lstinputlisting[language=julia]{./text/bayesian-brain/quantile-expectile-regression/021.jl}
\lstinputlisting[language=julia]{./text/bayesian-brain/quantile-expectile-regression/022.jl}
\lstinputlisting[language=julia]{./text/bayesian-brain/quantile-expectile-regression/023.jl}
\begin{figure}[ht]
	\centering
	\includegraphics[scale=0.8, max width=\linewidth]{./fig/bayesian-brain/neural-sampling/cell023.png}
	\caption{cell023.png}
	\label{cell023.png}
\end{figure}
\lstinputlisting[language=julia]{./text/bayesian-brain/quantile-expectile-regression/024.jl}
\lstinputlisting[language=julia]{./text/bayesian-brain/quantile-expectile-regression/025.jl}
\lstinputlisting[language=julia]{./text/bayesian-brain/quantile-expectile-regression/026.jl}
\begin{figure}[ht]
	\centering
	\includegraphics[scale=0.8, max width=\linewidth]{./fig/motor-learning/infinite-horizon-ofc/cell026.png}
	\caption{cell026.png}
	\label{cell026.png}
\end{figure}
ここでoptimisticな細胞($\tau=0.75$)は中央値よりも高い予測価値,pessimisticな細胞($\tau=0.25$)は中央値よりも低い予測価値に収束しています. つまり細胞の楽観度というものは,細胞が期待する報酬が大きいほど上がります.

同様のシミュレーションを今度は200個の細胞 (ユニット)で行います.報酬は0.1, 1, 2 μLのジュースがそれぞれ確率0.3, 0.6, 0.1で出るとします (Extended Data Fig.1と同じような分布にしています).なお,著者らはシミュレーションとマウスに対して\textbf{Variable-magnitude task}\index{Variable-magnitude task}
(異なる量の報酬(ジュース)が異なる確率で出る)と\textbf{Variable-probability task}\index{Variable-probability task} (一定量の報酬がある確率で出る)を行っています.以下はVariable-magnitude taskを行う,ということです.学習結果は次図のようになります.左はGround Truthの報酬分布で,右は$V_i$に対してカーネル密度推定
(KDE)することによって得た予測価値分布です.2つの分布はほぼ一致していることが分かります.
\lstinputlisting[language=julia]{./text/bayesian-brain/quantile-expectile-regression/028.jl}
\lstinputlisting[language=julia]{./text/bayesian-brain/quantile-expectile-regression/029.jl}
\lstinputlisting[language=julia]{./text/bayesian-brain/quantile-expectile-regression/030.jl}
\lstinputlisting[language=julia]{./text/bayesian-brain/quantile-expectile-regression/031.jl}
\begin{figure}[ht]
	\centering
	\includegraphics[scale=0.8, max width=\linewidth]{./fig/neuron-model/hodgkin-huxley/cell031.png}
	\caption{cell031.png}
	\label{cell031.png}
\end{figure}
そして$V_i$の経験累積分布関数(CDF)は$r$のサンプリングしたCDFとほぼ同一となっています (下図左).また,$\tau_i$の関数である$V_i$は\textbf{分位点関数 (quantile function)}\index{ぶんいてんかんすう (quantile function)@分位点関数 (quantile function)} または累積分布関数の逆関数 (inverse cumulative distribution function)となっています
(下図右).右の図を転置すると左の青い曲線とだいたい一致しそうなことが分かります.
\lstinputlisting[language=julia]{./text/bayesian-brain/quantile-expectile-regression/033.jl}
\subsubsection{sign関数を用いたDistributional RLと分位点回帰}

それでは,なぜ予測価値 $V_i$は$\tau_i$ 分位点に収束するのでしょうか.Extended Data Fig.1のように平衡点で考えてもよいのですが,後のために分位点回帰との関連について説明します.

実はDistributional RL (かつ,RPEの応答関数にsign関数を用いた場合)における予測報酬 $V_i$の更新式は,分位点回帰(Quantile
regression)を勾配法で行うときの更新式とほとんど同じです.分位点回帰では$\delta$の関数$\rho_{\tau}(\delta)$を次のように定義します.  \rho_{\tau}(\delta)=\left|\tau-\mathbb{I}_{\delta \leq 0}\right|\cdot |\delta|=\left(\tau-\mathbb{I}_{\delta
\leq 0}\right)\cdot \delta  そして,この関数を最小化することで回帰を行います.ここで$\tau$は分位点です.また$\delta=r-V$としておきます.今回,どんな行動をしても未来の報酬に影響はないので$\gamma=0$としています.\url{br/}
\url{br/}
ここで,  \frac{\partial \rho_{\tau}(\delta)}{\partial \delta}=\rho_{\tau}^{\prime}(\delta)=\left|\tau-\mathbb{I}_{\delta \leq 0}\right| \cdot \operatorname{sign}(\delta)  なので,$r$を観測値とすると, 
\frac{\partial \rho_{\tau}(\delta)}{\partial V}=\frac{\partial \rho_{\tau}(\delta)}{\partial \delta}\frac{\partial \delta(V)}{\partial V}=-\left|\tau-\mathbb{I}_{\delta \leq 0}\right| \cdot
\operatorname{sign}(\delta)  となります.ゆえに$V$の更新式は  V \leftarrow V - \beta\cdot\frac{\partial \rho_{\tau}(\delta)}{\partial V}=V+\beta \left|\tau-\mathbb{I}_{\delta \leq 0}\right| \cdot
\operatorname{sign}(\delta)  です.ただし,$\beta$はベースラインの学習率です.個々の$V_i$について考え,符号で場合分けをすると
 \begin{cases} V_{i} \leftarrow V_{i}+\beta\cdot |\tau_i|\cdot\operatorname{sign}\left(\delta_{i}\right)
&\text { for } \delta_{i}>0\\ V_{i} \leftarrow V_{i}+\beta\cdot |\tau_i-1|\cdot\operatorname{sign}\left(\delta_{i}\right) &\text { for } \delta_{i} \leq 0 \end{cases}  となります.$0 \leq
\tau_i \leq 1$であり,$\tau_i=\alpha_{i}^{+} / \left(\alpha_{i}^{+} + \alpha_{i}^{-}\right)$であることに注意すると上式は次のように書けます.  \begin{cases} V_{i} \leftarrow V_{i}+\beta\cdot
\frac{\alpha_{i}^{+}}{\alpha_{i}^{+}+\alpha_{i}^{-}}\cdot\operatorname{sign}\left(\delta_{i}\right) &\text { for } \delta_{i}>0\\ V_{i} \leftarrow V_{i}+\beta\cdot
\frac{\alpha_{i}^{-}}{\alpha_{i}^{+}+\alpha_{i}^{-}}\cdot\operatorname{sign}\left(\delta_{i}\right) &\text { for } \delta_{i} \leq 0 \end{cases}  これは前節で述べたDistributional
RLの更新式とほぼ同じです.いくつか違う点もありますが,RPEが正の場合と負の場合に更新される値の比は同じとなっています.

このようにRPEの応答関数にsign関数を用いた場合,報酬分布を上手く符号化することができます.しかし実際のドパミンニューロンはsign関数のような生理的に妥当でない応答はせず,RPEの大きさに応じた活動をします.そこで次節ではRPEの応答関数を線形にしたときの話をします.
\subsubsection{Expectile モデルとドパミンニューロンからの報酬分布のDecoding}

\subsubsection{RPEに対する応答が線形なモデルとExpectile回帰}
節の最後で述べたようにドパミンニューロンの活動はsign関数ではなく線形な応答をする,とした方が生理学的に妥当である (発火率を表現するならば$f(\delta)=c+\delta\quad(c > 0)$とした方が良いだろうが).それでは予測価値の更新式を 

 
\begin{cases} V_{i}(x) \leftarrow V_{i}(x)+\alpha_{i}^{+}
\delta_{i} &\text{for } \delta_{i} \gt 0\\ V_{i}(x) \leftarrow V_{i}(x)+\alpha_{i}^{-} \delta_{i} &\text{for } \delta_{i} \leq 0 \end{cases} 


とした場合は,分位点回帰ではなく何に対応するのだろうか.結論から言えば,この場合は **エクスペクタイル回帰(Expectile
regression)\textbf{ と同じになる.expectileという用語自体はexpectationとquantileを合わせたような概念,というところから来ている.}\index{ とおなじになる.expectileというようごじたいはexpectationとquantileをあわせたようながいねん,というところからきている.@ と同じになる.expectileという用語自体はexpectationとquantileを合わせたような概念,というところから来ている.}中央値(median)に対する分位数(quantile)が,平均(mean)あるいは期待値(expectation)に対するexpectileの関係と同じ** であると捉えると良いです.
もう少し言えば,前者は誤差のL1ノルム, 後者はL2ノルムの損失関数を最小化することにより得られる.

分位点回帰で用いた損失関数は


\rho_{\tau}(\delta)=\left|\tau-\mathbb{I}_{\delta \leq 0}\right|\cdot |\delta|


だったが,最後の$|\delta|$を$\delta^2$として, 


\rho^E_{\tau}(\delta)=\left|\tau-\mathbb{I}_{\delta \leq
0}\right|\cdot \delta^2


とする.これを微分すれば 

 
\frac{\partial \rho^E_{\tau}(\delta)}{\partial \delta}=\rho_{\tau}^{E\prime}(\delta)=2 \cdot \left|\tau-\mathbb{I}_{\delta \leq 0}\right| \cdot \delta 


となり,上記の予測価値の更新式がExpectile回帰の損失関数から導けることが分かる.

\paragraph{報酬分布のデコーディング (decoding)}
それで,RPEの応答を線形とした場合は報酬分布を上手く学習できるのかという話ですが,実はRPEの応答をsign関数とした場合と同じように学習後の予測価値の分布を求めても報酬分布は復元されません (簡単な修正で確認できます).そこで報酬分布をデコーディングする方法を考えます.

デコーデイングには各細胞が学習した予測価値(またはreversal points) $V_i$, asymmetries $\tau_i$, および報酬分布(ただし報酬の下限と上限からの一様分布)からのサンプル $z_m (m=1,2,\cdots,
M)$を用います.$N$を推定する$V_i$の数,$M=100$を1つの報酬サンプル集合$\{z_m\}$内の要素数としたとき,次の損失関数を最小にする集合$\{z_m\}$を求めます.  \mathcal{L}(z, V, \tau)=\frac{1}{M} \sum_{m-1}^{M} \sum_{n=1}^{N}\left|\tau_{n}-\mathbb{I}_{z_{m} \leq
V_{n}}\right|\left(z_{m}-V_{n}\right)^{2}  ここで,集合$\{z_m\}$は20000回サンプリングするとします.損失関数$\mathcal{L}$を最小化する集合の分布が推定された報酬分布となっているので,それをplotします.以下はその結果とコードです
(このコードはほとんど著者実装のままです).灰色が元の報酬分布で,紫がデコーデイングされた分布です.完全とはいきませんが,ある程度は推定できていることが分かります.
\subsection{参考文献}
\begin{itemize}
\item https://en.wikipedia.org/wiki/Quantile_regression
\item Das, K., Krzywinski, M. & Altman, N. Quantile regression. Nat Methods 16, 451–452 (2019) doi:[10.1038/s41592-019-0406-y](https://www.nature.com/articles/s41592-019-0406-y)
\item Quantile and Expectile Regressions ([pdf](https://freakonometrics.hypotheses.org/files/2017/05/erasmus-1.pdf))
\item Dabney, W., Kurth-Nelson, Z., Uchida, N. *et al.* A distributional code for value in dopamine-based reinforcement learning. *Nature* (2020). [https://doi.org/10.1038/s41586-019-1924-6](https://www.nature.com/articles/s41586-019-1924-6)
\item Watabe-Uchida, M. et al. Whole-Brain Mapping of Direct Inputs to Midbrain Dopamine Neurons. Neuron 74, 5, 858 - 873 (2012). [https://doi.org/10.1016/j.neuron.2012.03.017](https://www.cell.com/neuron/fulltext/S0896-6273(12)00281-4)[ ](https://www.cell.com/neuron/fulltext/S0896-6273(12)00281-4)
\item Eshel, N., Tian, J., Bukwich, M. *et al.* Dopamine neurons share common response function for reward prediction error. *Nat Neurosci* \textbf{19,}\index{19,} 479–486 (2016). [https://doi.org/10.1038/nn.4239](https://www.nature.com/articles/nn.4239)
\item Schultz, W., Dayan, P., Montague, P.R. A neural substrate of prediction and reward. *Science*. 275, 1593-9 (1997). [doi:10.1126/science.275.5306.1593](https://science.sciencemag.org/content/275/5306/1593.long)
\item Chang, C., Esber, G., Marrero-Garcia, Y. *et al.* Brief optogenetic inhibition of dopamine neurons mimics endogenous negative reward prediction errors. *Nat Neurosci* \textbf{19,}\index{19,} 111–116 (2016) [doi:10.1038/nn.4191](https://www.nature.com/articles/nn.4191)  
\item Bayer, H.M., Lau, B., Glimcher, P.W. Statistics of midbrain dopamine neuron spike trains in the awake primate. *J Neurophysiol*. \textbf{98}\index{98}(3):1428-39 (2007). [https://doi.org/10.1152/jn.01140.2006](https://www.physiology.org/doi/full/10.1152/jn.01140.2006)
\item Eshel, N., Bukwich, M., Rao, V. *et al.* Arithmetic and local circuitry underlying dopamine prediction errors. *Nature* \textbf{525,}\index{525,} 243–246 (2015). [https://doi.org/10.1038/nature14855](https://www.nature.com/articles/nature14855)
\end{itemize}

\clearpage
\addcontentsline{toc}{chapter}{\bibname}
\bibliography{参考文献} 

\clearpage
\addcontentsline{toc}{chapter}{\indexname}
\printindex
\end{document}