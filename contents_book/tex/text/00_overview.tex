第1章:はじめに
\begin{itemize}
\item 本書の目的と構成
\item Julia言語の使用法
\item 基礎的数学
\end{itemize}
第2章:発火率モデルと局所学習則
\begin{itemize}
\item 神経細胞の生理
\item 発火率モデルとHebb則 ※
\item 線形回帰※
\item ロジスティック回帰とパーセプトロン※
\item 主成分分析
\item 独立成分分析
\item 自己組織化マップと競合学習
\end{itemize}
第3章:エネルギーベースモデル
\begin{itemize}
\item Hopfield モデル
\item Boltzmann マシン
\item スパース符号化モデル
\item 予測符号化モデル
\end{itemize}
第4章:ニューラルネットワークと貢献度分配問題
\begin{itemize}
\item 貢献度分配問題と誤差逆伝播法
\item 非対称な逆向き投射による誤差伝播
\item 予測符号化による活動と結合の共調整
\item 摂動を用いた学習則
\end{itemize}
第5章:再帰型ニューラルネットワークと経時的貢献度分配問題
\begin{itemize}
\item 経時的誤差逆伝播法 (BPTT)
\item 実時間リカレント学習 (RTRL)
\item 適格度トレースによるRTRLの近似※
\end{itemize}
第6章:ニューロンとシナプスの生物物理学的モデル
\begin{itemize}
\item コンダクタンスベースモデル※
\end{itemize}
  - FitzHugh-Nagumoモデル
\begin{itemize}
\item 積分発火モデル※
\item Izhikevich モデル※
\item ゲイン調整と四則演算
\item マルチコンパートメントモデル※
\item Inter-spike interval モデル (点過程モデルの方がいいか※)
\item シナプスの生理とCurrent/Conductance-based シナプス
\item 指数関数型シナプスモデル
\item 動力学シナプスモデル
\item 短期的シナプス可塑性
\end{itemize}
第7章:スパイキングニューラルネットワークの学習則
\begin{itemize}
\item ランダム回路網の構成
\item STDP則と競合学習
\item 代理勾配法
\item 適格性伝播法※
\end{itemize}
第8章:リザバーコンピューティング
\begin{itemize}
\item エコーステートネットワーク※
\item FORCE法 (rate, spiking)※
\end{itemize}
第9章:神経回路によるベイズ推論
\begin{itemize}
\item ベイズ脳仮説と不確実性の表現
\item ベイズ線形回帰
\item 確率的集団符号化
\item マルコフ連鎖モンテカルロ法
\item 神経サンプリング
\end{itemize}
第10章:運動学習と最適制御
\begin{itemize}
\item 躍度最小モデル
\item 終点誤差分散最小モデル
\item 最適フィードバック制御モデル
\item 無限時間最適フィードバック制御モデル
\end{itemize}
第11章:強化学習
\begin{itemize}
\item 強化学習とマルコフ決定過程
\item 状態価値の推定
\item 価値ベース法
\item 方策ベース法
\item 分布型強化学習
\item 内発的動機付け
\end{itemize}
第12章:神経・神経回路の発生・構造的モデル※
\begin{itemize}
\item 神経突起の成長モデル
\item シナプス結合強度の不均一性
\end{itemize}
