\documentclass[titlepage]{ltjsbook}
\usepackage[
  paperheight=232truemm, paperwidth=182truemm,
  top=20truemm, bottom=15truemm, inner=15truemm, outer=15truemm
  ]{geometry}

%\documentclass[tombow, paper={182truemm, 232truemm}, titlepage]{ltjsbook}
\usepackage{amsmath}
\usepackage{amsfonts}
\usepackage{amssymb}
\usepackage{mathtools}
\usepackage{mathrsfs}

\usepackage{textgreek}
\usepackage[luatex]{graphicx} 
\usepackage[svgnames]{xcolor}
\usepackage{sty/julia-syntax-highlighting} % 
\usepackage{sty/indexing} % 

\usepackage[export]{sty/adjustbox} % added

\usepackage{fancyhdr}
\pagestyle{fancy}
\fancyfoot{}
\fancyhead[RO, LE]{\thepage}
\fancyhead[LO]{\nouppercase{\leftmark}}
\fancyhead[RE]{\nouppercase{\rightmark}}

%\renewcommand{\chaptermark}[1]{\markboth{#1}{} }
\renewcommand{\chaptermark}[1]{\markboth{第\ \thechapter\ 章. ~#1}{}}
% \renewcommand{\chaptermark}[1]{\markboth{\MakeUppercase{第\chaptername \thechapter 章.\ #1}}{}}
% \renewcommand{\headrulewidth}{0pt}

\usepackage{hyperref}

% https://ja.overleaf.com/learn/latex/Bibliography_management_with_bibtex
\usepackage[
    backend=biber,
    bibencoding=utf8,
    style=authoryear-comp, 
    url=false,
    isbn=true,
    doi=true,
    natbib=true, 
    alldates=year,
    maxcitenames=2,
    uniquelist=false, 
    sorting=nty,
    sortcites=true,
    giveninits=true,
    terseinits=false,
    refsegment=chapter
]{biblatex}

% \addbibresource{bibfiles/appendix-references.bib}
% \addbibresource{bibfiles/bayesian-brain-references.bib}
% \addbibresource{bibfiles/energy-based-model-references.bib}
% \addbibresource{bibfiles/introduction-references.bib}
% \addbibresource{bibfiles/local-learning-rule-references.bib}
% \addbibresource{bibfiles/motor-learning-references.bib}
% \addbibresource{bibfiles/neuron-model-references.bib}
% \addbibresource{bibfiles/neuronal-computation-references.bib}
% \addbibresource{bibfiles/reinforcement-learning-references.bib}
% \addbibresource{bibfiles/solve-credit-assignment-problem-references.bib}
% \addbibresource{bibfiles/synapse-model-references.bib}
\addbibresource{../references/08_motor-learning.bib}

\DeclareNameAlias{author}{last-first}
\AtEveryBibitem{\clearlist{language}}
\renewbibmacro{in:}{}

% https://stackoverflow.com/questions/69682457/extended-links-in-citations
% \makeatletter
% \renewbibmacro*{cite}{%
%   \printtext[bibhyperref]{\iffieldundef{shorthand}
%     {\ifthenelse{\ifnameundef{labelname}\OR\iffieldundef{labelyear}}
%        {\usebibmacro{cite:label}%
%         \setunit{\printdelim{nonameyeardelim}}}
%        {\printnames{labelname}%
%         \setunit{\printdelim{nameyeardelim}}}%
%      \usebibmacro{cite:labeldate+extradate}}
%     {\usebibmacro{cite:shorthand}}}}
% \makeatother

\newcommand{\jl}{\lstinline[language=julia]}

\title{\Huge \textbf{Juliaで作って学ぶ計算論的神経科学}}
\author{\huge 山本 拓都}
\date{\huge \today} 

\begin{document}
%\maketitle
\setcounter{tocdepth}{2}
\tableofcontents
\clearpage
\chapter{運動制御}
\section{終点誤差分散最小モデル}
HarrisおよびWolpertは制御信号の大きさに従い,ノイズが生じるモデルを提案した.さらにこのモデルにおいて,状態の分散が可能な限り小さくなるような制御信号を求めた.これを終点誤差分散最小モデル (minimum-variance model) と呼ぶ \citep{Harris1998-gj}.

終点誤差分散最小モデルは状態$\mathbf{x}_t\in \mathbb{R}^n$, 制御信号$\mathbf{u}_t \in \mathbb{R}^p$とし,$\mathbf{A}\in \mathbb{R}^{n\times n}$, $\mathbf{B}\in \mathbb{R}^{n \times p}$とすると,
\begin{equation}
\mathbf{x}_{t+1} = \mathbf{A} \mathbf{x}_t + \mathbf{B}\mathbf{u}_t (1+\boldsymbol{\xi}_t)
\end{equation}
と表せる.ただし,$\boldsymbol{\xi}_t \sim \mathcal{N}(0, k\mathbf{I})\ (k>0)$である.このため,$\mathbf{u}_t (1+\xi_t)$の平均は $\mathbf{u}_t$, 分散共分散行列は $k\mathbf{u}_t \mathbf{u}_t^\top$となる.$\mathbf{x}_t$を過去の状態 $\mathbf{x}_{t'}\ (t'=0, \ldots, t-1)$で表すと,
\begin{align}
\mathbf{x}_{t} &= \mathbf{A} \mathbf{x}_{t-1} + \mathbf{B}\mathbf{u}_{t-1} (1+\boldsymbol{\xi}_{t-1})\\
&=\mathbf{A}^2 \mathbf{x}_{t-2} + \mathbf{A}\mathbf{B}\mathbf{u}_{t-2} (1+\boldsymbol{\xi}_{t-2}) + \mathbf{B}\mathbf{u}_{t-1} (1+\boldsymbol{\xi}_{t-1})\\
&=\cdots\\
&=\mathbf{A}^{t} \mathbf{x}_{0} + \sum_{t'=0}^{t-1} \mathbf{A}^{t-t'-1}\mathbf{B}\mathbf{u}_{t'} (1+\boldsymbol{\xi}_{t'})
\end{align}
となるので,$\mathbf{x}_t$の平均と分散共分散行列はそれぞれ,
\begin{align}
\mathbb{E}\left[\mathbf{x}_{t}\right]&=\mathbf{A}^{t} \mathbf{x}_{0}+\sum_{t'=0}^{t-1} \mathbf{A}^{t-t'-1} \mathbf{B} \mathbf{u}_{t'}\\
\operatorname{Cov}\left[\mathbf{x}_{t}\right]&=k\sum_{t'=0}^{t-1}\left(\mathbf{A}^{t-t'-1} \mathbf{B}\right) \mathbf{u}_{t'} \mathbf{u}_{t'}^\top \left(\mathbf{A}^{t-t'-1} \mathbf{B}\right)^{\top}
\end{align}
となる.制御信号の時系列 $\{\mathbf{u}_t\}$が与えられている場合,状態 $\mathbf{x}_t$の平均と分散共分散行列は,$\mathbb{E}\left[\mathbf{x}_{0}\right]=\mathbf{x}_0, \operatorname{Cov}\left[\mathbf{x}_{0}\right]=\mathbf{0}\in\mathbb{R}^{n\times n}$として,
\begin{align}
\mathbb{E}\left[\mathbf{x}_{t}\right] &=\mathbf{A}\mathbb{E}\left[\mathbf{x}_{t-1}\right] + \mathbf{B} \mathbf{u}_{t-1}\\
\operatorname{Cov}\left[\mathbf{x}_{t}\right]&=\mathbf{A}\operatorname{Cov}\left[\mathbf{x}_{t-1}\right]\mathbf{A}^\top + k\mathbf{B} \mathbf{u}_{t-1} \mathbf{u}_{t-1}^\top \mathbf{B}^\top
\end{align}
と逐次的に計算が可能である.

このようなモデルにおいて,次の条件を満たす制御信号を求めることを考える.まず,初期状態を$\mathbf{x}_0$, 目標状態を $\mathbf{x}_f$とする.また,運動時間を $T_m$, 運動後時間 (post-movement time) を $T_p$とする.よって1試行にかかる時間は$T:=T_m + T_p$となる.以下では時間は離散化されており,$T_m, T_p, T$は自然数を取るとする.運動後の停留期間である時刻 $T_m\leq t \leq T$において,状態の平均が目標状態と一致する,すなわち
\begin{equation}
\mathbb{E}\left[\mathbf{x}_{t}\right] = \mathbf{x}_f\quad (T_m\leq t \leq T)
\end{equation}
を満たし,位置の分散
\begin{equation}
\mathcal{F}=\sum_{i\in \mathrm{Pos.}}\left[\sum_{t=T_m}^{T} \operatorname{Cov}\left[\mathbf{x}_{t}\right]\right]_{i, i}
\end{equation}
を最小にするような制御信号 $\mathbf{u}_t$を求める.ただし,$\mathrm{Pos.}$は状態 $\mathbf{x}_t$の中で位置を表す次元の番号 (インデックス) の集合を意味し,$[\cdot]_{i,i}$は行列の$(i,i)$成分を取り出す操作を意味する.この最適化問題を(躍度最小モデルの際にも用いた)等式制約下の二次計画問題で解くことを考える.二次計画問題で解くには,最小化する目的関数と等式制約をそれぞれ
\begin{align}
&{\text{目的関数}}\quad {\frac {1}{2}}\mathbf{u}^\top \mathbf{P}\mathbf{u} +\mathbf{q} ^{\top}\mathbf{u}\\
&{\text{等式制約}}\quad \mathbf{C}\mathbf{u} =\mathbf{d}
\end{align}
の形にする必要がある.ただし,$\mathbf{P}, \mathbf{C}$は行列,$\mathbf{q}, \mathbf{d}$はベクトルである.簡単のため,$p=1$の場合を考慮すると,$\mathbf{u}_t \to u_{t} \in \mathbb{R}$となる.状態信号の時系列をベクトル化し,$\mathbf{u}=[u_t]_{t=0, \ldots, T-1} \in \mathbf{R}^{T}$とする.また,後の結果に影響しないため,$k=1$とする.さらに位置のインデックスを$\mathrm{Pos.}=\{1\}$のみとする.この条件の下,式変形を行うと,目的関数 $\mathcal{F}$は
\begin{align}
\mathcal{F}=\left[\sum_{t=T_m}^{T} \operatorname{Cov}\left[\mathbf{x}_{t}\right]\right]_{1,1}
&=\left[\sum_{t=T_m}^{T}\sum_{t'=0}^{t-1}u_{t'}^2\left(\mathbf{A}^{t-t'-1} \mathbf{B}\right) \left(\mathbf{A}^{t-t'-1} \mathbf{B}\right)^{\top}\right]_{1,1}\\
&=\sum_{t'=0}^{T-1} u_{t'}^2 \sum_{t=\max(t'+1, T_m)}^{T} \left[\left(\mathbf{A}^{t-t'-1} \mathbf{B}\right)\left(\mathbf{A}^{t-t'-1} \mathbf{B}\right)^{\top} \right]_{1,1}
\end{align}
と書ける.最後の式変形は $u_{t'}^2$を二重総和の外に出すために行った.この操作は次の図における横方向と縦方向の和の順番を交換することに該当する.

ここで $V_{t'}:=\sum_{t=\max(t'+1, T_m)}^{T} \left[\left(\mathbf{A}^{t-t'-1} \mathbf{B}\right)\left(\mathbf{A}^{t-t'-1} \mathbf{B}\right)^{\top} \right]_{1,1}$とすると,$\mathbf{P}=\mathrm{diag}(V_0, \ldots, V_{T-1})\in \mathbf{R}^{T\times T}$および $\mathbf{q}=\mathbf{0} \in \mathbf{R}^{T}$と置くことで,$\mathcal{F}=\mathbf{u}^\top \mathbf{P}\mathbf{u}+\mathbf{q} ^{\top}\mathbf{u}$と書ける.この場合,第2項は0であるので,第1項の係数は結果に影響しない.

次に等式制約を求める.$\mathbb{E}\left[\mathbf{x}_{t}\right] = \mathbf{x}_f\quad (T_m\leq t \leq T)$を変形すると,
\begin{equation}
\sum_{t'=0}^{t-1} \mathbf{A}^{t-t'-1} \mathbf{B} u_{t'}=\mathbf{x}_f-\mathbf{A}^{t} \mathbf{x}_{0}
\end{equation}
となる.左辺について
\begin{equation}
\mathbf{C}_{(t-T_m)n+1:(t-T_m+1)n+1,\ t'}=
\begin{cases}
    \mathbf{A}^{t-t'-1} \mathbf{B} & (0\leq t'\leq t-1) \\
    \mathbf{0} & (t\leq t'\leq T-1)
\end{cases}\in \mathbb{R}^n 
\end{equation}
および,右辺について
\begin{equation}
\mathbf{d}_{(t-T_m)n+1:(t-T_m+1)n+1}=\mathbf{x}_f-\mathbf{A}^{t} \mathbf{x}_{0} \in \mathbb{R}^n 
\end{equation}
とすることで,等式制約が書き下せる.ただし,$[\cdot]_{i:j}$はベクトルあるいは行列の $i$番目から $j$番目までを取り出す操作を意味する.このように,$\mathbf{P}, \mathbf{q}, \mathbf{C}, \mathbf{d}$を設定すると,等式制約下の二次計画問題を用いて $\mathbf{u}$を求めることができる.

\printbibliography[segment=\therefsegment,heading=subbibliography,title={参考文献}]
\addcontentsline{toc}{section}{参考文献}
\end{document}